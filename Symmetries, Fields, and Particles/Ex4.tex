\documentclass{article}

\usepackage{header}
%%%%%%%%%%%%%%%%%%%%%%%%%%%%%%%%%%%%%%%%%%%%%%%%%%%%%%%%
%Preamble

\title{Symmetries, Fields, and Particles Example Sheet 4}
\author{Linden Disney-Hogg}
\date{December 2018}

%%%%%%%%%%%%%%%%%%%%%%%%%%%%%%%%%%%%%%%%%%%%%%%%%%%%%%%%
%%%%%%%%%%%%%%%%%%%%%%%%%%%%%%%%%%%%%%%%%%%%%%%%%%%%%%%%
\begin{document}

\maketitle
\tableofcontents

%%%%%%%%%%%%%%%%%%%%%%%%%%%%%%%%%%%%%%%%%%%%%%%%%%%%%%%%
\section{Question 1} 
The Cartan matrix for $B_2$ is 
\eq{
A = \begin{pmatrix} 2 & -2 \\ -1 & 2 \end{pmatrix}
}
so \eq{
w_1 = \alpha_1 + \alpha_2 \\
w_2 = \frac{1}{2} \alpha_1 + \alpha_2 
}
The fundamental representation is $R_{(1,0)}$
Following the algorithm to generate the roots finds 
\eq{
R_{(1,0)} = \set{(1,0),(-1,2),(0,0), (1,-2),(-1,0)} \\
R_{(0,1)} = \set{(0,1),(1,-1),(-1,1), (0,-1)}
}
For the adjoint representation $\Lambda = 2w_2$ (as adjoint representation is the highest weight representation) giving 
\[
R_{(0,2)} = \set{(0,2),(1,0),(-1,2),(2,-2),(0,0),(1,-2),(-2,2),(-1,),(0,-2)}
\]
Now the adjoint representation should have dimension 10 and we have seen $|R_{(0,2)}|=9$. Hence one root must be degenerate. However, as 
\eq{
ad_{H_i}(H_j) = 0 
}
and $\dim CSA  = 2$ it can be seen that the degeneracy is only for the zero root and so done. 

%%%%%%%%%%%%%%%%%%%%%%%%%%%%%%%%%%%%%%%%%%%%%%%%%%%%%%%%
%%%%%%%%%%%%%%%%%%%%%%%%%%%%%%%%%%%%%%%%%%%%%%%%%%%%%%%%
\section{Question 2}
The idea is to demonstrate that in the basis shown, the indices in weight diagram exactly correspond to the isocharge and hyperspin. 
%%%%%%%%%%%%%%%%%%%%%%%%%%%%%%%%%%%%%%%%%%%%%%%%%%%%%%%%
%%%%%%%%%%%%%%%%%%%%%%%%%%%%%%%%%%%%%%%%%%%%%%%%%%%%%%%%
\section{Question 3}
Consider $A_2$ with Cartan matrix 
\eq{
A = \begin{pmatrix} 2 & -1 \\ -1 & 2 \end{pmatrix}
}
so 
\eq{
w_1 = \frac{1}{3}(2\alpha_{(1)} + \alpha_{(2)})\\
w_2 = \frac{1}{3}(\alpha_{(1)} + 2\alpha_{(2)}) 
}
giving 
\eq{
R_{(3,0)} = \set{(3,0),(1,1),(-1,2),(-3,3),(2,-1),(0,0),(-2,1),(1,-2),(-1,-1),(0,-3)}
}
This can be spotted to be 
\eq{
R_{(3,0)} = = \bm{4} \oplus \bm{3} \oplus \bm{2} \oplus \bm{1}
}

%%%%%%%%%%%%%%%%%%%%%%%%%%%%%%%%%%%%%%%%%%%%%%%%%%%%%%%%
%%%%%%%%%%%%%%%%%%%%%%%%%%%%%%%%%%%%%%%%%%%%%%%%%%%%%%%%
\section{Question 4}
%%%%%%%%%%%%%%%%%%%%%%%%%%%%%%%%%%%%%%%%%%%%%%%%%%%%%%%%
\subsection{(i)}
Take $A_2 = \mf{su}(2)_\mbb{C}$. Then 

\eq{
\bm{3}\otimes\bar{\bm{3}} &= \set{(1,0),(-1,1),(0,-1)} \times \set{(0,1),(1,-1),(-1,0)} \\ &=\set{(1,1,),(2,-1),(0,0)^3,(-1,2),(-2,1),(0,-2),(-1,1),(1,-2)}
}
then look at the rep with heighest weight (1,1) gets 
\eq{
\set{(1,1),(-1,2),(2,-1),(0,0)^2,(1,-2),(-2,1),(-1,1)}
}
so 
\eq{
\bm{3}\otimes\bar{\bm{3}} = 
\set{(1,1),(-1,2),(2,-1),(0,0)^2,(1,-2),(-2,1),(-1,1)}
 \cup \set{(0,0)} \\
\Rightarrow \bm{3}\otimes\bar{\bm{3}} = \bm{8} \oplus \bm{1}
}
%%%%%%%%%%%%%%%%%%%%%%%%%%%%%%%%%%%%%%%%%%%%%%%%%%%%%%%%
\subsection{(ii)}
Repeating for $\bm{3} \otimes \bm{3} \otimes \bm{3}$ gives
\eq{
\bm{3} \otimes \bm{3} &= \set{(1,0),(-1,1),(0,-1)} \times \set{(1,0),(-1,1),(0,-1)}  \\
&= \set{(2,0),(0,1)(1,-1),(0,1),(-2,2),(-1,0),(1,-1),(-1,0),(0,-2)}
}
Now 
\eq{
\bm{6} = \set{(2,0),(0,1),(-2,2),(1,-1),(-1,0),(0,-2)} \\
\bar{\bm{3}} = \set{(0,1),(1,-1),(-1,0)}
}
so 
\eq{
\bm{3} \otimes \bm{3} \otimes \bm{3} &= \bm{3} \otimes (\bm{6} \oplus \bar{\bm{3}} ) \\
&= (\bm{3} \otimes \bm{6}) \oplus (\bm{3} \otimes \bar{\bm{3}})
}
Now it can be found $\bm{3}\otimes\bm{6} = \bm{10} \oplus \bm{8}$ so 
\[
\bm{3} \otimes \bm{3} \otimes \bm{3} = \bm{10} \oplus \bm{8} \oplus \bm{8} \oplus \bm{1} 
\]

%%%%%%%%%%%%%%%%%%%%%%%%%%%%%%%%%%%%%%%%%%%%%%%%%%%%%%%%
%%%%%%%%%%%%%%%%%%%%%%%%%%%%%%%%%%%%%%%%%%%%%%%%%%%%%%%%
\section{Question 5}
The non-trivial lowest dimension representation for $B_2$ is 
\eq{
R_{(0,1)}^2 = \sum{(0,2),(1,0),(-1,2),(0,0),(1,0),(2,-2),(0,0),(1,-2),(-1,2),(0,0),(-2,2),(-1,0),(0,0),(1,-2),(-1,0),(0,-2)}
}
and 
\eq{
R_{(0,2)} = \set{(0,2),(1,0),(2,-2),(-1,2),(0,0)^2,(-2,2),(1,-2),(0,-2),(-1,0)}
}
giving 
\eq{
R_{(0,1)}^2\setminus R_{(0,2)} = \set{(1,0),(-1,2),(0,0),(1,-2),(-1,0),(0,0)} = \bm{5} \oplus \bm{1}
}
so 
\[
\bm{4} \otimes \bm{4} = \bm{10} \oplus \bm{5} \oplus \bm{1}
\]

%%%%%%%%%%%%%%%%%%%%%%%%%%%%%%%%%%%%%%%%%%%%%%%%%%%%%%%%
%%%%%%%%%%%%%%%%%%%%%%%%%%%%%%%%%%%%%%%%%%%%%%%%%%%%%%%%
\section{Question 6}
Consider the gauge group $G$ with gague field 
\eq{
A_\mu : \mbb{R}^{3,1} \to \mf{g}
}
that transforms as 
\eq{
A_\mu \to A_\mu^\prime = g A_\mu g^{-1} - (\del_\mu g)g^{-1}
}
for 
\eq{
g : \mbb{R}^{3,1} \to G
}
Take $\eps \ll 1$. 
\eq{
\exp(\eps X ) \in G \text{ for } X \in \mf{g} \\
\frac{d}{d\eps} \exp(\eps X ) |_{\eps=0} = X \in \mf{g} \\
\frac{d}{d\eps} g\exp(\eps X )g^{-1} |_{\eps=0} = gXg^{-1} \in \mf{g} \\
}
Let $g(\eps)\in G$ with $g(0)=g_0$ then  
\eq{
\frac{d}{d\eps} g(\eps)g^{-1}_0 |_{\eps=0} = g^\prime(0)g_0^{-1} \in \mf{g}
}
So 
\eq{
A_\mu^\prime &= \exp(\eps X) A_\mu \exp(-\eps X) - \eps(\del_\mu X) \exp(\eps X)\exp(-\eps X) \\
&= A_\mu - \eps (\comm[A_\mu]{X}-\del_\mu X) \text{ to order } \eps
}

%%%%%%%%%%%%%%%%%%%%%%%%%%%%%%%%%%%%%%%%%%%%%%%%%%%%%%%%
%%%%%%%%%%%%%%%%%%%%%%%%%%%%%%%%%%%%%%%%%%%%%%%%%%%%%%%%
\section{Question 7}
Fundamental scalar fields transform as 
\eq{
\phi_F \to \phi_F^\prime = g\phi_F \\
\phi_A \to \phi_A^\prime = g\phi_A g^{-1}
}
Now from lectures 
\eq{
D_\mu = \del_\mu + R(A_\mu) 
}
in some rep and $D_\mu \phi$ transforms as $\phi$. Hence 
\eq{
D_\mu^F \phi_F =\del_\mu \phi_F + A_\mu \phi_F \\
D_\mu^A \phi_A  = \del_\mu \phi_A + \comm[A_\mu]{\phi_A}
}
So under a gauge transform 
\eq{
(D_\mu \phi_F)^\dagger D^\mu \phi_F \to &(g D_\mu \phi_F)^\dagger (g D_\mu \phi_F) \\
&= (D_\mu \phi_F)^\dagger g^\dagger g D_\mu \phi_F \\
&= (D_\mu \phi_F)^\dagger D_\mu \phi_F
}
So 
\eq{
\tr (D_\mu \phi_A)^\dagger D^\mu \phi_A  \to &\tr (gD_\mu \phi_A g^{-1})^\dagger g D^\mu \phi_A g^{-1} \\
&= \tr g D_\mu \phi_A^\dagger g^\dagger g D_\mu \phi_A g^{-1} 
&= \tr (D_\mu \phi_A)^\dagger D^\mu \phi_A 
}

%%%%%%%%%%%%%%%%%%%%%%%%%%%%%%%%%%%%%%%%%%%%%%%%%%%%%%%%
%%%%%%%%%%%%%%%%%%%%%%%%%%%%%%%%%%%%%%%%%%%%%%%%%%%%%%%%
\section{Question 8}
\eq{
\comm[D_\mu]{D_\nu} \phi_A &= \comm[\del_\mu A_\mu - \del_\nu A_\mu]{ \phi_A} + \comm[A_\mu]{\comm[A_\nu]{\phi_A}} -  \comm[A_\nu]{\comm[A_\mu]{\phi_A}} \\
&= \comm[\del_\mu A_\mu - \del_\nu A_\mu + {\comm[A_\mu]{A_\nu}}]{\phi_A} \\
&= \comm[F_{\mu\nu}]{\phi_A}
}
Now under gauge transform 
\eq{
\comm[F_{\mu\nu}]{\phi_A} &= \comm[D_\mu]{D_\nu} \phi_A  \\
&\to \comm[F^\prime_{\mu\nu}]{g \phi_A g^{-1}}
}

%%%%%%%%%%%%%%%%%%%%%%%%%%%%%%%%%%%%%%%%%%%%%%%%%%%%%%%%
%%%%%%%%%%%%%%%%%%%%%%%%%%%%%%%%%%%%%%%%%%%%%%%%%%%%%%%%
\section{Question 9}
Recall 
\eq{
\kappa(X,Y) = \tr ad_X \circ ad_Y
}
and 
\eq{
\mc{L} \sim \kappa(F_{\mu\nu},F^{\mu\nu})
}
Notate $(X,Y) = \tr(X,Y)$. Then 
\eq{
\tr (\comm[X]{Y} Z) &= \tr(XYZ - YXZ) \\
&= \tr(YZX-YXZ) \\
&= \tr(Y\comm[Z,X])
}
so $\tr$ is ad-invariant. Now taking the bases 
\eq{
\set{(T_{ij})^a_b = \delta^a_i \delta_{bj}}
}
gives 
\eq{
ad_X(T_{ij}) = X^{li}T_{lj} - X^{jl}T_{il}
}
so 
\eq{
(ad_X \circ ad_Y)T_{ij} &= [(XY)^{ri}\delta_{sj} + (XY)^{js}\delta_{ri} - X^{ri}Y^{js}-X^{js}Y^{ri}]T_{rs} \\
&= M^{rs,ij}T_{rs}
}
so 
\eq{
\tr ad_X \circ ad_Y &= M^{ij,ij} \\
&= 2N \tr(XY) - 2\tr(x)\tr(Y)
}
Now $G=SU(N) \Rightarrow \tr(X)=0$, so 
\eq{
\tr ad_X \circ ad_Y = 2N \tr(XY) = \kappa(X,Y)
}


\end{document}