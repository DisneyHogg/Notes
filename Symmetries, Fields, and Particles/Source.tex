\documentclass{article}

%%%%%%%%%%%%%%%%%%%%%%%%%%%%%%%%%%%%%%%%%%%%%%%%%%%%%%%%
%Packages

\usepackage[utf8]{inputenc}
\usepackage{amsmath}
\usepackage{amsthm}
\usepackage{amssymb}
\usepackage{hyperref}

%%%%%%%%%%%%%%%%%%%%%%%%%%%%%%%%%%%%%%%%%%%%%%%%%%%%%%%%
%Definitions

\newtheorem{theorem}{Theorem}[section]
\newtheorem{corollary}{Corollary}[theorem]
\newtheorem{lemma}[theorem]{Lemma}
\newtheorem*{remark}{Remark}
\newtheorem{definition}{Definition}[subsection]
\newtheorem{fact}{Fact}[subsection]
\newtheorem*{idea}{Idea}

\DeclareMathOperator{\spn}{Span}
\DeclareMathOperator{\tr}{Tr}
\DeclareMathOperator{\SU}{SU}

\newcommand{\bam}[1]{\textbf{#1}}
\newcommand{\mf}[1]{\mathfrak{#1}}
\newcommand{\mbb}[1]{\mathbb{#1}}
\newcommand{\comm}[2][]{\left[ #1, #2 \right]}
\newcommand{\be}{\begin{equation}}
\newcommand{\ee}{\end{equation}}
\newcommand{\set}[1]{\lbrace #1 \rbrace}

%%%%%%%%%%%%%%%%%%%%%%%%%%%%%%%%%%%%%%%%%%%%%%%%%%%%%%%%
%Preamble

\title{Symmetries, Fields, and Particles Revision Notes}
\author{Linden Disney-Hogg}
\date{December 2018}

%%%%%%%%%%%%%%%%%%%%%%%%%%%%%%%%%%%%%%%%%%%%%%%%%%%%%%%%
%%%%%%%%%%%%%%%%%%%%%%%%%%%%%%%%%%%%%%%%%%%%%%%%%%%%%%%%
\begin{document}

\maketitle
\tableofcontents

\section{Introduction}
A brief overview of some key ideas, concepts, and facts that I find useful in revising SFP. 

%%%%%%%%%%%%%%%%%%%%%%%%%%%%%%%%%%%%%%%%%%%%%%%%%%%%%%%%
%%%%%%%%%%%%%%%%%%%%%%%%%%%%%%%%%%%%%%%%%%%%%%%%%%%%%%%%
\section{Lie Groups}
\subsection{Basics}
\begin{definition}[Lie Group]
A \bam{Lie group} is a group that has a smooth manifold structure such that the group operations are smooth functions on the manifold. 
\end{definition}

\begin{idea}
Lie groups are introduced here to give a foundation to the idea of a continuous symmetry group, which will act in some way. It is a consequence of the Peter-Weyl theorem that every compact Lie groups is isomorphic to a subgroup of $GL(n,\mbb{C})$ for some $n$ so often one can restrict their thought process to matrix Lie groups. 
\end{idea}

\begin{idea}
When showing that something is a structure is a Lie group, typically the hardest step is showing that it is a manifold. One approach is to find an explicit parametrisation of the manifold, often a good approach if the group has very obvious parameters upon which it depends (e.g. the Heisenberg group). Alternatively, one can use the preimage theorem, useful if the group is defined by some constraints (e.g. the orthogonal group). 
\end{idea}

%%%%%%%%%%%%%%%%%%%%%%%%%%%%%%%%%%%%%%%%%%%%%%%%%%%%%%%%
%%%%%%%%%%%%%%%%%%%%%%%%%%%%%%%%%%%%%%%%%%%%%%%%%%%%%%%%
\section{Lie Algebras}

%%%%%%%%%%%%%%%%%%%%%%%%%%%%%%%%%%%%%%%%%%%%%%%%%%%%%%%%
\subsection{Basics}

\begin{definition}[Lie Algebra] 

A \bam{Lie algebra} is a vector space, equipped with a bracket that:
\begin{itemize}
    \item is bilinear
    \item is antisymmetric
    \item obeys the Jacobi identity
\end{itemize}
\end{definition}

\begin{idea}
Given a matrix Lie group $G$, it has a natural associated Lie algebra $\mf{g}=T_{I}G$ with bracket given by the matrix commutator. This will be the prototypical Lie algebra to think of. The Lie algebra corresponding to a Lie group is useful to work with as we can take a basis, which will commute under addition, rather than work with a difficult presentation of the group.
Using the exponential map, we can map back into the original Lie group. This is not generally injective or surjective, but map injectively onto the connected component containing the identity. 
\end{idea}

\begin{fact}
The dimension of the tangent space to a manifold is equal to the dimension of the manifold. Hence, a Lie group and its associated Lie algebra have the same dimension
\end{fact}

\begin{definition}[Structure Constants]
Given a basis $\lbrace T^a \rbrace$ of a Lie algebra, the \bam{structure constants} $f^{ab}_c$ are defined such that 
\[
\comm[T^a]{T^b}=f^{ab}_c T^c
\]
\end{definition}
%%%%%%%%%%%%%%%%%%%%%%%%%%%%%%%%%%%%%%%%%%%%%%%%%%%%%%%%
\subsection{Ideals and Simplicity}

\begin{definition}[ideal]
An \bam{ideal} is a subalgebra $\mf{h} \leq \mf{g}$ s.t.
\[
\forall X\in\mf{g}, \; \forall Y\in\mf{h} \quad  \comm[X]{Y}\in\mf{h}
\]
\end{definition}

\begin{definition}[Derived Algebra]
The \bam{derived algebra} of a Lie algebra $\mf{g}$ is 
\[
i(\mf{g})=\set{\comm[X]{Y} : X,Y\in\mf{g}}
\]
\end{definition}

\begin{definition}[Centre]
The \bam{centre} of a Lie algebra $\mf{g}$ is 
\[
\zeta(\mf{g})=\set{X\in\mf{g} : \forall Y\in\mf{g} \; \comm[X]{Y}=0}
\]
\end{definition}

\begin{definition}[Simple Lie Algebra]
A Lie Algebra is \bam{simple} if it contains no non-trivial ideals.
\end{definition}

\begin{idea}
An ideal is to a Lie algebra what a normal subgroup is to a Lie group. Hence simple Lie algebras are the analogy of simple Lie groups.  
\end{idea}

\begin{definition}[Semi-Simple Lie Algebra]
A Lie Algebra is \bam{semi-simple} if it contains no non-trivial abelian ideals. 
\end{definition}

%%%%%%%%%%%%%%%%%%%%%%%%%%%%%%%%%%%%%%%%%%%%%%%%%%%%%%%%
\subsection{Adjoint Map}

\begin{definition}[Adjoint Map]
The \bam{adjoint map} of an element $X\in\mf{g}$ is $ad_{X}:\mf{g}\to\mf{g}$ defined such that for $Y\in\mf{g}$
\[
ad_{X}\left(Y\right)=\comm[X]{Y}
\]
\end{definition}

\begin{definition}[ad-diagonalisable]
An element $X\in\mf{g}$ is \bam{ad-diagonalisable} if the linear map $ad_X$ is diagonalisable. 
\end{definition}

\begin{definition}[Killing Form]
The \bam{Killing form} is a map $\kappa:\mf{g}\times\mf{g}\to\mbb{C}$ defined such that for $X,Y\in\mf{g}$
\[
\kappa\left(X,Y\right)=\tr\left(ad_X \circ ad_Y\right)
\]
It is symmetric and bilinear 
\end{definition}

\begin{theorem}
Let $\mf{g}$ be a Lie algebra and $\kappa$ the Killing form. Then
\[
\forall X,Y,Z\in\mf{g} \quad \kappa\left(\comm[Z]{X},Y\right)+\kappa\left(X,\comm[Z]{Y}\right)=0
\]
\end{theorem}

%%%%%%%%%%%%%%%%%%%%%%%%%%%%%%%%%%%%%%%%%%%%%%%%%%%%%%%%
\subsection{Cartan Subalgebras}

\begin{definition}[Cartan Subalgebra]
A \bam{Cartan subalgebra} $\mf{h}\leq\mf{g}$ is a maximal abelian subalgebra containing only ad-diagonalisable elements. It is typically written as $\mf{h}=\spn$
\end{definition}

\begin{fact}
All Cartan subalgebras of a given Lie algebra have the same dimension. Though Cartan subgalgebras are not unique, they are all conjugate. 
\end{fact}

\begin{idea}
It is a fact that Cartan subalgebras correspond to maximal toral subgroups of the corresponding Lie group. This gives a route to visualisation of what the subalgebra might be. 
\end{idea}

%%%%%%%%%%%%%%%%%%%%%%%%%%%%%%%%%%%%%%%%%%%%%%%%%%%%%%%%
\begin{definition}[Compact Type]
A real Lie algebra $\mf{g}_\mathbb{R}$ is of \bam{compact type} if $\exists$ a basis in which $\kappa^{ij}=-K\delta^{ij}$ for some $K\in\mathbb{R}^{+}$.
\end{definition}

\begin{definition}[Complexification and Real Form]
Given a  real Lie algebra $\mf{g}_\mathbb{R}\equiv\spn_{\mathbb{R}}\lbrace T^a : a=1, \dots, D \rbrace$ the \bam{complexification}
is $\mf{g}_\mathbb{C}\equiv\spn_{\mathbb{C}}\lbrace T^a : a=1, \dots, D \rbrace$. Conversely, given such a $\mf{g}_\mathbb{C}$, $\mf{g}_\mathbb{R}$ is called a \bam{real form}.  
\end{definition}

%%%%%%%%%%%%%%%%%%%%%%%%%%%%%%%%%%%%%%%%%%%%%%%%%%%%%%%%
%%%%%%%%%%%%%%%%%%%%%%%%%%%%%%%%%%%%%%%%%%%%%%%%%%%%%%%%
\section{Representations}

%%%%%%%%%%%%%%%%%%%%%%%%%%%%%%%%%%%%%%%%%%%%%%%%%%%%%%%%
\subsection{Basics}

\begin{definition}[Representations]
A \bam{representation} of a Lie group $G$ is a smooth group homomorphism 
\[
D:G\to GL\left(V\right)
\]
where $V$ is a $n$-dimensional vector space called the \bam{representation space}.
In analogy a representation of a Lie algebra is a Lie algebra homomorphism 
\[
d:\mf{g} \to \mf{gl}\left(V\right)
\]
\end{definition}

\begin{theorem}
A representation $D$ of a Lie group $G$ induces a representation on the the corresponding Lie algebra $\mf{g}$ as such: For $X\in\mf{g}$ let $g:(-\epsilon, \epsilon)\to G$ be a curve in $G$ defined such that $g(0)=e$ the identity of $G$ and $g'(0)=X$. This curve necessarily exists for some $\epsilon>0$. Then define $d$ by \[
d(X)\equiv\frac{d}{dt} D\left(g(t)\right) \vert_{t=0}
\]
Conversely, a representation of a Lie algebra $\mf{g}$ induces a representation in some neighbourhood of $e$ in $G$ given by 
\[
D(\exp{X})\equiv\exp{d(X)}
\]
\end{theorem}

\begin{definition}[Dimension]
The \bam{dimension} of a representation is the dimension of the representation space.
\end{definition}

\begin{definition}[Trivial Representation] 
The \bam{trivial representation} is the unique one dimensional representation that sends every element to the identity. 
\end{definition}

\begin{definition}[Fundamental Representation]
For a matrix Lie group $G\leq GL(V)$ the \bam{fundamental representation} is the identity map $R=id_V$. Note $dim\left(R\right)=dim\left(V\right)$.
\end{definition}

\begin{definition}[Adjoint Representation]
Given a Lie algebra $\mf{g}$ the \bam{adjoint representation} is $d_{adj}:\mf{g}\to\mf{g}$, $d_{adj}\left(X\right)=ad_X $. Note $dim\left(d_{adj}\right)=dim\left(\mf{g}\right)$
\end{definition}

\begin{definition}[Faithful Representation]
A representation is \bam{faithful} if it is injective. 
\end{definition}

\begin{definition}[Isomorphic Representations]
Two representations $R_1$, $R_2$ of a Lie algebra $\mf{g}$ are \bam{isomorphic}, written $R_1 \cong R_2$ if $\exists S$ an invertible matrix such that 
\[
\forall X\in\mf{g} \quad R_2\left(X\right)=S R_1\left(X\right) S^{-1}
\]
\end{definition}

%%%%%%%%%%%%%%%%%%%%%%%%%%%%%%%%%%%%%%%%%%%%%%%%%%%%%%%%
\subsection{Reducibility}
\begin{definition}[Invariant Subspace]
Given a representation $R$ of Lie algebra $\mf{g}$ with representation space $V$, an \bam{invariant subspace} is $U\leq V$ such that
\[
\forall X\in\mf{g}, \; \forall u\in U \quad R\left(X\right)u\in U
\]
\end{definition}

\begin{definition}[Irreducible Representation]
A representation is \bam{irreducible} if it has no invariant subspaces. 
\end{definition}

%%%%%%%%%%%%%%%%%%%%%%%%%%%%%%%%%%%%%%%%%%%%%%%%%%%%%%%%
%%%%%%%%%%%%%%%%%%%%%%%%%%%%%%%%%%%%%%%%%%%%%%%%%%%%%%%%
\section{Useful Facts}

%%%%%%%%%%%%%%%%%%%%%%%%%%%%%%%%%%%%%%%%%%%%%%%%%%%%%%%%
\subsection{Miscellaneous}
\begin{theorem}
If $X$ is a complex square matrix then 
\[
\det{e^X}=e^{\tr{X}}
\]
\end{theorem}

\begin{theorem}[Baker-Campbell-Hausdorff Formula]
Let $\mf{g}$ be a Lie algebra, and $X,Y\in\mf{g}$. Then under sufficient existence conditions $e^X e^Y=e^Z$ where
\[
Z=X+Y+\frac{1}{2}\comm[X]{Y}+\frac{1}{12}\left(\comm[X]{\comm[X]{Y}}+\comm[Y]{\comm[Y]{X}}\right)
\]
\end{theorem}

\begin{corollary}[Lie Product Formula]
For $X,Y\in\mf{g}$, a Lie algebra
\[
e^{X+Y}=\lim_{N\to\infty}\left( e^{\frac{X}{N}} e^{\frac{Y}{N}} \right)^N
\]
\end{corollary}
%%%%%%%%%%%%%%%%%%%%%%%%%%%%%%%%%%%%%%%%%%%%%%%%%%%%%%%%
\subsection{Pauli Matrices}

\begin{definition}[Pauli Matrices]
The \bam{Pauli matrices} are

\begin{align*}
\sigma_1 &= \begin{pmatrix} 0 & 1 \\ 1 & 0\end{pmatrix}  \\
\sigma_2 &= \begin{pmatrix} 0 & -i \\ i & 0\end{pmatrix}  \\
\sigma_3 &= \begin{pmatrix} 1 & 0 \\ 0 & -1\end{pmatrix}  
\end{align*}

Note they are all Hermitian and traceless.
\end{definition}

\begin{fact}
\be
\sigma_i \sigma_j = \delta_{ij}I +i\epsilon_{ijk}\sigma_k
\ee
\end{fact}

%%%%%%%%%%%%%%%%%%%%%%%%%%%%%%%%%%%%%%%%%%%%%%%%%%%%%%%%
%%%%%%%%%%%%%%%%%%%%%%%%%%%%%%%%%%%%%%%%%%%%%%%%%%%%%%%%
\section{SU(2)}
\begin{definition}[SU($n$)]
The \bam{special unitary group} of dimension $n$ is
\[
\SU\left(n\right)=\lbrace U\in GL\left(n,\mbb{C}\right) : U^\dagger U=I, \det{U}=1 \rbrace
\]
\end{definition}
\noindent SU($n$) is a Lie group The corresponding Lie algebra is 
\[
\mf{su}\left(n\right)=\lbrace Z\in GL\left(n,\mbb{C}\right) : Z^\dagger+Z=0, \tr{Z}=0 \rbrace
\]
The real dimension is
\[
\underbrace{2\times\frac{1}{2}(n-1)n}_{\text{off diagonal elements}}+\underbrace{n}_{\text{diagonal elements}}-\underbrace{1}_{\text{trace constraint}}=n^2-1
\]

%%%%%%%%%%%%%%%%%%%%%%%%%%%%%%%%%%%%%%%%%%%%%%%%%%%%%%%
%%%%%%%%%%%%%%%%%%%%%%%%%%%%%%%%%%%%%%%%%%%%%%%%%%%%%%%%
\section{Cartan Classification}
\subsection{Root Geometry}

\begin{definition}[Roots]
The roots of a Lie Algebra are $\alpha\in\Phi$, vectors with components $\alpha^i$ eigenvalues of $ad_{H^i}$.
\end{definition}

\begin{definition}[$\alpha$-string through $\beta$]
The \bam{$\alpha$-string through $\beta$} is 
\[
S_{\alpha, \beta}=\lbrace \beta+\rho\alpha\in\Phi : \rho\in\mbb{Z} \rbrace
\]
\end{definition}

\begin{fact}
Every complex semi-simple Lie algebra of finite dimension has a real form of compact type.
\end{fact}

\end{document}
