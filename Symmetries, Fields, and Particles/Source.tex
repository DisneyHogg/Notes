\documentclass{article}

%%%%%%%%%%%%%%%%%%%%%%%%%%%%%%%%%%%%%%%%%%%%%%%%%%%%%%%%
%Packages

\usepackage[utf8]{inputenc}
\usepackage{amsmath}
\usepackage{amsthm}
\usepackage{amssymb}
\usepackage{hyperref} % Used for refernces and links. 
\usepackage{bm} % Used to bold vectors. 
\usepackage{faktor} % Used to write quotients nicely.

%%%%%%%%%%%%%%%%%%%%%%%%%%%%%%%%%%%%%%%%%%%%%%%%%%%%%%%%
%Definitions

\newtheorem{theorem}{Theorem}[subsection]
\newtheorem{corollary}{Corollary}[theorem]
\newtheorem{lemma}[theorem]{Lemma}
\newtheorem*{remark}{Remark}
\newtheorem{definition}{Definition}[subsection]
\newtheorem{fact}{Fact}[subsection]
\newtheorem*{idea}{Idea}

\DeclareMathOperator{\spn}{Span}
\DeclareMathOperator{\tr}{Tr}
\DeclareMathOperator{\SU}{SU}
%\DeclareMathOperator{\dim}{dim}
\DeclareMathOperator{\rank}{rank}

\newcommand{\bam}[1]{\textbf{#1}}
\newcommand{\mf}[1]{\mathfrak{#1}}
\newcommand{\mbb}[1]{\mathbb{#1}}
\newcommand{\comm}[2][]{\left[ #1, #2 \right]}
\newcommand{\be}{\begin{equation}}
\newcommand{\ee}{\end{equation}}
\newcommand{\set}[1]{\lbrace #1 \rbrace}

%%%%%%%%%%%%%%%%%%%%%%%%%%%%%%%%%%%%%%%%%%%%%%%%%%%%%%%%
%Preamble

\title{Symmetries, Fields, and Particles Revision Notes}
\author{Linden Disney-Hogg}
\date{December 2018}

%%%%%%%%%%%%%%%%%%%%%%%%%%%%%%%%%%%%%%%%%%%%%%%%%%%%%%%%
%%%%%%%%%%%%%%%%%%%%%%%%%%%%%%%%%%%%%%%%%%%%%%%%%%%%%%%%
\begin{document}

\maketitle
\tableofcontents

\section{Introduction}
A brief overview of some key ideas, concepts, and facts that I find useful in revising SFP. 

%%%%%%%%%%%%%%%%%%%%%%%%%%%%%%%%%%%%%%%%%%%%%%%%%%%%%%%%
%%%%%%%%%%%%%%%%%%%%%%%%%%%%%%%%%%%%%%%%%%%%%%%%%%%%%%%%
\section{Lie Groups}
\subsection{Basics}
\begin{definition}[Lie Group]
A \bam{Lie group} is a group that has a smooth manifold structure such that the group operations are smooth functions on the manifold. 
\end{definition}

\begin{idea}
Lie groups are introduced here to give a foundation to the idea of a continuous symmetry group, which will act in some way. It is a consequence of the Peter-Weyl theorem that every compact Lie groups is isomorphic to a subgroup of $GL(n,\mbb{C})$ for some $n$ so often one can restrict their thought process to matrix Lie groups.
\end{idea}

\begin{idea}
When showing that something is a structure is a Lie group, typically the hardest step is showing that it is a manifold. One approach is to find an explicit parametrisation of the manifold, often a good approach if the group has very obvious parameters upon which it depends (e.g. the Heisenberg group). Alternatively, one can use the preimage theorem, useful if the group is defined by some constraints (e.g. the orthogonal group).
\end{idea}

\begin{theorem}[Closed Subgroup Theorem]
If $G$ is a Lie group and $H\leq G$ is closed, then $H$ is an embedded Lie group with the subspace topology. 
\end{theorem}

%%%%%%%%%%%%%%%%%%%%%%%%%%%%%%%%%%%%%%%%%%%%%%%%%%%%%%%%
%%%%%%%%%%%%%%%%%%%%%%%%%%%%%%%%%%%%%%%%%%%%%%%%%%%%%%%%
\section{Lie Algebras}

%%%%%%%%%%%%%%%%%%%%%%%%%%%%%%%%%%%%%%%%%%%%%%%%%%%%%%%%
\subsection{Basics}

\begin{definition}[Lie Algebra] 

A \bam{Lie algebra} is a vector space, equipped with a bracket that:
\begin{itemize}
    \item is bilinear
    \item is antisymmetric
    \item obeys the Jacobi identity
\end{itemize}
\end{definition}

\begin{idea}
Given a matrix Lie group $G$, it has a natural associated Lie algebra $\mf{g}=T_{I}G$ with bracket given by the matrix commutator. This will be the prototypical Lie algebra to think of. The Lie algebra corresponding to a Lie group is useful to work with as we can take a basis, which will commute under addition, rather than work with a difficult presentation of the group.
Using the exponential map, we can map back into the original Lie group. This is not generally injective or surjective, but map injectively onto the connected component containing the identity. 
\end{idea}

\begin{fact}
The dimension of the tangent space to a manifold is equal to the dimension of the manifold. Hence, a Lie group and its associated Lie algebra have the same dimension
\end{fact}

\begin{definition}[Structure Constants]
Given a basis $\lbrace T^a \rbrace$ of a Lie algebra, the \bam{structure constants} $f^{ab}_c$ are defined such that 
\[
\comm[T^a]{T^b}=f^{ab}_c T^c
\]
\end{definition}

\begin{definition}[Complexification and Real Form]
Given a  real Lie algebra $\mf{g}_\mathbb{R}\equiv\spn_{\mathbb{R}}\lbrace T^a : a=1, \dots, D \rbrace$ the \bam{complexification}
is $\mf{g}_\mathbb{C}\equiv\spn_{\mathbb{C}}\lbrace T^a : a=1, \dots, D \rbrace$. Conversely, given such a $\mf{g}_\mathbb{C}$, $\mf{g}_\mathbb{R}$ is called a \bam{real form}.  
\end{definition}
%%%%%%%%%%%%%%%%%%%%%%%%%%%%%%%%%%%%%%%%%%%%%%%%%%%%%%%%
\subsection{Ideals and Simplicity}

\begin{definition}[ideal]
An \bam{ideal} is a subalgebra $\mf{h} \leq \mf{g}$ s.t.
\[
\forall X\in\mf{g}, \; \forall Y\in\mf{h} \quad  \comm[X]{Y}\in\mf{h}
\]
\end{definition}

\begin{definition}[Derived Algebra]
The \bam{derived algebra} of a Lie algebra $\mf{g}$ is 
\[
i(\mf{g})=\set{\comm[X]{Y} : X,Y\in\mf{g}}
\]
\end{definition}

\begin{definition}[Centre]
The \bam{centre} of a Lie algebra $\mf{g}$ is 
\[
\zeta(\mf{g})=\set{X\in\mf{g} : \forall Y\in\mf{g} \; \comm[X]{Y}=0}
\]
\end{definition}

\begin{definition}[Simple Lie Algebra]
A Lie Algebra is \bam{simple} if it contains no non-trivial ideals.
\end{definition}

\begin{idea}
An ideal is to a Lie algebra what a normal subgroup is to a Lie group. Hence simple Lie algebras are the analogy of simple Lie groups.  
\end{idea}

\begin{definition}[Semi-Simple Lie Algebra]
A Lie Algebra is \bam{semi-simple} if it contains no non-trivial abelian ideals. 
\end{definition}

%%%%%%%%%%%%%%%%%%%%%%%%%%%%%%%%%%%%%%%%%%%%%%%%%%%%%%%%
\subsection{Adjoint Map}

\begin{definition}[Adjoint Map]
The \bam{adjoint map} of an element $X\in\mf{g}$ is $ad_{X}:\mf{g}\to\mf{g}$ defined such that for $Y\in\mf{g}$
\[
ad_{X}\left(Y\right)=\comm[X]{Y}
\]
\end{definition}

\begin{definition}[ad-diagonalisable]
An element $X\in\mf{g}$ is \bam{ad-diagonalisable} if the linear map $ad_X$ is diagonalisable. 
\end{definition}

\begin{definition}[Killing Form]
The \bam{Killing form} is a map $\kappa:\mf{g}\times\mf{g}\to\mbb{C}$ defined such that for $X,Y\in\mf{g}$
\[
\kappa\left(X,Y\right)=\tr\left(ad_X \circ ad_Y\right)
\]
In components 
\begin{align*}
    \kappa(X,Y) &= \kappa^{ab}X_a Y_b \\
    \kappa^{ab} &= f^{ac}_d f^{bd}_c
\end{align*}
It is symmetric and bilinear 
\end{definition}

\begin{theorem}
Let $\mf{g}$ be a Lie algebra and $\kappa$ the Killing form. Then
\[
\forall X,Y,Z\in\mf{g} \quad \kappa\left(\comm[Z]{X},Y\right)+\kappa\left(X,\comm[Z]{Y}\right)=0
\]
\end{theorem}

\begin{theorem}[Cartan]
Let $\mf{g}$ be a Lie algebra with Killing form $\kappa$. Then 
\[
\kappa \, \text{non-degenerate} \Leftrightarrow \mf{g} \, \text{semi-simple}
\]
\end{theorem}

%%%%%%%%%%%%%%%%%%%%%%%%%%%%%%%%%%%%%%%%%%%%%%%%%%%%%%%%
\subsection{Cartan Subalgebras}

\begin{definition}[Cartan Subalgebra]
A \bam{Cartan subalgebra} $\mf{h}\leq\mf{g}$ is a maximal abelian subalgebra containing only ad-diagonalisable elements. It is typically written as $\mf{h}=\spn\lbrace H^i \rbrace$. Write $\dim{\mf{h}}=\rank{\mf{g}}=r$
\end{definition}

\begin{fact}
All Cartan subalgebras of a given Lie algebra have the same dimension. Though Cartan subgalgebras are not unique, they are all conjugate. 
\end{fact}

\begin{idea}
It is a fact that Cartan subalgebras correspond to maximal toral subgroups of the corresponding Lie group. This gives a route to visualisation of what the subalgebra might be. A Cartan subalgebra is defined such that representations of the basis $\lbrace H^i \rbrace$ are simultaneously diagonalisable. 
\end{idea}

%%%%%%%%%%%%%%%%%%%%%%%%%%%%%%%%%%%%%%%%%%%%%%%%%%%%%%%%
\begin{definition}[$\kappa^{ij}$]
Given a Cartan subalgebra $\mf{h}=\spn\lbrace H^i \rbrace$ and Killing form $\kappa$ the notation will be used $\kappa^{ij}$ for the matrix elements of $\kappa |_{\mf{h}\times\mf{h}}$,
\[
\kappa^{ij}=\kappa(H^i, H^j)
\]
\end{definition}

\begin{definition}[Compact Type]
A real Lie algebra $\mf{g}_\mathbb{R}$ is of \bam{compact type} if $\exists$ a basis in which $\kappa^{ij}=-K\delta^{ij}$ for some $K\in\mathbb{R}^{+}$.
\end{definition}

\begin{fact}
Every complex semi-simple Lie algebra of finite dimension has a real form of compact type.
\end{fact}

%%%%%%%%%%%%%%%%%%%%%%%%%%%%%%%%%%%%%%%%%%%%%%%%%%%%%%%%
%%%%%%%%%%%%%%%%%%%%%%%%%%%%%%%%%%%%%%%%%%%%%%%%%%%%%%%%
\section{Representations}

%%%%%%%%%%%%%%%%%%%%%%%%%%%%%%%%%%%%%%%%%%%%%%%%%%%%%%%%
\subsection{Basics}

\begin{definition}[Representations]
A \bam{representation} of a Lie group $G$ is a smooth group homomorphism 
\[
D:G\to GL\left(V\right)
\]
where $V$ is a $n$-dimensional vector space called the \bam{representation space}.
In analogy a representation of a Lie algebra is a Lie algebra homomorphism 
\[
d:\mf{g} \to \mf{gl}\left(V\right)
\]
\end{definition}

\begin{definition}[Dimension]
The \bam{dimension} of a representation is the dimension of the representation space.
\end{definition}

\begin{definition}[Trivial Representation] 
The \bam{trivial representation} is the unique one dimensional representation that sends every element to the identity. 
\end{definition}

\begin{definition}[Fundamental Representation]
For a matrix Lie group $G\leq GL(V)$ the \bam{fundamental representation} is the identity map $R=id_V$. Note $dim\left(R\right)=dim\left(V\right)$.
\end{definition}

\begin{definition}[Adjoint Representation]
Given a Lie algebra $\mf{g}$ the \bam{adjoint representation} is $d_{adj}:\mf{g}\to\mf{gl}(\mf{g})$, $d_{adj}\left(X\right)=ad_X $. Note $dim\left(d_{adj}\right)=dim\left(\mf{g}\right)$
\end{definition}

\begin{definition}[Faithful Representation]
A representation is \bam{faithful} if it is injective. 
\end{definition}

\begin{definition}[Isomorphic Representations]
Two representations $R_1$, $R_2$ of a Lie algebra $\mf{g}$ are \bam{isomorphic}, written $R_1 \cong R_2$ if $\exists S$ an invertible matrix such that 
\[
\forall X\in\mf{g} \quad R_2\left(X\right)=S R_1\left(X\right) S^{-1}
\]
\end{definition}
%%%%%%%%%%%%%%%%%%%%%%%%%%%%%%%%%%%%%%%%%%%%%%%%%%%%%%%%
\subsection{Constructing Representations}

\begin{theorem}
A representation $D$ of a Lie group $G$ induces a representation on the the corresponding Lie algebra $\mf{g}$ as such: For $X\in\mf{g}$ let $g:(-\epsilon, \epsilon)\to G$ be a curve in $G$ defined such that $g(0)=e$ the identity of $G$ and $g'(0)=X$. This curve necessarily exists for some $\epsilon>0$. Then define $d$ by \[
d(X)\equiv\frac{d}{dt} D\left(g(t)\right) \vert_{t=0}
\]
Conversely, a representation of a Lie algebra $\mf{g}$ induces a representation in some neighbourhood of $e$ in $G$ given by 
\[
D(\exp{X})\equiv\exp{d(X)}
\]
\end{theorem}

\begin{definition}[Direct Sum Representation]
Given representations $R_1, R_2$ with representation spaces $V_1, V_2$ the \bam{direct sum representation} is $R_1 \oplus R_2$ with representation space $V_1 \oplus V_2$ acting by 
\[
(R_1 \oplus R_2)(X)(v_1 \oplus v_2)=(R_1(X)v_1)\oplus(R_2(X)v_2)
\]
\end{definition}

\begin{definition}[Tensor Product Representation]
Given representations $R_1$, $R_2$ with representation spaces $V_1, V_2$ the \bam{tensor product representation} is $R_1 \otimes R_2$ with representation space $V_1 \otimes V_2$ with
\[
(R_1 \otimes R_2)(X)=R_1(X)\otimes I_{V_2}+I_{V_1}\otimes R_2(X)
\]
\end{definition}

\begin{idea}
Considering a representation as a visualisation of the action of the algebra induced by a  group on a representation space, then $R_1 \oplus R_2$ can be viewed as two non-interacting systems, whereas $R_1 \otimes R_2$ can be viewed as two entangled systems.
\end{idea}

%%%%%%%%%%%%%%%%%%%%%%%%%%%%%%%%%%%%%%%%%%%%%%%%%%%%%%%%
\subsection{Reducibility}
\begin{definition}[Invariant Subspace]
Given a representation $R$ of Lie algebra $\mf{g}$ with representation space $V$, an \bam{invariant subspace} is $U\leq V$ such that
\[
\forall X\in\mf{g}, \; \forall u\in U \quad R\left(X\right)u\in U
\]
\end{definition}

\begin{definition}[Irreducible Representation]
A representation is \bam{irreducible} if it has no invariant subspaces. 
\end{definition}

\begin{definition}[Full Reducible]
A representation is \bam{fully reducible} if it can be expressed as the direct sum of irreducible representations. 
\end{definition}

\begin{fact}
If $R_i$ are finite dimensional irreducible representations of a simple Lie algebra for $i=1,\dots,m$ then
\[
R_1 \otimes R_2 \otimes \dots \otimes R_m
\]
is fully reducible. 
\end{fact}

%%%%%%%%%%%%%%%%%%%%%%%%%%%%%%%%%%%%%%%%%%%%%%%%%%%%%%%%
\subsection{Weights}
\begin{definition}[Weights]
Given a representation $R$ of a Lie algebra with Cartan subalgebra $\mf{h}=\spn\lbrace H^i \rbrace$ the \bam{weights} of the representation are the eigenvalues $S_R=\lbrace \lambda : \exists v\in V \, R(H^i)v=\lambda^i v\rbrace$
\end{definition}

\begin{fact}
Weights can be viewed as maps in $\mf{h}^\ast$ given by 
\begin{align*}
\lambda &:\mf{h}\to\mbb{C} \\
\lambda &: e_i H^i \mapsto e_i \lambda^i
\end{align*}
Thus 
\[
R(H)v=\lambda(H)v
\]
\end{fact}

\begin{idea}
This idea is made more explicit in the concept of roots, which are the weights of the adjoint representation. 
\end{idea}

\begin{fact}
Given weights set $S_R$
\[
V=\bigoplus_{\lambda\in S_R} V_\lambda
\]
where $V_\lambda=\lbrace v\in V : R(H^i)v=\lambda^i v\rbrace$
\end{fact}
%%%%%%%%%%%%%%%%%%%%%%%%%%%%%%%%%%%%%%%%%%%%%%%%%%%%%%%%
%%%%%%%%%%%%%%%%%%%%%%%%%%%%%%%%%%%%%%%%%%%%%%%%%%%%%%%%
\section{Useful Facts}

%%%%%%%%%%%%%%%%%%%%%%%%%%%%%%%%%%%%%%%%%%%%%%%%%%%%%%%%
\subsection{Miscellaneous}
\begin{theorem}
If $X$ is a complex square matrix then 
\[
\det{e^X}=e^{\tr{X}}
\]
\end{theorem}

\begin{theorem}[Baker-Campbell-Hausdorff Formula]
Let $\mf{g}$ be a Lie algebra, and $X,Y\in\mf{g}$. Then under sufficient existence conditions $e^X e^Y=e^Z$ where
\[
Z=X+Y+\frac{1}{2}\comm[X]{Y}+\frac{1}{12}\left(\comm[X]{\comm[X]{Y}}+\comm[Y]{\comm[Y]{X}}\right)+\dots
\]
\end{theorem}

\begin{corollary}[Lie Product Formula]
For $X,Y\in\mf{g}$, a Lie algebra
\[
e^{X+Y}=\lim_{N\to\infty}\left( e^{\frac{X}{N}} e^{\frac{Y}{N}} \right)^N
\]
\end{corollary}
%%%%%%%%%%%%%%%%%%%%%%%%%%%%%%%%%%%%%%%%%%%%%%%%%%%%%%%%
\subsection{Pauli Matrices}

\begin{definition}[Pauli Matrices]
The \bam{Pauli matrices} are

\begin{align*}
\sigma_1 &= \begin{pmatrix} 0 & 1 \\ 1 & 0\end{pmatrix}  \\
\sigma_2 &= \begin{pmatrix} 0 & -i \\ i & 0\end{pmatrix}  \\
\sigma_3 &= \begin{pmatrix} 1 & 0 \\ 0 & -1\end{pmatrix}  
\end{align*}

Note they are all Hermitian and traceless.
\end{definition}

\begin{fact}
\be
\sigma_i \sigma_j = \delta_{ij}I +i\epsilon_{ijk}\sigma_k
\ee
\end{fact}

%%%%%%%%%%%%%%%%%%%%%%%%%%%%%%%%%%%%%%%%%%%%%%%%%%%%%%%%
%%%%%%%%%%%%%%%%%%%%%%%%%%%%%%%%%%%%%%%%%%%%%%%%%%%%%%%%
\section{SU(2)}
%%%%%%%%%%%%%%%%%%%%%%%%%%%%%%%%%%%%%%%%%%%%%%%%%%%%%%%%
\subsection{Definition and Basic Properties}

\begin{definition}[U($n$)]
The \bam{unitary group} of dimension $n$ is 
\[
U\left(n\right)=\lbrace U\in GL\left(n,\mbb{C}\right) : U^\dagger U=I \rbrace
\]
It is a path connected, compact, Lie group.
\end{definition}

\begin{definition}[SU($n$)]
The \bam{special unitary group} of dimension $n$ is
\[
SU\left(n\right)=\lbrace U\in U(n) : \det{U}=1 \rbrace
\]
\end{definition}
\noindent Through the $\det$ homomorphism, it can be seen $SU(n)$ is a closed subgroup of $U(n)$, hence also a Lie group.  The corresponding Lie algebra is 
\[
\mf{su}\left(n\right)=\lbrace Z\in GL\left(n,\mbb{C}\right) : Z^\dagger+Z=0, \tr{Z}=0 \rbrace
\]
The real dimension is
\[
\underbrace{2\times\frac{1}{2}(n-1)n}_{\text{off diagonal elements}}+\underbrace{n}_{\text{diagonal elements}}-\underbrace{1}_{\text{trace constraint}}=n^2-1
\]
In the case $n=2$ we have 
\[
SU(2)=\lbrace   \begin{pmatrix} \alpha & -\overline{\beta} \\ \beta & \overline{\alpha} \end{pmatrix}  : \alpha,\beta\in\mathbb{C} , |\alpha|^2+|\beta|^2=1  \rbrace
\]
This can be expressed as, for $A\in SU(2)$
\[
A=a_0 I +i\bm{a}\cdot\bm{\sigma}
\]
where $\bm{a}=(a_1, a_2, a_3)$, $\bm{\sigma}=(\sigma_1, \sigma_2, \sigma_3)$, and $a_0^2+|\bm{a}|^2=1$. Hence $SU(2)\cong S^3$. In addition, by parametrising $SU(2)$ by the $a_i$, it can be seen that $\lbrace i\sigma_i\rbrace$ forms a basis of $\mf{su}(2)$. It is typical to normalise this basis to $\lbrace T^a=-\frac{1}{2} i\sigma_a \rbrace$. 

\begin{fact}
The structure constants in this basis $\lbrace T^a \rbrace$ are 
\[
f^{ab}_c=\epsilon_{abc}
\]
\end{fact}

\begin{fact}
The function 
\begin{align*}
     & R : SU(2) \to SO(3) \\
     & R(A)_{ij} = \frac{1}{2}\tr\left( \sigma_i A \sigma_j A^\dagger \right)
\end{align*}
Is a double cover of $SO(3)$, with $R(A)=R(-A)$, and inverse given by 
\[
A=\pm \frac{I+\sigma_i R(A)_{ij} \sigma_j}{2\sqrt{1+\tr R(A)}}
\]
Hence $SO(3)\cong\faktor{SU(2)}{\mbb{Z}_2}$
\end{fact}

%%%%%%%%%%%%%%%%%%%%%%%%%%%%%%%%%%%%%%%%%%%%%%%%%%%%%%%%
\subsection{Representations}

\begin{definition}[Cartan-Weyl basis of $\mf{su}_\mbb{C}(2)$]
$\lbrace H, E_\pm \rbrace$, where $H=\sigma_3$ and $E_\pm=\frac{1}{2}(\sigma_1\pm i\sigma_2 )$, is the \bam{Cartan-Weyl basis} for $\mf{su}_\mbb{C}(2)$. The elements satisfy the commutation relations 
\begin{align*}
\comm[H]{E_\pm} &= \pm 2E_\pm \\
\comm[E_+]{E_-} &= H
\end{align*}
\end{definition} 

\begin{fact}
$H$ spans a Cartan subalgebra of $\mf{su}_\mbb{C}(2)$, ad-diagonalisable w.r.t the Cartan Weyl basis, with eigenvalues $0, \pm 2$ respectively. 
\end{fact}

\begin{theorem}
Finite dimensional irreducible representations of $\mf{su}_\mbb{C}(2)$ with $R(H)$ diagonalisable are determined by the highest weight $\Lambda\in\mbb{N}_0$. Such a representation is called $R_\Lambda$. It has weight set 
\[
S_\Lambda=\lbrace \Lambda-2n : n=0,\dots,\Lambda \rbrace
\]
The eigenvectors of $R_\Lambda(H)$ are $v_{\Lambda-2n}$ defined such that 
\begin{align*}
R_\Lambda(H) v_\lambda &= \lambda v_\lambda \\
R_\Lambda(E_+) v_\Lambda &= 0 \\
 \left(R_\Lambda(E_-)\right)^n v_{\Lambda} &= v_{\Lambda-2n} \\
 R_\Lambda(E_+) v_{\Lambda-2n} &= r_n v_{\Lambda-2n}
\end{align*}
with 
\[
r_n=(\Lambda+1-n)n
\]
\end{theorem}
Note that $\dim(R_\Lambda)=\Lambda+1$, so 
\begin{itemize}
    \item $R_0$ is the trivial representation $d_0$.
    \item $R_1$ is the fundamental representation.
    \item $R_2$ is the adjoint representation. 
\end{itemize}
%%%%%%%%%%%%%%%%%%%%%%%%%%%%%%%%%%%%%%%%%%%%%%%%%%%%%%%
%%%%%%%%%%%%%%%%%%%%%%%%%%%%%%%%%%%%%%%%%%%%%%%%%%%%%%%%
\section{Cartan Classification}
\'Elie Cartan classified finite dimensional, simple, complex Lie algebras in 1894. This sections will therefore restrict to treatment of such Lie algebras. Note that $\mf{su}_\mbb{C}(2)$ will be a prototypical such Lie algebra. 

%%%%%%%%%%%%%%%%%%%%%%%%%%%%%%%%%%%%%%%%%%%%%%%%%%%%%%%%
\subsection{Cartan-Weyl Basis}

\begin{definition}[Cartan-Weyl Basis]
Given a Lie algebra $\mf{g}$, with Cartan subalgebra $\mf{h}=\spn\lbrace H^i : i=1,\dots,r\rbrace$, let $\lbrace E^\alpha : \alpha\in\Phi \rbrace$ be the simultaneous eigenvectors of $ad_{H^i}$ s.t. 
\[
ad_{H^i}\left(E^\alpha \right)=\alpha^i E^\alpha
\]
Then 
\[
B=\lbrace H^i : i=1,\dots,r\rbrace \cup \lbrace E^\alpha : \alpha\in\Phi \rbrace
\]
is the \bam{Cartan-Weyl basis} for $\mf{g}$.
\end{definition}

\begin{definition}[Roots]
The \bam{roots} of a Lie Algebra are $\alpha\in\Phi$.
These are the weights corresponding to the adjoint representation. 
\end{definition}

\begin{fact}
The roots are all non-degenerate i.e. the eigenspace is 1 dimensional. 
\end{fact}

\begin{theorem}
Let $\kappa$ be the Killing form on $\mf{g}$. Then, 
\begin{itemize}
    \item $\forall H\in\mf{h}, \forall\alpha\in\Phi, \; \kappa\left(H,E^\alpha\right)=0 $
    \item $\forall\alpha,\beta\in\Phi, \alpha+\beta\neq0, \; \kappa\left(E^\alpha,E^\beta\right)=0$
    \item $\forall H\in\mf{h}, \exists H^\prime\in\mf{h}, \; \kappa\left(H,H^\prime\right)\neq0$
    \item $\alpha\in\Phi \Rightarrow -\alpha\in\Phi \text{ and } \kappa\left(E^\alpha,E^{-\alpha}\right)\neq0$
\end{itemize}
Hence $\kappa|_{\mf{h}\times\mf{h}}$ is non-degenerate, so invertible. 
\end{theorem}

\begin{fact}
Given $\alpha\in\Phi$, $\lbrace \lambda : \lambda\alpha\in\Phi \rbrace = \lbrace \pm1 \rbrace$. Note one direction of this inclusion follows from the previous result. 
\end{fact}

\begin{definition}[Dual Space Inner Product]
The Killing form gives the dual space $\mf{h}^\ast$ a natural inner product $(\cdot,\cdot):\mf{h}^\ast \times \mf{h}^\ast \to \mbb{C}$ defined by 
\[
(\alpha, \beta) = \left(\kappa^{-1}\right)_{ij} \alpha^i \beta^j = k^{ij} \alpha_i \beta_j
\]
Defining $\alpha_i=\left( \kappa^{-1} \right)_{ij} \alpha^j$. Note that it is immediate that the bracket is symmetric and bilinear. Positive definiteness follows from the fact
\[
(\alpha, \alpha)=\sum_{\delta\in\Phi} (\alpha, \delta)^2
\]
\end{definition}

\begin{definition}[Coroots]
Given a root $\alpha\in\Phi$ the \bam{coroot} is 
\[
\alpha^\vee=\frac{2}{(\alpha,\alpha)}\alpha
\]
Note $\left( \alpha^\vee \right)^\vee=\alpha$
\end{definition}

\begin{definition}[$H^\alpha$]
Define 
\[
    H^\alpha = \frac{\comm[E^\alpha]{E^{-\alpha}}}{\kappa\left(E^\alpha,E^{-\alpha}\right)}
\]
In components
\[
H^\alpha = \left( \kappa^{-1}\right)_{ij} \alpha^j H^i
\]
\end{definition}

\begin{theorem}[Cartan-Weyl Basis Algebra]
The algebra of the Cartan-Weyl basis is 
\begin{align*}
    \comm[H^i]{H^j} &= 0 \\
    \comm[H^i]{E^\alpha} &= 0 \\
    \comm[E^\alpha]{E^\beta} &= \left\{ \begin{array}{lc} N_{\alpha,\beta}E^{\alpha+\beta} & \alpha+\beta\in\Phi \\
    \kappa\left(E^\alpha,E^{-\alpha}\right)H^\alpha & \alpha+\beta=0 \\
    0 & \text{otherwise}
    \end{array} \right. \\
\end{align*}
\end{theorem}

\begin{fact}
$\forall H\in\mf{h}$, $\alpha,\beta\in\Phi$
\begin{itemize}
    \item $\kappa\left(H^\alpha,H\right)=\alpha(H)$
    \item $\kappa\left(H^\alpha,E^\beta\right)=0$
    \item $\comm[H^\alpha]{H}=0$
    \item $\comm[H^\alpha]{E^\beta}=(\alpha,\beta)E^\beta$
\end{itemize}
\end{fact}

\begin{theorem}
Letting 
\begin{align*}
    e^\alpha &= \sqrt{\frac{2}{(\alpha,\alpha)\kappa\left(E^\alpha,E^{-\alpha}\right)}}E^\alpha \\
    h^\alpha &= \frac{2}{(\alpha,\alpha)}H^\alpha
\end{align*}
yields the algebra
\begin{align*}
    \comm[h^\alpha]{h^\beta} &= 0 \\
    \comm[h^\alpha]{e^\beta} &= \frac{2(\alpha,\beta)}{(\alpha,\alpha)}e^\beta \\
    \comm[e^\alpha]{e^\beta} &= \left\{ \begin{array}{lc} n_{\alpha,\beta}e^{\alpha+\beta} & \alpha+\beta\in\Phi \\
    h^\alpha & \alpha+\beta=0 \\
    0 & \text{otherwise}
    \end{array} \right.\\
\end{align*}
\end{theorem}

\begin{definition}[$sl(2)_\alpha$]
For $\alpha\in\Phi$ there is a $\mf{su}_\mbb{C}(2)$ subalgebra 
\[
sl(2)_\alpha=\spn\lbrace h^\alpha, e^{\pm\alpha}\rbrace
\]
\end{definition}


%%%%%%%%%%%%%%%%%%%%%%%%%%%%%%%%%%%%%%%%%%%%%%%%%%%%%%%%
\subsection{Root Geometry}

\begin{definition}[$\alpha$-string through $\beta$]
The \bam{$\alpha$-string through $\beta$} is 
\[
S_{\alpha, \beta}=\lbrace \beta+\rho\alpha\in\Phi : \rho\in\mbb{Z} \rbrace
\]
The corresponding subspace of $\mf{g}$ is 
\[
V_{\alpha, \beta}=\left\{ e^{\beta+\rho\alpha} : \beta+\rho\alpha\in S_{\alpha, \beta} \right\}
\]
The \bam{length} of the string is 
\[
l_{\alpha,\beta}=|S_{\alpha,\beta}|
\]
\end{definition}
\begin{fact}
$V_{\alpha, \beta}$ is a representation space of the adjoint representation $R=d_{adj}$of $sl(2)_\alpha$. The non-degeneracy of the roots ensures that the weight set 
\begin{align*}
    S_R &= \left\{ \frac{2(\alpha,\beta)}{(\alpha,\alpha)}+2\rho : \beta+\rho\alpha\in S_{\alpha, \beta} \right\} \\
     &= \lbrace -\Lambda,-\Lambda+2,\dots,\Lambda\rbrace=S_\Lambda \\
\end{align*}
Hence $R_{\alpha,\beta}=\frac{2(\alpha,\beta)}{(\alpha,\alpha)} \in\mbb{Z}$
\end{fact}

\begin{fact}
Writing
\[
S_{\alpha, \beta}=\lbrace \beta+\rho\alpha\in\Phi : n_- \leq \rho \leq n_+ \rbrace
\]
with $n_+ \geq 0$ and $n_- \leq 0$ yields
\begin{itemize}
    \item $l_{\alpha,\beta}=n_+-n_-+1$
    \item $R_{\alpha,\beta}=\frac{2(\alpha,\beta)}{(\alpha,\alpha)} = -(n_+ + n_-) \in\mbb{Z}$
\end{itemize}
\end{fact}

\begin{definition}[Weyl Reflections and Group]
For $\alpha\in\Phi$, the \bam{Weyl reflection} in the hyperplane orthogonal to $\alpha$ is 
\begin{align*}
    & w_\alpha : \mf{h}^\ast \to \mf{h}^\ast \\
    & w_\alpha (x) = x - \frac{2(\alpha,x)}{(\alpha,\alpha)}\alpha
\end{align*}
Noting 
\begin{itemize}
    \item $(w_\alpha)^2=id$
    \item $w_\alpha (\alpha) = -\alpha$
    \item $(\alpha,x)=0 \Rightarrow w_\alpha (x) = x$
\end{itemize}
It can be seen $w_\alpha$ is indeed a reflection in the hyperplane orthogonal to $\alpha$, and hence is an isometry. The \bam{Weyl group} is 
\[
W=\lbrace w_\alpha : \alpha\in\Phi \rbrace
\]
with composition as the group operation. Using the previous fact it can be seen $w_\alpha$ restricts to an isometry on $\Phi$. Hence $W$ acts via permutation on $\Phi$, and so $W\leq S_\Phi$
\end{definition}

\begin{fact}
Under the assumption that $ad_{H^i}, ad_{H^j}$ are simultaneously diagonalisable $\forall i,j$ the Killing form, scaled by a factor $\frac{1}{N}$, is given by 
\[
\kappa^{ij}=\frac{1}{N} \sum_{\delta\in\Phi} \delta^i\delta^j
\]
Hence 
\[
(\alpha, \beta)= \frac{1}{N}\sum_{\delta\in\Phi} (\alpha, \delta) (\beta, \delta)
\]
\end{fact}

\begin{fact}
\[
\forall \alpha, \beta\in\Phi \; (\alpha, \beta)\in\mbb{R}
\]
\end{fact}

\begin{theorem}
$\mf{h}^\ast=\spn_\mbb{C}\lbrace \alpha\in\Phi \rbrace$. Hence $r$ roots may be chosen to form a basis $\lbrace \alpha_{(i)} : i=1,\dots,r \rbrace$
\end{theorem}

\begin{definition}[$\mf{h}^\ast_\mbb{R}$]
Given a basis of roots $\lbrace \alpha_{(i)} : i=1,\dots,r \rbrace$, $\mf{h}^\ast_\mbb{R}$ is defined as 
\[
\mf{h}^\ast_\mbb{R} = \spn_\mbb{R} \lbrace \alpha_{(i)} : i=1,\dots,r \rbrace
\]
\end{definition}

\begin{fact}
$\Phi\subset\mf{h}^\ast_\mbb{R}$
\end{fact}

\begin{theorem}
$(\cdot,\cdot) : \mf{h}^\ast_\mbb{R} \times \mf{h}^\ast_\mbb{R} \to \mbb{R}$ is a Euclidean inner product. 
\end{theorem}

\begin{definition}[Length and angle between roots]
Given $\alpha\in\Phi$ the \bam{length} of $\alpha$ is defined as 
\[
|\alpha|=(\alpha,\alpha)^\frac{1}{2}
\]
Given also $\beta\in\Phi$, the \bam{angle} between $\alpha$ and $\beta$ is $\phi$ defined such that 
\[
(\alpha, \beta) = |\alpha||\beta|\cos{\phi}
\]
\end{definition}

\begin{theorem}
\[
\cos\phi=\pm\frac{\sqrt{n}}{2}
\]
for $n\in\lbrace 0,1,2,3,4 \rbrace$
\end{theorem}

\begin{definition}[Positive roots]
Given a hyperplane $H\leq\mf{h}^\ast$ s.t. $H\cap\Phi=\emptyset$ and $\mf{h}^\ast$ separated into $\mf{h}^\ast_\pm$ let 
\[
\Phi_\pm=\Phi\cap\mf{h}^\ast_\pm
\]
Define the \bam{positive roots} to be $\alpha\in\Phi_+$
\end{definition}

\begin{fact} Let $\Phi_\pm$ be as defined.
\begin{itemize}
    \item $\alpha\in\Phi_\pm \Rightarrow -\alpha\in\Phi_\mp$
    \item $\alpha,\beta\in\Phi_\pm,\, \alpha+\beta\in\Phi \Rightarrow \alpha+\beta\in\Phi_\pm$
\end{itemize}
\end{fact}

\begin{definition}[Simple Roots]
A roots $\delta$ is \bam{simple} if 
\begin{itemize}
\item $\delta\in\Phi_+$
\item $\nexists \alpha,\beta\in\Phi_+ \, s.t. \, \delta=\alpha+\beta$
\end{itemize}
The set of simple roots is $\Phi_S$
\end{definition}

\begin{theorem}[Properties of simple roots]
Let $\alpha,\beta\in\Phi_S$, $\alpha\neq\beta$. Then
\begin{itemize}
    \item $\alpha-\beta\notin\Phi$
    \item $l_{\alpha,\beta}=1-\frac{2(\alpha,\beta)}{(\alpha,\alpha)}$
    \item $(\alpha,\beta)\leq0$
    \item $\Phi_+\subset\spn_{\mbb{N}_0}\Phi_S$
    \item The simple roots are linearly independent.
    \item $|\Phi_S|=r$
\end{itemize}
Hence $\Phi_S=\lbrace \alpha_{(i)} : i=1,\dots,r \rbrace$ is a basis for $\mf{h}^\ast_\mbb{R}$
\end{theorem}

%%%%%%%%%%%%%%%%%%%%%%%%%%%%%%%%%%%%%%%%%%%%%%%%%%%%%%%%
\subsection{Cartan Matrix and Dynkin Diagrams}

\begin{definition}[Cartan Matrix]
The \bam{Cartan matrix} $A$ is an $r \times r$ matrix with elements 
\[
A^{ij}=\frac{2(\alpha_{(i)},\alpha_{(j)})}{(\alpha_{(j)},\alpha_{(j)})}
\]
\end{definition}

\begin{definition}[Chevalley Basis]
For each $\alpha_{(i)}\in\Phi_S$ the basis of $sl(2)_{\alpha_{(i)}}$
\[
\left\{ h^i=h^{\alpha_{(i)}}, e^i_\pm=e^{\pm\alpha_{(i)}} \right\}
\]
is the \bam{Chevalley basis}
\end{definition}

\begin{theorem}
In the Chevalley basis
\begin{itemize}
    \item $\comm[h^i]{h^j}=0$
    \item $\comm[h^i]{e^j_\pm}=\pm A^{ji} e^j_\pm$
    \item $\comm[e^i_+]{e^j_-}=\delta_{ij} h^i$
\end{itemize}
\end{theorem}

\begin{theorem}[Serre Relation]
The \bam{Serre relation} is 
\[
\left( ad_{e^i_\pm}\right)^{1-A^{ji}} e^j_\pm=0
\]
\end{theorem}

\begin{idea}
The previous two results show that the Cartan Matrix completely determines a finite dimensional, simple, complex Lie algebra. 
\end{idea}

\begin{theorem}[Properties of the Cartan Matrix]
The Cartan matrix satisfies
\begin{itemize}
    \item $\forall i \, A^{ii}=2$
    \item $A^{ij}=0 \Rightarrow A^{ji}=0$
    \item for $i \neq j \; A^{ij}\in\mbb{Z}_{\leq0}$
    \item $|A^{ij}|\leq4$
    \item $\det{A}>0$
    \item $A$ irreducible (for simple Lie algebras)
\end{itemize}
\end{theorem}

\begin{definition}[Dynkin Diagrams]
The \bam{Dynkin diagrams} are constructed from the Cartan matrix as follows:
\begin{itemize}
    \item Draw a node for each simple root.
    \item Connect the $i^{\text{th}}$ and $j^\text{th}$ root with $\max\lbrace |A^{ij}|, |A^{ji}| \rbrace$ lines. 
    \item If the roots have different length, draw an arrow from the longer root to the shorter. 
\end{itemize}
\end{definition}

%%%%%%%%%%%%%%%%%%%%%%%%%%%%%%%%%%%%%%%%%%%%%%%%%%%%%%%%
\subsection{Lattices}

\begin{definition}[Root Lattice]
Given a Lie algebra $\mf{g}$ with simple roots $\Phi_S=\lbrace \alpha_{(i)} : i=1,\dots,r \rbrace$ the \bam{root lattice} is 
\[
\mathcal{L} [\mf{g}] = \left\{ \sum_{i=1}^r m_i \alpha_{(i)} : m_i \in\mbb{Z} \right\}
\]
\end{definition}

\begin{definition}[Coroot Lattice]
The \bam{coroot lattice} is given by 
\[
\mathcal{L}^{\bigvee} [\mf{g}] =\left\{ \sum_{i=1}^r m_i \alpha^\vee_{(i)} : m_i \in\mbb{Z} \right\}
\]
\end{definition}

\begin{definition}[Quantisation Condition]
Given a representation $R$ of the $sl(2)_\alpha$ subalgbera, and $v\in V_\lambda$ for $\lambda$ a weight of the representation,
\[
R(h^\alpha)v=\frac{2(\alpha,\lambda)}{(\alpha,\alpha)}v
\]
Hence as the weights of $sl(2)_\alpha$ are integers
\[
\frac{2(\alpha,\lambda)}{(\alpha,\alpha)} \in \mbb{Z}
\]
\end{definition}

\begin{definition}[Weight Lattice]
The \bam{weight lattice}, $\mathcal{L}_W [\mf{g}]$, is the dual of the coroot lattice, defined as 
\[
\mathcal{L}_W [\mf{g}] = \left( \mathcal{L}^{\bigvee} [\mf{g}] \right)^\ast = \left\{ \lambda\in\mf{h}^\ast_\mbb{R} : \forall \mu\in\mathcal{L}^{\bigvee} [\mf{g}] \, (\lambda, \mu) \in \mbb{Z} \right\}
\]
\end{definition}

\begin{theorem}
Note 
\[
(\lambda, \alpha^\vee_{(i)})=\frac{2(\alpha_{(i)},\lambda)}{(\alpha_{(i)},\alpha_{(i)})} \in \mbb{Z}
\]
Hence all weights Lie in the weight lattice. 
\end{theorem}

\begin{definition}[Fundamental Weights]
The dual to the coroot basis, i.e. the weights $\omega_{(i)}$ such that 
\[
(\alpha_{(i)}, \omega_{(j)})=\delta_{ij}
\]
are the \bam{fundamental weights}
\end{definition}

\begin{fact}
Writing $\omega_{(i)}=\sum_{j=1}^r B^{ij} \alpha_{(j)}$ yields $\delta_{ij}=A^{ki}B_{jk}$. Hence 
\[
\alpha_{(i)}=\sum_{j=1}^r A^{ij} \omega_{(j)}
\]
where $A^{ij}$ is the Cartan matrix. 
\end{fact}

\begin{definition}[Highest Weight]
Given a finite dimensional representation, the \bam{highest weight} is $\Lambda$ such that 
\[
\forall v_\Lambda\in V_\Lambda, \, \forall \alpha\in\Phi_+ \quad R(E^\alpha)v_\Lambda=0
\]
If the representation is irreducible, all other weight are generated as product of $R(E^{-\alpha})$ for $\alpha\in\Phi_+$. 
\end{definition}

\begin{theorem}[Theorem of the Highest Weight]
Every finite dimension, irreducible, representation has a highest weight. Moreover, if two such representation have the same highest weight, they are isomorphic. Call the representation $R_\Lambda$. 
\end{theorem}

\begin{fact}
Given a representation R, for $v\in V_\lambda$, 
\[
\lambda+\alpha\in S_R \Rightarrow R(E^\alpha)v \in V_{\lambda+\alpha}
\]
Hence if $R=R_\Lambda$ all weights are of the form 
\[
\lambda=\Lambda=\mu
\]
for $\mu=\sum_{i=1}^r \mu_i \alpha_{(i)}$, $\mu_i\in\mbb{N}_0$. Hence writing 
\[
\lambda=\sum_{i=1}^r \lambda^i \omega_{(i)}
\]
gives that
\[
\forall m_{(i)}\in\mbb{Z}, \, 0\leq m_{(i)}\leq \lambda^i, \quad \lambda-m_{(i)}\alpha_{(i)}\in S_R
\]
\end{fact}


\end{document}
