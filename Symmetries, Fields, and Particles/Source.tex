\documentclass{article}
\usepackage[utf8]{inputenc}
\usepackage{amsmath}
\usepackage{amsthm}
\usepackage{amssymb}
\usepackage{hyperref}

\newtheorem{theorem}{Theorem}[section]
\newtheorem{corollary}{Corollary}[theorem]
\newtheorem{lemma}[theorem]{Lemma}
\newtheorem*{remark}{Remark}
\newtheorem{definition}{Definition}[section]
\newtheorem*{fact}{Fact}

\DeclareMathOperator{\spn}{Span}
\DeclareMathOperator{\tr}{Tr}

\newcommand{\bam}[1]{\underline{\bf{#1}}}
\newcommand{\mf}[1]{\mathfrak{#1}}
\newcommand{\mbb}[1]{\mathbb{#1}}
\newcommand{\comm}[2][]{\left[ #1, #2 \right]}

\title{Symmetries Fields and Particles Revision Notes}
\author{Linden Disney-Hogg}
\date{December 2018}

\begin{document}

\maketitle
\tableofcontents

\section{Introduction}
A brief overview of some key ideas concepts and facts that I find useful in revising SFP. 

%%%%%%%%%%%%%%%%%%%%%%%%%%%%%%%%%%%%%%%%%%%%%%%%%%%%%%%%

\section{Definitions - Lie Algebras}

\begin{definition}[ideal]
An \bam{ideal} is a subalgebra $\mf{h} \leq \mf{g}$ s.t.
\[
\forall X\in\mf{g}, \; \forall Y\in\mf{h} \quad  \comm[X]{Y}\in\mf{h}
\]
\end{definition}

\begin{definition}[Simple Lie Algebra]
A Lie Algebra is \bam{simple} if it contains no non-trivial ideals.
\end{definition}

\begin{definition}[Semi-Simple Lie Algebra]
A Lie Algebra is \bam{semi-simple} if it contains no non-trivial abelian ideals. 
\end{definition}

\begin{definition}[Compact Type]
A real Lie algebra $\mf{g}_\mathbb{R}$ is of \bam{compact type} if $\exists$ a basis in which $\kappa^{ij}=-K\delta^{ij}$ for some $K\in\mathbb{R}^{+}$.
\end{definition}

\begin{definition}[Adjoint Map]
The \bam{adjoint map} of an element $X\in\mf{g}$ is $ad_{X}:\mf{g}\to\mf{g}$ defined such that for $Y\in\mf{g}$
\[
ad_{X}\left(Y\right)=\comm[X]{Y}
\]
\end{definition}

\begin{definition}[ad-diagonalisable]
An element $X\in\mf{g}$ is \bam{ad-diagonalisable} if the linear map $ad_X$ is diagonalisable. 
\end{definition}

\begin{definition}[Killing Form]
The \bam{killing form} is a map $\kappa:\mf{g}\times\mf{g}\to\mbb{C}$ defined such that for $X,Y\in\mf{g}$
\[
\kappa\left(X,Y\right)=\tr\left(ad_X \circ ad_Y\right)
\]
\end{definition}

\begin{definition}[Complexification and Real Form]
Given a  real Lie algebra $\mf{g}_\mathbb{R}\equiv\spn_{\mathbb{R}}\lbrace T^a : a=1, \dots, D \rbrace$ the \bam{complexification}
is $\mf{g}_\mathbb{C}\equiv\spn_{\mathbb{C}}\lbrace T^a : a=1, \dots, D \rbrace$. Conversely, given such a $\mf{g}_\mathbb{C}$, $\mf{g}_\mathbb{R}$ is called a \bam{real form}.  
\end{definition}

\begin{definition}[Cartan Subalgebra]
A \bam{Cartan subalgebra} is a maximal abelian subalgebra containing only ad-diagonalisable elements.  
\end{definition}

%%%%%%%%%%%%%%%%%%%%%%%%%%%%%%%%%%%%%%%%%%%%%%%%%%%%%%%%

\section{Definitions - Representations}

\begin{definition}[Representations]
A \bam{representation} of a Lie group $G$ is a smooth group homomorphism 
\[
D:G\to GL\left(V\right)
\]
where $V$ is a $n$-dimensional vector space called the \bam{representation space}.
In analogy a representation of a Lie algebra is a Lie algebra homomorphism 
\[
d:\mf{g} \to \mf{gl}\left(V\right)
\]
\end{definition}

\begin{definition}[Dimension]
The \bam{dimension} of a representation is the dimension of the representation space.
\end{definition}

\begin{definition}[Faithful Representation]
A representation is \bam{faithful} if it is injective. 
\end{definition}

\begin{definition}[Isomorphic Representations]
Two representations $R_1$, $R_2$ of a Lie algebra $\mf{g}$ are \bam{isomorphic}, written $R_1 \cong R_2$ if $\exists S$ an invertible matrix such that 
\[
\forall X\in\mf{g} \quad R_2\left(X\right)=S R_1\left(X\right) S^{-1}
\]
\end{definition}

\begin{definition}[Invariant Subspace]
Given a representation $R$ of Lie algebra $\mf{g}$ with representation space $V$, an \bam{invariant subspace} is $U\leq V$ such that
\[
\forall X\in\mf{g}, \; \forall u\in U \quad R\left(X\right)u\in U
\]
\end{definition}

\begin{definition}[Irreducible Representation]
A representation is \bam{irreducible} if it has no invariant subspaces. 
\end{definition}

%%%%%%%%%%%%%%%%%%%%%%%%%%%%%%%%%%%%%%%%%%%%%%%%%%%%%%%%

\section{SU(2)}




%%%%%%%%%%%%%%%%%%%%%%%%%%%%%%%%%%%%%%%%%%%%%%%%%%%%%%%

\section{Cartan Classification}
\subsection{Root Geometry}

\begin{definition}[Roots]
The roots of a Lie Algebra are $\alpha\in\Phi$, vectors with components $\alpha^i$ eigenvalues of $ad_{H^i}$.
\end{definition}

\begin{definition}[$\alpha$-string through $\beta$]
The \bam{$\alpha$-string through $\beta$} is 
\[
S_{\alpha, \beta}=\lbrace \beta+\rho\alpha\in\Phi : \rho\in\mbb{Z} \rbrace
\]
\end{definition}

\begin{fact}
Every complex semi-simple Lie algebra of finite dimension has a real form of compact type.
\end{fact}



\end{document}
