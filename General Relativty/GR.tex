\documentclass{article}

\usepackage{header}
%%%%%%%%%%%%%%%%%%%%%%%%%%%%%%%%%%%%%%%%%%%%%%%%%%%%%%%%
%Preamble

\title{General Relativity Revision Notes}
\author{Linden Disney-Hogg}
\date{January 2019}

%%%%%%%%%%%%%%%%%%%%%%%%%%%%%%%%%%%%%%%%%%%%%%%%%%%%%%%%
%%%%%%%%%%%%%%%%%%%%%%%%%%%%%%%%%%%%%%%%%%%%%%%%%%%%%%%%
\begin{document}

\maketitle
\tableofcontents

%%%%%%%%%%%%%%%%%%%%%%%%%%%%%%%%%%%%%%%%%%%%%%%%%%%%%%%%
%%%%%%%%%%%%%%%%%%%%%%%%%%%%%%%%%%%%%%%%%%%%%%%%%%%%%%%%
\section{Introduction}
A brief overview of some key ideas, concepts, and facts that I find useful in revising GR.
%%%%%%%%%%%%%%%%%%%%%%%%%%%%%%%%%%%%%%%%%%%%%%%%%%%%%%%%
%%%%%%%%%%%%%%%%%%%%%%%%%%%%%%%%%%%%%%%%%%%%%%%%%%%%%%%%
\section{Connections}

\begin{definition}[Vierbein Fields]
The \bam{veirbein fields} are defined as $e^\mu_a$ such that, given a metric $g_{ab}$
\[
g_ab = e^\mu_a e^\nu_b \eta_{\mu\nu}
\]
\end{definition}

\begin{definition}[Lorentz and Spacetime indices]
Let Greek indices represent \bam{Lorentz indices} and Latin indices represent \bam{spacetime indices}. These are converted between via 
\[
V^\mu = e^\mu_a V^a \Leftrightarrow V^a = e^a_\mu V^\mu
\]
where 
\[
e^a_\mu=g^{ab} \eta_{\mu\nu} e^\nu_b
\]
\end{definition}

\begin{definition}[Pseudo-orthonormal Basis]
Define the pseudo-orthonormal basis of one forms $\set{E^\mu}$ by 
\[
E^\mu = e^\mu_a dx^a
\]
In this basis 
\[
ds^2 = g_{ab} dx^a dx^b = \eta_{\mu\nu} E^{\mu} E^{\nu}
\]
\end{definition}

\begin{definition}[Spin Connection]
Define the \bam{spin connection} $\omega\indices{_\mu^\nu_\rho}$ with the covariant derivative 
\[
\nabla_\mu V^\nu = \del_\mu V^\nu + \omega\indices{_\mu^\nu_\rho} V^\rho
\]
subject to the constraints
\[
\nabla_a e^b_\nu = 0
\]
which gives the condition
\[
\omega_{\lambda\tau\nu} =e_\lambda^a e_{b\tau}\left(\partial_a e_\nu^b+\Gamma_{ac}^b e_\nu^c \right)
\]
as well as having a vanishing torsion tensor 
\[
\nabla_\mu e^a_\nu - \nabla_\nu e^a_\mu = T\indices{^\rho_\mu_\nu} e^a_\rho
\]
which gives 
\[
\omega_{\mu\rho\sigma} = -\omega_{\mu\sigma\rho}
\]
\end{definition}

\begin{definition}[Riemann Tensor]
Define the Riemann tensor in terms of $\omega$ as 
\[
(\nabla_\mu \nabla_\nu - \nabla_\nu \nabla_\mu ) V_\rho = R\indices{_\mu_\nu_\rho^\sigma}(\omega) V_\sigma
\]
\end{definition}

\begin{theorem}
\[
R_{abcd}(\Gamma) = e^\mu_a e^\mu_b e^\rho_c e^\sigma_d R_{\mu\nu\rho\sigma}(\omega)
\]
\end{theorem}

\begin{definition}[Ricci Spin Coefficients]
Define the \bam{Ricci spin coefficients} $c\indices{^\mu_\rho_\sigma}$ by 
\[
dE^\mu = \frac{1}{2} c\indices{^\mu_\rho_\sigma} E^\rho \wedge E^\sigma
\]
\end{definition}

\begin{definition}[Connection 1-form]
Define the \bam{connection 1-form} by 
\[
\omega\indices{^\mu_\nu} = \omega\indices{^\mu_\nu_\rho}E^\rho
\]
\end{definition}

\begin{definition}[Torsion 2-form]
Define the \bam{torsion 2-form} by 
\[
\Theta^\mu = \frac{1}{2} T\indices{^\mu_\rho_\sigma} E^\rho \wedge E^\sigma
\]
\end{definition}

\begin{definition}[Curvature 2-form]
Define the \bam{curvature 2-form} by 
\[
\Omega\indices{^\mu_\nu} = \frac{1}{2} R\indices{^\mu_\nu_\rho_\sigma}(\omega) E^\rho \wedge E^\sigma
\]
\end{definition}

\begin{theorem}[Cartan's Equations of Structure]
Cartan's first and second equations of structure are 
\begin{itemize}
    \item $dE^\mu +  \omega\indices{^\mu_\nu} E^\nu = \Theta^\mu = 0 $ 
    \item $\Omega\indices{^\mu_\nu} = d\omega\indices{^\mu_\nu} +  \omega\indices{^\mu_\rho} \wedge  \omega\indices{^\rho_\nu}$
\end{itemize}
\end{theorem}






\end{document}