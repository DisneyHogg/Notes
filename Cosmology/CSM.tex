\documentclass{article}
\usepackage{header}
%%%%%%%%%%%%%%%%%%%%%%%%%%%%%%%%%%%%%%%%%%%%%%%%%%%%%%%%
%Preamble

\title{Cosmology Notes}
\author{Linden Disney-Hogg}
\date{January 2019}

%%%%%%%%%%%%%%%%%%%%%%%%%%%%%%%%%%%%%%%%%%%%%%%%%%%%%%%%
%%%%%%%%%%%%%%%%%%%%%%%%%%%%%%%%%%%%%%%%%%%%%%%%%%%%%%%%
\begin{document}

\maketitle
\tableofcontents

\section{Introduction}
%%%%%%%%%%%%%%%%%%%%%%%%%%%%%%%%%%%%%%%%%%%%%%%%%%%%%%%%
%%%%%%%%%%%%%%%%%%%%%%%%%%%%%%%%%%%%%%%%%%%%%%%%%%%%%%%%

\section{Conventions and prerequisites}

\begin{definition}[Natural Units]
\bam{Natural units} are used where
\[
c=\hbar=k_B = 1
\]
$[c]=LT^{-1}$, $[\hbar]=L^2 M T^{-1}$, $[k_B] = ML^2 T^{-2} K^{-1}$, so in natural units
\[
L=T=K^{-1}=M^{-1}
\]
Units are therefore given in \bam{mass dimension}, e.g. if $[f]=M^d$, write $[f]=d$.
\end{definition}

\begin{fact}
A table of common values and their mass dimension is given below. 
\begin{center}$
\begin{array}{ccc}
    \text{Quantity} & \text{Symbol} & \text{Mass Dimension} \\
    \hline
    \hline
    \text{Energy} & E & 1 \\
    \text{Energy Density} & \rho & 4 \\
    \text{Pressure} & P & 4 \\
    \text{Scale Factor} & a & 0 \\
    \text{Hubble Rate} & H & 1 \\
    \text{Newton's Constant} & G & -2 \\
\end{array}
$\end{center}
\end{fact}

\begin{definition}[Plank Mass]
The \bam{Plank mass} is 
\eq{
M_{pl} = (8 \pi G)^{-\frac{1}{2}}3
}
\end{definition}

\begin{definition}[Mostly plus signature]
The signature that will be used throughout these notes will be the \bam{mostly plus signature}. As these notes will only consider 1+3 dimensional spacetime, as these are all the dimensions that have been observed, this take the form $(-,+,+,+)$. 
\end{definition}
%%%%%%%%%%%%%%%%%%%%%%%%%%%%%%%%%%%%%%%%%%%%%%%%%%%%%%%%
%%%%%%%%%%%%%%%%%%%%%%%%%%%%%%%%%%%%%%%%%%%%%%%%%%%%%%%%
\section{General Relativity}
Some of the tools from General Relativity will be useful in discussions of cosmology. For a better idea of these results, see the GR notes, and these will be referenced where appropriate. 

\begin{definition}[Energy Momentum Tensor]
Given an action $S$ with respect to a metric $g_{ab}$ the \bam{energy momentum tensor} is 
\eq{
T^{ab} = \frac{2}{\sqrt{-g}} \frac{\delta S_{matter}}{g_{ab}}
}
where $g=\det g_{ab}$ and 
\eq{
S_{matter} = \int \mc{L}_{matter} g^{\frac{1}{2}} d^4 x
}
is the matter action. 
\end{definition}

\begin{definition}[Perfect Fluid]
A \bam{perfect fluid} is a medium in which at each point there exist a Local Inertial Frame (LIF). 
\end{definition}

\begin{fact}
In a comoving LIF the energy momentum tensor must be diagonal and isotropic hence 
\eq{
T^\mu_\nu = \diag(-\rho,P,P,P)
}
Hence for a general frame moving with velocity $u^\mu$ s.t $u^\mu u_\mu = - 1$ 
\eq{
T^{\mu\nu} =(\rho+p) u^\mu u^\nu +g^{\mu\nu}p
}
\end{fact}


\begin{definition}[Lie Derivative]
If under an infinitesimal transform $x^\mu \to x^\mu + \eps V^\mu$ a tensor transform as $T \to T^\prime$ then the \bam{Lie Derivative} of $T$ in the $V$ direction is 
\eq{
(\mc{L}_V T)(x) = \lim_{\eps\to 0} \frac{T\indices{^\dots_\dots}(x) - {T^\prime}\indices{^\dots_\dots}(x)}{\eps}
}
Some examples are 
\eq{
\mc{L}_V \phi &= V^\mu \del_\mu \phi \\
\mc{L}_V W^\mu &= V^\nu \nabla_\nu W^\mu - W^\nu \nabla_\nu V^\mu \\
\mc{L}_V W_\mu &= V^\nu \nabla_\nu W^\mu + W_\nu \nabla_\mu V^\nu \\ 
\mc{L}_V T_{\mu\nu} &= V^\rho \nabla_\rho T_{\mu\nu} + T_{\rho\nu} \nabla_\mu V^\rho + T_{\mu\rho} \nabla_\nu V^\rho
}
The most important example is for $g_{\mu\nu}$ the metric tensor where
\eq{
\mc{L}_V g_{\mu\nu} = \nabla_\mu V_\nu + \nabla_\nu V_\mu
}
\end{definition}

\begin{definition}[Isometry]
A transform $x \to x^\prime$ is an \bam{isometry} if under the transform 
\eq{
g_{\mu\nu} \to g^\prime_{\mu\nu} \; g^\prime_{\mu\nu}(x) = g_{\mu\nu}(x)
}
\end{definition}

\begin{definition}[Killing Vector]
A vector $\xi^\mu$ is a \bam{Killing vector} if 
\eq{
\mc{L}_\xi g_{\mu\nu} = \nabla_\mu \xi_\nu + \nabla_\nu \xi_\mu =0
}
Motion along integral curves of $\xi$ are then isometries. 
\end{definition}

\begin{prop}
For a Killing vector $\xi$
\eq{
\nabla_\rho \nabla_\sigma \xi_\mu = R\indices{_\lambda_\sigma_\mu_\rho} \xi^\lambda
}
\end{prop}
\begin{corollary}
A Killing vector field  is determined uniquely by specifying $\xi_\mu ,\nabla_\nu \xi^\mu$ at a point. Hence a $D$ dimensional spacetime has at most 
\[
D + \frac{1}{2}D(D-1) = \frac{1}{2}D(D+1)
\]
isometries. 
\end{corollary}

\begin{definition}[Maximally Symmetric Spactime]
A $D$ dimensional spacetime is \bam{maximally symmetric} if it admits $\frac{1}{2}D(D+1)$ independent Killing vectors. 
\end{definition}

\begin{prop}
In maximally symmetric spacetime the Riemann tensor is 
\eq{
R_{\mu\nu\rho\sigma} = K g_{\mu(\sigma}g_{\nu\rho)}
R = -D(D-1) K 
}
Hence the Ricci scalar uniquely characterises a maximally symmetric spacetime. 
\end{prop}

\begin{fact}
If $N\leq M$ is a maximally symmetric subspace then the line element on $M$ can be expressed as the \bam{warp product} 
\eq{
ds^2 = g_{ab}(x) dx^a dx^b + f(x) \tilde{g}_{ij}(y) dy^i dy^j
}
where $y$ are coordinate on $N$ and $x$ coordinates on the rest of $M\setminus N$
\end{fact}

\begin{definition}[Einstein-Hilbert Action]
The \bam{Einstein - Hilbert action} is 
\eq{
S_{EH} = \frac{1}{2}M_{pl}^2 \int (R-2\Lambda) g^{\frac{1}{2}} d^4 x
}
\end{definition}

\begin{definition}[Einstein Equations]
The \bam{Einstein Equations} are a set of 10 non-linear PDEs for the metric in the presence of some energy momentume tensor, and they are 
\eq{
M_{pl}^2 ( R_{ab} - \frac{1}{2}R g_{ab} + \Lambda g_{ab} = T_{ab}
}
\end{definition}

\begin{theorem}
The Einstein equations are obtained from finding when the action 
\eq{
S = S_{EH} + S_{Matter}
}
is stationary, i.e. $\delta S = 0$ with respect to variations in the metric. 
\end{theorem}

\begin{prop}[Contracted Bianchi Identity]
\eq{
\nabla^a(R_{ab} -  \frac{1}{2}R g_{ab} ) = \nabla^a R_{ab} - \frac{1}{2} \nabla_b R = 0
}
\end{prop}
\begin{corollary}
The energy momentum tensor is conserved, i.e. 
\eq{
\nabla_a T^{ab} = 0 
}
\end{corollary}

%%%%%%%%%%%%%%%%%%%%%%%%%%%%%%%%%%%%%%%%%%%%%%%%%%%%%%%%
%%%%%%%%%%%%%%%%%%%%%%%%%%%%%%%%%%%%%%%%%%%%%%%%%%%%%%%%
\section{FLRW}
%%%%%%%%%%%%%%%%%%%%%%%%%%%%%%%%%%%%%%%%%%%%%%%%%%%%%%%%
\subsection{Metric}
On large scales, at any given time the universe looks \emph{homogeneous} and \emph{isotropic}, and hence it has a maximally symmetric constant time hypersurface. As a result the line element is 
\eq{
ds^2 = dt^2 + a(t)^2 \tilde{g}_{ij}(x) dx^i dx^j
}
Note $x$ are \emph{comoving} coordinates such that, for a spacelike separation $\Delta x$
\eq{
|\Delta x|_{phys} = a |\Delta x|
}

\begin{definition}[Hubble Parameter]
Define the \bam{Hubble parameter} $H$
\eq{
H= \frac{\dot{a}}{a}
}
\end{definition}

\begin{definition}[FLRW metric]
The metric induced on the homogeneous isotropic spacetime is the \bam{Friedmann-Lemaitre-Robertson-Walker Metric} 
\eq{
ds^2 &= - dt^2 + a(t)^2 \left[ \frac{dr^2}{1-Kr^2} + r^2 d\Omega^2 \right] \\
 &= - dt^2 + a(t)^2 \left[ d\chi^2 + f(\chi) d\Omega^2 \right]
}
where $K = 0, \pm1$, and 
\eq{
f(\chi) = \left\{ \begin{array}{lc} \sinh^2 \chi  & K = -1 \\
    \chi^2 & K=0 \\
    0\sin^2 \chi & K=1
    \end{array} \right.
}
\end{definition}

\begin{definition}[Conformal time]
Introduce \bam{conformal time} $\tau$ which satisfies 
\[
d\tau = \frac{dt}{a(t)} \Leftrightarrow \tau = \int^t \frac{dt^\prime}{a(t^\prime)}
\]
\end{definition}

In conformal time the FLRW metric becomes 
\eq{
ds^2 = a^2 [ -d\tau^2 + d\chi^2 + f(\chi) d\Omega^2 ]
}

%%%%%%%%%%%%%%%%%%%%%%%%%%%%%%%%%%%%%%%%%%%%%%%%%%%%%%%%
\subsection{Evolution equations}

\begin{definition}[Continuity Equation]
The conservation of the energy momentum tensor implies 
\eq{
\dot{\rho} + 3H(\rho + p) = 0
}
\end{definition}

\begin{definition}[Equation of State]
The \bam{equation of state} is a relation $p=p(\rho)$. In this course the equation of state will be taken to be 
\eq{
p = w\rho
}
where the constant $w$ depends on the fluid.
\end{definition}

\begin{prop}
If the equation of state is $p = w\rho$ then 
\eq{
\dot{\rho} + 3 \frac{\dot{a}}{a}(1+w) \rho = 0 \Rightarrow \rho \propto a^{-3(1+w)}
}
\end{prop}

The examples encountered in this course are 
\begin{center}$
\begin{array}{ccc}
    \text{Fluid} & w & \text{Relation} \\
    \hline
    \hline
    \text{Non-relativistic matter (dust)} & 0 & \rho\propto a^{-3}  \\
    \text{Radiation} & \frac{1}{3} & \rho\propto a^{-4} \\
    \text{Dark energy (cosmological constant)} & -1 & \rho \text{ constant} \\
\end{array}
$\end{center}

\begin{definition}[Friedmann equation]
In the FLRW metric the Einstein equations reduce to 
\eq{
3M_{pl}^2 ( H^2 + \frac{K}{a^2} ) = \sum_i \rho_i
}
this is the \bam{Friedmann equation}. 
\end{definition}

\begin{definition}[Critical Density]
Define the \bam{critical density} to be 
\eq{
\rho_c = 3 M_{pl}^2 H^2
}
writing 
\eq{
\Omega_i &= \frac{\rho_i}{\rho_c} \\
\Omega_K &= \frac{3M_{pl}^2 K}{\rho_c a^2} = \frac{K}{H^2 a^2}
}
the Friedmann equation becomes 
\eq{
1-\Omega_K = \sum_i \Omega_i
}
It is common notation to let $\Omega_{i,0} = \Omega_i |_{t=t_0}$
\end{definition}

%%%%%%%%%%%%%%%%%%%%%%%%%%%%%%%%%%%%%%%%%%%%%%%%%%%%%%%%
\subsection{Possible Universes}

\begin{example}
Consider a flat $(K=0)$ universe, with only one contribution to the energy density. Then 
\eq{
3M_{pl}^2 \left(\frac{\dot{a}}{a}\right)^2 &= \rho_0 \left(\frac{a}{a_0}\right)^{-3(1+w)} \\
\Rightarrow \frac{\dot{a}}{a} &\propto a^{-\frac{3(1+w)}{2}} \\
\frac{da}{a} a^{\frac{3(1+w)}{2}} &= dt \\
\Rightarrow a(t) &= \left\{ \begin{array}{cc} \left[ \frac{3}{2}(1+w) H_0 t \right]^{\frac{2}{3(1+w)}} & w\neq -1 \\ Ae^{H_0 t} & w = -1 \end{array} \right.
}
Hence \begin{itemize}
    \item For non-relativistic matter, $a \propto t^{\frac{2}{3}}$
    \item For radiation, $a \propto t^\frac{1}{2}$
    \item For dark energy, $a \propto e^{H_0 t}$
\end{itemize}

In this case 
\eq{
H = \left\{ \begin{array}{cc}  \frac{2}{3(1+w)}\frac{1}{t}  & w\neq -1 \\ H_0 & w = -1 \end{array} \right.
}
\end{example}

\begin{prop}
If $K=0$, the age of the universe can be calculates as 
\eq{
t_{age} &= \int dt \\
 &= \int \frac{da}{\dot{a}} \\
 &= \int \frac{da}{a H} \\
 &= \frac{1}{H_0} \int \frac{da}{a} \left[  \frac{\sum_i \rho_i}{3 M_{pl}^2 H_0^2} \right]^{-\frac{1}{2}} \\
 &= \frac{1}{H_0} \int \frac{da}{a} \left[ \Omega_\Lambda + \Omega_{m,0}a^{-3} + \Omega_{r,0} a^{-4} \right]^{-\frac{1}{2}}
}
\end{prop}


%%%%%%%%%%%%%%%%%%%%%%%%%%%%%%%%%%%%%%%%%%%%%%%%%%%%%%%%
\subsection{Distance}

\begin{definition}[Redshift]
Suppose that a wave is emitted with wavelength $\lambda_e$ and observed with wavelength $ \lambda_o$. Then the \bam{redshift} $z$ is defined by 
\eq{
1+z = \frac{\lambda_o}{\lambda_e}
}
\end{definition}

\begin{definition}[Cosmological redshift]
For a photon emitted at time $t_e < t_0$ the redshift due to the expansion of the universe is the \bam{cosmological redshift}
\eq{
1+z = \frac{a_0}{a_e}
}
\end{definition}

\begin{definition}[Comoving Distance]
The \bam{comoving distance} $\chi(t_i,t_f)$ is the distance travelled by a photon between times $t_i,t_f$. Note that for a radially moving null geodesic $ds^2 = a^2[-d\tau^2 + d\chi^2] = 0$ so $d\chi=d\tau$, so 
\eq{
\chi(t_i,t_f) &= \int d\tau \\
&= \int_{t_i}^{t_f} \frac{dt}{a} \\
&= \int \frac{da}{a^2 H} \\
&= \int \frac{dz}{H(z)}
}
Writing this then in terms of the redshift 
\eq{
\chi(z) = \int_0^z \frac{dz}{H(z)}
}
\end{definition}

\begin{definition}[Luminosity]
The \bam{intrinsic luminosity} $L$ of an object is the amount of energy radiated per unit time. The \bam{observed luminosity} $l$ is the amount of energy radiated per unit time per unit surface arriving at an observer some distance from the object.
\end{definition}

\begin{definition}[Luminosity Distance]
The observed and instrinsic luminosity are related by 
\eq{
l = \frac{L}{4\pi d_L^2}
}
which defines the $\bam{luminosity distance}$ $d_L$
 \end{definition}
 
 \begin{prop}
 The luminosity distance and comoving distance are related by 
 \eq{
 d_L(z) = (1+z) a_o \chi(z)
 }
 \end{prop}
\begin{proof}
\eq{
l = \frac{L}{4\pi (\chi a_o)^2}\left(frac{a_e}{a_o}\right)^2
}
\end{proof}

\begin{definition}[Angular Diameter Distance]
For an object of size $s$ which subtends an angle $\theta$ from the point of view of the observer, the \bam{angular diameter distance} $d_A$ to the object is 
\eq{
d_A = \frac{s}{\theta}
}
As the angle the object takes up in the sky will be stretched by a factor of (1+z), the angular diameter distance is related to the comoving distance by 
\eq{
d_A(z) = \frac{\chi(z)}{1+z}
}
\end{definition}

\begin{definition}[Particle Horizon]
The \bam{particle horizon} $d_{p.h.}$ is is the greatest distance a photon a could have travelled since the beginning of time $t_i$. Any object a distance $d>d_{p.h.}$ from an observer away can never have communicated with said observer. 
\eq{
d_{p.h.}(t) = a(t) \chi(t,t_i) = a(t) \int_{t_i}^t \frac{dt^\prime}{a(t^\prime)}
}
\end{definition}



%%%%%%%%%%%%%%%%%%%%%%%%%%%%%%%%%%%%%%%%%%%%%%%%%%%%%%%%
%%%%%%%%%%%%%%%%%%%%%%%%%%%%%%%%%%%%%%%%%%%%%%%%%%%%%%%%
\section{Observables and statistical properties}
%%%%%%%%%%%%%%%%%%%%%%%%%%%%%%%%%%%%%%%%%%%%%%%%%%%%%%%%
\subsection{Observables}
\subsubsection*{Large scale structure}
\begin{definition}[Fractional Overdensity]
Let $n_g$ be the galaxy number density. Then define the \bam{fractional overdensity} as 
\[
\delta_g(\bm{x}) = \frac{n_g(\bm{x})-\bar{n}_g}{\bar{n}_g}
\]
where $\bar{n}_g$. 
\end{definition}

It is typical to assume a linear relation ship between the galaxy density and matter density 
\[
n_g = b \rho_m
\]
where b is the \bam{linear galaxy bias}. This gives the \bam{matter density perturbation}
\[
\delta_g(\bm{x}) = b \times \frac{\rho_m(\bm{x})-\bar{\rho}_m}{\bar{\rho}_m} = b \times \delta_m(\bm{x})
\]

\begin{definition}[Two point correlation function]
Given a general field $f$ and some probability of having the field configuration $Pr[f]$ the \bam{two point correlation function} is 
\eq{
\xi^f(\bm{x},\bm{y}) = \braket{f(\bm{x})f(\bm{y})} = \int \mc{D}f f(\bm{x})f(\bm{y}) Pr[f]
}
\end{definition}

\begin{lemma}
Imposing homogeneity and isotropy of the universe gives 
\eq{
\braket{f(\bm{k})f(\bm{k}^\prime)} = (2\pi)^3 P^f(|\bm{k}|) \delta^{(D)}(\bm{k} + \bm{k}^\prime)
}
for the Fourier coefficients $\braket{f(\bm{k})f(\bm{k}^\prime)}$ such that 
\eq{
 \xi^f(\bm{x},\bm{y}) = \braket{f(\bm{x})f(\bm{y})} = \int \frac{d\bm{k}}{(2\pi)^3} \frac{d\bm{k}^\prime}{(2\pi)^3} e^{-i\bm{k}\cdot\bm{x} -i \bm{k}^\prime \cdot \bm{y}} \braket{f(\bm{k})f(\bm{k}^\prime)}
}
This defines the \bam{power spectrum} $P^f$ of the field. It is common to also define the \bam{dimensionless power spectrum} 
\eq{
\Delta^f(k) = \frac{k^3}{2\pi^2} P^f(k)
}
\end{lemma}

\begin{definition}[Gaussain Random Fields]
A \bam{Gaussian random field} is one whose probablity funcitonal takes the form 
\eq{
Pr[f] \propto \frac{\exp -f_i \xi_{ij} f_j}{\sqrt{\det \xi_{ij}}}
}
for 
\eq{
\braket{f_i f_j} = \xi_{ij}
}
\end{definition}

\subsubsection*{CMB}
\begin{idea}
The CMB is "pretty much" linear, which makes it nice to work with. 
\end{idea}

\begin{definition}[Flat Sky approximation]
The CMB temperature $T=T(\hat{\bm{n}})$ is defined on a sphere. However, on a sufficiently small patch the \bam{flat-sky approximation} can be used, taking $T(\bm{l})$ to be the 2D Fourier transform of $T$, then the power spectrum is given by 
\eq{
\braket{T(\bm{l})T(\bm{l}^\prime)} = (2\pi)^3 C_l \delta^{(D)}(\bm{l} + \bm{l}^\prime )
}
\end{definition}

%%%%%%%%%%%%%%%%%%%%%%%%%%%%%%%%%%%%%%%%%%%%%%%%%%%%%%%%
\subsection{Assumptions}
It will be assumed throughout subsequent sections that all observational predictions are generated by "well known" physics if the initial density fluctuation are 
\begin{itemize}
    \item Scale invariant : $P \propto k^{-4 + n_s}$ with $n_s \approx 1$. 
    \item Adiabatic. 
    \item Gaussian.
\end{itemize}
These can be justified through inflation. 

%%%%%%%%%%%%%%%%%%%%%%%%%%%%%%%%%%%%%%%%%%%%%%%%%%%%%%%%
\subsection{Newtonian Perturbation}

For a non-relativistic fluid with mass density $\rho$, pressure $P$, and velocity $\bm{u}$, the determining equations are 
\eq{
\del_t \rho + \grad_{\bm{r}} \cdot (\rho \bm{u}) = 0 \quad \text{(continuity equation)} \\
\del_t \bm{u} + \bm{u} \cdot \grad_{\bm{r}} \bm{u} = -\frac{1}{\rho} \grad_{\bm{r}} P - \grad_{\bm{r}} \Phi \quad \text{(Euler equation)} \\
\nabla_{\bm{r}}^2 \Phi = 4\pi G \rho \quad \text{(Poisson's equation)}
}
For a comoving observer the physical coordinate $\bm{r}$ is related to the comoving coordinate $\bm{x}$ by 
\eq{
\bm{r}(t) = a(t) \bm{x} \\
\Rightarrow \bm{u} = \dot{\bm{r}} = H\bm{r}
}
Hence 
\eq{
(\pd{t})_{\bm{r}} = (\pd{t})_{\bm{x}} - H\bm{x} \cdot \grad \\
\grad_{\bm{r}} = \frac{1}{a} \grad
}
letting $\grad = \grad_{\bm{x}}$. \\
Now writing a perturbation as 
\eq{
\rho \to \bar{\rho} + \delta\rho = \bar{\rho}(1+\delta) \\
P \to \bar{P} + \delta P \\
\bm{u} \to Ha\bm{x} + \bm{v} \\
\Phi \to \bar{\Phi} + \delta Phi
}
and substituting in gives, to zeroth and first order 
\eq{
\pd[\bar{rho}]{t}+3H\bar{\rho} = 0 \quad \text{(continuity for matter)}\\
\dot{\delta} = -\frac{1}{a}\grad\cdot\bm{v} \quad \text{(continuity equation)} \\
\dot{\bm{v}} + H\bm{v} = -\frac{1}{a\bar{\rho}}\grad \delta P -\frac{\grad \Phi}{a} \quad \text{(Euler equation)} \\
\nabla^2 \Phi = 4\pi G a^1 \bar{\rho} \delta \quad \text{(Poisson's equation)}
}
Combining these gives 
\begin{align} \label{eq:CSM:1}
\ddot{\delta} + 2H\dot{\delta} - \frac{1}{a^2\bar{\rho}}\nabla^2 \delta P - 4\pi G \bar{\rho}\delta = 0
\end{align}

\begin{definition}[Barotropic Fluid]
A barotropic fluid is one where $P=P(\rho)$. In such a fluid the speed of sound is 
\[
c_s = \sqrt{\pd[P]{\rho}}
\]
\end{definition}

For a barotropic fluid \ref{eq:CSM:1} gives 
\eq{
\ddot{\delta} + 2H\dot{\delta} - \frac{c_s^2}{a^2}\nabla^2 \delta  - 4\pi G \bar{\rho}\delta = 0
}
and so Fourier transforming 
\eq{
\ddot{\delta} + 2H\dot{\delta} + \left[ \frac{c_s^2}{a^2}k^2 - 4\pi G \bar{\rho} \right]\delta = 0
}

\begin{definition}[Jean's wavenumber]
The \bam{Jean's wavenumber and scale} are
\eq{
k_J - \frac{\sqrt{4\pi G \bar{\rho}a^2}}{c_s} \\
\lambda_J = \frac{2\pi a }{k_J} = c_s \sqrt{\frac{\pi}{G\bar{\rho}}}
}
\end{definition}

Perturbations on a scale smaller than the Jean's scale, i.e. $k>k_J$ the euqation is essentially damped oscillation, whereas for large perturbations $k<k_J$ tge perturbation grows as a power law. 

\begin{idea}
This power law growth should be seen as the case where there is insufficient pressure support to stop the gravitational collapse of a perturbation, as the speed of such a wave is too slow to repsond in sufficient collapse time. This can be seen by estimating the collapse time as 
\eq{
t_f \sim \frac{1}{\sqrt{G\bar{\rho}}}
}
and the time for a pressure wave to cross a perturbation radius $R$ as 
\eq{
t_{sc} = \frac{R}{c_s}
}
For pressure support we would then want 
\eq{
t_{sc} < t_f  \\ 
\Rightarrow R \lesssim \lambda_J
}
\end{idea}

\begin{example}[Dark matter evolution]
Take $\delta=\delta_m$. Assume that $c_s=0$ in dark matter. 
\subsubsection*{Dark matter domination}
In dark matter domination, the Hubble rate is determined only by dark matter. Then as seen before $a\propto t^{\frac{2}{3}}$,  $H=\frac{2}{3t}$ and $H^2=\frac{8\pi G\bar{\rho}}{3}$
\eq{
\ddot{\delta}_m + \frac{4}{3t} \dot{\delta}_m - \frac{2}{3t^2}\delta_m = 0
}
Power law solutions are $\delta \propto t^\frac{2}{3}, t^{-1}$, so growing modes are $\delta \propto a$. 

\subsubsection*{Radiation Domination}
During radiation domination, the total energy density sources the Hubble growth, so 
\eq{
\ddot{\delta}_m + 2H\dot{\delta}_m   - 4\pi G \sum_i \bar{\rho}_i\delta_i = 0
}
It will be shown that on subhorizon scales photon perturbations oscillates rapidly with respect to the time scale of structure formation, and so will not contribute overall. Now in radiation domination $a\propto t^\frac{1}{2}$, $H=\frac{1}{2t}$, so 
\eq{
\ddot{\delta}_m + \frac{1}{t}\dot{\delta}_m   - 4\pi G \bar{\rho}_m\delta_m = 0
}
In the absence of pressure the time space must be fixed so 
\eq{
\ddot{\delta}_m \sim H \dot{\delta}_m \sim H^2\delta_m \sim \frac{8\pi G \rho_r }{3}\delta_m \gg 4\pi G \bar{\rho}_m \delta_m
}
and the equation reduces to 
\eq{
\ddot{\delta}_m + \frac{1}{t} \dot{\delta}_m = 0 
}
Hence the solutions are $\delta_m \propto 1, \log t$ 
giving for the growing mode 
\eq{
\delta_m \log a 
}
\subsubsection*{Dark energy domination}
In dark energy domination, as $\Lambda$ is constant, and $H\gg 4\pi G\bar{\rho}_m$ the equation becomes 
\eq{
\ddot{\delta}_m + 2H \dot{\delta}_m = 0 
}
and so $\delta_m \propto 1, e^{-2Ht}$ so in dark energy domination perturbations stop growing. 
\end{example}

%%%%%%%%%%%%%%%%%%%%%%%%%%%%%%%%%%%%%%%%%%%%%%%%%%%%%%%%
\subsection{Relativistic Perturbation}

\begin{definition}[Horizon scale]
The \bam{conformal horizon scale} is $\mc{H}^{-1}$, where
\eq{
\mc{H} = \frac{a^\prime}{a}
}
is the conformal hubble parameter
A mode is \bam{superhorizon} if $k^{-1}\gg \mc{H}^{-1}$, and it is \bam{subhorizon} if $k^{-1}\ll \mc{H}^{-1}$.
\end{definition}

\begin{fact}
Superhorizon modes are not in causal contact with themselves, and so cannot evolve dynamically. 
\end{fact}

Now for a relativistic perturbation, the following first order equations are 
\eq{
\delta^\prime + 3\mc{H} \left( \frac{\delta P}{\delta \rho} - \frac{\bar{P}}{\bar{\rho}} \right) \delta = - \left( 1 + \frac{\bar{P}}{\bar{\rho}}\right) (\grad \cdot \bm{v} - 3\Phi^\prime) \quad \text{(Conservation of stress energy)} \\
\bm{v}^\prime + 3\mc{H} \left(\frac{1}{3} - \frac{\bar{P}}{\bar{\rho}} \right) \bm{v} = -\frac{\grad \delta P}{\bar{\rho}+\bar{P}} - \grad\Phi \quad \text{(Euler equation)} \\
}
and the Einstein equations 
\begin{align} \label{eq:CSM:2}
\nabla^2 \Phi - 3\mc{H}(\Phi^\prime + \mc{H}\Phi)  &= 4\pi G a^2 \delta \rho \\
\Phi^\prime + \mc{H}\Phi &= -4\pi G a^2 (\bar{\rho} + \bar{P}) v \\
\Phi^{\prime\prime} + 3 \mc{H} \Phi^\prime + (2\mc{H}^\prime + \mc{H}^2 ) \Phi &= 4\pi G a^2 \delta P 
\end{align}
where $\bm{v} = \grad v$.
\begin{definition}[Comoving gauge density contrast]
Define the \bam{Comoving gauge density contrast} $\Delta$ by
\eq{
\Delta = \delta - 3\mc{H}\left(1 + \frac{\bar{P}}{\bar{\rho}} \right) v
}
\end{definition}

Substituting for the CGDC gives 
\eq{
\nabla^2 \Phi = 4\pi G a^2 \bar{\rho} \Delta
}

\begin{definition}[Comoving curvature perturbation]
The \bam{comoving curvature perturbation} $\mc{R}$ is the curvature perturbation on a comoving hyperfusrface and is defined by 
\[
\mc{R} = -\Phi + \mc{H}v
\]
Using \ref{eq:CSM:2} this can be written as 
\eq{
\mc{R} = -\Phi - \frac{\mc{H}(\Phi^\prime + \mc{H} \Phi}{4\pi G a^2 (\bar{\rho} + \bar{P})}
}
\end{definition}

\begin{prop}
$\mc{R}$ is conserved on superhorizon scales
\end{prop}
\begin{proof}
It can be shown 
\eq{
-4\pi G a^2 (\bar{\rho} + \bar{P}) \mc{R}^\prime = \underbrace{4\pi G a^2 \mc{H} (\delta P - \frac{\bar{P}^\prime}{\bar{\rho}^\prime} \delta \rho)}_{=0 \text{ for adiabatic pert}} + \underbrace{\mc{H} \frac{\bar{P}^\prime}{\bar{\rho}^\prime} \nabla^2 \Phi}_{\sim \mc{H}k^2 \mc{R}}
}
Hence 
\eq{
\frac{d \log \mc{R}}{d\log a} \sim (\frac{k}{\mc{H}})^2
}
small on superhorizon scales. 
\end{proof}

%%%%%%%%%%%%%%%%%%%%%%%%%%%%%%%%%%%%%%%%%%%%%%%%%%%%%%%%
%%%%%%%%%%%%%%%%%%%%%%%%%%%%%%%%%%%%%%%%%%%%%%%%%%%%%%%%

\begin{remark}
In cosmology, Bessel functions like to turn up (why?). Hence it can sometimes be useful to try turn an euqaiotn into the form of Bessel's equation 
\[
\frac{d^2y}{dx^2} + \frac{1}{x} \frac{dy}{dx} + \left( 1-\frac{\nu^2}{x^2} \right ) = 0 
\]
\end{remark}


%%%%%%%%%%%%%%%%%%%%%%%%%%%%%%%%%%%%%%%%%%%%%%%%%%%%%%%%
%%%%%%%%%%%%%%%%%%%%%%%%%%%%%%%%%%%%%%%%%%%%%%%%%%%%%%%%
\section{Quantum Inflationary Origin}

\begin{definition}[Mukhanov Sasaki Equation]
The \bam{MSE} is 
\[
f ^ { \prime \prime } - \nabla ^ { 2 } f - \frac { a ^ { \prime \prime } } { a } f = 0
\]
\end{definition}

\end{document}