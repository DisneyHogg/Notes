\documentclass{article}
\usepackage{header}
%%%%%%%%%%%%%%%%%%%%%%%%%%%%%%%%%%%%%%%%%%%%%%%%%%%%%%%%
%Preamble

\title{Cosmology Notes}
\author{Linden Disney-Hogg}
\date{January 2019}

%%%%%%%%%%%%%%%%%%%%%%%%%%%%%%%%%%%%%%%%%%%%%%%%%%%%%%%%
%%%%%%%%%%%%%%%%%%%%%%%%%%%%%%%%%%%%%%%%%%%%%%%%%%%%%%%%
\begin{document}

\maketitle
\tableofcontents

\section{Introduction}
%%%%%%%%%%%%%%%%%%%%%%%%%%%%%%%%%%%%%%%%%%%%%%%%%%%%%%%%
%%%%%%%%%%%%%%%%%%%%%%%%%%%%%%%%%%%%%%%%%%%%%%%%%%%%%%%%
\section{Evolution equations}

\begin{definition}[Conformal time]
Introduce \bam{conformal time} $\tau$ which satisfies 
\[
d\tau = \frac{dt}{a(t)} \Leftrightarrow \tau = \int^t \frac{dt^\prime}{a(t^\prime)}
\]
\end{definition}


%%%%%%%%%%%%%%%%%%%%%%%%%%%%%%%%%%%%%%%%%%%%%%%%%%%%%%%%
%%%%%%%%%%%%%%%%%%%%%%%%%%%%%%%%%%%%%%%%%%%%%%%%%%%%%%%%
\section{Observables and statistical properties}
%%%%%%%%%%%%%%%%%%%%%%%%%%%%%%%%%%%%%%%%%%%%%%%%%%%%%%%%
\subsection{Observables}
\subsubsection*{Large scale structure}
\begin{definition}[Fractional Overdensity]
Let $n_g$ be the galaxy number density. Then define the \bam{fractional overdensity} as 
\[
\delta_g(\bm{x}) = \frac{n_g(\bm{x})-\bar{n}_g}{\bar{n}_g}
\]
where $\bar{n}_g$. 
\end{definition}

It is typical to assume a linear relation ship beteween the galaxy density and matter density 
\[
n_g = b \rho_m
\]
where b is the \bam{linear galaxy bias}. This gives the \bam{matter density perturbation}
\[
\delta_g(\bm{x}) = b \times \frac{\rho_m(\bm{x})-\bar{\rho}_m}{\bar{\rho}_m} = b \times \delta_m(\bm{x})
\]

\begin{definition}[Two point correlation function]
Given a general field $f$ and some probability of having the field configuration $Pr[f]$ the \bam{two point correlation function} is 
\eq{
\xi^f(\bm{x},\bm{y}) = \braket{f(\bm{x})f(\bm{y})} = \int \mc{D}f f(\bm{x})f(\bm{y}) Pr[f]
}
\end{definition}

\begin{lemma}
Imposing homogeneity and isotropy of the universe gives 
\eq{
\braket{f(\bm{k})f(\bm{k}^\prime)} = (2\pi)^3 P^f(|\bm{k}|) \delta^{(D)}(\bm{k} + \bm{k}^\prime)
}
for the Fourier coefficients $\braket{f(\bm{k})f(\bm{k}^\prime)}$ such that 
\eq{
 \xi^f(\bm{x},\bm{y}) = \braket{f(\bm{x})f(\bm{y})} = \int \frac{d\bm{k}}{(2\pi)^3} \frac{d\bm{k}^\prime}{(2\pi)^3} e^{-i\bm{k}\cdot\bm{x} -i \bm{k}^\prime \cdot \bm{y}} \braket{f(\bm{k})f(\bm{k}^\prime)}
}
This defines the \bam{power spectrum} $P^f$ of the field. It is common to also define the \bam{dimensionless power spectrum} 
\eq{
\Delta^f(k) = \frac{k^3}{2\pi^2} P^f(k)
}
\end{lemma}

\begin{definition}[Gaussain Random Fields]
A \bam{Gaussian random field} is one whose probablity funcitonal takes the form 
\eq{
Pr[f] \propto \frac{\exp -f_i \xi_{ij} f_j}{\sqrt{\det \xi_{ij}}}
}
for 
\eq{
\braket{f_i f_j} = \xi_{ij}
}
\end{definition}

\subsubsection*{CMB}
\begin{idea}
The CMB is "pretty much" linear, which makes it nice to work with. 
\end{idea}

\begin{definition}[Flat Sky approximation]
The CMB temperature $T=T(\hat{\bm{n}})$ is defined on a sphere. However, on a sufficiently small patch the \bam{flat-sky approximation} can be used, taking $T(\bm{l})$ to be the 2D Fourier transform of $T$, then the power spectrum is given by 
\eq{
\braket{T(\bm{l})T(\bm{l}^\prime)} = (2\pi)^3 C_l \delta^{(D)}(\bm{l} + \bm{l}^\prime )
}
\end{definition}

%%%%%%%%%%%%%%%%%%%%%%%%%%%%%%%%%%%%%%%%%%%%%%%%%%%%%%%%
\subsection{Assumptions}
It will be assumed throughout subsequent sections that all observational predictions are generated by "well known" physics if the initial density fluctuation are 
\begin{itemize}
    \item Scale invariant : $P \propto k^{-4 + n_s}$ with $n_s \approx 1$. 
    \item Adiabatic. 
    \item Gaussian.
\end{itemize}
These can be justified through inflation. 

%%%%%%%%%%%%%%%%%%%%%%%%%%%%%%%%%%%%%%%%%%%%%%%%%%%%%%%%
\subsection{Newtonian Perturbation}

For a non-relativistic fluid with mass density $\rho$, pressure $P$, and velocity $\bm{u}$, the determining equations are 
\eq{
\del_t \rho + \grad_{\bm{r}} \cdot (\rho \bm{u}) = 0 \quad \text{(continuity equation)} \\
\del_t \bm{u} + \bm{u} \cdot \grad_{\bm{r}} \bm{u} = -\frac{1}{\rho} \grad_{\bm{r}} P - \grad_{\bm{r}} \Phi \quad \text{(Euler equation)} \\
\nabla_{\bm{r}}^2 \Phi = 4\pi G \rho \quad \text{(Poisson's equation)}
}
For a comoving observer the physical coordinate $\bm{r}$ is related to the comoving coordinate $\bm{x}$ by 
\eq{
\bm{r}(t) = a(t) \bm{x} \\
\Rightarrow \bm{u} = \dot{\bm{r}} = H\bm{r}
}
Hence 
\eq{
(\pd{t})_{\bm{r}} = (\pd{t})_{\bm{x}} - H\bm{x} \cdot \grad \\
\grad_{\bm{r}} = \frac{1}{a} \grad
}
letting $\grad = \grad_{\bm{x}}$. \\
Now writing a perturbation as 
\eq{
\rho \to \bar{\rho} + \delta\rho = \bar{\rho}(1+\delta) \\
P \to \bar{P} + \delta P \\
\bm{u} \to Ha\bm{x} + \bm{v} \\
\Phi \to \bar{\Phi} + \delta Phi
}
and substituting in gives, to zeroth and first order 
\eq{
\pd[\bar{rho}]{t}+3H\bar{\rho} = 0 \quad \text{(continuity for matter)}\\
\dot{\delta} = -\frac{1}{a}\grad\cdot\bm{v} \quad \text{(continuity equation)} \\
\dot{\bm{v}} + H\bm{v} = -\frac{1}{a\bar{\rho}}\grad \delta P -\frac{\grad \Phi}{a} \quad \text{(Euler equation)} \\
\nabla^2 \Phi = 4\pi G a^1 \bar{\rho} \delta \quad \text{(Poisson's equation)}
}
Combining these gives 
\begin{align} \label{eq:CSM:1}
\ddot{\delta} + 2H\dot{\delta} - \frac{1}{a^2\bar{\rho}}\nabla^2 \delta P - 4\pi G \bar{\rho}\delta = 0
\end{align}

\begin{definition}[Barotropic Fluid]
A barotropic fluid is one where $P=P(\rho)$. In such a fluid the speed of sound is 
\[
c_s = \sqrt{\pd[P]{\rho}}
\]
\end{definition}

For a barotropic fluid \ref{eq:CSM:1} gives 
\eq{
\ddot{\delta} + 2H\dot{\delta} - \frac{c_s^2}{a^2}\nabla^2 \delta  - 4\pi G \bar{\rho}\delta = 0
}
and so Fourier transforming 
\eq{
\ddot{\delta} + 2H\dot{\delta} + \left[ \frac{c_s^2}{a^2}k^2 - 4\pi G \bar{\rho} \right]\delta = 0
}

\begin{definition}[Jean's wavenumber]
The \bam{Jean's wavenumber and scale} are
\eq{
k_J - \frac{\sqrt{4\pi G \bar{\rho}a^2}}{c_s} \\
\lambda_J = \frac{2\pi a }{k_J} = c_s \sqrt{\frac{\pi}{G\bar{\rho}}}
}
\end{definition}

Perturbations on a scale smaller than the Jean's scale, i.e. $k>k_J$ the euqation is essentially damped oscillation, whereas for large perturbations $k<k_J$ tge perturbation grows as a power law. 

\begin{idea}
This power law growth should be seen as the case where there is insufficient pressure support to stop the gravitational collapse of a perturbation, as the speed of such a wave is too slow to repsond in sufficient collapse time. This can be seen by estimating the collapse time as 
\eq{
t_f \sim \frac{1}{\sqrt{G\bar{\rho}}}
}
and the time for a pressure wave to cross a perturbation radius $R$ as 
\eq{
t_{sc} = \frac{R}{c_s}
}
For pressure support we would then want 
\eq{
t_{sc} < t_f  \\ 
\Rightarrow R \lesssim \lambda_J
}
\end{idea}

\begin{example}[Dark matter evolution]
Take $\delta=\delta_m$. Assume that $c_s=0$ in dark matter. 
\subsubsection*{Dark matter domination}
In dark matter domination, the Hubble rate is determined only by dark matter. Then as seen before $a\propto t^{\frac{2}{3}}$,  $H=\frac{2}{3t}$ and $H^2=\frac{8\pi G\bar{\rho}}{3}$
\eq{
\ddot{\delta}_m + \frac{4}{3t} \dot{\delta}_m - \frac{2}{3t^2}\delta_m = 0
}
Power law solutions are $\delta \propto t^\frac{2}{3}, t^{-1}$, so growing modes are $\delta \propto a$. 

\subsubsection*{Radiation Domination}
During radiation domination, the total energy density sources the Hubble growth, so 
\eq{
\ddot{\delta}_m + 2H\dot{\delta}_m   - 4\pi G \sum_i \bar{\rho}_i\delta_i = 0
}
It will be shown that on subhorizon scales photon perturbations oscillates rapidly with respect to the time scale of structure formation, and so will not contribute overall. Now in radiation domination $a\propto t^\frac{1}{2}$, $H=\frac{1}{2t}$, so 
\eq{
\ddot{\delta}_m + \frac{1}{t}\dot{\delta}_m   - 4\pi G \bar{\rho}_m\delta_m = 0
}
In the absence of pressure the time space must be fixed so 
\eq{
\ddot{\delta}_m \sim H \dot{\delta}_m \sim H^2\delta_m \sim \frac{8\pi G \rho_r }{3}\delta_m \gg 4\pi G \bar{\rho}_m \delta_m
}
and the equation reduces to 
\eq{
\ddot{\delta}_m + \frac{1}{t} \dot{\delta}_m = 0 
}
Hence the solutions are $\delta_m \propto 1, \log t$ 
giving for the growing mode 
\eq{
\delta_m \log a 
}
\subsubsection*{Dark energy domination}
In dark energy domination, as $\Lambda$ is constant, and $H\gg 4\pi G\bar{\rho}_m$ the equation becomes 
\eq{
\ddot{\delta}_m + 2H \dot{\delta}_m = 0 
}
and so $\delta_m \propto 1, e^{-2Ht}$ so in dark energy domination perturbations stop growing. 
\end{example}

%%%%%%%%%%%%%%%%%%%%%%%%%%%%%%%%%%%%%%%%%%%%%%%%%%%%%%%%
\subsection{Relativistic Perturbation}

\begin{definition}[Horizon scale]
The \bam{conformal horizon scale} is $\mc{H}^{-1}$, where
\eq{
\mc{H} = \frac{a^\prime}{a}
}
is the conformal hubble parameter
A CMB mode it \bam{superhorizon} if $k\ll aH$
\end{definition}


\begin{remark}
In cosmology, Bessel functions like to turn up (why?). Hence it can sometimes be useful to try turn an euqaiotn into the form of Bessel's equation 
\[
\frac{d^2y}{dx^2} + \frac{1}{x} \frac{dy}{dx} + \left( 1-\frac{\nu^2}{x^2} \right ) = 0 
\]
\end{remark}


%%%%%%%%%%%%%%%%%%%%%%%%%%%%%%%%%%%%%%%%%%%%%%%%%%%%%%%%
%%%%%%%%%%%%%%%%%%%%%%%%%%%%%%%%%%%%%%%%%%%%%%%%%%%%%%%%
\section{Quantum Inflationary Origin}

\begin{definition}[Mukhanov Sasaki Equation]
The \bam{MSE} is 
\[
f ^ { \prime \prime } - \nabla ^ { 2 } f - \frac { a ^ { \prime \prime } } { a } f = 0
\]
\end{definition}

\end{document}