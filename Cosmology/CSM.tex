\documentclass{article}
\usepackage{header}
%%%%%%%%%%%%%%%%%%%%%%%%%%%%%%%%%%%%%%%%%%%%%%%%%%%%%%%%
%Preamble

\title{Cosmology Notes}
\author{Linden Disney-Hogg}
\date{January 2019}

%%%%%%%%%%%%%%%%%%%%%%%%%%%%%%%%%%%%%%%%%%%%%%%%%%%%%%%%
%%%%%%%%%%%%%%%%%%%%%%%%%%%%%%%%%%%%%%%%%%%%%%%%%%%%%%%%
\begin{document}

\maketitle
\tableofcontents

\section{Introduction}
These are notes, hopefully condensed for more efficient revision of cosmology. Highlighted equations in math environment (in red) indicates a possible error that needs to be reviewed. Text that is highlighted (in yellow) indicates an assumption/ statement assumed from farther physics. 
%%%%%%%%%%%%%%%%%%%%%%%%%%%%%%%%%%%%%%%%%%%%%%%%%%%%%%%%
%%%%%%%%%%%%%%%%%%%%%%%%%%%%%%%%%%%%%%%%%%%%%%%%%%%%%%%%

\section{Conventions and prerequisites}

\begin{definition}[Natural Units]
\bam{Natural units} are used where
\[
c=\hbar=k_B = 1
\]
$[c]=LT^{-1}$, $[\hbar]=L^2 M T^{-1}$, $[k_B] = ML^2 T^{-2} K^{-1}$, so in natural units
\[
L=T=K^{-1}=M^{-1}
\]
Units are therefore given in \bam{mass dimension}, e.g. if $[f]=M^d$, write $[f]=d$.
\end{definition}

\begin{fact}
A table of common values and their mass dimension is given below. 
\begin{center}$
\begin{array}{ccc}
    \text{Quantity} & \text{Symbol} & \text{Mass Dimension} \\
    \hline
    \hline
    \text{Length/Distance} & L/D & -1 \\
    \text{Volume} & V & -3 \\
    \text{Measure} & d^n x & -n \\
    \text{Derivative} & \del_\mu & 1 \\
    \text{Time} & t & -1 \\
    \text{Temperature} & T & 1 \\
    \text{Energy} & E & 1 \\
    \text{Energy Density} & \rho & 4 \\
    \text{Momentum} & P/q & 1 \\
    \text{Pressure} & p & 4 \\
    \text{Entropy} & S & 0 \\
    \text{Entropy Density} & s & 3 \\
    \text{Scale Factor} & a & 0 \\
    \text{Hubble Rate} & H & 1 \\
    \text{Curvature parameter} & K & 2 \\
    \text{Newton's Constant} & G & -2 \\
    \text{Metric} & g_{ab},g^{ab} & 0 \\
    \text{Action} & S & 0 \\
    \text{Lagrangian Density} & \mc{L} & 4 \\
    \text{Christoffel Symbol} & \Gamma^a_{bc} & 1 \\
    \text{Riemann Tensor} & R\indices{^a_b_c_d} & 2 \\
    \text{Ricci Tensor/Scalar} & R_{ab} / R & 2 \\
    \text{Cosmological Constant} & \Lambda & 2 \\
    \text{Inflaton Field} & \phi & 0 
\end{array}
$\end{center}
\end{fact}

\begin{definition}[Plank Mass]
The \bam{Plank mass} is 
\eq{
M_{pl} = (8 \pi G)^{-\frac{1}{2}}
}
\end{definition}

\begin{definition}[Mostly plus signature]
The signature that will be used throughout these notes will be the \bam{mostly plus signature}. As these notes will only consider 1+3 dimensional spacetime, as these are all the dimensions that have been observed, this takes the form $(-,+,+,+)$. 
\end{definition}
%%%%%%%%%%%%%%%%%%%%%%%%%%%%%%%%%%%%%%%%%%%%%%%%%%%%%%%%
%%%%%%%%%%%%%%%%%%%%%%%%%%%%%%%%%%%%%%%%%%%%%%%%%%%%%%%%
\section{General Relativity}
Some of the tools from General Relativity will be useful in discussions of cosmology. For a better idea of these results, see the GR notes, and these will be referenced where appropriate. 

\begin{definition}[Energy Momentum Tensor]
Given an action $S$ with respect to a metric $g_{ab}$ the \bam{energy momentum tensor} is 
\eq{
T^{ab} = \frac{2}{\sqrt{-g}} \frac{\delta S_{matter}}{\delta g_{ab}}
}
where $g=\det g_{ab}$ and 
\eq{
S_{matter} = \int \mc{L}_{matter} g^{\frac{1}{2}} d^4 x
}
is the matter action. 
\end{definition}

\begin{definition}[Perfect Fluid]
A \bam{perfect fluid} is a medium in which at each point there exist a Local Inertial Frame (LIF). 
\end{definition}

\begin{fact}
In a comoving LIF the energy momentum tensor must be diagonal and isotropic hence 
\eq{
T^\mu_\nu = \diag(-\rho,P,P,P)
}
Hence for a general frame moving with velocity $u^\mu$ s.t $u^\mu u_\mu = - 1$ 
\eq{
T^{\mu\nu} =(\rho+p) u^\mu u^\nu +g^{\mu\nu}p
}
\end{fact}


\begin{definition}[Lie Derivative]
If under an infinitesimal transform $x^\mu \to x^\mu + \eps V^\mu$ a tensor transform as $T \to T^\prime$ then the \bam{Lie Derivative} of $T$ in the $V$ direction is 
\eq{
(\mc{L}_V T)(x) = \lim_{\eps\to 0} \frac{T\indices{^\dots_\dots}(x) - {T^\prime}\indices{^\dots_\dots}(x)}{\eps}
}
Some examples are 
\eq{
\mc{L}_V \phi &= V^\mu \del_\mu \phi \\
\mc{L}_V W^\mu &= V^\nu \nabla_\nu W^\mu - W^\nu \nabla_\nu V^\mu \\
\mc{L}_V W_\mu &= V^\nu \nabla_\nu W^\mu + W_\nu \nabla_\mu V^\nu \\ 
\mc{L}_V T_{\mu\nu} &= V^\rho \nabla_\rho T_{\mu\nu} + T_{\rho\nu} \nabla_\mu V^\rho + T_{\mu\rho} \nabla_\nu V^\rho
}
The most important example is for $g_{\mu\nu}$ the metric tensor where
\eq{
\mc{L}_V g_{\mu\nu} = \nabla_\mu V_\nu + \nabla_\nu V_\mu
}
\end{definition}

\begin{definition}[Isometry]
A transform $x \to x^\prime$ is an \bam{isometry} if under the transform 
\eq{
g_{\mu\nu} \to g^\prime_{\mu\nu}, \; g^\prime_{\mu\nu}(x) = g_{\mu\nu}(x)
}
\end{definition}

\begin{definition}[Killing Vector]
A vector $\xi^\mu$ is a \bam{Killing vector} if 
\eq{
\mc{L}_\xi g_{\mu\nu} = \nabla_\mu \xi_\nu + \nabla_\nu \xi_\mu =0
}
Motion along integral curves of $\xi$ are then isometries. 
\end{definition}

\begin{prop}
For a Killing vector $\xi$
\eq{
\nabla_\rho \nabla_\sigma \xi_\mu = R\indices{_\lambda_\sigma_\mu_\rho} \xi^\lambda
}
\end{prop}
\begin{corollary}
A Killing vector field  is determined uniquely by specifying $\xi_\mu ,\nabla_\nu \xi^\mu$ at a point. Hence a $D$ dimensional spacetime has at most 
\[
D + \frac{1}{2}D(D-1) = \frac{1}{2}D(D+1)
\]
isometries. 
\end{corollary}

\begin{definition}[Maximally Symmetric Spacetime]
A $D$ dimensional spacetime is \bam{maximally symmetric} if it admits $\frac{1}{2}D(D+1)$ independent Killing vectors. 
\end{definition}

\begin{prop}
In maximally symmetric spacetime the Riemann tensor is 
\eq{
R_{\mu\nu\rho\sigma} = K g_{\mu(\sigma}g_{\nu\rho)}
R = -D(D-1) K 
}
Hence the Ricci scalar uniquely characterises a maximally symmetric spacetime. 
\end{prop}

\begin{fact}
If $N\leq M$ is a maximally symmetric subspace then the line element on $M$ can be expressed as the \bam{\hl{warp product}}
\eq{
ds^2 = g_{ab}(x) dx^a dx^b + f(x) \tilde{g}_{ij}(y) dy^i dy^j
}
where $y$ are coordinate on $N$ and $x$ coordinates on the rest of $M\setminus N$
\end{fact}

\begin{definition}[Einstein-Hilbert Action]\label{def:CSM:EinsteinHilbertAction}
The \bam{Einstein - Hilbert action} is 
\eq{
S_{EH} = \frac{1}{2}M_{pl}^2 \int (R-2\Lambda) g^{\frac{1}{2}} d^4 x
}
\end{definition}

\begin{definition}[Einstein Equations]
The \bam{Einstein Equations} are a set of 10 non-linear PDEs for the metric in the presence of some energy momentum tensor, and they are 
\eq{
M_{pl}^2 ( R_{ab} - \frac{1}{2}R g_{ab} + \Lambda g_{ab}) = T_{ab}
}
$\Lambda$ is a cosmological constant, and is ignored typically in general relativity. 
\end{definition}

\begin{prop}[Trace Reversed Einstein Equations]
In $D>2$ dimensions this can be written as 
\eq{
M_{pl}^2( R - \frac{D}{2}R + D\Lambda) = T
}
with $T=T\indices{^a_a}$. Hence 
\eq{
R &= \frac{2}{D-2}\left[ D\Lambda - M_{pl}^{-2} T \right] \\
\Rightarrow R_{ab} &=  \frac{1}{D-2}\left[ D\Lambda - M_{pl}^{-2} T \right] g_{ab} -\Lambda g_{ab} + M_{pl}^{-2}T_{ab} \\
&= \frac{2\Lambda}{D-2}g_{ab} + M_{pl}^{-2}\left[ T_{ab} - \frac{1}{D-2} T g_{ab} \right]
}
in $D=4$ this becomes 
\eq{
R_{ab} = \Lambda g_{ab} + M_{pl}^{-2} \left[ T_{ab} -\frac{1}{2}Tg_{ab} \right] 
}
\end{prop}

\begin{theorem}
The Einstein equations are obtained from finding when the action 
\eq{
S = S_{EH} + S_{Matter}
}
is stationary, i.e. $\delta S = 0$ with respect to variations in the metric. 
\end{theorem}

\begin{prop}[Contracted Bianchi Identity]
\eq{
\nabla^a(R_{ab} -  \frac{1}{2}R g_{ab} ) = \nabla^a R_{ab} - \frac{1}{2} \nabla_b R = 0
}
\end{prop}
\begin{corollary}
The energy momentum tensor is conserved, i.e. 
\eq{
\nabla_a T^{ab} = 0 
}
\end{corollary}
%%%%%%%%%%%%%%%%%%%%%%%%%%%%%%%%%%%%%%%%%%%%%%%%%%%%%%%%
\subsection{de Sitter Spacetime}

\begin{definition}[de Sitter Spacetime]
Consider $\mbb{R}^{1,4}$ with metric 
\eq{
ds^2 = -dV^2 +dW^2 + dX^2 + dY^2 + dZ^2 
}
and the metric induced on the surface 
\eq{
-V^2 + W^2 + X^2 + Y^2+ Z^2 = \frac{3}{\Lambda}
}
Any spacetime that has this induced metric is \bam{de Sitter spacetime}.
\end{definition}

\begin{prop}
The metric on de Sitter spacetime can be written as 
\eq{
ds^2 = - d\tau^2 + \frac{3}{\Lambda}\left[d\chi^2 + \sin^2 \chi \left( d\theta^2 + \sin^2\theta d\phi^2 \right) \right]
}
with 
\eq{
V &= \sqrt{\frac{3}{\Lambda}}\sinh \left( \tau\sqrt{\frac{3}{\Lambda}} \right) \\
W &= \sqrt{\frac{3}{\Lambda}}\cosh \left( \tau\sqrt{\frac{3}{\Lambda}} \right)\cos\chi \\
X &= \sqrt{\frac{3}{\Lambda}}\cosh \left( \tau\sqrt{\frac{3}{\Lambda}} \right)\sin\chi \cos\theta \\
Y &= \sqrt{\frac{3}{\Lambda}}\cosh \left( \tau\sqrt{\frac{3}{\Lambda}} \right)\sin\chi \sin\theta \cos\phi \\
Z  &= -\sqrt{\frac{3}{\Lambda}}\cosh \left( \tau\sqrt{\frac{3}{\Lambda}} \right)\sin\chi \sin\theta \sin\phi
}
\end{prop}
\begin{proof}
Note that for constant $V$ 
\eq{
W^2 + X^2 + Y^2+ Z^2 = \frac{3}{\Lambda} + V^2
}
is a 3-sphere, hence has the metric 
\eq{
d\sigma^2 = d\chi^2 + \sin^2 \chi \left( d\theta^2 + \sin^2\theta d\phi^2 \right)
}
with coordinates 
\eq{
W &= k\cos\chi \\
X &= k\sin\chi \cos\theta \\
Y &= k\sin\chi \sin\theta \cos\phi \\
Z  &= -k\sin\chi \sin\theta \sin\phi
}
for $k=\sqrt{\frac{3}{\Lambda}+V^2}$. Hence letting $V=\sqrt{3}{\Lambda}\sinh v$. Choosing v to normalise $dV$ in the metric gives $v=\tau\sqrt{\frac{3}{\Lambda}}$. 
\end{proof}

\begin{prop}
Alternative coordinates can be found on de Sitter space such that the metric becomes 
\eq{
ds^2 &= -dt^2 + e^{2H_0 t} d\bm{x}^2 \\
 &= \frac{-d\tau^2 + d\bm{x}^2}{\tau^2 H_0^2}
}
where $H_0 = \sqrt{\frac{\Lambda}{3}}$ in a universe with $w=-1$. 
\end{prop}
%%%%%%%%%%%%%%%%%%%%%%%%%%%%%%%%%%%%%%%%%%%%%%%%%%%%%%%%
%%%%%%%%%%%%%%%%%%%%%%%%%%%%%%%%%%%%%%%%%%%%%%%%%%%%%%%%
\part{Homogenous Universe}
%%%%%%%%%%%%%%%%%%%%%%%%%%%%%%%%%%%%%%%%%%%%%%%%%%%%%%%%
%%%%%%%%%%%%%%%%%%%%%%%%%%%%%%%%%%%%%%%%%%%%%%%%%%%%%%%%
\section{FLRW}
%%%%%%%%%%%%%%%%%%%%%%%%%%%%%%%%%%%%%%%%%%%%%%%%%%%%%%%%
\subsection{Metric}
On large scales, at any given time the universe looks \emph{homogeneous} and \emph{isotropic}, and hence it has a maximally symmetric constant time hypersurface. As a result the line element is 
\eq{
ds^2 = dt^2 + a(t)^2 \tilde{g}_{ij}(x) dx^i dx^j
}
Note $x$ are \emph{comoving} coordinates such that, for a spacelike separation $\Delta x$
\eq{
|\Delta x|_{phys} = a |\Delta x|
}

\begin{definition}[Hubble Parameter]
Define the \bam{Hubble parameter} $H$
\eq{
H= \frac{\dot{a}}{a}
}
\end{definition}

\begin{definition}[FLRW metric]
The metric induced on the homogeneous isotropic spacetime is the \bam{Friedmann-Lemaitre-Robertson-Walker Metric} 
\eq{
ds^2 &= - dt^2 + a(t)^2 \left[ \frac{dr^2}{1-Kr^2} + r^2 d\Omega^2 \right] \\
 &= - dt^2 + a(t)^2 \left[ d\chi^2 + f(\chi) d\Omega^2 \right]
}
where $K = 0, \pm1$, and 
\eq{
f(\chi) = \left\{ \begin{array}{lc} \sinh^2 \chi  & K = -1 \\
    \chi^2 & K=0 \\
    \sin^2 \chi & K=1
    \end{array} \right.
}
\end{definition}

\begin{definition}[Conformal time]
Introduce \bam{conformal time} $\tau$ which satisfies 
\[
d\tau = \frac{dt}{a(t)} \Leftrightarrow \tau = \int^t \frac{dt^\prime}{a(t^\prime)}
\]
\end{definition}

\begin{definition}[Conformal Hubble Parameter]
In analogy with $H = \frac{\dot{a}}{a}$ the Hubble parameter, define the \bam{conformal Hubble parameter}
\eq{
\mc{H} = \frac{a^\prime}{a}
}
where $a^\prime = \frac{da}{d\tau} = a \dot{a}=a^2 H$. Hence 
\eq{
\mc{H} = aH
}
\end{definition}

In conformal time the FLRW metric becomes 
\eq{
ds^2 = a^2 [ -d\tau^2 + d\chi^2 + f(\chi) d\Omega^2 ]
}

%%%%%%%%%%%%%%%%%%%%%%%%%%%%%%%%%%%%%%%%%%%%%%%%%%%%%%%%
\subsection{Evolution equations}

\begin{definition}[Continuity Equation]
The conservation of the energy momentum tensor implies 
\eq{
\dot{\rho} + 3H(\rho + p) = 0
}
\end{definition}

\begin{definition}[Equation of State]
The \bam{equation of state} is a relation $p=p(\rho)$. In this course the equation of state will be taken to be 
\eq{
p = w\rho
}
where the constant $w$ depends on the fluid.
\end{definition}

\begin{prop}
If the equation of state is $p = w\rho$ then 
\eq{
\dot{\rho} + 3 \frac{\dot{a}}{a}(1+w) \rho = 0 \Rightarrow \rho \propto a^{-3(1+w)}
}
\end{prop}

The examples encountered in this course are 
\begin{center}$
\begin{array}{ccc}
    \text{Fluid} & w & \text{Relation} \\
    \hline
    \hline
    \text{Non-relativistic matter (dust)} & 0 & \rho\propto a^{-3}  \\
    \text{Radiation} & \frac{1}{3} & \rho\propto a^{-4} \\
    \text{Dark energy (cosmological constant)} & -1 & \rho \text{ constant} \\
\end{array}
$\end{center}

\begin{definition}[Friedmann equation]
In the FLRW metric the Einstein equations reduce to 
\eq{
3M_{pl}^2 \pround{ H^2 + \frac{K}{a^2} } = \sum_i \rho_i
}
this is the \bam{Friedmann equation}. 
\end{definition}

\begin{definition}[Raychaudhuri equation]
Differentiating and rearranging the Friedmann equation given the \bam{Raychaudhuri equation}
\eq{
\frac{\ddot{a}}{a} = \frac{-1}{6M_{pl}^2} (\rho + 3P)
}
\end{definition}

\begin{definition}[Critical Density]
Define the \bam{critical density} to be 
\eq{
\rho_c = 3 M_{pl}^2 H^2
}
writing 
\eq{
\Omega_i &= \frac{\rho_i}{\rho_c} \\
\Omega_K &= -\frac{3M_{pl}^2 K}{\rho_c a^2} = -\frac{K}{H^2 a^2}
}
the Friedmann equation becomes 
\eq{
1-\Omega_K = \sum_i \Omega_i
}
It is common notation to let $\Omega_{i,0} = \Omega_i |_{t=t_0}$, and use $\Omega_{i,0} h^2$ where $h$ is defined by 
\eq{
H_0 = 100 h \frac{\text{km}}{\text{sec Mpc}}
} 
in order to absorb error in the measurement of $H_0$. 
\end{definition}

%%%%%%%%%%%%%%%%%%%%%%%%%%%%%%%%%%%%%%%%%%%%%%%%%%%%%%%%
\subsection{Distance}\footnote{See this link \href{http://article.sciencepublishinggroup.com/pdf/10.11648.j.ijass.20150304.13.pdf}{http://article.sciencepublishinggroup.com/pdf/10.11648.j.ijass.20150304.13.pdf} for more information. }

\begin{definition}[Redshift]
Suppose that a wave is emitted with wavelength $\lambda_e$ and observed with wavelength $ \lambda_o$. Then the \bam{redshift} $z$ is defined by 
\eq{
1+z = \frac{\lambda_o}{\lambda_e}
}
\end{definition}

\begin{definition}[Cosmological redshift]
For a photon emitted at time $t_e < t_0$ the redshift due to the expansion of the universe is the \bam{cosmological redshift}
\eq{
1+z = \frac{a_0}{a_e}
}
If we let $a_0=1$, then we get $1+z = \frac{1}{a(z)}$
\end{definition}

\begin{definition}[Comoving Distance]
The \bam{comoving distance} $\chi(t_i,t_f)$ is the distance travelled by a photon between times $t_i,t_f$. Note that for a radially moving null geodesic $ds^2 = a^2[-d\tau^2 + d\chi^2] = 0$ so $d\chi=d\tau$, so 
\eq{
\chi(t_i,t_f) &= \int d\tau \\
&= \int_{t_i}^{t_f} \frac{dt}{a} \\
&= \int_{a_i}^{a_f} \frac{da}{a^2 H} \\
&= \int_{z_f}^{z_i} \frac{dz}{H(z)}
}
Writing this then in terms of the redshift, then the distance from the earth to an object with redshift $z$ is   
\eq{
\chi(z) = \int_0^z \frac{dz}{H(z)}
}
\end{definition}

\begin{definition}[Luminosity]
The \bam{intrinsic luminosity} $L$ of an object is the amount of energy radiated per unit time. The \bam{observed luminosity} $l$ is the amount of energy radiated per unit time per unit surface arriving at an observer some distance from the object.
\end{definition}

\begin{definition}[Magnitude]
The \bam{magnitude} of the luminosity of an object is 
\eq{
m = -2.5 \log_{10}l + \text{const}
}
\end{definition}

\begin{definition}[Luminosity Distance]
The observed and intrinsic luminosity are related by 
\eq{
l = \frac{L}{4\pi d_L^2}
}
which defines the $\bam{luminosity distance}$ $d_L$
 \end{definition}
 
 \begin{prop}
 The luminosity distance and comoving distance are related by 
 \eq{
 d_L(z) = (1+z) a_o \chi(z)
 }
 \end{prop}
\begin{proof}
\eq{
l = \frac{L}{4\pi (\chi a_o)^2}\left(\frac{a_e}{a_o}\right)^2
}
with the factors coming from: the physical distance corresponding to the comoving distance at the observation time, the reduction in the energy of a photon due to redshift, and the reduced rate of photon arrival. 
\end{proof}

\begin{definition}[Angular Diameter Distance]
For an object of size $s$ which subtends an angle $\theta$ from the point of view of the observer, the \bam{angular diameter distance} $d_A$ to the object is 
\eq{
d_A = \frac{s}{\theta}
}
As the angle the object takes up in the sky will be stretched by a factor of (1+z), the angular diameter distance is related to the comoving distance by 
\eq{
d_A(z) = \frac{\chi(z)}{1+z}
}
\end{definition}

\begin{definition}[Particle Horizon]
The \bam{particle horizon} $d_{p.h.}$ is is the greatest distance a photon a could have travelled since the beginning of time $t_i$. Any object a distance $d>d_{p.h.}$ from an observer away can never have communicated with said observer. 
\eq{
d_{p.h.}(t) = a(t) \chi(t,t_i) = a(t) \int_{t_i}^t \frac{dt^\prime}{a(t^\prime)}
}
\end{definition}

\begin{definition}[Hubble Radius]
The \bam{comoving Hubble radius} is 
\eq{
r_H = \frac{1}{aH}
}
and so the physical Hubble radius is 
\eq{
r_H = \frac{1}{H}
}
\end{definition}

\begin{definition}[e-folds]
If, during expansion $a \to e^N a$, is it said that \bam{N e-folds of expansion have occurred}. $N$ can we written as a variable 
\eq{
dN= Hdt = d(\log a) \Leftrightarrow N = \log a + \text{ constant}
}
\end{definition}

%%%%%%%%%%%%%%%%%%%%%%%%%%%%%%%%%%%%%%%%%%%%%%%%%%%%%%%%
\subsection{Possible Universes}

\begin{example}
Consider a flat $(K=0)$ universe, with only one contribution to the energy density. Then 
\eq{
3M_{pl}^2 \left(\frac{\dot{a}}{a}\right)^2 &= \rho_0 \left(\frac{a}{a_0}\right)^{-3(1+w)} \\
\Rightarrow \frac{\dot{a}}{a} &\propto a^{-\frac{3(1+w)}{2}} \\
\frac{da}{a} a^{\frac{3(1+w)}{2}} &= dt \\
\Rightarrow a(t) &= \left\{ \begin{array}{cc} \left[ \frac{3}{2}(1+w) H_0 t \right]^{\frac{2}{3(1+w)}} & w\neq -1 \\ Ae^{H_0 t} & w = -1 \end{array} \right.
}
Hence \begin{itemize}
    \item For non-relativistic matter, $a \propto t^{\frac{2}{3}}$
    \item For radiation, $a \propto t^\frac{1}{2}$
    \item For dark energy, $a \propto e^{H_0 t}$
\end{itemize}

In this case 
\eq{
H &= \left\{ \begin{array}{cc}  \frac{2}{3(1+w)}\frac{1}{t}  & w\neq -1 \\ H_0 & w = -1 \end{array} \right. \\
}
For $w\neq -1$ this can be written as 
\eq{
H(a) &= H_0 \left(\frac{a_0}{a}\right)^\frac{3(1+w)}{2} \\
H(z) &= H_0 (1+z)^\frac{3(1+w)}{2}
}
\end{example}

\begin{prop}
If $K=0$, the age of the universe can be calculates as 
\eq{
t_{age} &= \int dt \\
 &= \int \frac{da}{\dot{a}} \\
 &= \int \frac{da}{a H} \\
 &= \frac{1}{H_0} \int \frac{da}{a} \left[  \frac{\sum_i \rho_i}{3 M_{pl}^2 H_0^2} \right]^{-\frac{1}{2}} \\
 &= \frac{1}{H_0} \int \frac{da}{a} \left[ \Omega_\Lambda + \Omega_{m,0}a^{-3} + \Omega_{r,0} a^{-4} \right]^{-\frac{1}{2}}
}
\end{prop}

\begin{example}[Curvature problem]\label{example:CSM:curvature problem}
\eq{
\dot{\Omega}_K = -\frac{2\ddot{a}}{\dot{a}} \Omega_K
}
Hence the behaviour of $|\Omega_K|$ depends on 
\eq{
\frac{\ddot{a}}{\dot{a}} = - \frac{1}{6M_{pl}^2}(1+3w) \frac{\rho}{H}
}
So, in an expanding universe 
\eq{
 1 + 3w > 0 \Leftrightarrow |\Omega_K| \text{ grows} \\
 1 + 3w < 0 \Leftrightarrow |\Omega_K| \text{ shrinks}
}
and hence during radiation and matter domination $|\Omega_K|$ grows. Observation puts 
\eq{
\Omega_{K,0} = 0.000 \pm, 0.005
}
This suggests a very large fine tuning of the initial value of $\Omega_K$
\end{example}

\begin{example}[Horizon Problem]\label{example:CSM:horizon problem}
In a single component universe with $w\in(0,\frac{1}{3})$ we find 
\eq{
\chi(z) &= \frac{1}{H_0} \int_0^z (1+z)^{-\frac{3(1+w)}{2}} \, dz \\
&= \frac{1}{H_0}\left[ -\frac{2}{1+3w} (1+z)^{-\frac{1+3w}{2}}\right]_0^z \\
&= \frac{1}{H_0}\frac{2}{1+3w}\left[1-(1+z)^{-\frac{1+3w}{2}}\right]
}
For $z\gg1$ we thus have 
\eq{
\chi(z) \approx \frac{\mc{O}(1)}{H_0}
}
Alternatively we can calculate the particle horizon from the beginning of the universe to redshift $z$ as 
\eq{
x_{p.h.}(z) &= \frac{1}{H_0} \int_z^\infty (1+z)^{-\frac{3(1+w)}{2}} \, dz\\
&=  \frac{1}{H_0}\left[ -\frac{2}{1+3w} (1+z)^{-\frac{1+3w}{2}}\right]_z^\infty \\
&= \frac{1}{H_0}\frac{2}{1+3w} (1+z)^{-\frac{1+3w}{2}} \\
&= \frac{2}{1+3w} \frac{1+z}{H(z)}
}
Hence for large $z$
\eq{
\frac{\chi(z)}{x_{p.h.}(z)} \approx (1+z)^{\frac{1+3w}{2}} \gg 1
}
Thus antipodal regions with large $z$ must be causally disconnected, as their comoving distance is much greater than the particle horizon. As a result, there is no explanation for the statistical isotropy observed in the universe. 
\end{example}
%%%%%%%%%%%%%%%%%%%%%%%%%%%%%%%%%%%%%%%%%%%%%%%%%%%%%%%%
%%%%%%%%%%%%%%%%%%%%%%%%%%%%%%%%%%%%%%%%%%%%%%%%%%%%%%%%

\section{Inflation}

\begin{definition}[Inflation]
\bam{Inflation} is a period of time during which $\dot{a},\ddot{a} > 0$. 
\end{definition}

Such a period would provide a solution to the curvature problem (example \ref{example:CSM:curvature problem}) and the horizon problem (example \ref{example:CSM:horizon problem}). 

\begin{definition}[Reheating]
\bam{Reheating} is the period when evolution changes from inflation to the decelerating hot Big Bang. 
\end{definition}

\begin{prop}
Approximately $50$ e-folds passed during inflation. 
\end{prop}
\begin{proof}
In order to solve the Horizon problem (example \ref{example:CSM:horizon problem}) it is necessary that the particle horizon is greater than the Hubble radius. For approximately de Sitter expansion, $H$ is approximately constant, so $d_{p.h.}\approx \frac{1}{H}$, thus comparing the \hl{comoving distances}\footnote{why the comoving distances?} 
\eq{
\frac{1}{a_i H_i} > \frac{1}{a_0 H_0}
}
where $i$ indicates the beginning of inflation. Thus letting $reh$ represent when reheating occurs, 
\eq{
\frac{a_{reh} H_{reh}}{a_i H_i} > \frac{a_{reh} H_{reh}}{a_0 H_0}
}
Hence using definition \ref{def:CSM:1} and the Friedmann equation in a flat universe 
\eq{
3M_{pl}^2 H^2 = \rho
}
gives 
\eq{
\frac{a_{reh} H_{reh}}{a_0 H_0} &= \pround{\frac{\frac{cst}{T_{CMB,0}}\sqrt{(g_\ast)_0 \frac{\pi^2}{90} } \frac{T_{CMB,0}^2}{M_{pl}}}{\frac{cst}{T_{reh}}\sqrt{(g_\ast)_{reh} \frac{\pi^2}{90} } \frac{T_{reh,0}^2}{M_{pl}}}}^{-1} \\
&\approx \frac{T_{reh}}{T_{CMB,0}}
}
Now $T_{CMB,0} \approx 2.7K \approx 0.2 meV = 2 \times 10^{-13} Gev $
\eq{
\Rightarrow \frac{a_{reh} H_{reh}}{a_0 H_0} \approx \highlight{5 \times 10^{23}}\footnote{value given in notes is $4\times 10^{21}$. Why?} \left( \frac{T_{reh}}{10^{10} GeV} \right)
}
\footnote{value given in notes is $4\times 10^{21}$. Why?}
Hence the number of e-folds is 
\eq{
\Delta N_{infl} &= \log{\frac{a_{reh}}{a_i}} \gtrapprox 50 + \log \frac{T_{reh}}{10^{10}}
}
Taking the uncertainty in $T_{reh}$ to be s.t. that $T_{reh} \in (1,10^{15}) GeV$ gives $\Delta N_{infl} \in (25,60)$. It will be often be taken that $\Delta N_{infl} \approx 50$ hence. 
\end{proof}

%%%%%%%%%%%%%%%%%%%%%%%%%%%%%%%%%%%%%%%%%%%%%%%%%%%%%%%%
\subsection{Slow-Roll Inflation}

\begin{definition}[Hubble Slow Roll Parameter]
Define the \bam{first Hubble slow roll parameter} $\eps$ such that 
\eq{
\frac{\ddot{a}}{a} = \dot{H}+H^2 = H^2(1-\eps) \Leftrightarrow \eps = -\frac{\dot{H}}{H^2}
}
In a single fluid universe 
\eq{
\eps = \frac{3(1+w)}{2}
}
Note that this can be written as 
\eq{
\eps = - \del_N \log H 
}
Then define the \bam{second Hubble slow roll parameter} $\eta$ by 
\eq{
\eta = \del_N \log \eps = \frac{\dot{\eps}}{H\eps}
}
Then the \bam{n\textsuperscript{th} Hubble slow roll parameter} $\xi_n$ is given inductively by 
\eq{
\xi_n = \del_N \log \xi_{n-1}
}
with $\xi_1 = \eps$, $\xi_2 = \eta$. 
\end{definition}

Inflation requires $\eps < 1 \Rightarrow w < -\frac{1}{3}$. It will be seen later that scale invariance follows from a spacetime background that is approximately de Sitter, i.e. $H$ approximately constant. Hence consider the case $0< \eps \ll 1$. 

\begin{definition}[Slow Roll Inflation]
\bam{Slow roll inflation} is inflation during which $\dot{H}$ remains small throughout.
\end{definition}

\begin{prop}
For slow roll inflation, it is sufficient that 
\eq{
\forall n, \quad  0 < \xi_n \ll 1
}
\end{prop}
\begin{proof}
For slow roll inflation it is required that $0 < \eps \ll 1$ as already discussed. Taylor expanding $\eps(N)$ from the beginning of inflation $N=N_\ast$ gives \eq{
\eps(N) - \eps(N_\ast) &= \pd[\eps]{N}|_{N_\ast}(N-N_\ast) + \pds[\eps]{N}|_{N_\ast} + \mc{O}(\del_N^3 \eps) \\
&= \eps\left[ \eta(N-N_\ast) + \eta \xi_2 \frac{(N-N_\ast)^2}{2} + \dots \right]
}
so to ensure that $\eps$ remains small it is required $\eta \Delta N_{infl}, \eta \xi_2 \Delta N_{infl}, \dots < 1$, so it is enough to have $0<\xi_n < \frac{1}{\Delta N_{infl}} \ll 1$ for all $n$. 
\end{proof}

%%%%%%%%%%%%%%%%%%%%%%%%%%%%%%%%%%%%%%%%%%%%%%%%%%%%%%%%
\subsection{Single Fields Inflation}

Consider the action
\eq{
S = \frac{1}{2} M_{pl}^2 \int \left[ R + M_{pl}^{-2}\left( -\del_\mu \phi \del^\mu \phi - 2V(\phi) \right) \right] g^\frac{1}{2} d^4 x 
}
where $\phi$ is some scalar field. To ensure consistency with the symmetries of FLRW it must be that $\phi = \phi(t)$.  Note the similarity to the Einstein Hilbert action \ref{def:CSM:EinsteinHilbertAction} if 
\eq{
\Lambda = M_{pl}^{-2} \left( \frac{1}{2} \del_\mu \phi \del^\mu \phi + V(\phi) \right)
}
The energy momentum tensor of this action is 
\eq{
T_{\mu\nu} &= \del_\mu \phi \del_\nu \phi + g_{\mu\nu} \left[ -\frac{1}{2} \del_\sigma \phi \del^\sigma \phi - V(\phi) \right] \\
&= (\rho + p) u_\mu u_\nu + g_{\mu\nu}p
}
making the identifications 
\eq{
\rho &= -\frac{1}{2}\del_\mu \phi \del^\mu \phi + V(\phi) \\
p &= -\frac{1}{2}\del_\mu \phi \del^\mu \phi - V(\phi) \\
u_\mu &= \frac{\del_\mu \phi}{\sqrt{-\del_\mu \phi \del^\mu \phi}}
}
These reduce to 
\eq{
\rho &= \frac{1}{2} \dot{\phi}^2 + V(\phi) \\
p &=  \frac{1}{2} \dot{\phi}^2 - V(\phi) \\
u_\mu &= (1,\bm{0})
}
as $\phi=\phi(t)$. 

The Friedmann equations are then 
\eq{
\ddot{\phi} + 3H\dot{\phi} + V^\prime(\phi) = 0 \text{ (continuity equation)}\\
3H^2 M_{pl}^2 = \frac{1}{2} \dot{\phi}^2 + V(\phi)  \text{ (evolution equations)}
}

\begin{prop}\label{prop:CSM:1}
\eq{
-\dot{H} M_{pl}^2 = \frac{1}{2}\dot{\phi}^2
}
\end{prop}
\begin{proof}
Differentiate the evolution equation and sub in the continuity equation. 
\end{proof}

\begin{definition}
In scalar field inflation, the first three slow roll parameters are 
\begin{align}\label{eq:CSM:13}
\eps_V &= \frac{1}{2} M_{pl}^2 \left( \frac{V^\prime}{V} \right)^2 \\
\eta_V &= M_{pl}^2 \frac{V^{\prime\prime}}{V} \\
\xi_{3V} &= M_{pl}^4 \frac{V^\prime V^{\prime\prime}}{V^2}
\end{align}
\end{definition}

\begin{prop}
\eq{
V = (3-\eps) H^2 M_{pl}^2
}
and hence 
\eq{
\eps_V &= \frac{\eps(\eta-2\eps+6)^2}{4(\eps-3)^2} \\
\eta_V &= \frac{\eta(\eta+2\xi_3+6)-2\eps(5\eta+12)+8\eps^2}{4(\eps-3)}
}
\end{prop}
\begin{proof}
The evolution equation becomes 
\eq{
V = H^2 M_{pl}^2 (3-\eps)
}
through substitution for $\frac{1}{2}\dot{\phi}^2$. Then 
\eq{
V^\prime \dot{\phi} &= 2H \dot{H} M_{pl}^2 (3-\eps) - \dot{\eps} H^2 M_{pl}^2 \\
&= M_{pl}^2 H^3 \eps (2\eps-6 - \eta) \\
\Rightarrow {V^\prime}^2 \dot{\phi}^2 &= M_{pl}^4 H^6 \eps^2 (\eta - 2\eps + 6)^2  \\
\Rightarrow {V^\prime}^2 &= 2M_{pl}^2 H^4 \eps (\eta - 2\eps + 6)^2 \\
\frac{1}{2} M_{pl}^2 \pround{\frac{V^\prime}{V}}^2 &= \frac{\eps(\eta - 2\eps + 6)^2}{(\eps-3)^2 }
}
\end{proof}

\begin{corollary}
For $\eps,\eta \ll 1$
\eq{
\eps \approx \eps_V \\
\eta \approx 4\eps_V - 2\eta_V
}
\end{corollary}

\begin{definition}
Write $X = -\frac{1}{2} \del_\mu \phi \del^\mu \phi = \frac{1}{2}\dot{\phi}^2$. Then 
\eq{
\rho = X+V \\
p = X-V \\
\dot{X}+6HX +V^\prime \dot{\phi} = 0 \\
-\dot{H} M_{pl}^2 = X
}
\end{definition}

Hence we can write 
\eq{
\eps = -\frac{\dot{H}}{H^2} = \frac{\frac{X}{M_{pl}^2}}{\frac{X+V}{3M_{pl}^2}} = \frac{3X}{X+V} \ll 1 \Rightarrow X \ll V
}
and so  
\begin{align}\label{eq:CSM:14}
3 M_{pl}^2 H^2 \approx V 
\end{align}
Hence 
\eq{
\eta = \frac{\dot{\eps}}{\eps H} = 2\eps + \frac{\dot{X}}{XH}
}
so 
\eq{
\eps,\eta \ll 1 \Rightarrow \dot{X} \ll XH \Rightarrow 2\ddot{\phi} \ll \dot{\phi}H
}
and so we can neglect the acceleration term giving 
\eq{
3 H \dot{\phi} &\approx -V^\prime \\
\Rightarrow \dot{\phi} &\approx - \frac{V^\prime M_{pl}}{\sqrt{3V}} \\
\Rightarrow t &\approx \int d\phi \, \frac{\sqrt{3V}}{V^\prime M_{pl}} + \text{const}
}
This equation can be inverted to give $\phi(t)$ a \bam{slow roll solution}, valid if $\eps,\eta \ll 1$

\subsubsection*{End of Inflation}
We shall \hl{assume that inflation ends at} $t_e$ where $\eps(t_e)\approx 1$. In addition, we \hl{assume that at the end of inflation} $\phi$ is at a minimum of V with 
\eq{
V(\phi_{min}) \approx (10^{-3} eV)^4 \approx 0 
}
Then the total of e-folds during inflation is 
\eq{
\Delta N &= \int_{N_i}^{N_e} dN \\
&= \int_{t_i}^{t_e} H dt \\
&= \int_{\phi_i}^{\phi_e} \frac{H}{\dot{\phi}} d\phi \\
&\approx \int_{\phi_i}^{\phi_e} d\phi \, \sqrt{\frac{V}{3M_{pl}^2}} \left( - \frac{V^\prime M_{pl}}{\sqrt{3V}} \right)^{-1} \\
&= \int_{\phi_e}^{\phi_i} d\phi \, \frac{V}{M_{pl}^2 V^\prime}
}

\begin{definition}
Define a field to be a \bam{small field} if during inflation $\Delta \phi < M_{pl}$ and a \bam{large field} if $\Delta \phi > M_{pl}$
\end{definition}

\begin{definition}
Let 
\eq{
\Lambda_\phi = \frac{V}{V^\prime}
}
\end{definition}
\hl{assuming} that $V$ remains relatively flat, so $V,V^\prime$ are approximately constant. Then 
\eq{
\Delta N &\approx \frac{V}{M_{pl}^2 V^\prime}\Delta \phi \\
\Rightarrow \frac{\Delta \phi}{M_{pl}} &\approx 50 \frac{M_{pl}}{\Lambda_\phi}
}
%%%%%%%%%%%%%%%%%%%%%%%%%%%%%%%%%%%%%%%%%%%%%%%%%%%%%%%%
%%%%%%%%%%%%%%%%%%%%%%%%%%%%%%%%%%%%%%%%%%%%%%%%%%%%%%%%

\section{Constituents of the Universe}
The five main components of the universe are 
\begin{itemize}
    \item Photons
    \item Baryons (Protons, Neutrons, and other 3 quark species)
    \item Neutrinos
    \item Dark Matter 
    \item Dark Energy
\end{itemize}
For a particle species to have sizeable density on its own today, that species must have a lifetime comparable to the lifetime of the universe. This includes photons, protons, and electrons. Species such as neutrons have retain their density by forming stable nuclei. 

%%%%%%%%%%%%%%%%%%%%%%%%%%%%%%%%%%%%%%%%%%%%%%%%%%%%%%%%
\subsection{Thermodynamics}
\begin{definition}
The 1st law of thermodynamics is 
\eq{
TdS = dE + pdV - \mu dN
}
where $E$ is the energy of the system, $T$ the temperature, $p$ pressure, $V$ volume, $\mu$ the chemical potential, and $N$ the number of the particle. 
\end{definition}

\subsubsection*{Equilibrium}
\begin{definition}[Kinetic Equilibrium]
A system of particles is in \bam{kinetic equilibrium} if the constituent particles exchange energy and momentum efficiently. They are thus in a maximum-entropy state with distributions given by 
\eq{
f_{BE/FD}(\bm{x},\bm{P},t) = \frac{1}{e^{\frac{E_{\bm{P}}-\mu}{T}}\mp1}
}
the phase space density for bosons (\bam{Bose-Einstein statistics}) and fermions (\bam{Fermi-Dirac statistics}) respectively. 
\end{definition}

\begin{definition}[Chemical Equilibrium]
An interaction is in \bam{chemical equilibrium} if the chemical potentials of the interacting species balance out, i.e in a reaction $1 + 2 \leftrightarrow 3 + 4$
\eq{
\mu_1 + \mu_2 = \mu_3 + \mu_4
}
\end{definition}

\begin{lemma}
From chemical equilibrium we can say 
\begin{itemize}
    \item $\mu_\gamma = 0$ (as photon number is not conserved)
    \item $\mu_X = -\mu_{\bar{X}}$ (as a particle and its antiparticle can annihilate to make photons)
\end{itemize}
\end{lemma}

\begin{definition}[Thermal Equilibrium]
An interaction is in \bam{thermal equilibrium} if it is in both kinetic and chemical potential. This forces all species to have the same temperature $T$. 
\end{definition}

\begin{definition}[Bose-Einstein/Fermi-Dirac statistics]
For a species of particles $a$ with comoving momentum $P$, then the energy momentum tensor and number density are 
\eq{
T^{\mu\nu}_a ( \bm{x},t) &= \frac{2}{\sqrt{-g}} g_a \int \frac{d^3\bm{P
}}{(2\pi)^3 2E_{\bm{P}}} P^\mu P^\nu f(\bm{x},\bm{P},t) \\
n_a ( \bm{x},t) &= \frac{2}{\sqrt{-g}} g_a \int \frac{d^3\bm{P
}}{(2\pi)^3 2E_{\bm{P}}}  f(\bm{x},\bm{P},t)
}
where 
\begin{itemize}
    \item $g_a$ is the degeneracy of the species, e.g. 2 for a photon
    \item $E_{\bm{P}} = \sqrt{m^2 + |\bm{P}|^2}$
\end{itemize}
\end{definition}

\begin{prop}
In flat FLRW space the components of the energy momentum tensor and number density take the form 
\eq{
\rho_a &= \frac{g_a}{2\pi^2} \int_0^\infty dq \, q^2 \frac{E(q)}{e^{\frac{E(q)-\mu}{T}}\mp1} \\
p_a &= \frac{g_a}{2\pi^2} \int_0^\infty dq  \, q^2 \frac{q^2}{3E(q)} \frac{1}{e^{\frac{E(q)-\mu}{T}}\mp1} \\
n_a &= \frac{g_a}{2\pi^2} \int_0^\infty dq \, q^2 \frac{1}{e^{\frac{E(q)-\mu}{T}}\mp1}
}
\end{prop}
\begin{proof}
In flat FLRW $\sqrt{-g} = a^3$ and $d^3\bm{P} = a^3 d^3\bm{q}$ for physical momentum $\bm{q}$. Then $\rho = T^0_0$ and $p=\frac{1}{3}T^i_i$. Then finally writing $d^3\bm{q}=q^2 dq d\Omega$ and completing the angular integral for a factor of $4\pi$ gets the answer. 
\end{proof}
%%%%%%%%%%%%%%%%%%%%%%%%%%%%%%%%%%%%%%%%%%%%%%%%%%%%%%%%
Suppose $T \ll E - \mu $, then 
\begin{align}\label{eq:CSM:4}
\frac{1}{e^\frac{E(q)-\mu}{T}\mp1} \approx  e^{-\frac{E(q)}{T}}e^{\frac{\mu}{T}} \ll 1 
\end{align}
So the integrals will be supported mainly where 
\eq{
& 0= \frac{d}{dq} q^2 e^{-\frac{\sqrt{m^2 + q^2}}{T}} = 2q e^{-\frac{\sqrt{m^2 + q^2}}{T}} - \frac{q^3}{T\sqrt{m^2+q^2}}e^{-\frac{\sqrt{m^2 + q^2}}{T}}\\
 \Rightarrow & 2-\frac{q^2}{T\sqrt{m^2+q^2}} = 0 \\
  \Rightarrow & q^4 -4T^2q^2-4T^2m^2 = 0 \\
  \Rightarrow & q^2 = 2T^2 \pround{1 + \sqrt{1 + \pround{\frac{m}{T}}^2}} \\
 \Rightarrow & \left\{ \begin{array}{cc}  q\approx 2T & m \ll T \\  q\approx \sqrt{2mT} & m\gg T \end{array} \right.
}

Hence 
\eq{
\sqrt{m^2+q^2} \approx \left\{ \begin{array}{cc} q & m \ll T \\ m+\frac{q^2}{2m} & m \gg T \end{array} \right.
}
%%%%%%%%%%%%%%%%%%%%%%%%%%%%%%%%%%%%%%%%%%%%%%%%%%%%%%%%
\subsubsection*{Relativistic Particles}
For relativistic particles, the expressions thus are 
\eq{
\rho_a &\approx \frac{g_a}{2\pi^2} \int_0^\infty dq \,  \frac{q^3}{e^{\frac{q}{T}}\mp1} \\
p_a &\approx \frac{1}{3}\frac{g_a}{2\pi^2} \int_0^\infty dq  \,   \frac{q^3}{e^{\frac{q}{T}}\mp1} \\
n_a &\approx \frac{g_a}{2\pi^2} \int_0^\infty dq \, \frac{q^2}{e^{\frac{q}{T}}\mp1}
}
making the substitution $y = \frac{q}{T}$ gives 
\eq{
\rho_a &\approx \frac{g_a T^4}{2\pi^2} \int_0^\infty dy \,  \frac{y^3}{e^{y}\mp1} \\
p_a &= \frac{1}{3} \rho_a \\
n_a &\approx \frac{g_a T^3}{2\pi^2} \int_0^\infty dy \, \frac{y^2}{e^y \mp1}
}

\begin{lemma}
\eq{
\int_0^\infty dy \, \frac{y^n}{e^y - 1} = \zeta(n+1) \Gamma(n+1)
}
and 
\eq{
\int dy \, \frac{y^n}{e^y + 1} = \left( 1- \frac{1}{2^n} \right) \int dy \, \frac{y^n}{e^y - 1}
}
\end{lemma}
\begin{proof}
For $\Re{z} > 0 $
\eq{
\Gamma(z) = \int_0^\infty t^{z-1} e^{-t} \, dt 
}
and for $\Re{s} > 1$
\eq{
\zeta(s) = \sum_{n=1}^\infty \frac{1}{n^s}
}
Hence for $Re{z} > 1$
\eq{
\zeta(z)\Gamma(z) &= \sum_{n=1}^\infty \int_0^\infty t^{z-1} n^{-z} e^{-t} \, dt  \\
\text{(let $t=n\tau$) } &= \sum_{n=1}^\infty \int_0^\infty \tau^{z-1} n^{-1} e^{-n\tau} \, nd\tau \\
&= \int_0^\infty \tau^{z-1} \left(\sum_{n=1}^\infty e^{-n\tau}\right) \, d\tau \\
&= \int_0^\infty d\tau \, \frac{\tau^{z-1}}{e^{\tau}-1} 
}
Now 
\eq{
\frac{1}{e^y+1} &= \frac{e^y-1}{(e^y+1)(e^y-1)} \\
&= \frac{(e^y+1)-2}{(e^y+1)(e^y-1)} \\
&= \frac{1}{e^y-1} - \frac{2}{e^{2y}-1} \\
\Rightarrow \int_0^\infty dy \, \frac{y^n}{e^y+1} &= \int_0^\infty dy \, \frac{y^n}{e^y-1} - 2\int_0^\infty dy \, \frac{y^n}{e^{2y}-1} \\
\text{(let $u=2y$) } &= \int_0^\infty dy \, \frac{y^n}{e^y-1} - 2\int_0^\infty \frac{1}{2}du \, \frac{2^{-n}u^n}{e^{u}-1} \\
&= \left(1 - \frac{1}{2^n} \right) \int_0^\infty dy \, \frac{y^n}{e^y-1}
}
\end{proof}

\begin{fact}
\eq{
\Gamma(n+1) &= n! \\
\zeta(3) &= 1.202\dots \quad \text{(no reasonable representation)} \\
\zeta(4) &= \frac{\pi^4}{90}
}
\end{fact}

Hence 
\eq{
\rho_a = 3p_a&= g_a \frac{\pi^2}{30}T^4 \left\{ \begin{array}{cc} 1 & \text{bosons} \\ \frac{7}{8} & \text{fermions} \end{array}\right. \\
n_a &= g_a \frac{\zeta(3)}{\pi^2} T^3  \left\{ \begin{array}{cc} 1 & \text{bosons} \\ \frac{7}{8} & \text{fermions} \end{array}\right.
}
\footnote{Note that fermions should have the lower density due to the Pauli exclusion principle.}
%%%%%%%%%%%%%%%%%%%%%%%%%%%%%%%%%%%%%%%%%%%%%%%%%%%%%%%%
\subsubsection*{Non-Relativistic Particles}
For non-relativistic particles the expressions thus are 
\eq{
\rho_a &\approx \frac{g_a}{2\pi^2}e^\frac{\mu-m}{T} \int_0^\infty dq \, q^2\left(m+\frac{q^2}{2m} \right) e^{-\frac{q^2}{2mT}} \\
p_a &\approx \frac{g_a}{2\pi^2}e^\frac{\mu-m}{T} \int_0^\infty dq \, q^2 \frac{q^2}{3\left( m+\frac{q^2}{2m} \right)} e^{-\frac{q^2}{2mT}} \\
n_a &\approx \frac{g_a}{2\pi^2}e^\frac{\mu-m}{T} \int_0^\infty dq \, q^2 e^{-\frac{q^2}{2mT}}
}
and then the integrals can be done analytically integrating by parts, e.g.
\eq{
n_a &= \frac{g_a}{2\pi^2}e^\frac{\mu-m}{T} \int_0^\infty \sqrt{2mT} \cdot du \cdot 2mT u^2 e^{-u^2} \\
&= \frac{g_a}{2\pi^2}e^\frac{\mu-m}{T} (2mT)^\frac{3}{2} \left\{ \left[-\frac{1}{2}u e^{-u^2}\right]_0^\infty + \int_0^\infty \frac{1}{2}e^{-u^2} \, du \right\} \\
&= g_a \left(\frac{mT}{2\pi}\right)^\frac{3}{2} e^\frac{\mu-m}{T} \quad \text{using } \int_0^\infty e^{-u^2} \, du = \frac{\sqrt{\pi}}{2} 
}
to give 
\eq{
n_a &= g_a \left( \frac{mT}{2\pi} \right)^\frac{3}{2} e^\frac{\mu-m}{T} \\
\rho_a &= g_a \left( \frac{mT}{2\pi} \right)^\frac{3}{2} e^\frac{\mu-m}{T} \left(m+\frac{3}{2}T \right) = n_a \left(m+\frac{3}{2}T \right)  \\
p_a &= g_a \left( \frac{mT}{2\pi} \right)^\frac{3}{2} e^\frac{\mu-m}{T} T = n_a T 
}
Note that if relativistic and non relativistic particles are in thermal equilibrium and $\mu \ll T$ then 
\eq{
\frac{\rho_{\text{non-rel}}}{\rho_\text{rel}} \propto e^{-\frac{m}{T}} \left(\frac{T}{m}\right)^\frac{5}{2} \ll 1
}
hence it is sensible to neglect non-relativistic particles' contribution to energy density, pressure, and entropy density in thermal equilibrium. 

\begin{definition}[Effective Number of Relativistic Degrees of Freedom]\label{def:CSM:1}
Define the \bam{effective number of relativistic degrees of freedom} $g_\ast = g_\ast(T)$ such that the total energy density is
\eq{
\rho = g_\ast(T) \frac{\pi^2}{30} T^4
}
In thermal equilibrium, this is just
\eq{
g_\ast = \sum_{\text{bosons}} g_a + \frac{7}{8} \sum_{\text{fermions}} g_a
}
Out of thermal equilibrium, we must allow for the fact that the species may be at different temperatures, so 
\eq{
g_\ast(T) = \sum_{\text{bosons}} g_a\pround{\frac{T_i}{T}}^4 + \frac{7}{8} \sum_{\text{fermions}} g_a \pround{\frac{T_i}{T}}^4
}
\end{definition}

\begin{example}
A possible way to use this definition is to, during radiation domination, note 
\eq{
0 &= \dot{\rho} + 4H\rho \quad \text{from the continuity equation} \\
&\propto \dot{T} + HT \quad \text{using $\rho\propto T^4$}
}
In a flat universe the Friedmann equation gives 
\begin{align}\label{eq:CSM:5}
H = \sqrt{\frac{\rho}{3 M_{pl}^2}} = \sqrt{g_\ast \frac{\pi^2}{90} } \frac{T^2}{M_{pl}}
\end{align}
Hence the continuity can be rewritten as 
\eq{
\dot{T} &= - T^3 \frac{\pi}{M_{pl}} \sqrt{\frac{g_\ast}{90}} \\
\Rightarrow T(t) &= \left( \frac{5}{2g_\ast} \right)^\frac{1}{4} \sqrt{\frac{3M_{pl}}{\pi t}} \propto \frac{1}{a} \\
\Rightarrow t &= \sqrt{\frac{5}{2g_\ast}} \frac{3M_{pl}}{\pi T^2}
}
\end{example}

\begin{prop}
In equilibrium
\eq{
S = \frac{(\rho+p)V}{T} \Leftrightarrow s = \frac{\rho+p}{T}
}
where $s=\frac{S}{V}$ is the \bam{entropy density}.
\end{prop}
\begin{proof}
From the first law
\eq{
T dS &= d(\rho V) + pdV \\
&= V d\rho + (\rho + p) dV
}
As the situation is in equilibrium, $\rho=\rho(T), p=p(T)$, so 
\begin{align} \label{eq:CSM:3}
TdS = V \frac{d\rho}{dT} dT + (\rho + p) dV
\end{align}
and from the symmetry of partial derivatives 
\eq{
\pd{V} \pd[S]{T} &= \pd{T} \pd[S]{V} \\
\Rightarrow \pd{V} \left( \frac{V}{T} \frac{d\rho}{dT} \right) &= \pd{T} \left(\frac{\rho+p}{T} \right) \\
\Rightarrow \frac{1}{T} \frac{d\rho}{dT} &= - \frac{\rho+p}{T^2} + \frac{1}{T} \left( \frac{d\rho}{dT} + \frac{dp}{dT} \right) \\
\Rightarrow \frac{dp}{dT} &= \frac{\rho+p}{T}
}
Now using \ref{eq:CSM:3} 
\eq{
dS &= \frac{1}{T} d[(\rho+p)V] - \frac{V}{T} dp \\
&= \frac{1}{T} d[(\rho+p)V] - \frac{\rho+p}{T^2}dT \\
&= d\left[ \frac{(\rho+p)V}{T} \right] \\
\Rightarrow S &= \frac{(\rho+p)V}{T} \\
\Rightarrow s &= \frac{S}{V} = \frac{\rho+p}{T}
}
(up to a constant)
\end{proof}

\begin{prop}
Entropy is conserved in equilibrium.
\end{prop}
\begin{proof}
The entropy 
\eq{
S = \frac{(\rho + p)V}{T}
}
But $\rho,p \propto T^4$, $T\propto a^{-1}$, and $V\propto a^3$. Hence $S \propto a^0$ so constant. Instead, if it is assumed entropy is constant, then for constant $g_\ast$, $T\propto a^{-1}$. 
\end{proof}

%%%%%%%%%%%%%%%%%%%%%%%%%%%%%%%%%%%%%%%%%%%%%%%%%%%%%%%%
\subsection{Boltzmann equation}
Consider a 2-2 scattering process $1+2 \leftrightarrow 3+4$. 

\begin{theorem}
The ODE that describes the time evolution of the number density of 1 particles, $n_1$, is 
\eq{
a^{-3} \frac{d(a^3 n_1)}{dt} = \braket{\sigma v} n_1^{(0)} n_2^{(0)} \left[ \frac{n_3 n_4}{n_3^{(0)} n_4^{(0)}} - \frac{n_1 n_2}{n_1^{(0)} n_2^{(0)}} \right]
}
\end{theorem}
\begin{proof}
\eq{
a^{-3} \frac{d(a^3 n_1)}{dt} =& \int \left( \prod_i \underbrace{\frac{d^3 \bm{p}_i}{(2\pi)^3 2E_i}}_{\text{LI measure}} \right) \underbrace{(2\pi)^4\delta^{(3)}(\bm{p}_1 + \bm{p}_2 - \bm{p}_3 -\bm{p}_4)\delta(E_1 + E_2 - E_3 - E_4)}_{\text{4-momentum conservation}} |\mc{M}|^2 \\
& \times \{ f_3 f_4 [1\pm f_2][1 \pm f_1] - f_1 f_2 [1\pm f_3][1 \pm f_4]\}
}
where $E_i = E_{\bm{p}_i}$, $|\mc{M}|$ is the \hl{scattering amplitude from quantum theory}, and $\{ f_3 f_4 \dots \}$ is a term \hl{corresponding to interactions} where the $f_i$ are the density functions. 
\hl{Assume the reaction takes place rapidly} such that $f_i = f_{FD/BE}$. Considering temperatures $T \ll E_\mu$, so using \ref{eq:CSM:4} 
\eq{
\{ f_3 f_4 [1\pm f_2][1 \pm f_1] - f_1 f_2 [1\pm f_3][1 \pm f_4]\} &\approx e^{-\frac{E_3+E_4}{T}}e^{\frac{\mu_3+\mu_4}{T}}-e^{-\frac{E_1+E_2}{T}}e^{\frac{\mu_1+\mu_2}{T}} \\
&= e^{-\frac{E_1+E_2}{T}}\left[ e^{\frac{\mu_3+\mu_4}{T}} - ^{\frac{\mu_1+\mu_2}{T}} \right] \quad \text{using energy conservation}
}
Now defining $n_i^{(0)} = n_i |_{\mu=0}$ we have 
\eq{
e^\frac{\mu_i}{T} &= \frac{n_i}{n_i^{(0)}} \\
\Rightarrow \{ f_3 f_4 [1\pm f_2][1 \pm f_1] - f_1 f_2 [1\pm f_3][1 \pm f_4]\} &= e^{-\frac{E_1+E_2}{T}} \left[ \frac{n_3 n_4}{n_3^{(0)} n_4^{(0)}} - \frac{n_1 n_2}{n_1^{(0)} n_2^{(0)}} \right]
}
Finally, defining the \bam{thermally average cross section}
\eq{
\braket{\sigma v} = \frac{1}{n_1^{(0)} n_2^{(0)}} \int \left( \prod_i \frac{d^3 \bm{p}_i}{(2\pi)^3 2E_i} \right) (2\pi)^4\delta^{(3)}(\bm{p}_1 + \bm{p}_2 - \bm{p}_3 -\bm{p}_4)\delta(E_1 + E_2 - E_3 - E_4) |\mc{M}|^2 e^{-\frac{E_1+E_2}{T}}
}
yields the desired result
\end{proof}

\begin{definition}
The \bam{reaction rate} is 
\eq{
\Gamma = n_2^{(0)} \braket{\sigma v}
}
This gives 
\eq{
\underbrace{a^{-3} \frac{d(a^3 n_1)}{dt}}_{=\mc{O}(H)} = \underbrace{\braket{\sigma v} n_1^{(0)} n_2^{(0)} \left[ \frac{n_3 n_4}{n_3^{(0)} n_4^{(0)}} - \frac{n_1 n_2}{n_1^{(0)} n_2^{(0)}} \right]}_{=\mc{O}(\Gamma)}
}
\end{definition}

Note that
\begin{itemize}
    \item $\Gamma \gg H \Rightarrow$ reaction is fast and efficient, and so quickly reaches the chemical equilibrium 
    \eq{
    \frac{n_3 n_4}{n_3^{(0)} n_4^{(0)}} = \frac{n_1 n_2}{n_1^{(0)} n_2^{(0)}} \Leftrightarrow \mu_1 + \mu_2 = \mu_3 + \mu_4 \quad \text{(\bam{Saha equation})}
    }
    \item $\Gamma \ll H \Rightarrow$ reaction is too slow for expansion of the universe, and solutions becomes 
    \eq{
    a^{-3} \frac{d(a^3 n_1)}{dt}=0 \Rightarrow n_i(t) \approx n_i(a_\ast) \left( \frac{a_\ast}{a(t)}\right)^{-3}
    }
    where $a_\ast$ is when last $\Gamma \approx H$. 
\end{itemize}

Dimensional analysis finds $\Gamma \propto T$ at high temperatures, and so $\frac{\Gamma}{H}\propto\frac{1}{T}$.

%%%%%%%%%%%%%%%%%%%%%%%%%%%%%%%%%%%%%%%%%%%%%%%%%%%%%%%%
\subsection{Big Bang Nucleosynthesis (BBN)}

%%%%%%%%%%%%%%%%%%%%%%%%%%%%%%%%%%%%%%%%%%%%%%%%%%%%%%%%
\subsubsection*{Neutron Abundance}
\hl{Three reactions mediate neutron abundance at MeV energies} 
\eq{
n + \nu_e &\leftrightarrow p^+ + e^- \\
n + \bar{e}^+ &\leftrightarrow p^+ + \bar{\nu}_e \\
n \rightarrow p^+ &+e^- + \bar{\nu}_e
}
\begin{definition}[Fractional Abundance]
Define the \bam{fractional abundance} of neutrons to be 
\eq{
X_n = \frac{n_n}{n_n+n_p}
}
\end{definition}

\hl{Assume now the leptons are in equilibrium}, so $n_l = n_l^{(0)}$. Then 
\eq{
a^{-3} \frac{d(a^3 n_n)}{dt} = n_l^{(0)} \braket{\sigma v} \left\{ \frac{n_p n_n^{(0)}}{n_p^{(0)}} -n_n \right\}
}
Now substitute for $X_n$, using that $(n_p + n_n)a^3$ is conserved and letting $\lambda_{np}=n_l^{(0)} \braket{\sigma v}$
\eq{
a^{-3} \frac{d(a^3 X_n(n_n + n_p))}{dt} &= \lambda_{np} \left\{ (1-X_n)(n_n+n_p)\frac{\left(\frac{m_n T}{2\pi}\right)^\frac{3}{2} e^\frac{m_n}{T}}{\left(\frac{m_p T}{2\pi}\right)^\frac{3}{2} e^\frac{m_p}{T}}-X_n(n_n+n_p) \right\} \\
\Rightarrow \frac{dX_n}{dt} &\approx \lambda_{np} \left[ (1-X_n) e^{-\frac{Q}{T}} - X_n \right] \quad \text{letting } Q=m_n-m_p
}
Let $x=\frac{Q}{T}$ so 
\eq{
\frac{dx}{dt}=-x\frac{\dot{T}}{T} = xH \quad \text{using} \quad T \propto a^{-1}
}
then 
\be\label{eq:CSM:16}
\frac{dX_n}{dx} = \frac{\lambda_{np}(x)}{xH(x)}\left[ e^-x - X_n(1+e^{-x})\right]
\ee
\hl{It is an exercise in quantum field theory to find that }
\eq{
\lambda_{np}(x) = \frac{255}{t_{life}x^5}(12+6x+x^2)
}
with $t_{\text{life}} \sim 887s\sim 15\text{min}$ the neutron lifetime. \\
Finally at this time, 
\eq{
g_\ast = \underbrace{(1\times2)}_{1 \text{ photon}} + \frac{7}{8}[\underbrace{(3\times1)}_{3 \text{ neutrinos}} + \underbrace{(3\times 1)}_{3 \text{ antineutrinos}}+\underbrace{(1\times 2)}_{1 \text{ electron}} + \underbrace{(1\times2)}_{1 \text{ positron}}] = 10.75
}
Hence it is possible to calculate 
\eq{
H(x) &= \sqrt{\frac{g_\ast}{90}} \frac{T^2 \pi}{M_{pl}} \quad \text{from \ref{eq:CSM:5}} \\
&= \frac{1}{x^2}\sqrt{\frac{g_\ast}{90}} \frac{Q^2 \pi}{M_{pl}} \\
&= \frac{H(1)}{x^2} \approx \frac{1.1 s^{-1}}{x^2}
}
The solution to \ref{eq:CSM:16} may then be found numerically. 

\begin{remark}
When $T\sim 0.1 MeV$, neutron decay now becomes important (this corresponds to when the interaction time is comparable to the decay time). At this point to take account for decay, multiply $n_n$ by $e^{-\frac{t}{t_{\text{life}}}}$
\end{remark}

%%%%%%%%%%%%%%%%%%%%%%%%%%%%%%%%%%%%%%%%%%%%%%%%%%%%%%%%
\subsubsection*{Light Element Formation}
\hl{The approximation taken for light element formation will be that it is instantaneous at equilibrium}. Deuterium $D$ is mediated by the process \eq{
n + p \leftrightarrow D + \gamma 
}
\eq{
\text{Equilibrium} \Rightarrow \left[ \frac{n_D n_\gamma}{n_D^{(0)} n_\gamma^{(0)}} - \frac{n_n n_p}{n_n^{(0)} n_p^{(0)}} \right] 
}
\hl{Photons have negligible chemical potential} $\Rightarrow n_\gamma = n_\gamma^{(0)}$, so 
\eq{
\frac{n_D}{n_n n_p} &= \frac{n_D^{(0)}}{n_n^{(0)} n_p^{(0)}}\\
&= \frac{3}{4}\left( \frac{2\pi m_D}{m_n m_p T} \right)^\frac{3}{2} e^{\frac{(m_n + m_p-m_D)}{T}} \\
& \approx \frac{3}{4}\left( \frac{4\pi}{m_p T} \right)^\frac{3}{2} e^{\frac{B_D}{T}} \quad \text{letting } B_D = m_n+m_p-m_D
}

\begin{definition}[Baryon-to-Photon Ratio]
Define the \bam{baryon-to-photon ratio} 
\eq{
\eta_b = \frac{n_b}{n_\gamma}
}
\end{definition}

So assuming $n_n\sim n_p\sim n_b$ the \bam{\hl{baryon number density}} yields

\eq{
\frac{n_D}{n_b} & \approx n_b \frac{3}{4}\left( \frac{4\pi}{m_p T} \right)^\frac{3}{2} e^{\frac{B_D}{T}}\\
& \approx \eta_b \psquare{2 \frac{\zeta(3)}{\pi^2}T^3} \cdot \frac{3}{4}\left( \frac{4\pi}{m_p T} \right)^\frac{3}{2} e^{\frac{B_D}{T}} \\
& \approx \eta_b \left( \frac{T}{m_p}\right)^\frac{3}{2} e^\frac{B_D}{T}
}
%%%%%%%%%%%%%%%%%%%%%%%%%%%%%%%%%%%%%%%%%%%%%%%%%%%%%%%%
%%%%%%%%%%%%%%%%%%%%%%%%%%%%%%%%%%%%%%%%%%%%%%%%%%%%%%%%


\part{Cosmological Perturbation Theory}
%%%%%%%%%%%%%%%%%%%%%%%%%%%%%%%%%%%%%%%%%%%%%%%%%%%%%%%%
%%%%%%%%%%%%%%%%%%%%%%%%%%%%%%%%%%%%%%%%%%%%%%%%%%%%%%%%
\section{Observables and statistical properties}
%%%%%%%%%%%%%%%%%%%%%%%%%%%%%%%%%%%%%%%%%%%%%%%%%%%%%%%%
\subsection{Observables}
\subsubsection*{Large scale structure}
\begin{definition}[Fractional Overdensity]
Let $n_g$ be the galaxy number density. Then define the \bam{fractional overdensity} as 
\[
\delta_g(\bm{x}) = \frac{n_g(\bm{x})-\bar{n}_g}{\bar{n}_g}
\]
where $\bar{n}_g$ is the average value. 
\end{definition}

It is typical to assume a linear relation ship between the galaxy density and matter density 
\[
n_g = b \rho_m
\]
where b is the \bam{linear galaxy bias}. This gives the \bam{matter density perturbation}
\[
\delta_g(\bm{x}) = b \times \frac{\rho_m(\bm{x})-\bar{\rho}_m}{\bar{\rho}_m} = b \times \delta_m(\bm{x})
\]

\begin{definition}[Two point correlation function]
Given a general field $f$ and some probability of having the field configuration $Pr[f]$, the \bam{two point correlation function} is 
\eq{
\xi^f(\bm{x},\bm{y}) = \braket{f(\bm{x})f(\bm{y})} = \int \mc{D}f f(\bm{x})f(\bm{y}) Pr[f]
}
\end{definition}

\begin{lemma}
Imposing homogeneity and isotropy of the universe gives 
\eq{
\braket{f(\bm{k})f(\bm{k}^\prime)} = (2\pi)^3 P^f(|\bm{k}|) \delta^{(D)}(\bm{k} + \bm{k}^\prime)
}
for the Fourier coefficients $\braket{f(\bm{k})f(\bm{k}^\prime)}$ such that 
\begin{align}\label{eq:CSM:15}
 \xi^f(\bm{x},\bm{y}) = \braket{f(\bm{x})f(\bm{y})} &= \int \frac{d\bm{k}}{(2\pi)^3} \frac{d\bm{k}^\prime}{(2\pi)^3} e^{-i\bm{k}\cdot\bm{x} -i \bm{k}^\prime \cdot \bm{y}} \braket{f(\bm{k})f(\bm{k}^\prime)} \\
 &=\int \frac{d\bm{k}}{(2\pi)^3} \frac{d\bm{k}^\prime}{(2\pi)^3} e^{-i\bm{k}\cdot\bm{x} -i \bm{k}^\prime \cdot \bm{y}}  (2\pi)^3 P^f(|\bm{k}|) \delta^{(D)}(\bm{k} + \bm{k}^\prime) \\
 &= \int \frac{d\bm{k}}{(2\pi)^3} P^f(|\bm{k}|) e^{-i\bm{k} \cdot (\bm{x} - \bm{y})}
\end{align}
This defines the \bam{power spectrum} $P^f$ of the field. It is common to also define the \bam{dimensionless power spectrum} 
\eq{
\highlight{\Delta^f(k)} = \frac{k^3}{2\pi^2} P^f(k)
}
\end{lemma}

\begin{definition}[Gaussian Random Fields]
A \bam{Gaussian random field} is one whose probability functional takes the form 
\eq{
Pr[f] \propto \frac{\exp -f_i \xi_{ij} f_j}{\sqrt{\det \xi_{ij}}}
}
for 
\eq{
\braket{f_i f_j} = \xi_{ij}
}
\end{definition}

\begin{definition}[Flat Sky approximation]
The CMB temperature $T=T(\hat{\bm{n}})$ is defined on a sphere. However, on a sufficiently small patch the \bam{flat-sky approximation} can be used, taking $T(\bm{l})$ to be the 2D Fourier transform of $T$, then the power spectrum is given by 
\eq{
\braket{T(\bm{l})T(\bm{l}^\prime)} = (2\pi)^3 C_l \delta^{(D)}(\bm{l} + \bm{l}^\prime )
}
\end{definition}

%%%%%%%%%%%%%%%%%%%%%%%%%%%%%%%%%%%%%%%%%%%%%%%%%%%%%%%%
\subsection{Assumptions}
It will be assumed throughout subsequent sections that all observational predictions are generated by "well known" physics if the initial density fluctuation are 
\begin{itemize}
    \item Scale invariant : $P^f \propto k^{-4 + n_s}$ with $n_s \approx 1 \Rightarrow \Delta^f \propto k^{-1+n_s}$. 
    \item Adiabatic. 
    \item Gaussian.
\end{itemize}
These can be justified through inflation. 

%%%%%%%%%%%%%%%%%%%%%%%%%%%%%%%%%%%%%%%%%%%%%%%%%%%%%%%%
%%%%%%%%%%%%%%%%%%%%%%%%%%%%%%%%%%%%%%%%%%%%%%%%%%%%%%%%
\section{Decomposition}
We will consider perturbations about homogeneous flat FLRW spacetime, namely 
\eq{
g_{\mu\nu}(x,t) &= \bar{g}_{\mu\nu}(t) + h_{\mu\nu}(x,t) \\
T_{\mu\nu}(x,t) &= \bar{T}_{\mu\nu}(t) + \delta T_{\mu\nu}(x,t) 
}
where $|h| \ll |\bar{g}|$ and $|\delta T| \ll |\bar{T}|$. Write 
\eq{
\delta \rho (x,t) &= \rho(x,t) - \bar{\rho}(t) \\
\delta p(x,t) &= p(x,t) - \bar{p}(t)
}

\begin{theorem}
The linearised, trace reversed Einstein equations are 
\eq{
\highlight{\text{This may not be necessary}}
}
\end{theorem}

These are now entirely written in terms of $h_{\mu\nu}$ and $\delta T_{\mu\nu}$. 
%%%%%%%%%%%%%%%%%%%%%%%%%%%%%%%%%%%%%%%%%%%%%%%%%%%%%%%%
\subsection{Fourier Decomposition}

\begin{prop}
Fourier modes of a perturbation decouple. 
\end{prop}
\begin{proof}
Consider the general form of linear equations of motion
\eq{
\sum_A \mc{O}_A Pert_A(\bm{x},t) = 0
}
where $Pert_A$ is some perturbation $\set{h_{\mu\nu},\delta T_{\mu\nu}}$, enumerated by index $A$, and the $\mc{O}_A$ are linear differential operators.
Now 
\begin{itemize}
    \item Ensuring general covariance gives that the $\mc{O}_A$ must be constructed from covariant derivatives $\nabla_\mu$ and other tensorial objects. i.e 
    \eq{
    \mc{O}_A = \mc{O}_A(\del_t, \del_j)
    }
    \item The background is homogeneous the $\mc{O}_A$ cannot depend on $\bm{x}$
\end{itemize}
hence when the Fourier transform of the e.o.m is taken we get 
\eq{
0 &= \int d^3 \bm{x} \, e^{-i \bm{k} \cdot \bm{x}} \sum_A \mc{O}_A Pert_A(\bm{x},t) \\
&= \sum_A \tilde{\mc{O}}_A \tilde{Pert}_A(\bm{k},t)
}
where 
\eq{
\tilde{\mc{O}}_A &= \mc{O}_A(\del_t,ik_j) \\
\tilde{Pert}_A(\bm{k},t) &= \int d^3 \bm{x} \, e^{i\bm{k}\cdot\bm{x}} Pert_A(\bm{x},t)
}
This has change the e.o.m from 1 PDE to infinitely many ODEs, on for each $\bm{k}$, that are not coupled together. 
\end{proof}

Note that as $r = x_{phys} = a x$,  $k_{phys} = \frac{k}{a}$. 



%%%%%%%%%%%%%%%%%%%%%%%%%%%%%%%%%%%%%%%%%%%%%%%%%%%%%%%%
\subsection{Scalar-Vector-Tensor Decomposition (SVT)}

\begin{definition}[Hodge Decomposition]
Any vector $v_i$ can be decomposed as 
\eq{
v_i = w_i + \del_i \theta
}
where $\theta$ is the solution (that necessarily exists, e.g. by using Green's functions, on topologically trivial spaces) of 
\eq{
\del^i v_i = \nabla^2 \theta
}
and then 
\eq{
w_i = v_i - \del_i \theta
}
satisfies $\del^i w_i = 0$
\end{definition}

\begin{definition}[Tensor Decomposition]
A tensor $S_{ij}$ can be decomposed as 
\eq{
S_{ij} = \delta_{ij} A + \del_i\del_j B  + \del_{(i}C_{j)} + D_{ij}
}
with 
\eq{
\del_i C_i = 0 = D_{ii} = \del_i D_{ij} 
}
\end{definition}


\begin{theorem}
Perturbations with different helicities decouple.
\end{theorem}
\begin{proof}
Under a rotation, $\set{x^0,x^i} \to \set{{x^\prime}^0,{x^\prime}^{i^\prime}}=\set{x^0, R\indices{_i^{i^\prime}} x^i}$, so the transformations of tensors is governed by 
\eq{
\pd[{x^\prime}^{\mu^\prime}]{x^\mu} = \begin{pmatrix} 1 & \\ & R\indices{^{i^\prime}_i} \end{pmatrix}
}
Hence terms that contain no spacial indices are rotation-scalars, terms that contain 1 spatial index are rotation vector, and those with two are rotation tensors. \\
Now, it is impossible to construct a non-zero scalar from transverse vectors or transverse traceless tensors using only derivatives and background quantities at linear order. Hence all mixing vanishes. 
\end{proof}

\begin{definition}[SVT Decomposition]
The \bam{SVT decomposition} oh $h$ and $\delta T $ is 
\eq{
h_{00} &= -E \\
h_{i0} &= a\left[ \del_i F + G_i \right] \text{ s.t } \del^i G_i = 0 \\
h_{ij} &= a^2 \left[ \delta_{ij}  A + \del_i\del_j B  + \del_{(i}C_{j)} + D_{ij} \right] \text{ s.t. } \del_i C_i = 0 = D_{ii} = \del_i D_{ij} \\
\delta T_{00} &= -\bar{\rho} h_{00} + \delta \rho \\
\delta T_{i0} &= \bar{p} h_{0i} - (\bar{\rho} + \bar{p} ) \left[ \del_i \delta u + \delta u_i^V \right] \text{ s.t } \del^i \delta u_i^V = 0 \\
\delta T_{ij} &= \bar{p} h_{ij} + a^2 \left[ \delta_{ij} \delta p + \del_i \del_j \pi^S + \del_{(i}\pi_{j)}^V + \pi_{ij}^T \right] \text{ s.t. } \del_i \pi_i^V = 0 = \pi_{ii}^T = \del_i 
\pi_{ij}^T \\
u_\mu &= (u_0,u_i) = (-1+\delta u_0, \del_i \delta u + \delta u_i^V)
}
The $\pi$ terms are the \bam{anisotropic inertia}, and are zero for perfect fluids. 
\end{definition}



%%%%%%%%%%%%%%%%%%%%%%%%%%%%%%%%%%%%%%%%%%%%%%%%%%%%%%%%
\subsubsection{Vector Perturbations}
Selecting only the vector perturbations, and \hl{assuming the constituents of the universe behave like perfect fluid} so the anisotropic inertia can be neglected gives 
\eq{
\del_0 \left[ \left(\bar{\rho} + \bar{p} \right) \delta u_j^V \right] + 3H \left(\bar{\rho} + \bar{p} \right)\delta u_j^V = 0 
\Rightarrow \left(\bar{\rho} + \bar{p} \right) \delta u_j^V \propto a^{-3}
}
Substituting into the linearised Einstein equations gives 
\eq{
M_{pl}^{-2} \left(\bar{\rho} + \bar{p} \right) \delta u_j^V a = \frac{1}{2} \del^2 \left( G_j - \highlight{a} \dot{C}_j \right) \\
\Rightarrow   G_j - \highlight{a} \dot{C}_j \propto a^{-2}
}

%%%%%%%%%%%%%%%%%%%%%%%%%%%%%%%%%%%%%%%%%%%%%%%%%%%%%%%%
\subsubsection{Tensor Perturbations}
Selecting only the tensor perturbations, and neglecting anisotropic inertia gives 
\eq{
\ddot{D}_{ij} + 3H \dot{D}_{ij} - \frac{1}{a^2} \del^2 D_{ij} = 0
}
or in terms of the Fourier transform 
\eq{
\ddot{\tilde{D}}_{ij} + 3H \dot{\tilde{D}}_{ij} - \frac{k^2}{a^2} \tilde{D}_{ij} = 0
}

\begin{definition}
Define the \bam{polarisation tensors} $\eps_{ij}^s$ for $s=1,2$ to be two independent solutions to 
\eq{
k^i \eps_{ij}^s(\bm{k}) = 0 = \eps_{ii}^s(\bm{k}) \\
\eps_{ij}^s \eps_{ji}^{s^\prime} = 2 \delta_{s s^\prime}
}
Then write 
\eq{
\tilde{D}_{ij}(t,\bm{k}) = \sum_{s=1,2} \eps_{ij}^s(\bm{k}) \mc{D}_s(t,k)
}
\end{definition}

Each polarisation evolves as 
\eq{
\ddot{\mc{D}}_s(t,k) + 3H \dot{\mc{D}}_s(t,k) - \frac{k^2}{a^2} \mc{D}_s(t,k) = 0
}

In the super Hubble regime $k_{phys} \ll H \Rightarrow k \ll aH$. 
\eq{
\Rightarrow \ddot{\mc{D}}_s(t,k) + 3H \dot{\mc{D}}_s(t,k) \approx 0 \Rightarrow \dot{\mc{D}}_s \propto a^{-3} \\
\Rightarrow \mc{D}_s(t,k) = A_s(k) + B_s(k) a(t)^\frac{3(w-1)}{2}
}

In the sub Hubble regime, $k_{phys} \gg H \Rightarrow k \gg aH$ and solve using the ansatz 
\eq{
\mc{D}_s(t,k) = X(t) \exp \left[ ik\int^t \frac{dt^\prime}{a(t^\prime)} \right] \\
\Rightarrow \mc{D}_s (t,k) = \frac{\bar{A}_s \cos k\tau + \bar{B}_s \sin k\tau}{a}
}
for $\tau = \int^t \frac{dt^\prime}{a(t^\prime)}$

%%%%%%%%%%%%%%%%%%%%%%%%%%%%%%%%%%%%%%%%%%%%%%%%%%%%%%%%
\subsubsection{Scalar Perturbations}
Taking the Newtonian gauge, and  selecting out just the scalar perturbations, taking $A=-2\Psi$, $E=2\Phi$ gives 
\eq{
- \frac { 1 } { 2 M _ { \mathrm { Pl } } ^ { 2 } } \left[ \delta \rho - \delta p - \nabla ^ { 2 } \pi ^ { S } \right] &= H \dot { \Phi } + \left( 4 H ^ { 2 } + 2 \frac { \ddot { a } } { a } \right) \Phi - \frac { \nabla ^ { 2 } \Psi } { a ^ { 2 } } + \ddot { \Psi } + 6 H \dot { \Psi } \\
- \frac { a ^ { 2 } } { M _ { \mathrm { Pl } } ^ { 2 } } \partial _ { i } \partial _ { j } \pi ^ { S } &= \partial _ { i } \partial _ { j } ( \Phi - \Psi ) \\
\frac { 1 } { 2 M _ { \mathrm { Pl } } ^ { 2 } } ( \overline { \rho } + \overline { p } ) \partial _ { i } \delta u &= - H \partial _ { i } \Phi - \partial _ { i } \dot { \Psi } \\
\frac { 1 } { 2 M _ { \mathrm { Pl } } ^ { 2 } } \left( \delta \rho + 3 \delta p + \nabla ^ { 2 } \pi ^ { S } \right) &= \frac { \nabla ^ { 2 } \Phi } { a ^ { 2 } } + 3 H \dot { \Phi } + 3 \ddot { \Psi } + 6 H \dot { \Psi } + 6 \frac { \ddot { a } } { a } \Phi
}
and the continuity equations 
\eq{
\delta p + \nabla ^ { 2 } \pi ^ { S } + \partial _ { 0 } [ ( \overline { \rho } + \overline { p } ) \delta u ] + 3 H ( \overline { \rho } + \overline { p } ) \delta u + ( \overline { \rho } + \overline { p } ) \Phi &= 0 \\
\delta \dot { \rho } + 3 H ( \delta \rho + \delta p ) + \nabla ^ { 2 } \left[ \frac { ( \overline { \rho } + \overline { p } ) } { a ^ { 2 } } \delta u + H \pi ^ { S } \right] - 3 ( \overline { \rho } + \overline { p } ) \dot { \Psi } &= 0
}
Neglecting anisotropic stress gives that $\Phi=\Psi$ is a solution

%%%%%%%%%%%%%%%%%%%%%%%%%%%%%%%%%%%%%%%%%%%%%%%%%%%%%%%%
\subsection{Gauge Transformations}

As GR is a generally covariant theory, we are able to perform a gauge transform $x^\mu \to {x^\prime}^\mu = x^\mu + \eps^\mu(x)$, where we hold the background constant and attribute all change in tensor to the perturbation. 

\begin{prop}
Under the gauge transform, the tensor $S = \bar{S} + \delta S$ varies as 
\eq{
\Delta(\delta S) = -\mc{L}_\eps S = -\mc{L}_\eps \bar{S} + \mc{O}(\eps^2)
}
so for a scalar $s$, vector $V^\mu$, and $h_{\mu\nu}$
\eq{
\Delta( \delta s) &= -\eps^0 \dot{\bar{s}} \\
\Delta( \delta V^\mu) &= -\eps^\nu \nabla \bar{V}^\mu + \bar{V}^\nu \nabla_\nu \eps^\mu \\
\Delta( h_{\mu\nu}) &= -\nabla_\mu \eps_\nu - \nabla_\nu \eps_\mu
}
\end{prop}

\begin{prop}
Under a gauge transform the SVT decomposition transforms as 
\eq{
\Delta A &= 2H \eps_0 \\
\Delta B &= -\frac{2}{a^2}\eps^S \\
\Delta C_i &= -\frac{1}{a^2} \eps_i^V \\
\Delta D_{ij} &= 0 \\
\Delta E &= 2\dot{\eps_0} \\
\Delta F &= \frac{1}{a} \left( -\eps_0 - \dot{\eps}^S + 2H \eps^S \right) \\
\Delta G_i &= \frac{1}{a} \left( -\dot{\eps}_i^V + 2H \eps_i^V \right) 
\Delta \delta \rho  \\
\Delta \delta p &= \dot{\bar{p}} \eps_0 \\
\Delta \delta u &= - \eps_0 \\
\Delta \pi^S &= \Delta \pi_i^V =\Delta \pi_{ij}^T = \Delta \delta u_i^V =0
}
decomposing the gauge transform as 
\eq{
\eps^\mu = (\eps^0 , \del^i \eps^S + {\eps^i}^V)
}
\end{prop}

\begin{definition}[Newtonian Gauge]
The \bam{Newtonian gauge} takes 
\eq{
\left\{ \begin{array}{c} \eps^S = \frac{a^2 B}{2} \\ \eps_0 = aF - \frac{a^2}{2} \dot{B} \end{array} \right. \Rightarrow \left\{ \begin{array}{c} B=0 \\ F=0 \end{array} \right.
}
In this gauge we find 
\eq{
h_{00} &= -2\Phi \\
h_{0i} &= 0 \\
h_{ii} &= -2a^2 \delta_{ij}\Psi \\
\delta T _ { 00 } &= 2 \overline { \rho } \Phi + \delta \rho \\
\delta T _ { 0 i } &= - ( \overline { \rho } + \overline { p } ) \partial _ { i } \delta u\\
\delta T _ { i j } &= a ^ { 2 } \partial _ { i j } \pi ^ { S } + \delta _ { i j } a ^ { 2 } ( \delta \rho - \overline { p } \Psi ) 
}
\footnote{\hl{How do we fix G to be 0?} }
\end{definition}


%%%%%%%%%%%%%%%%%%%%%%%%%%%%%%%%%%%%%%%%%%%%%%%%%%%%%%%%
\section{Adiabatic Modes}

\begin{definition}[Curvature Perturbations]
The \bam{curvature perturbation on comoving hypersurfaces} is 
\eq{
\mc{R} = \frac{A}{2} + H \delta u 
}
and the \bam{curvature perturbation on constant density hypersurfaces} is 
\eq{
\zeta = \frac{A}{2} - H \frac{\delta \rho}{\dot{\bar{\rho}}}
}
\end{definition}

\begin{prop}
In the Newtonian gauge 
\eq{
\zeta(k,t) = \mc{R}(k,t) + \frac{M_{pl}^2}{3a^2(\bar{\rho} + \bar{p})} k^2 A(k,t) 
}
\end{prop}

Note $\zeta-\mc{R} \propto \left(\frac{k}{aH}\right)^2$, so on super Hubble scales when $k\ll aH$ $\zeta-\mc{R} \approx0$.

\begin{theorem}
Outside the Hubble radius $\exists k_1, k_2, k_3$ s.t. 
\begin{itemize}
    \item $\dot{\mc{R}}(k_1,t) = 0 = \dot{\mc{R}}(k_2,t)$ (adiabatic modes) 
    \item $\mc{R}(k_1,t) \neq 0$
    \item $\dot{\tilde{D}}_{ij}(k_3) = 0$ and $\tilde{D}_{ij}(k_3) \neq 0$
\end{itemize}
\end{theorem}
\begin{proof}
We will prove to linear order in the Newtonian gauge. 
Consider the gauge transform that preserves the Newtonian Gauge 
\eq{
\eps_\mu = (\eps(t), a(t)^2 \omega_{ij} x^j)
}
This perturbs a flat FLRW metric to 
\eq{
\Phi &= -\dot{\eps} \\
\Psi &= H\eps - \frac{1}{3} \omega_{ii} \\
\delta p &= -\dot{\bar{p}}\eps \\
\delta \rho &= -\dot{\bar{\rho}}\eps \\
\delta u &= \eps \\
\pi^S &= 0 \\
D_{ij} &= -\omega_{(ij)} + \frac{2}{3} \delta_{ij} \omega_{kk}
}
This perturbation must be a solution of the Einstein equation as GR is a generally covariant theory. For it to be a physical solution it must decay at infinity. This requires the Fourier transform to be supported at $k \neq 0  $. Then if the solution does not decay, it is always possible to ensure it does by perturbing the solution slightly in Fourier space, and using that this remains a solution to the equation. Hence $\exists k$ such that $D_{ij}$ is constant and non-vanishing up to correction suppressed by $k^2$ in the super Hubble limit  \\
The solution is therefore only not physical if an equation of motion vanishes identically for $k=0$. This does occur for off diagonal term, so we impose for $k\neq 0$
\eq{
k_i k_j (\Phi- \Psi) = 0 \Rightarrow \Phi = \Psi
}
this fixes 
\eq{
\dot{\eps}+H\eps = \frac{1}{3} \omega_{kk} \Rightarrow \eps(t) = \frac{\omega_{kk}}{3a(t)} \int_T^t a(t^\prime) dt^\prime 
}
Substituting yields 
\eq{
\mc{R} = \frac{\omega_{kk}}{3}
}
Hence a solution with $\dot{\mc{R}}=0$, $\mc{R}\neq 0$ always exists by diffeomorphism invariance. (\hl{where did the second sol go})

\end{proof}


%%%%%%%%%%%%%%%%%%%%%%%%%%%%%%%%%%%%%%%%%%%%%%%%%%%%%%%%
\section{Newtonian Perturbation}

For a non-relativistic fluid with mass density $\rho$, pressure $P$, and velocity $\bm{u}$, the determining equations are 
\eq{
\del_t \rho + \grad_{\bm{r}} \cdot (\rho \bm{u}) = 0 \quad \text{(continuity equation)} \\
\del_t \bm{u} + \bm{u} \cdot \grad_{\bm{r}} \bm{u} = -\frac{1}{\rho} \grad_{\bm{r}} P - \grad_{\bm{r}} \Phi \quad \text{(Euler equation)} \\
\nabla_{\bm{r}}^2 \Phi = 4\pi G \rho \quad \text{(Poisson's equation)}
}
For a comoving observer the physical coordinate $\bm{r}$ is related to the comoving coordinate $\bm{x}$ by 
\eq{
\bm{r}(t) = a(t) \bm{x} \\
\Rightarrow \bm{u} = \dot{\bm{r}} = H\bm{r}
}
Hence 
\eq{
\pround{\pd{t}}_{\bm{r}} = \pround{\pd{t}}_{\bm{x}} - H\bm{x} \cdot \grad \\
\grad_{\bm{r}} = \frac{1}{a} \grad
}
letting $\grad = \grad_{\bm{x}}$. \\
Now writing a perturbation as 
\eq{
\rho \to \bar{\rho} + \delta\rho = \bar{\rho}(1+\delta) \\
P \to \bar{P} + \delta P \\
\bm{u} \to Ha\bm{x} + \bm{v} \\
\Phi \to \bar{\Phi} + \delta \Phi
}
and substituting in gives, to zeroth and first order 
\eq{
\pd[\bar{\rho}]{t}+3H\bar{\rho} = 0 \quad \text{(continuity for matter)}\\
\dot{\delta} = -\frac{1}{a}\grad\cdot\bm{v} \quad \text{(continuity equation)} \\
\dot{\bm{v}} + H\bm{v} = -\frac{1}{a\bar{\rho}}\grad \delta P -\frac{\grad \Phi}{a} \quad \text{(Euler equation)} \\
\nabla^2 \Phi = 4\pi G a^2 \bar{\rho} \delta \quad \text{(Poisson's equation)}
}
Combining these gives 
\begin{align} \label{eq:CSM:1}
\ddot{\delta} + 2H\dot{\delta} - \frac{1}{a^2\bar{\rho}}\nabla^2 \delta P - 4\pi G \bar{\rho}\delta = 0
\end{align}

\begin{definition}[Barotropic Fluid]
A barotropic fluid is one where $P=P(\rho)$. In such a fluid the speed of sound is 
\[
c_s = \sqrt{\pd[P]{\rho}}
\]
\end{definition}

For a barotropic fluid \ref{eq:CSM:1} gives 
\eq{
\ddot{\delta} + 2H\dot{\delta} - \frac{c_s^2}{a^2}\nabla^2 \delta  - 4\pi G \bar{\rho}\delta = 0
}
and so Fourier transforming 
\eq{
\ddot{\delta} + 2H\dot{\delta} + \left[ \frac{c_s^2}{a^2}k^2 - 4\pi G \bar{\rho} \right]\delta = 0
}

\begin{definition}[Jean's wavenumber]
The \bam{Jean's wavenumber and scale} are
\eq{
k_J &= \frac{\sqrt{4\pi G \bar{\rho}a^2}}{c_s} = \frac{2a\sqrt{\pi G\bar{\rho}}}{c_s}  \\
\lambda_J &= \frac{2\pi a }{k_J} = c_s \sqrt{\frac{\pi}{G\bar{\rho}}}
}
\end{definition}

Perturbations on a scale smaller than the Jean's scale, i.e. $k>k_J$ the equation is essentially damped oscillation, whereas for large perturbations $k<k_J$ the perturbation grows as a power law. 

\begin{idea}
This power law growth should be seen as the case where there is insufficient pressure support to stop the gravitational collapse of a perturbation, as the speed of such a wave is too slow to respond in sufficient collapse time. This can be seen by estimating the collapse time as 
\eq{
t_f \sim \frac{1}{\sqrt{G\bar{\rho}}}
}
and the time for a pressure wave to cross a perturbation radius $R$ as 
\eq{
t_{sc} = \frac{R}{c_s}
}
For pressure support we would then want 
\eq{
t_{sc} < t_f  \\ 
\Rightarrow R \lesssim \lambda_J
}
\end{idea}




%%%%%%%%%%%%%%%%%%%%%%%%%%%%%%%%%
\subsection{Dark matter domination}
In dark matter domination, the Hubble rate is determined only by dark matter. Then as seen before $a\propto t^{\frac{2}{3}}$,  $H=\frac{2}{3t}$ and $H^2=\frac{8\pi G\bar{\rho}_m}{3}$. Then assuming that $c_s=0$ in dark matter
\eq{
\ddot{\delta}_m + \frac{4}{3t} \dot{\delta}_m - \frac{2}{3t^2}\delta_m = 0
}
Power law solutions are $\delta \propto t^\frac{2}{3}, t^{-1}$, so growing modes are $\delta \propto a$. 

%%%%%%%%%%%%%%%%%%%%%%%%%%%%%%%%%
\subsection{Radiation Domination}
During radiation domination, the total energy density sources the Hubble growth, so 
\eq{
\ddot{\delta}_m + 2H\dot{\delta}_m   - 4\pi G \sum_i \bar{\rho}_i\delta_i = 0
}
It will be shown that on subhorizon scales photon perturbations oscillates rapidly with respect to the time scale of structure formation, and so will not contribute overall. Now in radiation domination $a\propto t^\frac{1}{2}$, $H=\frac{1}{2t}$, so 
\eq{
\ddot{\delta}_m + \frac{1}{t}\dot{\delta}_m   - 4\pi G \bar{\rho}_m\delta_m = 0
}
In the absence of pressure the time space must be fixed so 
\eq{
\ddot{\delta}_m \sim H \dot{\delta}_m \sim H^2\delta_m \sim \frac{8\pi G \rho_r }{3}\delta_m \gg 4\pi G \bar{\rho}_m \delta_m
}
and the equation reduces to 
\eq{
\ddot{\delta}_m + \frac{1}{t} \dot{\delta}_m = 0 
}
Hence the solutions are $\delta_m \propto 1, \log t$ 
giving for the growing mode 
\eq{
\delta_m \propto \log a 
}
%%%%%%%%%%%%%%%%%%%%%%%%%%%%%%%%%
\subsection{Dark energy domination}
In dark energy domination, as $\Lambda$ is constant, and $H\gg 4\pi G\bar{\rho}_m$ the equation becomes 
\eq{
\ddot{\delta}_m + 2H \dot{\delta}_m = 0 
}
and so $\delta_m \propto 1, e^{-2Ht}$ so in dark energy domination perturbations stop growing. 

%%%%%%%%%%%%%%%%%%%%%%%%%%%%%%%%%%%%%%%%%%%%%%%%%%%%%%%%
\section{Relativistic Perturbation}

\begin{definition}[Horizon scale]
The \bam{conformal horizon scale} is $\mc{H}^{-1}$, where
\eq{
\mc{H} = \frac{a^\prime}{a}
}
is the conformal Hubble parameter
A mode is \bam{superhorizon} if $k^{-1}\gg \mc{H}^{-1}$, and it is \bam{subhorizon} if $k^{-1}\ll \mc{H}^{-1}$.
\end{definition}

\begin{fact}
Superhorizon modes are not in causal contact with themselves, and so cannot evolve dynamically. 
\end{fact}

Now for a relativistic perturbation, the following first order equations are 
\begin{align}\label{eq:CSM:9}
\delta^\prime + 3\mc{H} \left( \frac{\delta P}{\delta \rho} - \frac{\bar{P}}{\bar{\rho}} \right) \delta = - \left( 1 + \frac{\bar{P}}{\bar{\rho}}\right) (\grad \cdot \bm{v} - 3\Phi^\prime) \quad \text{(Conservation of stress energy)} \\
\bm{v}^\prime + 3\mc{H} \left(\frac{1}{3} - \frac{\bar{P}}{\bar{\rho}} \right) \bm{v} = -\frac{\grad \delta P}{\bar{\rho}+\bar{P}} - \grad\Phi \quad \text{(Euler equation)} \\
\end{align}
and the Einstein equations 
\begin{align} \label{eq:CSM:2}
\nabla^2 \Phi - 3\mc{H}(\Phi^\prime + \mc{H}\Phi)  &= 4\pi G a^2 \delta \rho \\
\Phi^\prime + \mc{H}\Phi &= -4\pi G a^2 (\bar{\rho} + \bar{P}) v \\
\Phi^{\prime\prime} + 3 \mc{H} \Phi^\prime + (2\mc{H}^\prime + \mc{H}^2 ) \Phi &= 4\pi G a^2 \delta P 
\end{align}
where $\bm{v} = \grad v$.
\begin{definition}[Comoving gauge density contrast]
Define the \bam{Comoving gauge density contrast} $\Delta$ by
\eq{
\Delta = \delta - 3\mc{H}\left(1 + \frac{\bar{P}}{\bar{\rho}} \right) v
}
\end{definition}

Substituting for the CGDC gives 
\begin{equation}\label{eq:CSM:8}
\nabla^2 \Phi = 4\pi G a^2 \bar{\rho} \Delta
\end{equation}

\begin{definition}[Comoving curvature perturbation]
The \bam{comoving curvature perturbation} $\mc{R}$ is the curvature perturbation on a comoving hypersurface and is defined by 
\[
\mc{R} = -\Phi + \mc{H}v
\]
Using \ref{eq:CSM:2} this can be written as 
\eq{
\mc{R} = -\Phi - \frac{\mc{H}(\Phi^\prime + \mc{H} \Phi}{4\pi G a^2 (\bar{\rho} + \bar{P})}
}
and then recalling 
\eq{
\mc{H}^2 = \frac{1}{3M_{pl}^2} \bar{\rho}a^2 
}
gives 
\begin{equation}\label{eq:CSM:6}
\mc{R} = -\Phi - \frac{2}{3(1+w)} \left( \frac{\Phi^\prime}{\mc{H}} + \Phi \right)
\end{equation}
\end{definition}

\begin{prop}
$\mc{R}$ is conserved on superhorizon scales
\end{prop}
\begin{proof}
It can be shown 
\eq{
-4\pi G a^2 (\bar{\rho} + \bar{P}) \mc{R}^\prime = \underbrace{4\pi G a^2 \mc{H} (\delta P - \frac{\bar{P}^\prime}{\bar{\rho}^\prime} \delta \rho)}_{=0 \text{ for adiabatic pert}} + \underbrace{\mc{H} \frac{\bar{P}^\prime}{\bar{\rho}^\prime} \nabla^2 \Phi}_{\sim \mc{H}k^2 \mc{R}}
}
Hence 
\eq{
\frac{d \log \mc{R}}{d\log a} \sim \pround{\frac{k}{\mc{H}}}^2
}
small on superhorizon scales. 
\end{proof}

When one component of the universe dominates, $\bar{P} \approx w \bar{\rho} $ and $\delta P \approx w \delta \rho$, substituting in \ref{eq:CSM:2}, yields 
\begin{equation}\label{eq:CSM:7}
\Phi^{\prime \prime} + 3(1+w) \mc{H} \Phi + k^2 w \Phi = 0
\end{equation}
where we have used that 
\eq{
\mc{H}^2 + 2\mc{H} + 3w \mc{H}^2 = 0
} 
from the Friedmann equation, and taken the FT. 

%%%%%%%%%%%%%%%%%%%%%%%%%%%%%%%%%
\subsection{Superhorizon limit}
In the superhorizon limit of \ref{eq:CSM:7} we have that 
\eq{
\Phi^{\prime \prime} + 3(1+w) \mc{H} \Phi^\prime = 0 
}
This has only one growing mode solution $\Phi = \text{const}$ for fixed $w$. Using \ref{eq:CSM:6} we see 
\begin{align} \label{eq:CSM:10}
\mc{R} = - \frac{5+3w}{3+3w} \Phi
\end{align}
must be constant between matter domination and radiation domination, hence 
\eq{
\mc{R} = - \frac{3}{2} \Phi_{RD} = - \frac{5}{3} \Phi_{MD} \\
\Rightarrow \Phi_{MD} = \frac{9}{10} \Phi_{RD}
}
Then \ref{eq:CSM:2} gives 
\eq{
\delta &= \highlight{- \frac{2}{3} \frac{k^2 \Phi}{\mc{H}^2} - 2\frac{\Phi^\prime}{\mc{H}} - 2\Phi} \\
&\approx -2\Phi = \text{ const}
}
Hence initially after inflation, when $\delta \approx \delta_r$, \hl{assuming adiabatic perturbations}, 
\eq{
\highlight{\delta_m = \frac{3}{4} \delta_r} \approx - \frac{3}{2} \Phi_{RD}
}
so we have overall, in matter domination \begin{align}\label{eq:CSM:12}
    \delta_m = \frac{3}{4} \delta_r - 2\Phi_{MD}
\end{align}
%%%%%%%%%%%%%%%%%%%%%%%%%%%%%%%%%
\subsection{Potential Evolution}

\subsubsection*{Radiation Domination}
Taking $w=\frac{1}{3}$ in \ref{eq:CSM:7} gives 
\eq{
\Phi^{\prime \prime} + \frac{4}{\tau} \Phi^\prime  + \frac{k^2}{3} \Phi = 0
}

The solution is 
\begin{align}\label{eq:CSM:11}
\Phi_k = A_k \frac{j_1\pround{\frac{k\tau}{\sqrt{3}}}}{\frac{k\tau}{\sqrt{3}}} + B_k \frac{n_1\pround{\frac{k\tau}{\sqrt{3}}}}{\frac{k\tau}{\sqrt{3}}}
\end{align}
defined as 
\eq{
j_1(x) &= \frac{\sin x}{x^2} - \frac{\cos x}{x} \quad \text{ (Bessel function)} \\
n_1(x) &= - \frac{\cos{x}}{x^2} - \frac{\sin x}{x} \quad \text{(Neumann function)}\\ 
}
In the $x \to 0$ limit, $n_1(x)$ is singular and so $B_k = 0$ in order to match inflationary perturbations \footnote{\hl{why could it not just be very small? Or is this just effectively zero?}}. As $j_1(x) \sim \frac{x}{3}$ for small $x$, we find $A_k = -2\mc{R}_k(\tau=0) $
\eq{
\Rightarrow \Phi_k = -2\mc{R}_k(0) \frac{\sin\pround{\frac{k\tau}{\sqrt{3}}} -\frac{k\tau}{\sqrt{3}} \cos\pround{\frac{k\tau}{\sqrt{3}}}}{\pround{\frac{k\tau}{\sqrt{3}}}^3}
}
Recalling that during radiation domination $\tau = \frac{1}{\mc{H}}$, on subhorizon scales $k\tau = \frac{k}{\mc{H}} \gg 1$ so 
\eq{
\Phi_k \approx 6 \mc{R}_k(0) \frac{\cos\pround{\frac{k\tau}{\sqrt{3}}}}{(k\tau)^2}
}

\subsubsection*{Matter Domination}
Taking $w = 0$ in \ref{eq:CSM:7} gives 
\eq{
\Phi^{\prime \prime} = \frac{6}{\tau} \Phi^\prime = 0
}
The only growing mode solution to this is $\Phi = \text{ const}$. 


%%%%%%%%%%%%%%%%%%%%%%%%%%%%%%%%%
\subsection{Radiation Evolution}

Use again 
\eq{
\delta = - \frac{2}{3} \frac{k^2 \Phi}{\mc{H}^2} -\frac{2\Phi^\prime}{\mc{H}}-2\Phi
}

\subsubsection*{Radiation Domination}
With $\mc{H} = \frac{1}{\tau}$ gives 
\eq{
\delta_r = -\frac{2}{3} k^2 \tau^2 \Phi - 2\Phi^\prime \tau - 2\Phi
}
\begin{itemize}
    \item Superhorizon : $\delta_r = -2\Phi = \text{ const}$ 
    \item Subhorizon : 
    \eq{
    \delta_r = -\frac{2}{3} k^2 \tau^2 \Phi =-4\mc{R}(0) \cos \left( \frac{k\tau}{\sqrt{3}} \right)
    }
\end{itemize}

\subsubsection*{Matter Domination}

Substituting $\frac{\delta P}{\delta \rho} = \frac{\bar{P}}{\bar{\rho}} = \frac{1}{3} $ into 
\eq{
\delta _ { r } ^ { \prime } + 3 \mathcal { H } \left( \frac { \delta P } { \delta \rho } - \frac { \overline { P } } { \overline { \rho } } \right) \delta _ { r } &= - \left( 1 + \frac { \overline { P } } { \overline { \rho } } \right) \left( \nabla \cdot \mathbf { v } _ { r } - 3 \Phi ^ { \prime } \right) \\
\mathbf { v } _ { r } ^ { \prime } + 3 \mathcal { H } \left( \frac { 1 } { 3 } - \frac { \overline { P } } { \overline { \rho } } \right) \mathbf { v } _ { r } &= - \frac { \nabla \delta P } { \overline { \rho } + \overline { P } } - \nabla \Phi
}
gives 
\eq{
&\delta_r^\prime = - \frac{4}{3} \divergence \bm{v}_r \\
&\bm{v}_r^\prime = - \frac{1}{4} \grad \delta_r - \grad \Phi \\
\Rightarrow &\delta_r^{\prime \prime} - \frac{1}{3} \nabla^2 \delta_r = \frac{4}{3} \nabla^2 \Phi = \text{ const}
}
Writing this as 
\eq{
\left[ \delta_r(k,\tau) + 4\Phi_{MD}(k) \right]^{\prime \prime} + \frac{1}{3} k^2 \left[ \delta_r(k,\tau) + 4\Phi_{MD}(k) \right] = 0
}
we can see that $\delta_r$ exhibits oscillations about $\delta_r = -4\Phi_{MD}$. 

%%%%%%%%%%%%%%%%%%%%%%%%%%%%%%%%%
\subsection{Matter Evolution}

\subsubsection*{Superhorizon Evolution} 
Using equation \ref{eq:CSM:8} to write 
\eq{
\Delta_m = \frac{\nabla^2 \Phi}{4\pi G a^2 \bar{\rho}}
}
and recalling that, on superhorizon scale $\Phi$ is constant, so 
\begin{itemize}
    \item Radiation domination $\Rightarrow \bar{\rho} \propto a^{-4} \Rightarrow \Delta_m \propto a^2$
    \item Matter domination $\Rightarrow \bar{\rho} \propto a^{-3} \Rightarrow \Delta_m \propto a$
\end{itemize}


\subsubsection*{Subhorizon Evolution} 
Taking $P\approx 0$ in \ref{eq:CSM:9} gives 
\eq{
\delta_m^\prime = - \divergence \bm{v}_m + 3\Phi^\prime \\ 
\bm{v}_m^\prime + \mc{H} \bm{v}_m = - \grad \Phi
}
which gives 
\eq{
\delta_m^{\prime \prime} + \mc{H} \delta_m^\prime = \nabla^2 \Phi + 3(\Phi^{\prime \prime} + \mc{H} \Phi^\prime ) 
}
Then \hl{assuming} $\Delta \approx \delta $ 
\eq{
\delta_m^{\prime \prime} + \mc{H} \delta_m^\prime - 4\pi G \rho a^2 \delta_m = 0 \\
\Rightarrow \delta_m^{\prime \prime} + \mc{H} \delta_m^\prime - \frac{3}{2} \mc{H}^2 \delta_m = 0
}
Rewriting 
\eq{
\mc{H}^2 = \frac{H_0^2 \Omega_m^2}{\Omega_r} \left( \frac{1}{y} + \frac{1}{y^2} \right)
}
where $y = \frac{a}{a_{eq}}$. It was an example sheet exercise to show that this implies 
\eq{
\frac{d^2 \delta_m}{dy^2} + \frac{2+3y}{2y(1+y)} \frac{d\delta_m}{dy} - \frac{3}{2y(1+y)} \delta_m = 0 \quad \text{(Mezaros equation)}
}
The solutions are 
\eq{
\delta_m \propto \left\{ \begin{array}{c} 2+3y \\ (2+3y)\log \left( \frac{\sqrt{1+y}+1}{\sqrt{1+y}-1} \right) - 6\sqrt{1+y} \end{array} \right.
}
hence 
\begin{itemize}
    \item Radiation domination $ \Rightarrow \delta_m \propto \log a$ 
    \item Matter domination $ \Rightarrow \delta_m \propto a$
\end{itemize}


\subsubsection*{Dark Energy Contribution} At late times 
\eq{
\nabla^2 \Phi = 4 \pi G a^2 \bar{\rho}_m \Delta_m
}
since there are no fluctuations in the dark energy density\footnote{As dark energy has $w = -1$ so is constant density}. Then 
\eq{
\bar{\rho} \propto a^{-3} \Rightarrow \forall k \; \Phi \propto \frac{\Delta_m}{a}
}
We now consider the EE for potential evolution \ref{eq:CSM:2} with $\delta P \approx 0$ so 
\eq{
\Phi^{\prime \prime} + 3 \mc{H} \Phi^\prime + (2\mc{H}^\prime + \mc{H}^2) \Phi = 0 \\
\Rightarrow \left(\frac{\Delta_m}{a} \right)^{\prime \prime} + 3 \mc{H} \left(\frac{\Delta_m}{a} \right)^\prime + (2\mc{H}^\prime + \mc{H}^2) \left(\frac{\Delta_m}{a} \right) = 0 \\
\Rightarrow \Delta_m^{\prime\prime} + \mc{H} \Delta_m^\prime + (\mc{H}^\prime - \mc{H}^2) \Delta_m = 0  \\
\Rightarrow \Delta_m^{\prime\prime} + \mc{H} \Delta_m^\prime - 4 \pi H \bar{\rho}_m a^2 \Delta_m = 0
}
using the Friedmann equation $ \mc{H}^\prime - \mc{H}^2 = -4 \pi H \bar{\rho}_m a^2$. This corresponds to \ref{eq:CSM:1}, the Newtonian case, so we see as before that growth halts due to suppression from dark energy. 



\subsubsection*{Power Spectrum Evolution}
\hl{Assuming a scale invariant initial spectrum} $P_\mc{R} \propto k^{-3}$. Since on superhorizon scales $\Phi \propto \mc{R}$ (see equation \ref{eq:CSM:10}) $P_\Phi \propto k^{-3}$. Using equation \ref{eq:CSM:8} so say, in Fourier space 
\eq{
 \Delta_m \propto k^2 \Phi_k \\
 \Rightarrow P_\Delta \propto (k^2)^2 P_\Phi \propto k 
}
Now at early times (radiation domination) 
\begin{itemize}
    \item subhorizon $\Rightarrow k> k_{eq}$ and $\Delta_m \propto \log a $
    \item superhorizon $\Rightarrow k < k_{eq}$ and $\Delta_m \propto a^2$
\end{itemize}
Thus modes with $k > k_{eq}$ are suppressed relative to modes with $k < k_{eq}$ by a factor 
\eq{
\approx \frac{a_{he}^2}{a_{eq}^2} \log \frac{a_{eq}}{a_{he}}
}
where $he$ stands for horizon exit. Noting that up to horizon entry $k \propto \frac{1}{a}$ so the power spectrum is scaled by a relative factor of 
\eq{
\approx \left[\frac{k_{eq}^2}{k^2} \log \frac{k}{k_{eq}} \right]^2
}
(note $k = k_{he}$) so the final power spectrum is 
\eq{
P_\Delta(k) \propto \left\{ \begin{array}{cc} k^{-3} \left[[\log\left(\frac{k}{k_{eq}}\right)\right]^2 & k>k_{eq} \\ k & k<k_{eq} \end{array} \right.
}

%%%%%%%%%%%%%%%%%%%%%%%%%%%%%%%%%%%%%%%%%%%%%%%%%%%%%%%%
%%%%%%%%%%%%%%%%%%%%%%%%%%%%%%%%%%%%%%%%%%%%%%%%%%%%%%%%
\section{Cosmic Microwave Background (CMB)}
\begin{idea}
The CMB is "pretty much" linear, which makes it nice to work with. 
\end{idea}

\begin{definition}[Last Scattering Surface]
The \bam{last scattering surface} is the hypersurface at the point at which the electron-photon-proton plasma recombines leaving the universe transparent. Photons emitted from this surface constitute the the CMB. The last scattering surface is \hl{assumed} to have occurred at a time $\tau = \tau_\ast + \delta \tau$, at a comoving distance $\chi_\ast = \tau_0 - \tau_\ast$. 
\end{definition}


%%%%%%%%%%%%%%%%%%%%%%%%%%%%%%%%%%%%%%%%%%%%%%%%%%%%%%%%
\subsection{Relating CMB to observables}
Consider propagation in the conformal Newtonian gauge, where
\eq{
ds^2 = a(\tau)^2 \left[ (1+2\Phi) d\tau^2 - (1-2\Phi) d\bm{x}^2 \right]
}
Consider an observer at rest, i.e. $u^\nu \propto \delta^\nu_0$. Then to linear order we must have 
\eq{
u^\nu &= \frac{(1-\Phi) \delta^\nu_0}{a} \\
u_\nu &= a(1+\Phi) \delta_{0\nu}
}
Let a photon have momentum $p^\mu = \frac{d x^\mu}{d\lambda}$ where $\lambda$ is an affine parameter, so  
\eq{
E = p^\mu u_\mu = p^0a(1+\Phi) 
\Rightarrow p^0 = \frac{E(1-\Phi)}{a}
}
Then evaluating the 0 component of the geodesic equation $\frac{dp^\mu}{d\lambda} = \Gamma^\mu_{\rho\sigma} p^\rho p^\sigma$
\eq{
\Rightarrow \frac{dp^0}{d\lambda}\left( \frac{a^\prime}{a} + \Phi^\prime \right) p^0 p^0 + 2p^0 p^i \pd[\Phi]{x^i} - \left[ \frac{a^\prime}{a}(1-2\Phi) - \Phi^\prime \right] \frac{g_{ij}}{a^2} p^i p^j = 0
}
Then using 
\begin{itemize}
    \item $p^i \pd{x^i} = \frac{d}{d\lambda} - p^0 \pd{\tau}$
    \item $p_\mu p^\mu =0$ (as a photon)
    \item $p^ 0 = \frac{E(1-\Phi)}{a}$
\end{itemize}
this can we rewritten as  
\eq{
\frac{d[aE(1+\Phi)]}{d\lambda} &= 2E^2 \Phi^\prime \\
\Rightarrow \frac{ d\log (aE)}{d\tau } - \frac{d\Phi}{d\tau} + 2\Phi^\prime &= 0
}
Hence for a photon emitted at $\tau = \tau_e$ and observed at $\tau = \tau_0$ 
\eq{
E_0 = a(\tau_e) E(\tau_e) \left[ 1 + \underbrace{\Phi_e - \Phi_0}_{\text{redshift}} + \underbrace{\int_{\tau_e}^{\tau_0} 2\Phi^\prime \, d\tau}_{\substack{\text{Integrated } \\ \text{Sachs-Wolfe effect} }} \right]
}
If the last scattering surface instead has velocity $\bm{v}$ it can be shown $E_{\text{obs}} = E_0 ( 1- \hat{\bm{n}} \cdot \bm{v} ) $
\eq{
E_\text{obs} = a(\tau_e) E(\tau_e) \left[ 1 + \Phi_e - \Phi_0 - \hat{\bm{n}} \cdot \bm{v}  + \int_{\tau_e}^{\tau_0} 2\Phi^\prime \, d\tau \right]
}
As this redshifting is what causes the change in temperature of the CMB
\eq{
\frac{T_\text{obs}}{T_e} = a(\tau_e) \left[ 1 + \Phi_e - \Phi_0 - \hat{\bm{n}} \cdot \bm{v}  + \int_{\tau_e}^{\tau_0} 2\Phi^\prime \, d\tau \right]
}
We specifically want to consider the temperature of recombination $T_\ast$, i.e. at the last scattering surface. This temperature must be constant,as it is fixed by atomic physics, but the time at which this temperature is reached will vary.
\eq{
\rho_r ( \tau_\ast + \delta \tau) = (\bar{\rho}_r + \delta \rho_r)( \tau_\ast + \delta \tau) = \sigma T_\ast^4 \\
\Rightarrow \delta \rho_r = \bar{\rho}^\prime_r \delta \tau \\
\Rightarrow \delta \tau = - \frac{\delta \rho_r}{ \bar{\rho}^\prime_r} = - \frac{\delta \rho_r}{ \frac{d\bar{\rho}}{da} \times \frac{da}{d\tau}} = \frac{\delta_r}{4\mc{H}} \quad \text{as } \rho_r \propto a^{-4} \\
\Rightarrow a(\tau_e) = a(\tau_\ast + \delta \tau) = a(\tau_\ast) (1 + \mc{H} \delta \tau) = a(\tau_\ast) \left( 1 + \frac{\delta_r}{4} \right)
}
Hence 
\eq{
\frac{T_\text{obs}}{T_e} = a(\tau_\ast) \left[ 1 + \frac{\delta_r}{4} + \Phi_e - \Phi_0 - \hat{\bm{n}} \cdot \bm{v}  + \int_{\tau_e}^{\tau_0} 2\Phi^\prime \, d\tau \right]
}
Writing $\Delta T(\hat{\bm{n}}) = T(\hat{\bm{n}}) = \bar{T}$, $\bar{T} = a(\tau_\ast) T_\ast$. 
\eq{
\frac{\Delta T(\hat{\bm{n}})}{\bar{T}} =   \underbrace{\frac{\delta_r}{4} + \Phi_e}_{\substack{\text{Sachs-Wolfe} \\ \text{term}}} - \Phi_0 - \hat{\bm{n}} \cdot \bm{v}  + \int_{\tau_e}^{\tau_0} 2\Phi^\prime \, d\tau 
}

%%%%%%%%%%%%%%%%%%%%%%%%%%%%%%%%%%%%%%%%%%%%%%%%%%%%%%%%
\subsection{Power Spectrum}

Expand the CMB temperature in terms of spherically harmonic function $Y_{lm}(\hat{\bm{n}})$
\eq{
T ( \hat { \mathbf { n } } ) = \sum _ { l = 0 } ^ { \infty } \sum _ { m = - l } ^ { l } a _ { l m } Y _ { l m } ( \hat { \mathbf { n } } )
}
Given the orthonormality of the harmonics over the sphere, that is 
\eq{
\int d \hat { \mathbf { n } } \, Y _ { l m } ( \hat { \mathbf { n } } ) Y _ { l ^ { \prime } m ^ { \prime } } ^ { * } ( \hat { \mathbf { n } } ) = \delta _ { l l ^ { \prime } } \delta _ { m m ^ { \prime } }
}
we can express 
\eq{
a _ { l m } = \int d \hat { \mathbf { n } } \,  T ( \hat { \mathbf { n } } ) Y _ { l m } ^ { * } ( \hat { \mathbf { n } } )
}
The \emph{power spectrum} is then defined by 
\eq{
\left\langle a _ { l m } a _ { l ^ { \prime } m ^ { \prime } } ^ { * } \right\rangle = C _ { l } \delta _ { l l ^ { \prime } } \delta _ { m m ^ { \prime } }
}
By seeking to understand \hl{large scale fluctuations}\footnote{Why can we neglect the other terms on large scales} , we will focus only on 
\eq{
S(\bm{x}) =\frac{1}{4}\delta_r(\bm{x}) + \Phi_e(\bm{x})
}
It was shown previously that in linear theory the terms in $S$ can be related to $\mc{R}$, so we express 
\eq{
S(\bm{k},\tau) = T_S(k,\tau) \mc{R}(\bm{k},0)
}
where $T_S$ is some transfer function and we have Fourier transformed. Hence 
\eq{
\frac{\Delta T(\hat{\bm{n}})}{\bar{T}} = S(\bm{x} = \chi_\ast \hat{\bm{n}}, \tau_\ast ) = \int \frac{d^3 k }{(2\pi)^3} e^{-i \bm{k} \cdot \hat{\bm{n}} \chi_\ast} S(\bm{k},\tau)
}

\begin{prop}[Rayleigh Plane Wave Expansion]\label{prop:CSM:RayleighPlaneWaveExpansion}
\eq{
e ^ { i \mathbf { k } \cdot \mathbf { x } } = 4 \pi \sum _ { l, m } i ^ { l } j _ { l } ( k x ) Y _ { l m } ^ { * } ( \hat { \mathbf { k } } ) Y _ { l m } ( \hat { \mathbf { x } } )
}
where $j_l$ are the spherical Bessel functions. 
\end{prop}

\begin{remark}
In cosmology, Bessel functions like to turn up (why?) (e.g. prop \ref{prop:CSM:RayleighPlaneWaveExpansion}, equation \ref{eq:CSM:11}) . Hence it can sometimes be useful to try turn an equation into the form of Bessel's equation 
\[
\frac{d^2y}{dx^2} + \frac{1}{x} \frac{dy}{dx} + \left( 1-\frac{\nu^2}{x^2} \right )y = 0 
\]
\end{remark}

\begin{theorem}
\eq{
C_l &= 4\pi \int d(\log k) \, \Delta_\mc{R}(k)  [ T_S(k,\tau_\ast) j_l(k \chi_\ast)]^2
}
\end{theorem}
\begin{proof}
Using prop \ref{prop:CSM:RayleighPlaneWaveExpansion} gives 
\eq{
\frac{\Delta T(\hat{\bm{n}})}{\bar{T}} = 4\pi \int \frac { d ^ { 3 } k } { ( 2 \pi ) ^ { 3 } } T _ { S } \left( k , \tau _ { * } \right) \mathcal { R } ( \mathbf { k } , 0 ) \sum _ { l, m } i ^ { l } j _ { l } \left( k \chi _ { * } \right) Y _ { l m } ^ { * } ( \hat { \mathbf { k } } ) Y _ { l m } ( \hat { \mathbf { n } } )
}
Hence we can read off the the spherical harmonic coefficients 
\eq{
a _ { l m } = 4 \pi i ^ { l } \int \frac { d ^ { 3 } k } { ( 2 \pi ) ^ { 3 } } T _ { S } \left( k , \tau _ { * } \right) \mathcal { R } ( \mathbf { k } , 0 ) j _ { l } \left( k \chi _ { * } \right) Y _ { l m } ^ { * } ( \hat { \mathbf { k } } )
}
Hence 
\eq{
\braket{ a_{lm} a_{l^\prime m^\prime}^\ast} &= (4\pi)^2 i^{l-l^\prime} \int \frac{d^3 k }{(2\pi)^3} \int \frac{d^3 k^\prime }{(2\pi)^3} T_S(k,\tau_\ast) T_S(k^\prime,\tau_\ast) \braket{ \mc{R}(\bm{k},0) \mc{R}^\ast (\bm{k}^\prime,\tau_\ast) } j_l(k \chi_\ast) j_{l^\prime}(k^\prime \chi_\ast) Y_{lm}^\ast (\hat{\bm{k}}) Y_{l^\prime m^\prime} (\hat{\bm{k}}^\prime) \\
&= (4\pi)^2 i^{l-l^\prime} \int \frac{d^3 k }{(2\pi)^3} \int \frac{d^3 k^\prime }{(2\pi)^3} T_S(k,\tau_\ast) T_S(k^\prime,\tau_\ast) \left[ (2\pi)^3 \frac{2\pi^2}{k^3} \Delta_\mc{R} (k) \delta(\bm{k}+\bm{k}^\prime) \right] j_l(k \chi_\ast) j_{l^\prime}(k^\prime \chi_\ast) Y_{lm}^\ast (\hat{\bm{k}}) Y_{l^\prime m^\prime} (\hat{\bm{k}}^\prime) \\
&= (4\pi)^2 i^{l-l^\prime} \int \frac{d^3 k }{(2\pi)^3}  \frac{2\pi^2}{k^3} \Delta_\mc{R} (k) [ T_S(k,\tau_\ast)]^2 j_l(k \chi_\ast) j_{l^\prime}(k \chi_\ast) Y_{lm}^\ast (\hat{\bm{k}}) Y_{l^\prime m^\prime} (\hat{\bm{k}}) \\
&= 4\pi \delta_{l l^\prime} \delta_{m m^\prime} \int d(\log k) \, \Delta_\mc{R}(k)  [ T_S(k,\tau_\ast) j_l(k \chi_\ast)]^2 
}
Result follows. 
\end{proof}

\begin{corollary}
On superhorizon scales, \hl{assuming} recombination occurs during matter domination (i.e. $\Phi_e = -\frac{3}{5} \mc{R}$ from eq. \ref{eq:CSM:10}, $\delta_r=\frac{4}{3}\delta_m = -\frac{8}{3}\Phi_e = \frac{8}{5} \mc{R}$from eq. \ref{eq:CSM:12}  ) gives
\eq{
S &= -\frac{1}{5} \mc{R} \\
\Rightarrow T_S &= -\frac{1}{5} \\
\Rightarrow C_l &= \frac{4\pi}{25}\int d(\log k) \, \Delta_\mc{R}(k)  [j_l(k \chi_\ast)]^2
}
\end{corollary}

\begin{corollary}
Taking the power spectrum to be scale invariant, i.e. $P\propto k^{-4+1} \Rightarrow \Delta \propto k^0$ gives 
\eq{
C_l \propto \int d(\log k) \, [j_l(k \chi_\ast)]^2 \propto \frac{2\pi}{2l(l+1)}
} 
\end{corollary}

The limitations of this result are 
\begin{itemize}
    \item A large scales / low multipoles, we see a contribution from the ISW term. This is as, because the potentials decay during dark energy domination, a photon blueshifts more when falling into a potential then it redshifts when climbing out. 
    \item Baryons mean that potential wells are deeper where there is an overdensity. The opposite does not occur. Hence compression peaks have larger amplitudes than rarefaction peaks\footnote{This gives a way to measure the baryon density.}. 
    \item Photons and baryons do not perfectly couple in a fluid, instead photons diffuse slowly, smoothing out oscillations on larger scales, causing the power spectrum to fall off at high multipoles. 
\end{itemize}

%%%%%%%%%%%%%%%%%%%%%%%%%%%%%%%%%%%%%%%%%%%%%%%%%%%%%%%%
%%%%%%%%%%%%%%%%%%%%%%%%%%%%%%%%%%%%%%%%%%%%%%%%%%%%%%%%
\section{Quantum Inflationary Origin}


%%%%%%%%%%%%%%%%%%%%%%%%%%%%%%%%%%%%%%%%%%%%%%%%%%%%%%%%
\subsection{Classic Evolution of the Inflaton Field}

By taking the standard action for the inflaton field and constructing the field equations for a general $\phi(\bm{x},t)= \bar{\phi}(t) + \delta \phi(\bm{x},t)$ gives 
\eq{
\ddot{\phi} + 3H \dot{\phi} - \frac{1}{a^2} \nabla^2 \phi + V^\prime(\phi) = 0
}
or in conformal time 
\eq{
\phi^{\prime\prime} + 2\mc{H} \phi^\prime - \nabla^2 \phi + a^2 V^\prime(\phi) = 0
}
Now define 
\eq{
f(\bm{x},\tau)= a(\tau) \delta \phi(\bm{x},\tau)
}

substituting in gives 
\eq{
f^{\prime\prime} - \nabla^2 f + \left( a^2  V^{\prime\prime} - \frac{a^{\prime\prime}}{a} \right) f =0 
}

\begin{prop}
In slow roll inflation, we have
\eq{
\highlight{a^2 V^{\prime\prime} \ll \frac{a^{\prime\prime}}{a}}
}
\end{prop}
\begin{proof}
\eq{
a^{\prime\prime} =(a^2 H)^\prime = 2aa^\prime H + a^2 H^\prime
}
During slow roll inflation, $H$ is approximately constant, so we can neglect the second term to say  
\eq{
\frac{a^{\prime\prime}}{a} \approx 2a^\prime H = 2a^2 H^2 
}
Now using equations \ref{eq:CSM:13} and \ref{eq:CSM:14} gives 
\eq{
\frac{V^{\prime\prime}}{H^2}  \approx \frac{3M_{pl}^2 V^{\prime\prime}}{V} =3 \eta_V \ll 1
}
\end{proof}

Hence we define 

\begin{definition}[Mukhanov Sasaki Equation]
The \bam{MSE} is 
\[
f ^ { \prime \prime } - \nabla ^ { 2 } f - \frac { a ^ { \prime \prime } } { a } f = 0
\]
In Fourier space this is 
\eq{
f_k^{\prime\prime} + \left( k^2 - \frac{a^{\prime\prime}}{a} \right)f_k = 0
}
\end{definition}

Note on subhorizon scales this reduces to \eq{
f_k^{\prime \prime} + k^2 f = 0 
}
the equation for a simple harmonic oscillator. 

\begin{aside}
Note that the same result can be derived by substituting for $f$ straight in the action initially. The actions is 
\eq{
S = \int d^4 x  \, g^{\frac{1}{2}} \left( \frac{1}{2}\del_\mu \phi \del^\mu \phi - V(\phi) \right)
}
and taking the conformally flat FLRW metrix $ds^2 = a(\tau)^2 [d\tau^2 + d\bm{x}^2]$ gives 
\eq{
S = \int d\tau \, d^3 x \, a^4 \left( \frac{1}{2} \frac{1}{a^2} \psquare{(\del_\tau \phi)^2  + |\grad \phi|^2 } - V \right)
}
Substituting in $\phi = \bar{\phi} + \frac{f}{a}$ and collecting the terms of order 2 in $f$ gives 
\eq{
S^{(2)} = \frac{1}{2}\int d\tau \, d^3x \, \frac{1}{2} \left[ (f^\prime)^2 - |\grad f|^2 - \frac{a^{\prime\prime}}{a} f^2 \right] 
}
Applying Euler-Lagrange to this new integrand rederives the Mukhanov-Sasaki equation. 
\end{aside}

%%%%%%%%%%%%%%%%%%%%%%%%%%%%%%%%%%%%%%%%%%%%%%%%%%%%%%%%
\subsection{Quantisation of the Inflaton Field}

Taking $H$ approximately constant, so $a(t) \approx e^{Ht}$ gives $a(\tau) = - \frac{1}{H\tau}$, so 
\eq{
f_k^{\prime\prime} + \left( k^2 - \frac{2}{\tau^2}\right)f_k = 0
}
The previously found action gives a conjugate momentum to $f$ 
\eq{
\pi(\bm{x},\tau) = \frac{\delta \mc{L}}{\delta f^\prime} = f^\prime(\bm{x},t)
}

Canonically quantise the function by promoting $ f \to \hat{f}, \pi \to \hat{\pi}$, operators, and require 
\eq{
\comm[\hat{f}(\bm{x},\tau)]{\hat{\pi}(\tilde{\bm{y}},\tau)} = i \delta^3(\bm{x}-\bm{y}) \\
\hat{f}(\tau,\mathbf{x})=\int\frac{\mathrm{d}^3k}{(2\pi)^\frac{3}{2}} \left[ f_k^\ast (\tau) \hat{a}_{\bm{k}}^\dagger e^{-i\bm{k} \cdot\bm{x}}+f_k(\tau) \hat{a}_{ \bm{k}} e^{i \bm{k}\cdot \bm{x}} \right] \\
\hat{\pi}(\tau,\mathbf{y})=\int\frac{\mathrm{d}^3k}{(2\pi)^\frac{3}{2}} \left[ [f_k^\ast (\tau)]^\prime \hat{a}_{\bm{k}}^\dagger e^{-i\bm{k} \cdot\bm{x}}+[f_k(\tau)]^\prime \hat{a}_{ \bm{k}} e^{i \bm{k}\cdot \bm{x}} \right] 
}
Then the conditions
\eq{
\comm[\hat{a}_{\bm{p}}]{\hat{a}_{\bm{q}}^\dagger} = \delta^3(\bm{p}-\bm{q}) \\
\comm[\hat{a}_{\bm{p}}]{\hat{a}_{\bm{q}}} = 0 =\comm[\hat{a}_{\bm{p}}^\dagger]{\hat{a}_{\bm{q}}^\dagger}
}
are equivalent to 
\eq{
iW(f_k,f_k^\prime) = f_k (f_k^\ast)^\prime - f_{-k}^\ast (f_{-k})^\prime = 1
}
Combining this with the Mukhanov Sasaki equation fully determines 
\eq{
f_k(\tau) = \frac{e^{-ik\tau}}{\sqrt{2k}} \left( 1 - \frac{i}{k\tau}\right)
}
the \bam{Bunch Davies mode function}. 


%%%%%%%%%%%%%%%%%%%%%%%%%%%%%%%%%%%%%%%%%%%%%%%%%%%%%%%%
\subsection{Evaluating Perturbations from Inflation}

To understand perturbations resulting from inflation, we want to calculate the two point correlation functions 
\eq{
&\xi(r) = \braket{0 | \delta \hat{\phi}(\bm{x},\tau) \delta \hat{\phi}(\bm{x}+\bm{r},\tau)  | 0} \\
= & \frac{1}{a^2} \int\int \frac{d^3 k}{(2\pi)^\frac{3}{2}} \frac{d^3 k^\prime}{(2\pi)^\frac{3}{2}} \braket{0 | \left[ f_k^\ast (\tau) \hat{a}_{\bm{k}}^\dagger e^{-i\bm{k} \cdot\bm{x}}+f_k(\tau) \hat{a}_{ \bm{k}} e^{i \bm{k}\cdot \bm{x}} \right] \left[ f_{k^\prime}^\ast (\tau) \hat{a}_{\bm{k}^\prime}^\dagger e^{-i\bm{k}^\prime \cdot(\bm{x}+\bm{r})}+f_{k^\prime}(\tau) \hat{a}_{ \bm{k}^\prime} e^{i \bm{k}^\prime \cdot (\bm{x}+\bm{r})} \right] | 0 } \\
= & \frac{1}{a^2} \int\int \frac{d^3 k \, d^3 k^\prime}{(2\pi)^3} f_k(\tau) f_{k^\prime}^\ast (\tau) e^{i \bm{k} \cdot \bm{x}} e^{-i \bm{k}^\prime \cdot (\bm{x} + \bm{r})} \braket{0 | \hat{a}_{\bm{k}} \hat{a}_{\bm{k}^\prime}^\dagger | 0} \\
= & \frac{1}{a^2} \int \frac{d^3 k }{(2\pi)^3} |f_k(\tau)|^2 e^{-i \bm{k} \cdot \bm{r}}
}
Now from \ref{eq:CSM:15} we know 
\eq{
\highlight{\Delta_{\delta \phi}}  &= \frac{k^3}{2\pi^2} P(k) \\
&= \frac{k^3}{2\pi^2} \frac{|f_k(\tau)|^2}{a^2} \\
&= \frac{k^3}{2\pi^2} \frac{1}{2ka^2} \left( 1 + \frac{1}{k^2 \tau^2} \right) \\
&= \left( \frac{H}{2\pi}\right)^2 \left( 1 + \frac{k^2}{a^2 H^2} \right)
}
using $\tau = - \frac{1}{aH}$ in inflation. 
\begin{remark}
For $k < aH$, i.e. just after horizon exit, 
\eq{
\Delta_{\delta \phi} \approx \left( \frac{H}{2\pi}\right)^2
}
so perturbations are approximately scale invariant, as during inflation $H$ is constant. 
\end{remark}

\begin{fact}
\hl{From Baumann's notes} 
\eq{
\mc{R} = C - \frac{1}{3} \nabla^2 E - \mc{H} \frac{\delta \rho}{\bar{\rho}}
}
and so in the spatially flat gauge, where $C = 0 =E$
\eq{
\mc{R}(\bm{x}) = - \mc{H} \frac{\delta \phi(\bm{x})}{\bar{\phi}^\prime} \\
\Rightarrow \Delta_\mc{R}^2 = \left( \frac{\mc{H}}{\bar{\phi}^\prime} \right)^2 \Delta_{\delta\phi}^2 
}
\end{fact}

Hence with the definition of the slow roll parameter and using proposition \ref{prop:CSM:1}
\eq{
\eps = \frac{\frac{1}{2} \dot{\phi}^2}{H^2 M_{pl}^2} = \frac{{\phi^\prime}^2}{2 \mc{H}^2 M_{pl}^2} 
}
so we have 
\eq{
\Delta_{\mc{R}}^2 = \frac{1}{2\eps M_{pl}^2} \left(\frac{H}{2\pi} \right)^2
}
If we instead parametrise the power spectrum as 
\eq{
\Delta_{\mc{R}}^2 = A \left(\frac{k}{k_0} \right)^{n_s -1}
}
in order to investigate near scale invariance, we get 
\eq{
n_s - 1 = \frac{d(\log \Delta_{\mc{R}}^2)}{d(\log k)} =\frac{d(\log \Delta_{\mc{R}}^2)}{dN} \frac{dN}{d(\log k)}
}
Taking the scale invariant approximation gives 
\eq{
\frac{d(\log \Delta_{\mc{R}}^2)}{dN} = 2 \frac{d(\log H)}{dN} - \frac{d(\log \eps)}{dN} = -2\eps - \eta
}
Further, as near $k = aH, \log k = N + \log H $
\eq{
\Rightarrow &\frac{dN}{d(\log k)} = \left( \frac{d(\log k)}{dN} \right)^{-1} = ( 1 + \eps)^{-1} \approx 1-\eps \\
&n_s - 1 \approx -2\eps - \eta 
}
to leading order, 


\end{document}