\documentclass{article}
\usepackage{header}
%%%%%%%%%%%%%%%%%%%%%%%%%%%%%%%%%%%%%%%%%%%%%%%%%%%%%%%%
%Preamble

\title{Cosmology Notes}
\author{Linden Disney-Hogg}
\date{January 2019}

%%%%%%%%%%%%%%%%%%%%%%%%%%%%%%%%%%%%%%%%%%%%%%%%%%%%%%%%
%%%%%%%%%%%%%%%%%%%%%%%%%%%%%%%%%%%%%%%%%%%%%%%%%%%%%%%%
\begin{document}

\maketitle
\tableofcontents

\section{Introduction}



\section{CMB}
\begin{idea}
The CMB is "pretty much" linear, which makes it nice to work with. 
\end{idea}


\begin{definition}[Superhorizon]
A CMB mode it \bam{superhorizon} if $k\ll aH$
\end{definition}


\begin{remark}
In cosmology, Bessel functions like to turn up (why?). Hence it can sometimes be useful to try turn an euqaiotn into the form of Bessel's equation 
\[
\frac{d^2y}{dx^2} + \frac{1}{x} \frac{dy}{dx} + \left( 1-\frac{\nu^2}{x^2} \right ) = 0 
\]
\end{remark}



\section{Quantum Inflationary Origin}

\begin{definition}[Mukhanov Sasaki Equation]
The \bam{MSE} is 
\[
f ^ { \prime \prime } - \nabla ^ { 2 } f - \frac { a ^ { \prime \prime } } { a } f = 0
\]
\end{definition}

\end{document}