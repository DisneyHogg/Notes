\documentclass{article}
\usepackage{header}
%%%%%%%%%%%%%%%%%%%%%%%%%%%%%%%%%%%%%%%%%%%%%%%%%%%%%%%%
%Preamble

\title{Cosmology Notes}
\author{Linden Disney-Hogg}
\date{January 2019}

%%%%%%%%%%%%%%%%%%%%%%%%%%%%%%%%%%%%%%%%%%%%%%%%%%%%%%%%
%%%%%%%%%%%%%%%%%%%%%%%%%%%%%%%%%%%%%%%%%%%%%%%%%%%%%%%%
\begin{document}

\maketitle
\tableofcontents

\section{Introduction}
These are notes, hopefully condensed for more efficient revision of cosmology. Highlighted equations in math environment (in red) indicates a possible error that needs to be reviewed. Text that is highlighted (in yellow) indicates an assumption/ statement assumed from farther physics. 
%%%%%%%%%%%%%%%%%%%%%%%%%%%%%%%%%%%%%%%%%%%%%%%%%%%%%%%%
%%%%%%%%%%%%%%%%%%%%%%%%%%%%%%%%%%%%%%%%%%%%%%%%%%%%%%%%

\section{Conventions and prerequisites}

\begin{definition}[Natural Units]
\bam{Natural units} are used where
\[
c=\hbar=k_B = 1
\]
$[c]=LT^{-1}$, $[\hbar]=L^2 M T^{-1}$, $[k_B] = ML^2 T^{-2} K^{-1}$, so in natural units
\[
L=T=K^{-1}=M^{-1}
\]
Units are therefore given in \bam{mass dimension}, e.g. if $[f]=M^d$, write $[f]=d$.
\end{definition}

\begin{fact}
A table of common values and their mass dimension is given below. 
\begin{center}$
\begin{array}{ccc}
    \text{Quantity} & \text{Symbol} & \text{Mass Dimension} \\
    \hline
    \hline
    \text{Length/Distance} & L/D & -1 \\
    \text{Volume} & V & -3 \\
    \text{Measure} & d^n x & -n \\
    \text{Derivative} & \del_\mu & 1 \\
    \text{Time} & t & -1 \\
    \text{Temperature} & T & 1 \\
    \text{Energy} & E & 1 \\
    \text{Energy Density} & \rho & 4 \\
    \text{Momentum} & P/q & 1 \\
    \text{Pressure} & p & 4 \\
    \text{Entropy} & S & 0 \\
    \text{Entropy Density} & s & 3 \\
    \text{Scale Factor} & a & 0 \\
    \text{Hubble Rate} & H & 1 \\
    \text{Curvature parameter} & K & 2 \\
    \text{Newton's Constant} & G & -2 \\
    \text{Metric} & g_{ab},g^{ab} & 0 \\
    \text{Action} & S & 0 \\
    \text{Lagrangian Density} & \mc{L} & 4 \\
    \text{Christoffel Symbol} & \Gamma^a_{bc} & 1 \\
    \text{Riemann Tensor} & R\indices{^a_b_c_d} & 2 \\
    \text{Ricci Tensor/Scalar} & R_{ab} / R & 2 \\
    \text{Cosmological Constant} & \Lambda & 2 \\
    \text{Inflaton Field} & \phi & 0 
\end{array}
$\end{center}
\end{fact}

\begin{definition}[Plank Mass]
The \bam{Plank mass} is 
\eq{
M_{pl} = (8 \pi G)^{-\frac{1}{2}}
}
\end{definition}

\begin{definition}[Mostly plus signature]
The signature that will be used throughout these notes will be the \bam{mostly plus signature}. As these notes will only consider 1+3 dimensional spacetime, as these are all the dimensions that have been observed, this take the form $(-,+,+,+)$. 
\end{definition}
%%%%%%%%%%%%%%%%%%%%%%%%%%%%%%%%%%%%%%%%%%%%%%%%%%%%%%%%
%%%%%%%%%%%%%%%%%%%%%%%%%%%%%%%%%%%%%%%%%%%%%%%%%%%%%%%%
\section{General Relativity}
Some of the tools from General Relativity will be useful in discussions of cosmology. For a better idea of these results, see the GR notes, and these will be referenced where appropriate. 

\begin{definition}[Energy Momentum Tensor]
Given an action $S$ with respect to a metric $g_{ab}$ the \bam{energy momentum tensor} is 
\eq{
T^{ab} = \frac{2}{\sqrt{-g}} \frac{\delta S_{matter}}{\delta g_{ab}}
}
where $g=\det g_{ab}$ and 
\eq{
S_{matter} = \int \mc{L}_{matter} g^{\frac{1}{2}} d^4 x
}
is the matter action. 
\end{definition}

\begin{definition}[Perfect Fluid]
A \bam{perfect fluid} is a medium in which at each point there exist a Local Inertial Frame (LIF). 
\end{definition}

\begin{fact}
In a comoving LIF the energy momentum tensor must be diagonal and isotropic hence 
\eq{
T^\mu_\nu = \diag(-\rho,P,P,P)
}
Hence for a general frame moving with velocity $u^\mu$ s.t $u^\mu u_\mu = - 1$ 
\eq{
T^{\mu\nu} =(\rho+p) u^\mu u^\nu +g^{\mu\nu}p
}
\end{fact}


\begin{definition}[Lie Derivative]
If under an infinitesimal transform $x^\mu \to x^\mu + \eps V^\mu$ a tensor transform as $T \to T^\prime$ then the \bam{Lie Derivative} of $T$ in the $V$ direction is 
\eq{
(\mc{L}_V T)(x) = \lim_{\eps\to 0} \frac{T\indices{^\dots_\dots}(x) - {T^\prime}\indices{^\dots_\dots}(x)}{\eps}
}
Some examples are 
\eq{
\mc{L}_V \phi &= V^\mu \del_\mu \phi \\
\mc{L}_V W^\mu &= V^\nu \nabla_\nu W^\mu - W^\nu \nabla_\nu V^\mu \\
\mc{L}_V W_\mu &= V^\nu \nabla_\nu W^\mu + W_\nu \nabla_\mu V^\nu \\ 
\mc{L}_V T_{\mu\nu} &= V^\rho \nabla_\rho T_{\mu\nu} + T_{\rho\nu} \nabla_\mu V^\rho + T_{\mu\rho} \nabla_\nu V^\rho
}
The most important example is for $g_{\mu\nu}$ the metric tensor where
\eq{
\mc{L}_V g_{\mu\nu} = \nabla_\mu V_\nu + \nabla_\nu V_\mu
}
\end{definition}

\begin{definition}[Isometry]
A transform $x \to x^\prime$ is an \bam{isometry} if under the transform 
\eq{
g_{\mu\nu} \to g^\prime_{\mu\nu} \; g^\prime_{\mu\nu}(x) = g_{\mu\nu}(x)
}
\end{definition}

\begin{definition}[Killing Vector]
A vector $\xi^\mu$ is a \bam{Killing vector} if 
\eq{
\mc{L}_\xi g_{\mu\nu} = \nabla_\mu \xi_\nu + \nabla_\nu \xi_\mu =0
}
Motion along integral curves of $\xi$ are then isometries. 
\end{definition}

\begin{prop}
For a Killing vector $\xi$
\eq{
\nabla_\rho \nabla_\sigma \xi_\mu = R\indices{_\lambda_\sigma_\mu_\rho} \xi^\lambda
}
\end{prop}
\begin{corollary}
A Killing vector field  is determined uniquely by specifying $\xi_\mu ,\nabla_\nu \xi^\mu$ at a point. Hence a $D$ dimensional spacetime has at most 
\[
D + \frac{1}{2}D(D-1) = \frac{1}{2}D(D+1)
\]
isometries. 
\end{corollary}

\begin{definition}[Maximally Symmetric Spactime]
A $D$ dimensional spacetime is \bam{maximally symmetric} if it admits $\frac{1}{2}D(D+1)$ independent Killing vectors. 
\end{definition}

\begin{prop}
In maximally symmetric spacetime the Riemann tensor is 
\eq{
R_{\mu\nu\rho\sigma} = K g_{\mu(\sigma}g_{\nu\rho)}
R = -D(D-1) K 
}
Hence the Ricci scalar uniquely characterises a maximally symmetric spacetime. 
\end{prop}

\begin{fact}
If $N\leq M$ is a maximally symmetric subspace then the line element on $M$ can be expressed as the \bam{\hl{warp product}}
\eq{
ds^2 = g_{ab}(x) dx^a dx^b + f(x) \tilde{g}_{ij}(y) dy^i dy^j
}
where $y$ are coordinate on $N$ and $x$ coordinates on the rest of $M\setminus N$
\end{fact}

\begin{definition}[Einstein-Hilbert Action]\label{def:CSM:EinsteinHilbertAction}
The \bam{Einstein - Hilbert action} is 
\eq{
S_{EH} = \frac{1}{2}M_{pl}^2 \int (R-2\Lambda) g^{\frac{1}{2}} d^4 x
}
\end{definition}

\begin{definition}[Einstein Equations]
The \bam{Einstein Equations} are a set of 10 non-linear PDEs for the metric in the presence of some energy momentume tensor, and they are 
\eq{
M_{pl}^2 ( R_{ab} - \frac{1}{2}R g_{ab} + \Lambda g_{ab}) = T_{ab}
}
$\Lambda$ is a cosmological constant, and is ignored typically in general relativity. 
\end{definition}

\begin{prop}[Trace Reversed Einstein Equations]
In $D>2$ dimensions this can be written as 
\eq{
M_{pl}^2( R - \frac{D}{2}R + D\Lambda) = T
}
with $T=T\indices{^a_a}$. Hence 
\eq{
R &= \frac{2}{D-2}\left[ D\Lambda - M_{pl}^{-2} T \right] \\
\Rightarrow R_{ab} &=  \frac{1}{D-2}\left[ D\Lambda - M_{pl}^{-2} T \right] g_{ab} -\Lambda g_{ab} + M_{pl}^{-2}T_{ab} \\
&= \frac{2\Lambda}{D-2}g_{ab} + M_{pl}^{-2}\left[ T_{ab} - \frac{1}{D-2} T g_{ab} \right]
}
in $D=4$ this becomes 
\eq{
R_{ab} = \Lambda g_{ab} + M_{pl}^{-2} \left[ T_{ab} -\frac{1}{2}Tg_{ab} \right] 
}
\end{prop}

\begin{theorem}
The Einstein equations are obtained from finding when the action 
\eq{
S = S_{EH} + S_{Matter}
}
is stationary, i.e. $\delta S = 0$ with respect to variations in the metric. 
\end{theorem}

\begin{prop}[Contracted Bianchi Identity]
\eq{
\nabla^a(R_{ab} -  \frac{1}{2}R g_{ab} ) = \nabla^a R_{ab} - \frac{1}{2} \nabla_b R = 0
}
\end{prop}
\begin{corollary}
The energy momentum tensor is conserved, i.e. 
\eq{
\nabla_a T^{ab} = 0 
}
\end{corollary}
%%%%%%%%%%%%%%%%%%%%%%%%%%%%%%%%%%%%%%%%%%%%%%%%%%%%%%%%
\subsection{de Sitter Spacetime}

\begin{definition}[de Sitter Spacetime]
Consider $\mbb{R}^{1,4}$ with metric 
\eq{
ds^2 = -dV^2 +dW^2 + dX^2 + dY^2 + dZ^2 
}
and the metric induced on the surface 
\eq{
-V^2 + W^2 + X^2 + Y^2+ Z^2 = \frac{3}{\Lambda}
}
Any spacetime that has this induced metric is \bam{de Sitter spacetime}.
\end{definition}

\begin{prop}
The metric on de Sitter spacetime can be written as 
\eq{
ds^2 = - d\tau^2 + \frac{3}{\Lambda}\left[d\chi^2 + \sin^2 \chi \left( d\theta^2 + \sin^2\theta d\phi^2 \right) \right]
}
with 
\eq{
V &= \sqrt{\frac{3}{\Lambda}}\sinh \left( \tau\sqrt{\frac{3}{\Lambda}} \right) \\
W &= \sqrt{\frac{3}{\Lambda}}\cosh \left( \tau\sqrt{\frac{3}{\Lambda}} \right)\cos\chi \\
X &= \sqrt{\frac{3}{\Lambda}}\cosh \left( \tau\sqrt{\frac{3}{\Lambda}} \right)\sin\chi \cos\theta \\
Y &= \sqrt{\frac{3}{\Lambda}}\cosh \left( \tau\sqrt{\frac{3}{\Lambda}} \right)\sin\chi \sin\theta \cos\phi \\
Z  &= -\sqrt{\frac{3}{\Lambda}}\cosh \left( \tau\sqrt{\frac{3}{\Lambda}} \right)\sin\chi \sin\theta \sin\phi
}
\end{prop}
\begin{proof}
Note that for constant $V$ 
\eq{
W^2 + X^2 + Y^2+ Z^2 = \frac{3}{\Lambda} + V^2
}
is a 3-sphere, hence has the metric 
\eq{
d\sigma^2 = d\chi^2 + \sin^2 \chi \left( d\theta^2 + \sin^2\theta d\phi^2 \right)
}
with coordinates 
\eq{
W &= k\cos\chi \\
X &= k\sin\chi \cos\theta \\
Y &= k\sin\chi \sin\theta \cos\phi \\
Z  &= -k\sin\chi \sin\theta \sin\phi
}
for $k=\sqrt{\frac{3}{\Lambda}+V^2}$. Hence letting $V=\sqrt{3}{\Lambda}\sinh v$. Choosing v to normalise $dV$ in the metric gives $v=\tau\sqrt{\frac{3}{\Lambda}}$. 
\end{proof}

\begin{prop}
Alternative coordinates can be found on de Sitter space such that the metric becomes 
\eq{
ds^2 &= -dt^2 + e^{2H_0 t} d\bm{x}^2 \\
 &= \frac{-d\tau^2 + d\bm{x}^2}{\tau^2 H_0^2}
}
where $H_0 = \sqrt{\frac{\Lambda}{3}}$ in a universe with $w=-1$. 
\end{prop}

%%%%%%%%%%%%%%%%%%%%%%%%%%%%%%%%%%%%%%%%%%%%%%%%%%%%%%%%
%%%%%%%%%%%%%%%%%%%%%%%%%%%%%%%%%%%%%%%%%%%%%%%%%%%%%%%%
\section{FLRW}
%%%%%%%%%%%%%%%%%%%%%%%%%%%%%%%%%%%%%%%%%%%%%%%%%%%%%%%%
\subsection{Metric}
On large scales, at any given time the universe looks \emph{homogeneous} and \emph{isotropic}, and hence it has a maximally symmetric constant time hypersurface. As a result the line element is 
\eq{
ds^2 = dt^2 + a(t)^2 \tilde{g}_{ij}(x) dx^i dx^j
}
Note $x$ are \emph{comoving} coordinates such that, for a spacelike separation $\Delta x$
\eq{
|\Delta x|_{phys} = a |\Delta x|
}

\begin{definition}[Hubble Parameter]
Define the \bam{Hubble parameter} $H$
\eq{
H= \frac{\dot{a}}{a}
}
\end{definition}

\begin{definition}[FLRW metric]
The metric induced on the homogeneous isotropic spacetime is the \bam{Friedmann-Lemaitre-Robertson-Walker Metric} 
\eq{
ds^2 &= - dt^2 + a(t)^2 \left[ \frac{dr^2}{1-Kr^2} + r^2 d\Omega^2 \right] \\
 &= - dt^2 + a(t)^2 \left[ d\chi^2 + f(\chi) d\Omega^2 \right]
}
where $K = 0, \pm1$, and 
\eq{
f(\chi) = \left\{ \begin{array}{lc} \sinh^2 \chi  & K = -1 \\
    \chi^2 & K=0 \\
    0\sin^2 \chi & K=1
    \end{array} \right.
}
\end{definition}

\begin{definition}[Conformal time]
Introduce \bam{conformal time} $\tau$ which satisfies 
\[
d\tau = \frac{dt}{a(t)} \Leftrightarrow \tau = \int^t \frac{dt^\prime}{a(t^\prime)}
\]
\end{definition}

In conformal time the FLRW metric becomes 
\eq{
ds^2 = a^2 [ -d\tau^2 + d\chi^2 + f(\chi) d\Omega^2 ]
}

%%%%%%%%%%%%%%%%%%%%%%%%%%%%%%%%%%%%%%%%%%%%%%%%%%%%%%%%
\subsection{Evolution equations}

\begin{definition}[Continuity Equation]
The conservation of the energy momentum tensor implies 
\eq{
\dot{\rho} + 3H(\rho + p) = 0
}
\end{definition}

\begin{definition}[Equation of State]
The \bam{equation of state} is a relation $p=p(\rho)$. In this course the equation of state will be taken to be 
\eq{
p = w\rho
}
where the constant $w$ depends on the fluid.
\end{definition}

\begin{prop}
If the equation of state is $p = w\rho$ then 
\eq{
\dot{\rho} + 3 \frac{\dot{a}}{a}(1+w) \rho = 0 \Rightarrow \rho \propto a^{-3(1+w)}
}
\end{prop}

The examples encountered in this course are 
\begin{center}$
\begin{array}{ccc}
    \text{Fluid} & w & \text{Relation} \\
    \hline
    \hline
    \text{Non-relativistic matter (dust)} & 0 & \rho\propto a^{-3}  \\
    \text{Radiation} & \frac{1}{3} & \rho\propto a^{-4} \\
    \text{Dark energy (cosmological constant)} & -1 & \rho \text{ constant} \\
\end{array}
$\end{center}

\begin{definition}[Friedmann equation]
In the FLRW metric the Einstein equations reduce to 
\eq{
3M_{pl}^2 ( H^2 + \frac{K}{a^2} ) = \sum_i \rho_i
}
this is the \bam{Friedmann equation}. 
\end{definition}

\begin{definition}[Raychaudhuri equation]
Differentiating and rearranging the Friedmann equation given the \bam{Raychaudhuri equation}
\eq{
\frac{\ddot{a}}{a} = \frac{-1}{6M_{pl}^2} (\rho + 3P)
}
\end{definition}

\begin{definition}[Critical Density]
Define the \bam{critical density} to be 
\eq{
\rho_c = 3 M_{pl}^2 H^2
}
writing 
\eq{
\Omega_i &= \frac{\rho_i}{\rho_c} \\
\Omega_K &= -\frac{3M_{pl}^2 K}{\rho_c a^2} = -\frac{K}{H^2 a^2}
}
the Friedmann equation becomes 
\eq{
1-\Omega_K = \sum_i \Omega_i
}
It is common notation to let $\Omega_{i,0} = \Omega_i |_{t=t_0}$, and use $\Omega_{i,0} h^2$ where $h$ is defined by 
\eq{
H_0 = 100 h \frac{\text{km}}{\text{sec Mpc}}
} 
in order to absorb error in the measurement of $H_0$. 
\end{definition}

%%%%%%%%%%%%%%%%%%%%%%%%%%%%%%%%%%%%%%%%%%%%%%%%%%%%%%%%
\subsection{Distance}\footnote{See this link \href{http://article.sciencepublishinggroup.com/pdf/10.11648.j.ijass.20150304.13.pdf}{http://article.sciencepublishinggroup.com/pdf/10.11648.j.ijass.20150304.13.pdf} for more information. }

\begin{definition}[Redshift]
Suppose that a wave is emitted with wavelength $\lambda_e$ and observed with wavelength $ \lambda_o$. Then the \bam{redshift} $z$ is defined by 
\eq{
1+z = \frac{\lambda_o}{\lambda_e}
}
\end{definition}

\begin{definition}[Cosmological redshift]
For a photon emitted at time $t_e < t_0$ the redshift due to the expansion of the universe is the \bam{cosmological redshift}
\eq{
1+z = \frac{a_0}{a_e}
}
If we let $a_0=1$, then we get $1+z = \frac{1}{a(z)}$
\end{definition}

\begin{definition}[Comoving Distance]
The \bam{comoving distance} $\chi(t_i,t_f)$ is the distance travelled by a photon between times $t_i,t_f$. Note that for a radially moving null geodesic $ds^2 = a^2[-d\tau^2 + d\chi^2] = 0$ so $d\chi=d\tau$, so 
\eq{
\chi(t_i,t_f) &= \int d\tau \\
&= \int_{t_i}^{t_f} \frac{dt}{a} \\
&= \int_{a_i}^{a_f} \frac{da}{a^2 H} \\
&= \int_{z_f}^{z_i} \frac{dz}{H(z)}
}
Writing this then in terms of the redshift, then the distance from the earth to an object with redshit $z$ is   
\eq{
\chi(z) = \int_0^z \frac{dz}{H(z)}
}
\end{definition}

\begin{definition}[Luminosity]
The \bam{intrinsic luminosity} $L$ of an object is the amount of energy radiated per unit time. The \bam{observed luminosity} $l$ is the amount of energy radiated per unit time per unit surface arriving at an observer some distance from the object.
\end{definition}

\begin{definition}[Luminosity Distance]
The observed and instrinsic luminosity are related by 
\eq{
l = \frac{L}{4\pi d_L^2}
}
which defines the $\bam{luminosity distance}$ $d_L$
 \end{definition}
 
 \begin{prop}
 The luminosity distance and comoving distance are related by 
 \eq{
 d_L(z) = (1+z) a_o \chi(z)
 }
 \end{prop}
\begin{proof}
\eq{
l = \frac{L}{4\pi (\chi a_o)^2}\left(frac{a_e}{a_o}\right)^2
}
\end{proof}

\begin{definition}[Angular Diameter Distance]
For an object of size $s$ which subtends an angle $\theta$ from the point of view of the observer, the \bam{angular diameter distance} $d_A$ to the object is 
\eq{
d_A = \frac{s}{\theta}
}
As the angle the object takes up in the sky will be stretched by a factor of (1+z), the angular diameter distance is related to the comoving distance by 
\eq{
d_A(z) = \frac{\chi(z)}{1+z}
}
\end{definition}

\begin{definition}[Particle Horizon]
The \bam{particle horizon} $d_{p.h.}$ is is the greatest distance a photon a could have travelled since the beginning of time $t_i$. Any object a distance $d>d_{p.h.}$ from an observer away can never have communicated with said observer. 
\eq{
d_{p.h.}(t) = a(t) \chi(t,t_i) = a(t) \int_{t_i}^t \frac{dt^\prime}{a(t^\prime)}
}
\end{definition}

\begin{definition}[Hubble Radius]
The \bam{Hubble radius} is 
\eq{
r_H = \frac{1}{H}
}
\end{definition}

\begin{definition}[e-folds]
If, during expansion $a \to e^N a$, is it said that \bam{N e-fold of expansion have occured}. $N$ can we written as a variable 
\eq{
dN= Hdt = d(\log a) \Leftrightarrow N = \log a + \text{ constant}
}
\end{definition}

%%%%%%%%%%%%%%%%%%%%%%%%%%%%%%%%%%%%%%%%%%%%%%%%%%%%%%%%
\subsection{Possible Universes}

\begin{example}
Consider a flat $(K=0)$ universe, with only one contribution to the energy density. Then 
\eq{
3M_{pl}^2 \left(\frac{\dot{a}}{a}\right)^2 &= \rho_0 \left(\frac{a}{a_0}\right)^{-3(1+w)} \\
\Rightarrow \frac{\dot{a}}{a} &\propto a^{-\frac{3(1+w)}{2}} \\
\frac{da}{a} a^{\frac{3(1+w)}{2}} &= dt \\
\Rightarrow a(t) &= \left\{ \begin{array}{cc} \left[ \frac{3}{2}(1+w) H_0 t \right]^{\frac{2}{3(1+w)}} & w\neq -1 \\ Ae^{H_0 t} & w = -1 \end{array} \right.
}
Hence \begin{itemize}
    \item For non-relativistic matter, $a \propto t^{\frac{2}{3}}$
    \item For radiation, $a \propto t^\frac{1}{2}$
    \item For dark energy, $a \propto e^{H_0 t}$
\end{itemize}

In this case 
\eq{
H &= \left\{ \begin{array}{cc}  \frac{2}{3(1+w)}\frac{1}{t}  & w\neq -1 \\ H_0 & w = -1 \end{array} \right. \\
}
For $w\neq -1$ this can be written as 
\eq{
H(a) &= H_0 \left(\frac{a_0}{a}\right)^\frac{3(1+w)}{2} \\
H(z) &= H_0 (1+z)^\frac{3(1+w)}{2}
}
\end{example}

\begin{prop}
If $K=0$, the age of the universe can be calculates as 
\eq{
t_{age} &= \int dt \\
 &= \int \frac{da}{\dot{a}} \\
 &= \int \frac{da}{a H} \\
 &= \frac{1}{H_0} \int \frac{da}{a} \left[  \frac{\sum_i \rho_i}{3 M_{pl}^2 H_0^2} \right]^{-\frac{1}{2}} \\
 &= \frac{1}{H_0} \int \frac{da}{a} \left[ \Omega_\Lambda + \Omega_{m,0}a^{-3} + \Omega_{r,0} a^{-4} \right]^{-\frac{1}{2}}
}
\end{prop}

\begin{example}[Curvature problem]\label{example:CSM:curvature problem}
\eq{
\dot{\Omega}_K = -\frac{2\ddot{a}}{\dot{a}} \Omega_K
}
Hence the behaviour of $|\Omega_K|$ depends on 
\eq{
\frac{\ddot{a}}{\dot{a}} = - \frac{1}{6M_{pl}^2}(1+3w) \frac{\rho}{H}
}
So, in an expanding universe 
\eq{
 1 + 3w > 0 \Leftrightarrow |\Omega_K| \text{ grows} \\
 1 + 3w < 0 \Leftrightarrow |\Omega_K| \text{ shrinks}
}
and hence during radiation and matter domination $|\Omega_K|$ grows. Observation puts 
\eq{
\Omega_{K,0} = 0.000 \pm, 0.005
}
This suggests a very large fine tuning of the initial value of $\Omega_K$
\end{example}

\begin{example}[Horizon Problem]\label{example:CSM:horizon problem}
In a single component universe with $w\in(0,\frac{1}{3})$ we find 
\eq{
\chi(z) &= \frac{1}{H_0} \int_0^z (1+z)^{-\frac{3(1+w)}{2}} \\
&= \frac{1}{H_0}\left[ -\frac{2}{1+3w} (1+z)^{-\frac{1+3w}{2}}\right]_0^z \\
&= \frac{1}{H_0}\frac{2}{1+3w}\left[1-(1+z)^{-\frac{1+3w}{2}}\right]
}
For $z\gg1$ we thus have 
\eq{
\chi(z) \approx \frac{\mc{O}(1)}{H_0}
}
Alternatively we can calculated that particle horizon from the beginning of the universe to redshift $z$ as 
\eq{
x_{p.h.}(z) &= \frac{1}{H_0} \int_z^\infty (1+z)^{-\frac{3(1+w)}{2}} \\
&=  \frac{1}{H_0}\left[ -\frac{2}{1+3w} (1+z)^{-\frac{1+3w}{2}}\right]_z^\infty \\
&= \frac{1}{H_0}\frac{2}{1+3w} (1+z)^{-\frac{1+3w}{2}} \\
&= \frac{2}{1+3w} \frac{1+z}{H(z)}
}
Hence for large $z$
\eq{
\frac{\chi(z)}{x_{p.h.}(z)} \approx (1+z)^{\frac{1+3w}{2}} \gg 1
}
Thus antipodal regions with large $z$ must be causally disconnected, as their comoving distance is much greater than the particle horizon. As a result, there is no explanation for the statistical isotropy observed in the universe. 
\end{example}
%%%%%%%%%%%%%%%%%%%%%%%%%%%%%%%%%%%%%%%%%%%%%%%%%%%%%%%%
%%%%%%%%%%%%%%%%%%%%%%%%%%%%%%%%%%%%%%%%%%%%%%%%%%%%%%%%
\section{Constituents of the Universe}
The five main components of the universe are 
\begin{itemize}
    \item Photons
    \item Baryons (Protons, Neutrons, and other 3 quark species)
    \item Neutrinos
    \item Dark Matter 
    \item Dark Energy
\end{itemize}
For a particle species to have sizable density on its own today, that species must have a lifetime comparable to the lifetime of the universe. This includes photons, protons, and electrons. Species such as neutrons have retain their density by forming stable nuclei. 

%%%%%%%%%%%%%%%%%%%%%%%%%%%%%%%%%%%%%%%%%%%%%%%%%%%%%%%%
\subsection{Thermodynamics}
\begin{definition}
Neglecting the chemical potential, the 1st law of thermodynamics is 
\eq{
TdS = dE + pdV 
}
where $E$ is the energy of the system, $T$ the temperature, $p$ pressure and $V$ volume. 
\end{definition}

\begin{prop}
Letting $\rho$ be the energy density
\eq{
S = \frac{(\rho+p)V}{T} \Leftrightarrow s = \frac{\rho+p}{T}
}
where $s=\frac{S}{V}$ is the \bam{entropy density}.
\end{prop}
\begin{proof}
From the first law
\eq{
T dS &= d(\rho V) + pdV \\
&= V d\rho + (\rho + p) dV
}
If the situation is in equilibrium, $\rho=\rho(T), p=p(T)$, so 
\begin{align} \label{eq:CSM:3}
TdS = V \frac{d\rho}{dT} dT + (\rho + p) dV
\end{align}
and from the symmetry of partial derivatives 
\eq{
\pd{v} \pd[S]{T} &= \pd{T} \pd[S]{V} \\
\Rightarrow \pd{V} \left( \frac{V}{T} \frac{d\rho}{dT} \right) &= \pd{T} \left(\frac{\rho+p}{T} \right) \\
\Rightarrow \frac{1}{T} \frac{d\rho}{dT} &= - \frac{\rho+p}{T^2} + \frac{1}{T} \left( \frac{d\rho}{dT} + \frac{dp}{dT} \right) \\
\Rightarrow \frac{dp}{dT} = \frac{\rho+p}{T}
}
Now using \ref{eq:CSM:3} 
\eq{
dS &= \frac{1}{T} d[(\rho+p)V] - \frac{V}{T} dp \\
&= \frac{1}{T} d[(\rho+p)V] - \frac{\rho+p}{T^2}dT \\
&= d\left[ \frac{(\rho+p)V}{T} \right] \\
\Rightarrow S = \frac{(\rho+p)V}{T} \\
\Rightarrow s = \frac{S}{V} = \frac{\rho+p}{T}
}
\end{proof}

\begin{definition}[Bose-Einstein/Fermi-Dirac statistics]
For a species of particles $a$ with comoving momentum $P$, then the energy momentum tensor and number density are 
\eq{
T^{\mu\nu}_a ( \bm{x},t) &= \frac{2}{\sqrt{-g}} g_a \int \frac{d^3\bm{P
}}{(2\pi)^3 2E_{\bm{P}}} P^\mu P^\nu f(\bm{x},\bm{P},t) \\
n_a ( \bm{x},t) &= \frac{2}{\sqrt{-g}} g_a \int \frac{d^3\bm{P
}}{(2\pi)^3 2E_{\bm{P}}}  f(\bm{x},\bm{P},t)
}
where 
\begin{itemize}
    \item $g_a$ is the degeneracy of the species, e.g. 2 for a photon
    \item $E_{bm{P}} = \sqrt{m^2 + |\bm{P}|^2}$
\end{itemize}
and 
\eq{
f_{BE/FD}(\bm{x},\bm{P},t) = \frac{1}{e^{\frac{E_{\bm{P}}-\mu}{T}}\mp1}
}
is the phase space density for bosons (\bam{Bose-Einstein statistics}) and fermions (\bam{Fermi-Dirac statistics}) respectively. 
\end{definition}

\begin{prop}
In flat FLRW space the components of the energy momentum tensor and number density take the form 
\eq{
\rho_a &= \frac{g_a}{2\pi^2} \int_0^\infty dq \, q^2 \frac{E(q)}{e^{\frac{E(q)-\mu}{T}}\mp1} \\
p_a &= \frac{g_a}{2\pi^2} \int_0^\infty dq  \, q^2 \frac{q^2}{3E(q)} \frac{1}{e^{\frac{E(q)-\mu}{T}}\mp1} \\
n_a &= \frac{g_a}{2\pi^2} \int_0^\infty dq \, q^2 \frac{1}{e^{\frac{E(q)-\mu}{T}}\mp1}
}
\end{prop}
\begin{proof}
In flat FLRW $\sqrt{-g} = a^3$ and $d^3\bm{P} = a^3 d^3\bm{q}$ for physical momentum $\bm{q}$. Then $\rho = T^0_0$ and $p=\frac{1}{3}T^i_i$. Then finally writing $d^3\bm{q}=q^2 dq d\Omega$ and completing the angular integral for a factor of $4\pi$ gets the answer. 
\end{proof}
%%%%%%%%%%%%%%%%%%%%%%%%%%%%%%%%%%%%%%%%%%%%%%%%%%%%%%%%
Suppose $T \ll E - \mu $, then 
\begin{align}\label{eq:CSM:4}
q \frac{1}{e^\frac{E(q)-\mu}{T}\mp1} \approx  e^{-\frac{E(q)}{T}}e^{\frac{\mu}{T}} \ll 1 
\end{align}
So the integrals will be supported mainly where 
\eq{
& 0= \frac{d}{dq} q^2 e^{-\frac{\sqrt{m^2 + q^2}}{T}} = 2q e^{-\frac{\sqrt{m^2 + q^2}}{T}} - \frac{q^3}{T\sqrt{m^2+q^2}}e^{-\frac{\sqrt{m^2 + q^2}}{T}}\\
 \Rightarrow & 2-\frac{q^2}{T\sqrt{m^2+q^2}} = 0 \\
  \Rightarrow & q^4 -4T^2q^2-4T^2m^2 = 0 \\
 \Rightarrow & \left\{ \begin{array}{cc} \left(\frac{q^2}{4T^2}\right)^2 - \frac{q^2}{4T^2} \approx 0 \Rightarrow q\approx 2T & m \ll T \\ \left( \frac{q^2}{m} \right)^2 - 4T^2 \approx 0 \Rightarrow q\approx \sqrt{2mT} & m\gg T \end{array} \right.
}

Hence 
\eq{
\sqrt{m^2+q^2} \approx \left\{ \begin{array}{cc} q & m \ll T \\ m+\frac{q^2}{2m} & m \gg T \end{array} \right.
}
%%%%%%%%%%%%%%%%%%%%%%%%%%%%%%%%%%%%%%%%%%%%%%%%%%%%%%%%
\subsubsection*{Relativistic Particles}
For relativistic particles, the expressions thus are 
\eq{
\rho_a &\approx \frac{g_a}{2\pi^2} \int_0^\infty dq \,  \frac{q^3}{e^{\frac{q}{T}}\mp1} \\
p_a &\approx \frac{1}{3}\frac{g_a}{2\pi^2} \int_0^\infty dq  \,   \frac{q^3}{e^{\frac{q}{T}}\mp1} \\
n_a &\approx \frac{g_a}{2\pi^2} \int_0^\infty dq \, \frac{q^2}{e^{\frac{q}{T}}\mp1}
}
making the substitution $y = \frac{q}{T}$ gives 
\eq{
\rho_a &\approx \frac{g_a T^4}{2\pi^2} \int_0^\infty dy \,  \frac{y^3}{e^{y}\mp1} \\
p_a &= \frac{1}{3} \rho_a \\
n_a &\approx \frac{g_a T^3}{2\pi^2} \int_0^\infty dy \, \frac{y^2}{e^y \mp1}
}

\begin{lemma}
\eq{
\int_0^\infty dy \, \frac{y^n}{e^y - 1} = \zeta(n+1) \Gamma(n+1)
}
and 
\eq{
\int dy \, \frac{y^n}{e^y + 1} = \left( 1- \frac{1}{2^n} \right) \int dy \, \frac{y^n}{e^y - 1}
}
\end{lemma}
\begin{proof}
For $\Re{z} > 0 $
\eq{
\Gamma(z) = \int_0^\infty t^{z-1} e^{-t} \, dt 
}
and for $\Re{s} > 1$
\eq{
\zeta(s) = \sum_{n=1}^\infty \frac{1}{n^s}
}
Hence for $Re{z} > 1$
\eq{
\zeta(z)\Gamma(z) &= \sum_{n=1}^\infty \int_0^\infty t^{z-1} n^{-z} e^{-t} \, dt  \\
\text{(let $t=n\tau$) } &= \sum_{n=1}^\infty \int_0^\infty \tau^{z-1} n^{-1} e^{-n\tau} \, nd\tau \\
&= \int_0^\infty \tau^{z-1} \left(\sum_{n=1}^\infty e^{-n\tau}\right) \, d\tau \\
&= \int_0^\infty d\tau \, \frac{\tau^{z-1}}{e^{\tau}-1} 
}
Now 
\eq{
\frac{1}{e^y+1} 7= \frac{e^y-1}{(e^y+1)(e^y-1)} \\
&= \frac{(e^y+1)-2}{(e^y+1)(e^y-1)} \\
&= \frac{1}{e^y-1} - \frac{2}{e^{2y}-1} \\
\Rightarrow \int_0^\infty dy \, \frac{y^n}{e^y+1} &= \int_0^\infty dy \, \frac{y^n}{e^y-1} - 2\int_0^\infty dy \, \frac{y^n}{e^{2y}-1} \\
\text{(let $u=2y$) } &= \int_0^\infty dy \, \frac{y^n}{e^y-1} - 2\int_0^\infty \frac{1}{2}du \, \frac{2^{-n}u^n}{e^{u}-1} \\
&= \left(1 - \frac{1}{2^n} \right) \int_0^\infty dy \, \frac{y^n}{e^y-1}
}
\end{proof}

\begin{fact}
\eq{
\Gamma(n+1) &= n! \\
\zeta(3) &= 1.202\dots \quad \text{(no reasonable representation)} \\
\zeta(4) &= \frac{\pi^4}{90}
}
\end{fact}

Hence 
\eq{
\rho_a = 3p_a&= g_a \frac{\pi^2}{30}T^4 \left\{ \begin{array}{cc} 1 & \text{bosons} \\ \frac{7}{8} & \text{fermions}\footnote{Note that fermions should have the lower density due to the Pauli exclusion principle. } \end{array}\right. \\
n_a &= g_a \frac{\zeta(3)}{\pi^2} T^3  \left\{ \begin{array}{cc} 1 & \text{bosons} \\ \frac{7}{8} & \text{fermions} \end{array}\right.
}
%%%%%%%%%%%%%%%%%%%%%%%%%%%%%%%%%%%%%%%%%%%%%%%%%%%%%%%%
\subsubsection*{Non-Relativistic Particles}
For non-relativistic particles the expressions thus are 
\eq{
\rho_a &\approx \frac{g_a}{2\pi^2}e^\frac{\mu-m}{T} \int_0^\infty dq \, q^2\left(m+\frac{q^2}{2m} \right) e^{-\frac{q^2}{2mT}} \\
p_a &\approx \frac{g_a}{2\pi^2}e^\frac{\mu-m}{T} \int_0^\infty dq \, q^2 \frac{q^2}{3\left( m+\frac{q^2}{2m} \right)} e^{-\frac{q^2}{2mT}} \\
n_a &\approx \frac{g_a}{2\pi^2}e^\frac{\mu-m}{T} \int_0^\infty dq \, q^2 e^{-\frac{q^2}{2mT}}
}
and then the integrals can be done analytically integrating by parts, e.g.
\eq{
n_a &= \frac{g_a}{2\pi^2}e^\frac{\mu-m}{T} \int_0^\infty \sqrt{2mT} du \, 2mT u^2 e^{-u^2} \\
&= \frac{g_a}{2\pi^2}e^\frac{\mu-m}{T} (2mT)^\frac{3}{2} \left\{ \left[-\frac{1}{2}u e^{-u^2}\right]_0^\infty + \int_0^\infty \frac{1}{2}e^{-u^2} \, du \right\} \\
&= g_a \left(\frac{mT}{2\pi}\right)^\frac{3}{2} e^\frac{\mu-m}{T} \quad \text{using } \int_0^\infty e^{-u^2} \, du = \frac{\sqrt{\pi}}{2} 
}
to give 
\eq{
n_a &= g_a \left( \frac{mT}{2\pi} \right)^\frac{3}{2} e^\frac{\mu-m}{T} \\
\rho_a &= g_a \left( \frac{mT}{2\pi} \right)^\frac{3}{2} e^\frac{\mu-m}{T} \left(m+\frac{3}{2}T \right) = n_a \left(m+\frac{3}{2}T \right)  \\
p_a &= g_a \left( \frac{mT}{2\pi} \right)^\frac{3}{2} e^\frac{\mu-m}{T} T = n_a T 
}
Note that is relativistic and non relativistic particles are in thermal equilibrium and $\mu \ll T$ then 
\eq{
\frac{\rho_{\text{non-rel}}}{\rho_\text{rel}} \propto e^{-\frac{m}{T}} \left(\frac{T}{m}\right)^\frac{5}{2} \ll 1
}
hence it is sensible to neglect non-relativistic particles' contribution to energy density, pressure, and entropy density in thermal equilibrium. 

\begin{definition}[Effective Bosonic Degrees of Freedom]\label{def:CSM:1}
Define the \bam{effective number of bosonic degrees of freedom} $g_\ast$ by 
\eq{
g_\ast = \sum_{\text{bosons}} g_a + \frac{7}{8} \sum_{\text{fermions}} g_a
}
such that the total energy density is
\eq{
\rho = g_\ast \frac{\pi^2}{30} T^4
}
\end{definition}

\begin{example}
A possible way to use this definition is to, during radiation domination, note 
\eq{
0 &= \dot{\rho} + 4H\rho \quad \text{from the continutiy equation} \\
&= \propto \dot{T} + HT \quad \text{using $\rho\propto T^3$}
}
In a flat universe the Friedmann equation gives 
\begin{align}\label{eq:CSM:5}
H = \sqrt{\frac{\rho}{3 M_{pl}^2}} = \sqrt{g_\ast \frac{\pi^2}{90} } \frac{T^2}{M_{pl}}
\end{align}
Hence the continuity can be rewritten as 
\eq{
\dot{T} &= - T^3 \frac{\pi}{M_{pl}} \sqrt{\frac{g_\ast}{90}} \\
\Rightarrow T(t) &= \left( \frac{5}{2g_\ast} \right)^\frac{1}{4} \sqrt{\frac{3M_{pl}}{\pi t}} \propto \frac{1}{a} \\
\Rightarrow t &= \sqrt{\frac{5}{2g_\ast}} \frac{3M_{pl}}{\pi T^2}
}
\end{example}

\begin{prop}
Entropy is conserved
\end{prop}
\begin{proof}
The entropy 
\eq{
S = \frac{(\rho + p)V}{T}
}
But $\rho,p \propto T^4$ and $T\propto a^{-1}$. Hence $S \propto a^0$ so constant. Instead, if it is assumed entropy is constant, then for constant $g_\ast$ $T\propto a^{-1}$. 
\end{proof}

%%%%%%%%%%%%%%%%%%%%%%%%%%%%%%%%%%%%%%%%%%%%%%%%%%%%%%%%
\subsection{Boltzmann equation}
Consider a 2-2 scattering process $1+2 \leftrightarrow 3+4$. 

\begin{theorem}
The ODE that describes the time evolution of the number density of 1 particles, $n_1$, is 
\eq{
a^{-3} \frac{d(a^3 n_1)}{dt} = \braket{\sigma v} n_1^{(0)} n_2^{(0)} \left[ \frac{n_3 n_4}{n_3^{(0)} n_4^{(0)}} - \frac{n_1 n_2}{n_1^{(0)} n_2^{(0)}} \right]
}
\end{theorem}
\begin{proof}
\eq{
a^{-3} \frac{d(a^3 n_1)}{dt} =& \int \left( \prod_i \underbrace{\frac{d^3 \bm{p}_i}{(2\pi)^3 2E_i}}_{\text{LI measure}} \right) \underbrace{(2\pi)^4\delta^{(3)}(\bm{p}_1 + \bm{p}_2 - \bm{p}_3 -\bm{p}_4)\delta(E_1 + E_2 - E_3 - E_4)}_{\text{4-momentum conservation}} |\mc{M}|^2 \\
& \times \{ f_3 f_4 [1\pm f_2][1 \pm f_1] - f_1 f_2 [1\pm f_3][1 \pm f_4]\}
}
where $E_i = E_{\bm{p}_i}$, $|\mc{M}|$ is the \hl{scattering amplitude from quantum theory}, and $\{ f_3 f_4 \dots \}$ is a term \hl{corresponding to interactions} where the $f_i$ are the density functions. 
\hl{Assume the reaction takes place rapidly} such that $f_i = f_{FD/BE}$. Considering temperatures $T \ll E_\mu$, so using \ref{eq:CSM:4} 
\eq{
\{ f_3 f_4 [1\pm f_2][1 \pm f_1] - f_1 f_2 [1\pm f_3][1 \pm f_4]\} &\approx e^{-\frac{E_3+E_4}{T}}e^{\frac{\mu_3+\mu_4}{T}}-e^{-\frac{E_1+E_2}{T}}e^{\frac{\mu_1+\mu_2}{T}} \\
&= e^{-\frac{E_1+E_2}{T}}\left[ e^{\frac{\mu_3+\mu_4}{T}} - ^{\frac{\mu_1+\mu_2}{T}} \right] \quad \text{using energy conservation}
}
Now defining $n_i^{(0)} = n_i |_{\mu=0}$ we have 
\eq{
e^\frac{\mu_i}{T} &= \frac{n_i}{n_i^{(0)}} \\
\Rightarrow \{ f_3 f_4 [1\pm f_2][1 \pm f_1] - f_1 f_2 [1\pm f_3][1 \pm f_4]\} &= e^{-\frac{E_1+E_2}{T}} \left[ \frac{n_3 n_4}{n_3^{(0)} n_4^{(0)}} - \frac{n_1 n_2}{n_1^{(0)} n_2^{(0)}} \right]
}
Finally, defining the \bam{thermally average cross section}
\eq{
\braket{\sigma v} = \frac{1}{n_1^{(0)} n_2^{(0)}} \int \left( \prod_i \frac{d^3 \bm{p}_i}{(2\pi)^3 2E_i} \right) (2\pi)^4\delta^{(3)}(\bm{p}_1 + \bm{p}_2 - \bm{p}_3 -\bm{p}_4)\delta(E_1 + E_2 - E_3 - E_4) |\mc{M}|^2 e^{-\frac{E_1+E_2}{T}}
}
yields the desired result
\end{proof}

\begin{definition}
The \bam{reaction rate} is 
\eq{
\Gamma = n_2^{(0)} \braket{\sigma v}
}
This gives 
\eq{
\underbrace{a^{-3} \frac{d(a^3 n_1)}{dt}}_{=\mc{O}(H)} = \underbrace{\braket{\sigma v} n_1^{(0)} n_2^{(0)} \left[ \frac{n_3 n_4}{n_3^{(0)} n_4^{(0)}} - \frac{n_1 n_2}{n_1^{(0)} n_2^{(0)}} \right]}_{=\mc{O}(\Gamma)}
}
\end{definition}

Note that
\begin{itemize}
    \item $\Gamma \gg H \Rightarrow$ reaction is fast and efficient, and so quickly reaches the chemical equilibrium 
    \eq{
    \frac{n_3 n_4}{n_3^{(0)} n_4^{(0)}} = \frac{n_1 n_2}{n_1^{(0)} n_2^{(0)}} \Leftrightarrow \mu_1 + \mu_2 = \mu_3 + \mu_4 \quad \text{(Saha equation)}
    }
    \item $\Gamma \ll H \Rightarrow$ reaction is too slow for expansion of the universe, and solutions becomes 
    \eq{
    a^{-3} \frac{d(a^3 n_1)}{dt} \Rightarrow n_i(t) \approx n_i(a_\ast) \left( \frac{a_\ast}{a(t)}\right)^{-3}
    }
    where $a_\ast$ is when last $\Gamma \approx H$. 
\end{itemize}
%%%%%%%%%%%%%%%%%%%%%%%%%%%%%%%%%%%%%%%%%%%%%%%%%%%%%%%%
\subsection{Thermal History}

%%%%%%%%%%%%%%%%%%%%%%%%%%%%%%%%%%%%%%%%%%%%%%%%%%%%%%%%
\subsection{Big Bang Nucleosynthesis (BBN)}

%%%%%%%%%%%%%%%%%%%%%%%%%%%%%%%%%%%%%%%%%%%%%%%%%%%%%%%%
\subsubsection*{Neutron Abundance}
\hl{Three reactions mediate neutron abundance at MeV energies} 
\eq{
n + \nu_e &\leftrightarrow p^+ + e^- \\
n + \bar{e}^+ &\leftrightarrow p^+ + \bar{\nu}_e \\
n \rightarrow p^+ &+e^- + \bar{\nu}_e
}
\begin{definition}[Fractional Abundance]
Define the \bam{fractional abundance} of neutrons to be 
\eq{
X_n = \frac{n_n}{n_n+n_p}
}
\end{definition}

\hl{Assume now the leptons are in equilibrium}, so $n_l = n_l^{(0)}$. Then 
\eq{
a^{-3} \frac{d(a^3 n_n)}{dt} = n_l^{(0)} \braket{\sigma v} \left\{ \frac{n_p n_n^{(0)}}{n_p^{(0)}} -n_n \right\}
}
Now substitute for $X_n$, using that $(n_p + n_n)a^3$ is conserved and letting $\lambda_{np}=n_l^{(0)} \braket{\sigma v}$
\eq{
a^{-3} \frac{d(a^3 X_n(n_n + n_p))}{dt} &= \lambda_{np} \left\{ (1-X_n)(n_n+n_p)\frac{\left(\frac{-m_n T}{2\pi}\right)^\frac{3}{2} e^\frac{m_n}{T}}{\left(\frac{-m_p T}{2\pi}\right)^\frac{3}{2} e^\frac{m_p}{T}}-X_n(n_n+n_p) \right\} \\
\Rightarrow \frac{dX_n}{dt} &\approx \lambda_{np} \left[ (1-X_n) e^{-\frac{Q}{T}} - X_n \right] \quad \text{letting } Q=m_n-m_p
}
Let $x=\frac{Q}{T}$ so 
\eq{
\frac{dx}{dt}=-x\frac{\dot{T}}{T} = xH \quad \text{using } T \propto a^{-1}
}
then 
\eq{
\frac{dX_n}{dx} = \frac{\lambda_{np}(x)}{xH(x)}\left[ e^-x - X_n(1+e^{-x})\right]
}
\hl{It is an exercise in quantum field theory to find that }
\eq{
\lambda_{np}(x) = \frac{255}{t_{life}x^5}(12+6x+x^2)
}
with $t_{life} \sim 887s\sim 15min$ the neutron lifetime. \\
Finally at this time, 
\eq{
g_\ast = \underbrace{(1\times2)}_{1 \text{ photon}} + \frac{7}{8}[\underbrace{(3\times1)}_{3 \text{ neutrinos}} + \underbrace{(3\times 1)}_{3 \text{ antineutrinos}}+\underbrace{(1\times 2)}_{1 \text{ electron}} + \underbrace{(1\times2)}_{1 \text{ positron}}] = 10.75
}
Hence it is possible to calculate 
\eq{
H(x) &= \sqrt{\frac{g_\ast}{90}} \frac{T^2 \pi}{M_{pl}} \quad \text{from \ref{eq:CSM:5}} \\
&= \frac{1}{x^2}\sqrt{\frac{g_\ast}{90}} \frac{Q^2 \pi}{M_{pl}} \\
&= \frac{H(1)}{x^2} \approx \frac{1.1 s^{-1}}{x^2}
}
The solution may then be found numerically. 

\begin{remark}
When $T\sim 0.1 MeV$, neutron decay now becomes important (this correspond to when the interaction time is comprable to the decay time). At this point to take account for decay, multiply $n_n$ by $e^{-\frac{t}{t_{life}}}$
\end{remark}

%%%%%%%%%%%%%%%%%%%%%%%%%%%%%%%%%%%%%%%%%%%%%%%%%%%%%%%%
\subsubsection*{Light Element Formation}
\hl{The approximation taken for light element formation will be that it is instantaneous at equilibrium}. Deuterium $D$ is mediated by the process \eq{
n + p \leftrightarrow D + \gamma 
}
\eq{
\text{Equilibrium} \Rightarrow \left[ \frac{n_D n_\gamma}{n_D^{(0)} n_\gamma^{(0)}} - \frac{n_n n_p}{n_n^{(0)} n_p^{(0)}} \right] 
}
\hl{Photons have negligible chemical potential} $\Rightarrow n_\gamma = n_\gamma^{(0)}$, so 
\eq{
\frac{n_D}{n_n n_p} &= \frac{n_D^{(0)}}{n_n^{(0)} n_p^{(0)}}\\
&= \frac{3}{4}\left( \frac{2\pi m_D}{m_n m_p T} \right)^\frac{3}{2} e^{\frac{(m_n + m_p-m_D)}{T}} \\
& \approx \frac{3}{4}\left( \frac{4\pi}{m_p T} \right)^\frac{3}{2} e^{\frac{B_D}{T}} \quad \text{letting } B_D = m_n+m_p-m_D
}

\begin{definition}[Baryon-to-Photon Ratio]
Define the \bam{baryon-to-photon ratio} 
\eq{
\eta_b = \frac{n_b}{n_\gamma}
}
\end{definition}

So assuming $n_n\sim n_p\sim n_b$ the \bam{\hl{baryon number density}} yields

\eq{
\frac{n_D}{n_b} \approx \eta_b \left( \frac{T}{m_p}\right)^\frac{3}{2} e^\frac{B_D}{T}
}
%%%%%%%%%%%%%%%%%%%%%%%%%%%%%%%%%%%%%%%%%%%%%%%%%%%%%%%%
%%%%%%%%%%%%%%%%%%%%%%%%%%%%%%%%%%%%%%%%%%%%%%%%%%%%%%%%
\section{Inflation}

\begin{definition}[Inflation]
\bam{Inflation} is a period of time during which $\dot{a},\ddot{a} > 0$. 
\end{definition}

Such a period would provide a solution to the curvature problem (example \ref{example:CSM:curvature problem}) and the horizon problem (example \ref{example:CSM:horizon problem}). 

%%%%%%%%%%%%%%%%%%%%%%%%%%%%%%%%%%%%%%%%%%%%%%%%%%%%%%%%
\subsection{Slow-Roll Inflation}

\begin{definition}[Hubble Slow Roll Parameter]
Define the \bam{first Hubble slow roll parameter} $\eps$ such that 
\eq{
\frac{\ddot{a}}{a} = \dot{H}+H^2 = H^2(1-\eps) \Leftrightarrow \eps = -\frac{\dot{H}}{H^2}
}
In a single fluid universe 
\eq{
\eps = \frac{3(1+w)}{2}
}
Note that this can be written as 
\eq{
\eps = - \del_N \log H 
}
Then define the \bam{second Hubble slow roll parameter} $\eta$ by 
\eq{
\eta = \del_N \log \eps = \frac{\dot{\eps}}{H\eps}
}
Then the \bam{n\textsuperscript{th} Hubble slow roll parameter} $\xi_n$ is given inductively by 
\eq{
\xi_n = \del_N \log \xi_{n-1}
}
with $\xi_1 = \eps$, $\xi_2 = \eta$. 
\end{definition}

Inflation requires $\eps < 1 \Rightarrow w < -\frac{1}{3}$. It will be seen later that scale invariance follows from a spacetime background that is approximately de Sitter, i.e. $H$ approximately constant. Hence consider the case $0< \eps \ll 1$. 

\begin{definition}[Reheating]
\bam{Reheating} is the period when evolution changes from inflation to the decelerating hot Big Bang. 
\end{definition}

\begin{prop}
Approximately $50 e-$folds passed during inflation. 
\end{prop}
\begin{proof}
In order to solve the Horizon problem (example \ref{example:CSM:horizon problem}) it is necessary that the particle horizon is greater than the hubble radius. For approximately de Sitter expansion, $H$ is approximately constant, so $d_{p.h.}\approx \frac{1}{H}$, thus comparing the \hl{comoving distances}\footnote{why the comoving distances?} 
\eq{
\frac{1}{a_i H_i} > \frac{1}{a_0 H_0}
}
where $i$ indicates the beginning of inflation. Thus letting $reh$ represent when reheating occurs, 
\eq{
\frac{a_{reh} H_{reh}}{a_i H_i} > \frac{a_{reh} H_{reh}}{a_0 H_0}
}
Hence using definition \ref{def:CSM:1} and the Friedmann equation in a flat universe 
\eq{
3M_{pl}^2 H^2 = \rho
}
gives 
\eq{
\frac{a_{reh} H_{reh}}{a_0 H_0} &= (\frac{\frac{cst}{T_{CMB,0}}\sqrt{(g_\ast)_0 \frac{\pi^2}{90} } \frac{T_{CMB,0}^2}{M_{pl}}}{\frac{cst}{T_{reh}}\sqrt{(g_\ast)_{reh} \frac{\pi^2}{90} } \frac{T_{reh,0}^2}{M_{pl}}})^{-1} \\
&\approx \frac{T_{reh}}{T_{CMB,0}}
}
Now $T_{CMB,0} \approx 2.7K \approx 0.2 meV = 2 \times 10^{-13} Gev $
\eq{
\Rightarrow \frac{a_{reh} H_{reh}}{a_0 H_0} \approx \highlight{5 \times 10^{23}}\footnote{value given in notes is $4\times 10^{21}$. Why?} \left( \frac{T_{reh}}{10^{10} GeV} \right)
}
\footnote{value given in notes is $4\times 10^{21}$. Why?}
Hence the number of e-folds is 
\eq{
\Delta N_{infl} &= \log{\frac{a_{reh}}{a_i}} \gtrapprox 50 + \log \frac{T_{reh}}{10^{10}}
}
Taking the uncertainty in $T_{reh}$ to be s.t. that $T_{reh} \in (1,10^{15}) GeV$ gives $\Delta N_{infl} \in (25,60)$. It will be often be taken that $\Delta N_{infl} \approx 50$ hence. 
\end{proof}

\begin{definition}[Slow Roll Inflation]
\bam{Slow roll inflation} is inflation during which $\dot{H}$ remains small throughout.
\end{definition}

\begin{prop}
For slow roll inflation, it is sufficient that 
\eq{
\forall n  0 < \xi_n \ll 1
}
\end{prop}
\begin{proof}
For slow roll inflation it is required that $0 < \eps \ll 1$ as already discussed. No taylor expanding $\eps(N)$ from the beginning of inflation $N=N_\ast$ gives \eq{
\eps(N) - \eps(N_\ast) &= \pd[\eps]{N}|_{N_\ast}(N-N_\ast) + \pds[\eps]{N}|_{N_\ast} + \mc{O}(\del_N^3 \eps) \\
&= \eps\left[ \eta(N-N_\ast) + \eta \xi_2 \frac{(N-N_\ast)^2}{2} + \dots \right]
}
so to ensure that $\eps$ remains small it is required $\eta \Delta N_{infl}, \eta \xi_2 \Delta N_{infl}, \dots < 1$, so it is enough to have $0<\xi_n < \frac{1}{\Delta N_{infl}} \ll 1$ for all $n$. 
\end{proof}

%%%%%%%%%%%%%%%%%%%%%%%%%%%%%%%%%%%%%%%%%%%%%%%%%%%%%%%%
\subsection{Single Fields Inflation}

Consider the action
\eq{
S = \frac{1}{2} M_{pl}^2 \int \left[ R - M_{pl}^{-2}\left( -\del_\mu \phi \del^\mu \phi + V(\phi) \right) \right] g^\frac{1}{2} d^4 x 
}
where $\phi$ is some scalar field. To ensure consistency with the symmetries of FLRW it must be that $\phi = \phi(t)$.  Note the similarity to the Einstein Hilbert action \ref{def:CSM:EinsteinHilbertAction} if 
\eq{
\Lambda = \frac{1}{2}M_{pl}^2 \left( \del_\mu \phi \del^\mu \phi - V(\phi) \right)
}
The energy momentum tensor of this action is 
\eq{
T_{\mu\nu} &= \del_\mu \phi \del_\nu \phi - g_{\mu\nu} \left[ \frac{1}{2} \del_\sigma \phi \del^\sigma \phi - V(\phi) \right] \\
&= (\rho + p) u_\mu u_\nu + g_{\mu\nu}p
}
making the identifications 
\eq{
\rho &= -\frac{1}{2}\del_\mu \phi \del^\mu \phi + V(\phi) \\
p &= -\frac{1}{2}\del_\mu \phi \del^\mu \phi - V(\phi) \\
u_\mu &= \frac{\del_\mu \phi}{\sqrt{-\del_\mu \phi \del^\mu \phi}}
}
These reduce to 
\eq{
\rho &= \frac{1}{2} \dot{\phi}^2 + V(\phi) \\
p &=  \frac{1}{2} \dot{\phi}^2 - V(\phi) \\
u_\mu &= (1,\bm{0})
}
as $\phi=\phi(t)$. 

The Friedmann equations are then 
\eq{
\ddot{\phi} + 3H\dot{\phi} + V^\prime(\phi) = 0 \text{ (continuity equation)}\\
3H^2 M_{pl}^2 = \frac{1}{2} \dot{\phi}^2 + V(\phi)  \text{ (evolution equations)}
}

\begin{prop}
\eq{
-\dot{H} M_{pl}^2 = \frac{1}{2}\dot{\phi}^2
}
\end{prop}
\begin{proof}
Differentiate the evolution equation and sub in the continuity equation. 
\end{proof}

\begin{definition}
In scalar field inflation, the first three slow roll parameters are 
\eq{
\eps_V &= \frac{1}{2} M_{pl}^2 \left( \frac{V^\prime}{V} \right)^2 \\
\eta_V = M_{pl}^2 \frac{V^{\prime\prime}}{V} \\
\xi_{3V} &= M_{pl}^4 \frac{V^\prime V^{\prime\prime}}{V^2}
}
\end{definition}

\begin{prop}
\eq{
V = (3-\eps) H^2 M_{pl}^2
}
and hence 
\eq{
\eps_V &= \frac{\eps(\eta-2\eps+6)^2}{4(\eps-3)^2} \\
\eta_V &= \frac{\eta(\eta+2\xi_3+6)-2\eps(5\eta+12)+8\eps^2}{4(\eps-3)}
}
\end{prop}
\begin{proof}
\hl{DO THIS}
\end{proof}

\begin{corollary}
For $\eps,\eta \ll 1$
\eq{
\eps \approx \eps_V \\
\eta \approx 4\eps_V - 2\eta_V
}
\end{corollary}

\begin{definition}
Write $X = -\frac{1}{2} \del_\mu \phi \del^\mu \phi = \frac{1}{2}\dot{\phi}^2$. Then 
\eq{
\rho = X+V \\
p = X-V \\
\dot{X}+6HX +V^\prime \dot{\phi} = 0 \\
-\dot{H} M_{pl}^2 = X
}
\end{definition}

Hence we can write 
\eq{
\eps = -\frac{\dot{H}}{H^2} = \frac{\frac{X}{M_{pl}^2}}{\frac{X+V}{3M_{pl}^2}} = \frac{3X}{X+V} \ll 1 \Rightarrow X \ll V
}
and so  
\eq{
3 M_{pl}^2 H^2 \approx V 
}
Hence 
\eq{
\eta = \frac{\dot{\eta}}{\eps H} = 2\eps + \frac{\dot{X}}{XH}
}
so 
\eq{
\eps,\eta \ll 1 \Rightarrow \dot{X} \ll XH \Rightarrow 2\ddot{\phi} \ll \dot{\phi}H
}
and so we can neglect the acceleration term giving 
\eq{
3 H \dot{\phi} &\approx -V^\prime \\
\Rightarrow \dot{\phi} &\approx - \frac{V^\prime M_{pl}}{\sqrt{3V}} \\
\Rightarrow t &\approx \int d\phi \, \frac{\sqrt{3V}}{V^\prime M_{pl}} + \text{const}
}
This equation can be inverted to give $\phi(t)$ a \bam{slow roll solution}, valid if $\eps,\eta \ll 1$

\subsubsection*{End of Inflation}
We shall \hl{assume that inflation ends at} $t_e$ where $\eps(t_e)\approx 1$. In addition we \hl{assume that at the end of inflation} $\phi$ is at a minimum of V with 
\eq{
V(\phi_{min}) \approx (10^{-3} eV)^4 \approx 0 
}
Then the total of e-folds during inflation is 
\eq{
\Delta N &= \int_{N_i}^{N_e} dN \\
&= \int_{t_i}^{t_e} H dt \\
&= \int_{\phi_i}^{\phi_e} \frac{H}{\dot{\phi}} d\phi \\
&\approx \int_{\phi_i}^{\phi_e} d\phi \, \sqrt{\frac{V}{3M_{pl}^2}} \left( - \frac{V^\prime M_{pl}}{\sqrt{3V}} \right)^{-1} \\
&= \int_{\phi_e}^{\phi_i} d\phi \, \frac{V}{M_{pl}^2 V^\prime}
}

\begin{definition}
Define a field to be a \bam{small field} is during inflation $\Delta \phi < M_{pl}$ and a \bam{large field} if $\Delta \phi > M_{pl}$
\end{definition}

\begin{definition}
Let 
\eq{
\Lambda_\phi = \frac{V}{V^\prime}
}
\end{definition}
\hl{assuming} that $V$ remains relatively flat, so $V,V^\prime$ are approximately constant. Then 
\eq{
\Delta N &\approx \frac{V}{M_{pl}^2 V^\prime}\Delta \phi \\
\Rightarrow \frac{\Delta \phi}{M_{pl}} &\approx 50 \frac{M_{pl}}{\Lambda_\phi}
}
%%%%%%%%%%%%%%%%%%%%%%%%%%%%%%%%%%%%%%%%%%%%%%%%%%%%%%%%
%%%%%%%%%%%%%%%%%%%%%%%%%%%%%%%%%%%%%%%%%%%%%%%%%%%%%%%%
\section{Cosmological Perturbation Theory}
We will consider perturbations about homogeneous flat FLRW spacetime, namely 
\eq{
g_{\mu\nu}(x,t) &= \bar{g}_{\mu\nu}(t) + h_{\mu\nu}(x,t) \\
T_{\mu\nu}(x,t) &= \bar{T}_{\mu\nu}(t) + \delta T_{\mu\nu}(x,t) 
}
where $|h| \ll |\bar{g}|$ and $|\delta T| \ll |\bar{T}|$. Write 
\eq{
\delta \rho (x,t) &= \rho(x,t) - \bar{\rho}(t) \\
\delta p(x,t) &= p(x,t) - \bar{p}(t)
}

\begin{theorem}
The linearised, trace reversed Einstein equations are 
\eq{
\highlight{\text{This may not be necessary}}
}
\end{theorem}

These are now entirely written in terms of $h_{\mu\nu}$ and $\delta T_{\mu\nu}$. 
%%%%%%%%%%%%%%%%%%%%%%%%%%%%%%%%%%%%%%%%%%%%%%%%%%%%%%%%
\subsection{Fourier Decomposition}

\begin{prop}
Fourier modes of a perturbation decouple. 
\end{prop}
\begin{proof}
Consider the general form of linear equations of motion
\eq{
\sum_A \mc{O}_A Pert_A(\bm{x},t) = 0
}
where $Pert_A$ is some perturbation $\set{h_{\mu\nu},\delta T_{\mu\nu}}$, enumerated by index $A$, and the $\mc{O}_A$ are linear differential operators.
Now 
\begin{itemize}
    \item Ensuring general covariance gives that the $\mc{O}_A$ must be constructed from covariant derivatives $\nabla_\mu$ and other tensorial objects. i.e 
    \eq{
    \mc{O}_A = \mc{O}_A(\del_t, \del_j)
    }
    \item The background is homogeneous the $\mc{O}_A$ cannot depend on $\bm{x}$
\end{itemize}
hence when the Fourier transform of the e.o.m is taken we get 
\eq{
0 &= \int d^3 \bm{x} \, e^{-i \bm{k} \cdot \bm{x}} \sum_A \mc{O}_A Pert_A(\bm{x},t) \\
&= \sum_A \tilde{\mc{O}}_A \tilde{Pert}_A(\bm{k},t)
}
where 
\eq{
\tilde{\mc{O}}_A &= \mc{O}_A(\del_t,ik_j) \\
\tilde{Pert}_A(\bm{k},t) &= \int d^3 \bm{x} \, e^{i\bm{k}\cdot\bm{x}} 
}
This has change the e.o.m from 1 PDE to infinitely many ODEs, on for each $\bm{k}$, that are not coupled together. 
\end{proof}

Note that as $r = x_{phys} = a x$,  $k_{phys} = \frac{k}{a}$. 

%%%%%%%%%%%%%%%%%%%%%%%%%%%%%%%%%%%%%%%%%%%%%%%%%%%%%%%%
\subsection{Scalar-Vector-Tensor Decomposition (SVT)}

\begin{definition}[Hodge Decomposition]
Any vector $v_i$ can be decomposed as 
\eq{
v_i = w_i + \del_i \theta
}
where $\theta$ is the solution (that necessarily exists, e.g. by using Green's functions, on topologically trivial spaces) of 
\eq{
\del^i v_i = \nabla^2 \theta
}
and then 
\eq{
w_i = v_i - \del_i \theta
}
satisfies $\del^i w_i = 0$
\end{definition}

\begin{definition}[Tensor Decomposition]
A tensor $S_{ij}$ can be decomposed as 
\eq{
S_{ij} = \delta_{ij} A + \del_i\del_j B  + \del_{(i}C_{j)} + D_{ij}
}
with 
\eq{
\del_i C_i = 0 = D_{ii} = \del_i D_{ij} 
}
\end{definition}


\begin{theorem}
Perturbations with different helicities decouple.
\end{theorem}
\begin{proof}
Under a rotation, $\set{x^0,x^i} \to \set{{x^\prime}^0,{x^\prime}^{i^\prime}}=\set{x^0, R\indices{_i^{i^\prime}} x^i}$, so the transformations of tensors is governed by 
\eq{
\pd[{x^\prime}^{\mu^\prime}]{x^\mu} = \begin{pmatrix} 1 & \\ & R\indices{^{i^\prime}_i} \end{pmatrix}
}
Hence terms that contain no spacial indices are rotation-scalars, terms that contain 1 spatial index are rotation vector, and those with two are rotation tensors. \\
Now, it is impossible to construct a non-zero scalar from transverse vectors or transverse traceless tensors using only derivatives and background quantities at linear order. Hence all mixing vanishes. 
\end{proof}

\begin{definition}[SVT Decomposition]
The \bam{SVT decomposition} oh $h$ and $\delta T $ is 
\eq{
h_{00} &= -E \\
h_{i0} &= a\left[ \del_i F + G_i \right] \text{ s.t } \del^i G_i = 0 \\
h_{ij} &= a^2 \left[ \delta_{ij}  A + \del_i\del_j B  + \del_{(i}C_{j)} + D_{ij} \right] \text{ s.t. } \del_i C_i = 0 = D_{ii} = \del_i D_{ij} \\
\delta T_{00} &= -\bar{\rho} h_{00} + \delta \rho \\
\delta T_{i0} &= \bar{p} h_{0i} - (\bar{\rho} + \bar{p} ) \left[ \del_i \delta u + \delta u_i^V \right] \text{ s.t } \del^i \delta u_i^V = 0 \\
\delta T_{ij} &= \bar{p} h_{ij} + a^2 \left[ \delta_{ij} \delta p + \del_i \del_j \pi^S + \del_{(i}\pi_{j)}^V + \pi_{ij}^T \right] \text{ s.t. } \del_i \pi_i^V = 0 = \pi_{ii}^T = \del_i 
\pi_{ij}^T \\
u_\mu &= (u_0,u_i) = (-1+\delta u_0, \del_i \delta u + \delta u_i^V)
}
The $\pi$ terms are the \bam{anisotropic inertia}, and are zero for perfect fluids. 
\end{definition}

%%%%%%%%%%%%%%%%%%%%%%%%%%%%%%%%%%%%%%%%%%%%%%%%%%%%%%%%
\subsection{Gauge Transformations}

As GR is a generally covariant theory, we are able to perform a gauge transform $x^\mu \to {x^\prime}^\mu = x^\mu + \eps^\mu(x)$, where we hold the background constant and attribute all change in tensor to the perturbation. 

\begin{prop}
Under the gauge transform, the tensor $S = \bar{S} + \delta S$ varies as 
\eq{
\Delta(\delta S) = -\mc{L}_\eps S = -\mc{L}_\eps \bar{S} + \mc{O}(\eps^2)
}
so for a scalar $s$, vector $V^\mu$, and $h_{\mu\nu}$
\eq{
\Delta( \delta s) &= -\eps^0 \dot{\bar{s}} \\
\Delta( \delta V^\mu) &= -\eps^\nu \nabla \bar{V}^\mu + \bar{V}^\nu \nabla_\nu \eps^\mu \\
\Delta( h_{\mu\nu}) &= -\nabla_\mu \eps_\nu - \nabla_\nu \eps_\mu
}
\end{prop}

\begin{prop}
Under a gauge transform the SVT decomposition transforms as 
\eq{
\Delta A &= 2H \eps_0 \\
\Delta B &= -\frac{2}{a^2}\eps^S \\
\Delta C_i &= -\frac{1}{a^2} \eps_i^V \\
\Delta D_{ij} &= 0 \\
\Delta E &= 2\dot{\eps_0} \\
\Delta F &= \frac{1}{a} \left( -\eps_0 - \dot{\eps}^S + 2H \eps^S \right) \\
\Delta G_i &= \frac{1}{a} \left( -\dot{\eps}_i^V + 2H \eps_i^V \right) 
\Delta \delta \rho &= \dot{\bar{\rho}} \eps_0 \\
\Delta \delta p &= \dot{\bar{p}} \eps_0 \\
\Delta \delta u &= - \eps_0 \\
\Delta \pi^S &= \Delta \pi_i^V =\Delta \pi_{ij}^T = \Delta \delta u_i^V =0
}
decomposing the gauge transform as 
\eq{
\eps^\mu = (\eps^0 , \del^i \eps^S + {\eps^i}^V
}
\end{prop}

\begin{definition}[Newtonian Gauge]
The \bam{Newtonian gauge} takes 
\eq{
\left\{ \begin{array}{c} \eps^S = \frac{a^2 B}{2} \\ \eps_0 = aF - \frac{a^2}{2} \dot{B} \end{array} \right. \Rightarrow \left\{ \begin{array}{c} B=0 \\ F=0 \end{array} \right.
}
\end{definition}

%%%%%%%%%%%%%%%%%%%%%%%%%%%%%%%%%%%%%%%%%%%%%%%%%%%%%%%%
\subsection{Vector Perturbations}
Selecting only the vector perturbations, and \hl{assuming the constituents of the universe behave like perfect fluid} so the anisotropic inertia can be neglected gives 
\eq{
\del_0 \left[ \left(\bar{\rho} + \bar{p} \right) \delta u_j^V \right] + 3H \left(\bar{\rho} + \bar{p} \right)\delta u_j^V = 0 
\Rightarrow \left(\bar{\rho} + \bar{p} \right) \delta u_j^V \propto a^{-3}
}
Substituting into the linearised Einstein equations gives 
\eq{
M_{pl}^{-2} \left(\bar{\rho} + \bar{p} \right) \delta u_j^V a = \frac{1}{2} \del^2 \left( G_j - \highlight{a} \dot{C}_j \right) \\
\Rightarrow   G_j - \highlight{a} \dot{C}_j \propto a^{-2}
}

%%%%%%%%%%%%%%%%%%%%%%%%%%%%%%%%%%%%%%%%%%%%%%%%%%%%%%%%
\subsection{Tensor Perturbations}
Selecting only the tensor perturbations, and neglecting anisotropic inertia gives 
\eq{
\ddot{D}_{ij} + 3H \dot{D}_{ij} - \frac{1}{a^2} \del^2 D_{ij} = 0
}
or interms of the fourier transform 
\eq{
\ddot{\tilde{D}}_{ij} + 3H \dot{\tilde{D}}_{ij} - \frac{k^2}{a^2} \tilde{D}_{ij} = 0
}

\begin{definition}
Define the \bam{polarisation tensors} $\eps_{ij}^s$ for $s=1,2$ to be two independent solutions to 
\eq{
k^i \eps_{ij}^s(\bm{k}) = 0 = \eps_{ii}^s(\bm{k}) \\
\eps_{ij}^s \eps_{ji}^{s^\prime} = 2 \delta_{s s^\prime}
}
Then write 
\eq{
\tilde{D}_{ij}(t,\bm{k}) = \sum_{s=1,2} \eps_{ij}^s(\bm{k}) \mc{D}_s(t,k)
}
\end{definition}

Each polarisation evolves as 
\eq{
\ddot{\mc{D}}_s(t,k) + 3H \dot{\mc{D}}_s(t,k) - \frac{k^2}{a^2} \mc{D}_s(t,k) = 0
}

In the super Hubble regime $k_{phys} \ll H \Rightarrow k \ll aH$. 
\eq{
\Rightarrow \ddot{\mc{D}}_s(t,k) + 3H \dot{\mc{D}}_s(t,k) \approx 0 \Rightarrow \dot{\mc{D}}_s \propto a^{-3} \\
\Rightarrow \mc{D}_s(t,k) = A_s(k) + B_s(k) a(t)^\frac{3(w-1)}{2}
}

In the sub Hubble regime, $k_{phys} \gg H \Rightarrow k \gg aH$ and solve using the ansatz 
\eq{
\mc{D}_s(t,k) = X(t) \exp \left[ ik\int^t \frac{dt^\prime}{a(t^\prime)} \right] \\
\Rightarrow \mc{D}_s (t,k) = \frac{\bar{A}_s \cos k\tau + \bar{B}_s \sin k\tau}{a}
}
for $\tau = \int^t \frac{dt^\prime}{a(t^\prime)}$

%%%%%%%%%%%%%%%%%%%%%%%%%%%%%%%%%%%%%%%%%%%%%%%%%%%%%%%%
\subsection{Scalar Perturbations}
Taking the Newtonian gauge, and  selecting out just the scalar perturbations, taking $A=-2\Psi$, $E=2\Phi$ gives 
\eq{
- \frac { 1 } { 2 M _ { \mathrm { Pl } } ^ { 2 } } \left[ \delta \rho - \delta p - \nabla ^ { 2 } \pi ^ { S } \right] &= H \dot { \Phi } + \left( 4 H ^ { 2 } + 2 \frac { \ddot { a } } { a } \right) \Phi - \frac { \nabla ^ { 2 } \Psi } { a ^ { 2 } } + \ddot { \Psi } + 6 H \dot { \Psi } \\
- \frac { a ^ { 2 } } { M _ { \mathrm { Pl } } ^ { 2 } } \partial _ { i } \partial _ { j } \pi ^ { S } &= \partial _ { i } \partial _ { j } ( \Phi - \Psi ) \\
\frac { 1 } { 2 M _ { \mathrm { Pl } } ^ { 2 } } ( \overline { \rho } + \overline { p } ) \partial _ { i } \delta u &= - H \partial _ { i } \Phi - \partial _ { i } \dot { \Psi } \\
\frac { 1 } { 2 M _ { \mathrm { Pl } } ^ { 2 } } \left( \delta \rho + 3 \delta p + \nabla ^ { 2 } \pi ^ { S } \right) &= \frac { \nabla ^ { 2 } \Phi } { a ^ { 2 } } + 3 H \dot { \Phi } + 3 \ddot { \Psi } + 6 H \dot { \Psi } + 6 \frac { \ddot { a } } { a } \Phi
}
and the continuity equations 
\eq{
\delta p + \nabla ^ { 2 } \pi ^ { S } + \partial _ { 0 } [ ( \overline { \rho } + \overline { p } ) \delta u ] + 3 H ( \overline { \rho } + \overline { p } ) \delta u + ( \overline { \rho } + \overline { p } ) \Phi &= 0 \\
\delta \dot { \rho } + 3 H ( \delta \rho + \delta p ) + \nabla ^ { 2 } \left[ \frac { ( \overline { \rho } + \overline { p } ) } { a ^ { 2 } } \delta u + H \pi ^ { S } \right] - 3 ( \overline { \rho } + \overline { p } ) \dot { \Psi } &= 0
}
Neglecting anisotropic stress gives that $\Phi=\Psi$ is a solution

%%%%%%%%%%%%%%%%%%%%%%%%%%%%%%%%%%%%%%%%%%%%%%%%%%%%%%%%
\subsection{Adiabatic Modes}

\begin{definition}[Curvature Perturbations]
The \bam{curvature perturbation on comoving hypersurfaces} is 
\eq{
\mc{R} = \frac{A}{2} + H \delta u 
}
and the \bam{curvature perturbation on constant density hypersurfaces} is 
\eq{
\zeta = \frac{A}{2} - H \frac{\delta \rho}{\dot{\bar{\rho}}}
}
\end{definition}

\begin{prop}
In the Newtonian gauge 
\eq{
\zeta(k,t) = \mc{R}(k,t) + \frac{M_{pl}^2}{3a^2(\bar{\rho} + \bar{p})} k^2 A(k,t) 
}
\end{prop}

Note $\zeta-\mc{R} \propto \left(\frac{k}{aH}\right)^2$, so on super Hubble scales when $k\ll aH$ $\zeta-\mc{R} \approx0$.

\begin{theorem}
Outside the Hubble raidus $\exists k_1, k_2, k_3$ s.t. 
\begin{itemize}
    \item $\dot{\mc{R}}(k_1,t) = 0 \dot{\mc{R}}(k_2,t)$ (adiabatic modes) 
    \item $\mc{R}(k_1,t) \neq 0$
    \item $\dot{\tilde{D}}_{ij}(k_3) = 0$ and $\tilde{D}_{ij}(k_3) \neq 0$
\end{itemize}
\end{theorem}
\begin{proof}
We will prove to linear order in the Newtonian gauge. 
Consider the gauge transform that preserves the Newtonian Gauge 
\eq{
\eps_\mu = (\eps(t), a(t)^2 \omega_{ij} x^j)
}
This perturbs a flat FLRW metric to 
\eq{
\Phi &= -\dot{\eps} \\
\Psi &= H\eps - \frac{1}{3} \omega_{ii} \\
\delta p &= -\dot{\bar{p}}\eps \\
\delta \rho &= -\dot{\bar{\rho}}\eps \\
\delta u &= \eps \\
\pi^S &= 0 \\
D_{ij} &= -\omega_{(ij)} + \frac{2}{3} \delta_{ij} \omega_{kk}
}
This perturbation must be a solution of the Einstein equation as GR is a generally covariant theory. For it to be a physical solution it must decay at infinity. This requires the Fourier transform to be supported at $k \neq 0  $. Then if the solution does not decay, it is always possible to ensure it does by perturbing the solution slightly in Fourier space, and using that this remains a solution to the equation. Hence $\exists k$ such that $D_{ij}$ is constant and non-vanishing up to correction suppressed by $k^2$ in the super Hubble limit  \\
The solution is therefore only not physical if an equation of motion vanishes identically for $k=0$. This does occur for off diagonal term, so we impose for $k\neq 0$
\eq{
k_i k_j (\Phi- \Psi) = 0 \Rightarrow \Phi = \Psi
}
this fixes 
\eq{
\dot{\eps}+H\eps = \frac{1}{3} \omega_{kk} \Rightarrow \eps(t) = \frac{\omega_{kk}}{3a(t)} \int_T^t a(t^\prime) dt^\prime 
}
Substituting yields 
\eq{
\mc{R} = \frac{\omega_{kk}}{3}
}
Hence a solution with $\dot{\mc{R}}=0$, $\mc{R}\neq 0$ always exists by diffeomorphism invariance. (\hl{where did the second sol go})




\end{proof}

%%%%%%%%%%%%%%%%%%%%%%%%%%%%%%%%%%%%%%%%%%%%%%%%%%%%%%%%
%%%%%%%%%%%%%%%%%%%%%%%%%%%%%%%%%%%%%%%%%%%%%%%%%%%%%%%%
\section{Observables and statistical properties}
%%%%%%%%%%%%%%%%%%%%%%%%%%%%%%%%%%%%%%%%%%%%%%%%%%%%%%%%
\subsection{Observables}
\subsubsection*{Large scale structure}
\begin{definition}[Fractional Overdensity]
Let $n_g$ be the galaxy number density. Then define the \bam{fractional overdensity} as 
\[
\delta_g(\bm{x}) = \frac{n_g(\bm{x})-\bar{n}_g}{\bar{n}_g}
\]
where $\bar{n}_g$. 
\end{definition}

It is typical to assume a linear relation ship between the galaxy density and matter density 
\[
n_g = b \rho_m
\]
where b is the \bam{linear galaxy bias}. This gives the \bam{matter density perturbation}
\[
\delta_g(\bm{x}) = b \times \frac{\rho_m(\bm{x})-\bar{\rho}_m}{\bar{\rho}_m} = b \times \delta_m(\bm{x})
\]

\begin{definition}[Two point correlation function]
Given a general field $f$ and some probability of having the field configuration $Pr[f]$ the \bam{two point correlation function} is 
\eq{
\xi^f(\bm{x},\bm{y}) = \braket{f(\bm{x})f(\bm{y})} = \int \mc{D}f f(\bm{x})f(\bm{y}) Pr[f]
}
\end{definition}

\begin{lemma}
Imposing homogeneity and isotropy of the universe gives 
\eq{
\braket{f(\bm{k})f(\bm{k}^\prime)} = (2\pi)^3 P^f(|\bm{k}|) \delta^{(D)}(\bm{k} + \bm{k}^\prime)
}
for the Fourier coefficients $\braket{f(\bm{k})f(\bm{k}^\prime)}$ such that 
\eq{
 \xi^f(\bm{x},\bm{y}) = \braket{f(\bm{x})f(\bm{y})} = \int \frac{d\bm{k}}{(2\pi)^3} \frac{d\bm{k}^\prime}{(2\pi)^3} e^{-i\bm{k}\cdot\bm{x} -i \bm{k}^\prime \cdot \bm{y}} \braket{f(\bm{k})f(\bm{k}^\prime)}
}
This defines the \bam{power spectrum} $P^f$ of the field. It is common to also define the \bam{dimensionless power spectrum} 
\eq{
\Delta^f(k) = \frac{k^3}{2\pi^2} P^f(k)
}
\end{lemma}

\begin{definition}[Gaussian Random Fields]
A \bam{Gaussian random field} is one whose probability functional takes the form 
\eq{
Pr[f] \propto \frac{\exp -f_i \xi_{ij} f_j}{\sqrt{\det \xi_{ij}}}
}
for 
\eq{
\braket{f_i f_j} = \xi_{ij}
}
\end{definition}

\subsubsection*{CMB}
\begin{idea}
The CMB is "pretty much" linear, which makes it nice to work with. 
\end{idea}

\begin{definition}[Flat Sky approximation]
The CMB temperature $T=T(\hat{\bm{n}})$ is defined on a sphere. However, on a sufficiently small patch the \bam{flat-sky approximation} can be used, taking $T(\bm{l})$ to be the 2D Fourier transform of $T$, then the power spectrum is given by 
\eq{
\braket{T(\bm{l})T(\bm{l}^\prime)} = (2\pi)^3 C_l \delta^{(D)}(\bm{l} + \bm{l}^\prime )
}
\end{definition}

%%%%%%%%%%%%%%%%%%%%%%%%%%%%%%%%%%%%%%%%%%%%%%%%%%%%%%%%
\subsection{Assumptions}
It will be assumed throughout subsequent sections that all observational predictions are generated by "well known" physics if the initial density fluctuation are 
\begin{itemize}
    \item Scale invariant : $P \propto k^{-4 + n_s}$ with $n_s \approx 1$. 
    \item Adiabatic. 
    \item Gaussian.
\end{itemize}
These can be justified through inflation. 

%%%%%%%%%%%%%%%%%%%%%%%%%%%%%%%%%%%%%%%%%%%%%%%%%%%%%%%%
\subsection{Newtonian Perturbation}

For a non-relativistic fluid with mass density $\rho$, pressure $P$, and velocity $\bm{u}$, the determining equations are 
\eq{
\del_t \rho + \grad_{\bm{r}} \cdot (\rho \bm{u}) = 0 \quad \text{(continuity equation)} \\
\del_t \bm{u} + \bm{u} \cdot \grad_{\bm{r}} \bm{u} = -\frac{1}{\rho} \grad_{\bm{r}} P - \grad_{\bm{r}} \Phi \quad \text{(Euler equation)} \\
\nabla_{\bm{r}}^2 \Phi = 4\pi G \rho \quad \text{(Poisson's equation)}
}
For a comoving observer the physical coordinate $\bm{r}$ is related to the comoving coordinate $\bm{x}$ by 
\eq{
\bm{r}(t) = a(t) \bm{x} \\
\Rightarrow \bm{u} = \dot{\bm{r}} = H\bm{r}
}
Hence 
\eq{
(\pd{t})_{\bm{r}} = (\pd{t})_{\bm{x}} - H\bm{x} \cdot \grad \\
\grad_{\bm{r}} = \frac{1}{a} \grad
}
letting $\grad = \grad_{\bm{x}}$. \\
Now writing a perturbation as 
\eq{
\rho \to \bar{\rho} + \delta\rho = \bar{\rho}(1+\delta) \\
P \to \bar{P} + \delta P \\
\bm{u} \to Ha\bm{x} + \bm{v} \\
\Phi \to \bar{\Phi} + \delta Phi
}
and substituting in gives, to zeroth and first order 
\eq{
\pd[\bar{rho}]{t}+3H\bar{\rho} = 0 \quad \text{(continuity for matter)}\\
\dot{\delta} = -\frac{1}{a}\grad\cdot\bm{v} \quad \text{(continuity equation)} \\
\dot{\bm{v}} + H\bm{v} = -\frac{1}{a\bar{\rho}}\grad \delta P -\frac{\grad \Phi}{a} \quad \text{(Euler equation)} \\
\nabla^2 \Phi = 4\pi G a^1 \bar{\rho} \delta \quad \text{(Poisson's equation)}
}
Combining these gives 
\begin{align} \label{eq:CSM:1}
\ddot{\delta} + 2H\dot{\delta} - \frac{1}{a^2\bar{\rho}}\nabla^2 \delta P - 4\pi G \bar{\rho}\delta = 0
\end{align}

\begin{definition}[Barotropic Fluid]
A barotropic fluid is one where $P=P(\rho)$. In such a fluid the speed of sound is 
\[
c_s = \sqrt{\pd[P]{\rho}}
\]
\end{definition}

For a barotropic fluid \ref{eq:CSM:1} gives 
\eq{
\ddot{\delta} + 2H\dot{\delta} - \frac{c_s^2}{a^2}\nabla^2 \delta  - 4\pi G \bar{\rho}\delta = 0
}
and so Fourier transforming 
\eq{
\ddot{\delta} + 2H\dot{\delta} + \left[ \frac{c_s^2}{a^2}k^2 - 4\pi G \bar{\rho} \right]\delta = 0
}

\begin{definition}[Jean's wavenumber]
The \bam{Jean's wavenumber and scale} are
\eq{
k_J - \frac{\sqrt{4\pi G \bar{\rho}a^2}}{c_s} \\
\lambda_J = \frac{2\pi a }{k_J} = c_s \sqrt{\frac{\pi}{G\bar{\rho}}}
}
\end{definition}

Perturbations on a scale smaller than the Jean's scale, i.e. $k>k_J$ the euqation is essentially damped oscillation, whereas for large perturbations $k<k_J$ tge perturbation grows as a power law. 

\begin{idea}
This power law growth should be seen as the case where there is insufficient pressure support to stop the gravitational collapse of a perturbation, as the speed of such a wave is too slow to repsond in sufficient collapse time. This can be seen by estimating the collapse time as 
\eq{
t_f \sim \frac{1}{\sqrt{G\bar{\rho}}}
}
and the time for a pressure wave to cross a perturbation radius $R$ as 
\eq{
t_{sc} = \frac{R}{c_s}
}
For pressure support we would then want 
\eq{
t_{sc} < t_f  \\ 
\Rightarrow R \lesssim \lambda_J
}
\end{idea}

\begin{example}[Dark matter evolution]
Take $\delta=\delta_m$. Assume that $c_s=0$ in dark matter. 
\subsubsection*{Dark matter domination}
In dark matter domination, the Hubble rate is determined only by dark matter. Then as seen before $a\propto t^{\frac{2}{3}}$,  $H=\frac{2}{3t}$ and $H^2=\frac{8\pi G\bar{\rho}}{3}$
\eq{
\ddot{\delta}_m + \frac{4}{3t} \dot{\delta}_m - \frac{2}{3t^2}\delta_m = 0
}
Power law solutions are $\delta \propto t^\frac{2}{3}, t^{-1}$, so growing modes are $\delta \propto a$. 

\subsubsection*{Radiation Domination}
During radiation domination, the total energy density sources the Hubble growth, so 
\eq{
\ddot{\delta}_m + 2H\dot{\delta}_m   - 4\pi G \sum_i \bar{\rho}_i\delta_i = 0
}
It will be shown that on subhorizon scales photon perturbations oscillates rapidly with respect to the time scale of structure formation, and so will not contribute overall. Now in radiation domination $a\propto t^\frac{1}{2}$, $H=\frac{1}{2t}$, so 
\eq{
\ddot{\delta}_m + \frac{1}{t}\dot{\delta}_m   - 4\pi G \bar{\rho}_m\delta_m = 0
}
In the absence of pressure the time space must be fixed so 
\eq{
\ddot{\delta}_m \sim H \dot{\delta}_m \sim H^2\delta_m \sim \frac{8\pi G \rho_r }{3}\delta_m \gg 4\pi G \bar{\rho}_m \delta_m
}
and the equation reduces to 
\eq{
\ddot{\delta}_m + \frac{1}{t} \dot{\delta}_m = 0 
}
Hence the solutions are $\delta_m \propto 1, \log t$ 
giving for the growing mode 
\eq{
\delta_m \log a 
}
\subsubsection*{Dark energy domination}
In dark energy domination, as $\Lambda$ is constant, and $H\gg 4\pi G\bar{\rho}_m$ the equation becomes 
\eq{
\ddot{\delta}_m + 2H \dot{\delta}_m = 0 
}
and so $\delta_m \propto 1, e^{-2Ht}$ so in dark energy domination perturbations stop growing. 
\end{example}

%%%%%%%%%%%%%%%%%%%%%%%%%%%%%%%%%%%%%%%%%%%%%%%%%%%%%%%%
\subsection{Relativistic Perturbation}

\begin{definition}[Horizon scale]
The \bam{conformal horizon scale} is $\mc{H}^{-1}$, where
\eq{
\mc{H} = \frac{a^\prime}{a}
}
is the conformal hubble parameter
A mode is \bam{superhorizon} if $k^{-1}\gg \mc{H}^{-1}$, and it is \bam{subhorizon} if $k^{-1}\ll \mc{H}^{-1}$.
\end{definition}

\begin{fact}
Superhorizon modes are not in causal contact with themselves, and so cannot evolve dynamically. 
\end{fact}

Now for a relativistic perturbation, the following first order equations are 
\eq{
\delta^\prime + 3\mc{H} \left( \frac{\delta P}{\delta \rho} - \frac{\bar{P}}{\bar{\rho}} \right) \delta = - \left( 1 + \frac{\bar{P}}{\bar{\rho}}\right) (\grad \cdot \bm{v} - 3\Phi^\prime) \quad \text{(Conservation of stress energy)} \\
\bm{v}^\prime + 3\mc{H} \left(\frac{1}{3} - \frac{\bar{P}}{\bar{\rho}} \right) \bm{v} = -\frac{\grad \delta P}{\bar{\rho}+\bar{P}} - \grad\Phi \quad \text{(Euler equation)} \\
}
and the Einstein equations 
\begin{align} \label{eq:CSM:2}
\nabla^2 \Phi - 3\mc{H}(\Phi^\prime + \mc{H}\Phi)  &= 4\pi G a^2 \delta \rho \\
\Phi^\prime + \mc{H}\Phi &= -4\pi G a^2 (\bar{\rho} + \bar{P}) v \\
\Phi^{\prime\prime} + 3 \mc{H} \Phi^\prime + (2\mc{H}^\prime + \mc{H}^2 ) \Phi &= 4\pi G a^2 \delta P 
\end{align}
where $\bm{v} = \grad v$.
\begin{definition}[Comoving gauge density contrast]
Define the \bam{Comoving gauge density contrast} $\Delta$ by
\eq{
\Delta = \delta - 3\mc{H}\left(1 + \frac{\bar{P}}{\bar{\rho}} \right) v
}
\end{definition}

Substituting for the CGDC gives 
\eq{
\nabla^2 \Phi = 4\pi G a^2 \bar{\rho} \Delta
}

\begin{definition}[Comoving curvature perturbation]
The \bam{comoving curvature perturbation} $\mc{R}$ is the curvature perturbation on a comoving hyperfusrface and is defined by 
\[
\mc{R} = -\Phi + \mc{H}v
\]
Using \ref{eq:CSM:2} this can be written as 
\eq{
\mc{R} = -\Phi - \frac{\mc{H}(\Phi^\prime + \mc{H} \Phi}{4\pi G a^2 (\bar{\rho} + \bar{P})}
}
\end{definition}

\begin{prop}
$\mc{R}$ is conserved on superhorizon scales
\end{prop}
\begin{proof}
It can be shown 
\eq{
-4\pi G a^2 (\bar{\rho} + \bar{P}) \mc{R}^\prime = \underbrace{4\pi G a^2 \mc{H} (\delta P - \frac{\bar{P}^\prime}{\bar{\rho}^\prime} \delta \rho)}_{=0 \text{ for adiabatic pert}} + \underbrace{\mc{H} \frac{\bar{P}^\prime}{\bar{\rho}^\prime} \nabla^2 \Phi}_{\sim \mc{H}k^2 \mc{R}}
}
Hence 
\eq{
\frac{d \log \mc{R}}{d\log a} \sim (\frac{k}{\mc{H}})^2
}
small on superhorizon scales. 
\end{proof}

%%%%%%%%%%%%%%%%%%%%%%%%%%%%%%%%%%%%%%%%%%%%%%%%%%%%%%%%
%%%%%%%%%%%%%%%%%%%%%%%%%%%%%%%%%%%%%%%%%%%%%%%%%%%%%%%%

\begin{remark}
In cosmology, Bessel functions like to turn up (why?). Hence it can sometimes be useful to try turn an euqaiotn into the form of Bessel's equation 
\[
\frac{d^2y}{dx^2} + \frac{1}{x} \frac{dy}{dx} + \left( 1-\frac{\nu^2}{x^2} \right ) = 0 
\]
\end{remark}


%%%%%%%%%%%%%%%%%%%%%%%%%%%%%%%%%%%%%%%%%%%%%%%%%%%%%%%%
%%%%%%%%%%%%%%%%%%%%%%%%%%%%%%%%%%%%%%%%%%%%%%%%%%%%%%%%
\section{Quantum Inflationary Origin}

\begin{definition}[Mukhanov Sasaki Equation]
The \bam{MSE} is 
\[
f ^ { \prime \prime } - \nabla ^ { 2 } f - \frac { a ^ { \prime \prime } } { a } f = 0
\]
\end{definition}

\end{document}