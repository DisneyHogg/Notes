\documentclass{article}

\usepackage{header}
%%%%%%%%%%%%%%%%%%%%%%%%%%%%%%%%%%%%%%%%%%%%%%%%%%%%%%%%
%Preamble

\title{Advanced Quantum Field Theory Notes}
\author{Linden Disney-Hogg}
\date{January 2019}

%%%%%%%%%%%%%%%%%%%%%%%%%%%%%%%%%%%%%%%%%%%%%%%%%%%%%%%%
%%%%%%%%%%%%%%%%%%%%%%%%%%%%%%%%%%%%%%%%%%%%%%%%%%%%%%%%
\begin{document}

\maketitle
\tableofcontents

%%%%%%%%%%%%%%%%%%%%%%%%%%%%%%%%%%%%%%%%%%%%%%%%%%%%%%%%
%%%%%%%%%%%%%%%%%%%%%%%%%%%%%%%%%%%%%%%%%%%%%%%%%%%%%%%%
\section{Introduction}
%%%%%%%%%%%%%%%%%%%%%%%%%%%%%%%%%%%%%%%%%%%%%%%%%%%%%%%%
\section{Path Integral in Quantum Mechanics}
(See Osborn's notes, section 1.2). Set $\hbar = 1 = c$ for now, though later when restoring $\hbar$ through dimensional analysis 
\[
[\hbar]=[E][t]=[p][x]
\]
Take a Hamiltonian in 1 dimension 
\[
\hat{H} = H(\hat{x},\hat{p})
\]
with 
\[
\comm[\hat{x}]{\hat{p}}=i
\]
Assume for simplicity here that 
\[
\hat{H} = \frac{\hat{p}^2}{2m} + V(\hat{x})
\]
\begin{definition}[Schr\"odinger's equation]
\[
i \pd{t} \ket{\psi(t)} = \hat{H} \ket{\psi(t)}
\]
The formal solution given $\ket{\psi(0)}$ is 
\[
\ket{\psi(t)} = e^{-i\hat{H}t} \ket{\psi(0)}
\]
\end{definition}

Consider position eigenstates $\ket{x,t}$
\[
\hat{X}(t) \ket{x,t} = x\ket{x,t} 
\]
Normalise these such that 
\[
\braket{x^\prime, t| x,t} = \delta(x^\prime - x)
\]
In the Schr\"odinger picture, states depend on time, operators are treated as constant. Hence we have a fixed basis of eigenstates $\set{\ket{x}}$ of $\hat{X}$. The wavefunction is then 
\[
\psi(x,t) = \braket{x | \psi(t)}
\]
\[
\hat{H}\psi(x,t) = \left( -\frac{1}{2m} \pds{x} +V(x) \right) \psi(x,t)
\]

%%%%%%%%%%%%%%%%%%%%%%%%%%%%%%%%%%%%%%%%%%%%%%%%%%%%%%%%
\subsection{Path Integral}
We wish to express time evolution as a sum over all  trajectories, appropriately weighted. 
\[
\psi(x,t) = \braket{x | e^{-i\hat{H}t} | \psi(0)}
\]
Insert a complete set of states $1 = \int dx_0 \ket{x_0} \bra{x_0}$ 
\begin{align*}
    \psi(x,t) &= \int dx_0 \braket{x| e^{-i\hat{H}t} | x_0} \braket{x_0 | \psi(0)} \\ 
    &= \int dx_0 \, K(x,x_0 ; t) \psi(x_0,0)
\end{align*}
To evaluate $K$, break it into discrete steps $0=t_0 < t_1 < \dots < t_n < t_{n+1}=T$
\[
e^{-i\hat{H}T} = e^{-i\hat{H}(t_{n+1}-t_n)} \dots e^{-i\hat{H}(t_1-t_0)}
\]
and again insert complete sets of states between each operator 
\[
K(x,x_0 ; T) = \int \left[ \prod_{r=1}^n dx_r \braket{x_{r+1}|e^{-i\hat{H}(t_{r+1}-t_r)}|x_r}      \right] \braket{x_1 | e^{-i\hat{H} t_1} | x_0}
\]
i.e integrate over all positions $x_r$ for each point in time. \\
Look at free theory first , $V(x)=0$. 
\[
K_0(x,x^\prime;t) = \braket{x | e^{-i\frac{\hat{p}^2}{2m}t} | x^\prime }
\]
Insert a complete set of momentum eigenstates $\ket{p}$ 
\[
\int \frac{dp}{2\pi} \ket{p}\bra{p} = 1
\]
Recall these are plane waves $\braket{x|p} = e^{ipx}$. Then 
\[
K_0(x,x^\prime;t) = \int \frac{dp}{2\pi} e^{-i \frac{p^2 t}{2m}} e^{ip(x-x^\prime)}
\]
Complete the square letting $p^\prime = p-\frac{m(x-x^\prime)}{t}$

\begin{align*}
K_0(x,x^\prime;t) &= e^{i\frac{m(x-x^\prime)^2}{2t}} \int_{-\infty}^\infty \frac{dp^\prime}{2\pi} \exp \left[ - \frac{i(p^\prime)^2 t}{2m} \right] \\ 
&= e^{i\frac{m(x-x^\prime)^2}{2t}} \sqrt{\frac{m}{2\pi i t}} 
\end{align*}

Note as $t\to 0$ 
\[
\lim_{t\to 0 }  K_0(x,x^\prime;t) = \delta(x-x^\prime)
\]
agreeing with $\braket{x^\prime|x} = \delta(x-x^\prime)$. For $V(\hat{x}) \neq 0$, we need small time steps. Although 
\[
e^{\hat{A}}e^{\hat{B}} = \exp\left( \hat{A} +\hat{B} + \frac{1}{2}\comm[\hat{A}]{\hat{B}} + \dots \right) \neq e^{\hat{A} + \hat{B}}
\]
For small $\eps$
\[
e^{\eps\hat{A}}e^{\eps\hat{B}} = \exp\left( \eps\hat{A} +\eps\hat{B} + O(\eps^2) \right)
\]
So 
\[
e^{\eps(\hat{A} + \hat{B} ) } = e^{\eps\hat{A}}e^{\eps\hat{B}} ( 1 + O(\eps^2) ) 
\]
\[
\Rightarrow e^{\hat{A} + \hat{B} } = \lim_{n\to\infty} \left( e^{\frac{\hat{A}}{n}} e^{\frac{\hat{B}}{n}} \right)^n
\]
hence
\begin{align*}
\braket{x_{r+1} | e^{-i \hat{H} \delta t } | x_r } &= e^{-i V(x_r) \delta t} \braket{x_{r+1} | e^{-i \frac{\hat{p}^2 \delta t}{2m}} | x_r } \\
&= \sqrt{\frac{m}{2\pi i \delta t}} \exp \left[ \frac{i}{2} m \left( \frac{x_{r+1} - x_r}{\delta t } \right)^2 \delta t - i V(x_r) \delta t \right ]
\end{align*} 
With $T=n\delta t$
\[
K(x,x_0;T) = \int \left( \prod_{r=1}^n dx_r \right) \left( \frac{m}{2\pi i \delta t} \right)^{\frac{n+1}{2}} \exp \left\{ i \sum_{r=0}^n \left[ \frac{m}{2} \left( \frac{x_{r+1} - x_r}{\delta t } \right)^2 - V(x_r) \right] \delta t \right\} 
\]
Take the limit $n\to\infty$, $\delta t \to 0$ with $T$ fixed then exponent 
\[
\sum_{r=0}^n \left[ \right] \delta t \to \int_0^T \left[ \frac{m}{2} \dot{x}^2 - V(x) \right] dt = \int_0^T L dt 
\]
$L$ is the classical Lagrangian. Hence 
\[
K(x,x_0;T) = \braket{x | e^{-i\hat{H} T} | x_0} = \int \mc{D}x \; e^{iS[x]}
\]
where $S[x] = \int_0^T L(x,\dot{x}) dt$ is the classical action and $\mc{D}x = \lim \sqrt{} \prod \sqrt{} dx_r$.\\
If we restore $\hbar$, noting $[S] = [E][t] = [\hbar]$ so 
\[
K(x,x_0;T)  = \int \mc{D}x \; e^{i\frac{S[x]}{\hbar}}
\]
In the limit $\hbar \to 0$, i.e. the classical limit, the lowest frequency dominates the integral (Riemann-Lebesgue lemma) so the minimal action dominates. This coincides with Hamilton's principle from classical dynamics. 

Consider analytically continuing, "rotating" $t \to -i\tau$. Then 
\[
\braket{x | e^{-\frac{H\tau}{\hbar}} | x_0 } = \int \mc{D}x e^{-\frac{S}{\hbar}}
\]
In the $\hbar \to 0$ limit, it is clear that the integral is dominated by the classical path $x(t)$ that minimises $S[x]$. Thus quantum mechanics corresponds to QFT in $0+1$ dimensions. 

%%%%%%%%%%%%%%%%%%%%%%%%%%%%%%%%%%%%%%%%%%%%%%%%%%%%%%%%
\subsection{Path Integral Methods} 
The main ideas here are identical in 0-dimensions. $\phi:\set{pt}\to\mbb{R}$, i.e. $\phi$ a real variable. As with imaginary time, we study the "partition function"
\[
Z = \int_\mbb{R} d\phi e^{-\frac{S(\phi)}{\hbar}}
\]
$S(\phi)$ will be a polynomial, with the highest power even for convergence. The  correlation functions or expectation values are given by 
\[
\braket{ f } = \frac{1}{Z} \int d\phi f(\phi) e^{-\frac{S(\phi)}{\hbar}}
\]
where $f(\phi)$ should not grow $\hbar$ fast to ensure convergence. 

%%%%%%%%%%%%%%%%%%%%%%%%%%%%%%%%%%%%%%%%%%%%%%%%%%%%%%%%
\subsection{Free Field Theory}
Say we have $N$ real field $\set{\phi_a}$. For a free theory, the action is quadratic. 
\[
\Rightarrow S(\phi) = \frac{1}{2} M_{ab} \phi_a \phi_b = \frac{1}{2} \phi^T M \phi 
\]
where $M$ is $N\times N$ symmetric and positive definite. Thus we can diagonalise $M= P\Lambda P^T$. Thus 
\[
Z_0 = \int d^n \phi e^{-\frac{1}{2\hbar}\phi^T M \phi} = \frac{(2\pi\hbar)^{\frac{N}{2}}}{\sqrt{\det M}}
\]
Now introduce a source $J$ such that the action becomes 
\[
S(\phi) = \frac{1}{2} \phi^T M \phi  + J\cdot\phi
\]
Complete the square, writing $\tilde{\phi} = \phi + M^{-1}J$. Then 
\[
Z_0(J) = e^{\frac{1}{2\hbar}J^T M^{-1}J}Z_0
\]
This is called the \bam{generator function}.

\begin{example}
\begin{align*}
    \braket{ \phi_a \phi_b } &= \frac{1}{Z_0} \int d^N \phi \phi_a \phi_b \exp\left[ -\frac{1}{2\hbar}\phi^T M \phi - \frac{1}{\hbar} J\cdot \phi \right] \rvert_{J=0} \\
    &= \frac{1}{Z_0} \int d^N \phi \left( -\hbar \pd{J_a} \right) \left( -\hbar \pd{J_b } \right) \exp\left[ -\frac{1}{2\hbar}\phi^T M \phi - \frac{1}{\hbar} J\cdot \phi \right] \rvert_{J=0} \\
    &= \hbar^2 \frac{\del^2}{\del J_a \del J_b} \frac{1}{Z_0} Z_0 (J) |_{J=0} \\
    &= \hbar (M^{-1})_{ab}
\end{align*}
i.e. the two point function is the inverse of the quadratic part of the action, called the \bam{propagator}. \\
An alternative method is to consider 
\begin{align*}
    M_{ca} \braket{ \phi_c \phi_b } &= \frac{1}{Z_0} \int d^N \phi M_{ca} \phi_c \phi_b e^{ -\frac{1}{2\hbar}\phi^T M} \\
    &= \frac{-\hbar}{Z_0} \int d^N\phi \phi_b \pd{\phi_a} \left( e^{ -\frac{1}{2\hbar}\phi^T M} \right) \\
    &\text{Integrate by parts, assuming boundary terms vanish} \\
    &= \frac{\hbar}{Z_0} \int d^n \phi \pd[\phi_b]{\phi_a} e^{ -\frac{1}{2\hbar}\phi^T M} \\
    &= \hbar \frac{\delta_{ab}}{Z_0} Z_0 = \hbar \delta_{ab}
\end{align*}
\end{example}
More generally, let $\mc{L}(\phi)$ be some linear operator acting on $\phi_a$, i.e. $\mc{L}(\phi) = \ell_a \phi_a$. Then 
\[
\braket{ \mc{L}_1(\phi) \dots \mc{L}_p(\phi) } = (-\hbar)^p \prod_{i=1}^p  \left( \ell_i \pd{J} \right) \frac{Z_0(J)}{Z_0} \rvert_{J=0}
\]
\begin{itemize}
    \item If $p$ is odd, then the integrand is odd in at least one $\phi_a$ and the integral vanishes. 
    \item If $p=2k$ the terms that survive $J\to0$ involves $k$ factors of $M^{-1}$.
\end{itemize}

\begin{example}[4-point function]
\[
\braket{ \phi_a \phi_b \phi_c \phi_d } = \hbar^2 \left[ (M^{-1})_{ab}(M^{-1})_{cd}+(M^{-1})_{ac}(M^{-1})_{bd}+(M^{-1})_{ad}(M^{-1})_{bc} \right]
\]
This corresponds to the Feynmann diagram
\[
\tensor*[^a_c]{=}{^b_d} + \tensor*[^a_c]{|\phantom{x}|}{^b_d} + \tensor*[^a_c]{\times}{^b_d}
\]
\end{example}
In general the number of ways of pairing $2k$ elements is $\frac{(2k)!}{2^k k!}$.

\begin{remark}
For complex $\phi_a$, $M$ is Hermitian $\Rightarrow M^{-1}$ not symmetric $\Rightarrow$ order of indices matters. We denote 
\[
\braket{ \phi_a \phi_b^\ast } = \hbar (M^{-1})_{ab} = \tensor[_a]{\rightarrow}{_b}
\]
\end{remark}

\subsection{Interacting Theory}
Take a single, real scalar field $\varphi$. Let 
\[
S(\varphi) = \frac{1}{2} m^2 \varphi^2 + \frac{\lambda}{4!}\varphi^4
\]
Take $\lambda>0$ for stability, and $m^2>0$ s.t. minimum of S is at $\varphi = 0$. Expand about the minimum of $S$, equivalent to expanding   about $\hbar = 0$. 
\begin{align*}
Z &= \int d\varphi e^{-\frac{1}{\hbar}(\frac{1}{2} m^2 \varphi^2 + \frac{\lambda}{4!}\varphi^4)} \\
&= \int d\varphi e^{\frac{m^2\varphi^2}{2\hbar}} \sum_{n=0}^\infty \frac{1}{n!} \left( -\frac{\lambda}{\hbar 4!} \right)^n \varphi^{4n} \\
&= \frac{\sqrt{2\hbar}}{m} \sum_{n=0}^\infty \frac{1}{n!} \left( - \frac{\hbar \lambda}{4! m^4} \right)^n \cdot 2^{2n} \underbrace{\int_0^\infty dx e^{-x} x^{2n+\frac{1}{2}-1}}_{\Gamma(2n+1)=\frac{(4n)! \sqrt{\pi}}{4^{2n}(2n)!}} \quad \text{letting } x=\frac{m^2 \varphi^2}{2\hbar}\\
&= \frac{\sqrt{2\pi\hbar}}{m} \sum_{n=0}^N \left( -\frac{\lambda \hbar}{m^4}\right)^n \underbrace{\frac{1}{(4!)^n n!}}_{(1)} \underbrace{\frac{(4n)!}{2^{2n}(2n)!}}_{(2)}
\end{align*}
Recall stirling's approximation $ n! \approx e^{n\log n}$. This give $(1)\cdot(2) \approx e^{n\log n} \approx n!$. Factorial growth of coefficients give zero radius of convergence and hence the above is an asymptotic series. The true function can deviate from truncated series by some transcendental function. \\
Term $(1)$ comes from expanding $\frac{\lambda}{4! \hbar} \varphi^4$ term in $e^{-S}$. \\
Term $(2)$ is the number of ways of joining $4n$ elements into distinct pairs. \\
In Feynmann diagrams , we give a propagator a factor $-\frac{\hbar}{m^2}$ and a vertex a factor $-\frac{\lambda}{\hbar}$. 
$Z$ has no $\varphi$ dependence $\Leftrightarrow$ no external legs ("vacuum diagrams"). Let $D_n$ be the set of all labelled vacuum diagrams with $n$ vertices.
\begin{itemize}
    \item $D_1 = \set{\tensor*[^1_2]{\infty}{^3_4},\tensor*[^1_2]{8}{^3_4}, \tensor*[^1_2]{}{^3_4} } \Rightarrow |D_1| = 3$
\end{itemize}
Let $G_n$ be group which permutes each of the 4 fields at each vertex $(S_4)^n$, and also permutes the $n$ vertices $S_n$. Then 
\[
|G_n| = |S_4|^n |S_n| = (4!)^n n!
\]
\begin{align*}
\frac{Z}{Z_0} &= \sum_{n=0}^N \left(-\frac{\lambda\hbar}{m^4} \right)^n \frac{|D_n|}{|G_n|} \qquad Z_0 =\frac{\sqrt{2\pi\hbar}}{m} \\
&= 1 - \frac{\hbar\lambda}{8m^4} + \frac{35}{384}\frac{\hbar^2 \lambda^2}{m^8}+\dots
\end{align*}
\[
\frac{|D_n|}{|G_n|} = \frac{\text{sum over topologically distinct graphs}}{\text{symmetry factor}} = \sum_{\Gamma} \frac{1}{S_\Gamma}
\]
$\Gamma$ a distinct graph, free from labels, and $S_\Gamma$ the number of permutations of lines and vertices leaving $\Gamma$ invariant. 

\begin{example}
\begin{align*}
&\Gamma = 8 \Rightarrow S_\Gamma = \underbrace{2^2}_{\text{left/right for top/bottom}} \times \underbrace{2}_{\text{flip up/down}}=8 \\
&\Gamma = basketball \Rightarrow S_\Gamma = 2 \times 4! = 48 \\
&\Gamma = ooo \Rightarrow S_\Gamma = 16 \\
&\Gamma = 8 8 \Rightarrow S_\Gamma = 128
\end{align*}
\end{example}
Then 
\begin{align*}
    \frac{Z}{Z_0} &= \emptyset + 8 + basketball + ooo + 8 8 + \dots \\
    &= 1 - \frac{\hbar\lambda}{8m^4} + \frac{\hbar^2}{\lambda^2 m^6} \left( \frac{1}{48} + \frac{1}{16} + \frac{1}{128} \right) + \dots 
\end{align*}
In dimensions $d>0$, loops $\Rightarrow$ integrals over momenta. \\
With external source, the generating function 
\[
Z(j) = \int d\varphi \exp -\frac{1}{\hbar} \left( \frac{1}{2}m^2 \varphi^2 + \frac{\lambda}{4!}\varphi^4 + J\varphi \right) 
\]
\[
\braket{\varphi^2} = \frac{\hbar^2}{Z(0)} \pds{J} \left. Z(J) \right \rvert_{J=0}
\]
Rule for the source term: Expansion of $Z(J)$ includes $Z(0)$ vacuum diagrams, but also those ending here in $2,4,6,8,\dots$ sources. 
\begin{align*}
\braket{\varphi^2} = \frac{\hbar^2}{Z_0} \pds{J} \left.\left( 8 + \dots + \mc{l} + \mc{l}o +\dots +x+\dots \right)\right\rvert_{J=0} \\
= 1 + 10 + \dots 
\end{align*}



\end{document}