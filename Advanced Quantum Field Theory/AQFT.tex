\documentclass{article}

\usepackage{header}
%%%%%%%%%%%%%%%%%%%%%%%%%%%%%%%%%%%%%%%%%%%%%%%%%%%%%%%%
%Preamble

\title{Advanced Quantum Field Theory Notes}
\author{Linden Disney-Hogg}
\date{January 2019}

%%%%%%%%%%%%%%%%%%%%%%%%%%%%%%%%%%%%%%%%%%%%%%%%%%%%%%%%
%%%%%%%%%%%%%%%%%%%%%%%%%%%%%%%%%%%%%%%%%%%%%%%%%%%%%%%%
\begin{document}

\maketitle
\tableofcontents

\section{Introduction}

\section{Path Integral in Quantum Mechanics}
(See Osborn's notes, section 1.2). Set $\hbar = 1 = c$ for now, though later when restoring $\hbar$ through dimensional analysis 
\[
[\hbar]=[E][t]=[p][x]
\]
Take a Hamiltonian in 1 dimension 
\[
\hat{H} = H(\hat{x},\hat{p})
\]
with 
\[
\comm[\hat{x}]{\hat{p}}=i
\]
Assume for simplicity here that 
\[
\hat{H} = \frac{\hat{p}^2}{2m} + V(\hat{x})
\]
\begin{definition}[Schr\"odinger's equation]
\[
i \pd{t} \ket{\psi(t)} = \hat{H} \ket{\psi(t)}
\]
The formal solution given $\ket{\psi(0)}$ is 
\[
\ket{\psi(t)} = e^{-i\hat{H}t} \ket{\psi(0)}
\]
\end{definition}

Consider position eigenstates $\ket{x,t}$
\[
\hat{X}(t) \ket{x,t} = x\ket{x,t} 
\]
Normalise these such that 
\[
\braket{x^\prime, t| x,t} = \delta(x^\prime - x)
\]
In the Schr\"odinger picture, states depend on time, operators are treated as constant. Hence we have a fixed basis of eigenstates $\set{\ket{x}}$ of $\hat{X}$. The wavefunction is then 
\[
\psi(x,t) = \braket{x | \psi(t)}
\]
\[
\hat{H}\psi(x,t) = \left( -\frac{1}{2m} \pds{x} +V(x) \right) \psi(x,t)
\]

\subsection{Path Integral}
We wish to express time evolution as a sum over all  trajectories, appropriately weighted. 
\[
\psi(x,t) = \braket{x | e^{-i\hat{H}t} | \psi(0)}
\]
Insert a complete set of states $1 = \int dx_0 \ket{x_0} \bra{x_0}$ 
\begin{align*}
    \psi(x,t) &= \int dx_0 \braket{x| e^{-i\hat{H}t} | x_0} \braket{x_0 | \psi(0)} \\ 
    &= \int dx_0 \, K(x,x_0 ; t) \psi(x_0,0)
\end{align*}
To evaluate $K$, break it into discrete steps $0=t_0 < t_1 < \dots < t_n < t_{n+1}=T$
\[
e^{-i\hat{H}T} = e^{-i\hat{H}(t_{n+1}-t_n)} \dots e^{-i\hat{H}(t_1-t_0)}
\]
and again insert complete sets of states between each operator 
\[
K(x,x_0 ; T) = \int \left[ \prod_{r=1}^n dx_r \braket{x_{r+1}|e^{-i\hat{H}(t_{r+1}-t_r)}|x_r}      \right] \braket{x_1 | e^{-i\hat{H} t_1} | x_0}
\]
i.e integrate over all positions $x_r$ for each point in time. \\
Look at free theory first , $V(x)=0$. 
\[
K_0(x,x^\prime;t) = \braket{x | e^{-i\frac{\hat{p}^2}{2m}t} | x^\prime }
\]
Insert a complete set of momentum eigenstates $\ket{p}$ 
\[
\int \frac{dp}{2\pi} \ket{p}\bra{p} = 1
\]
Recall these are plane waves $\braket{x|p} = e^{ipx}$. Then 
\[
K_0(x,x^\prime;t) = \int \frac{dp}{2\pi} e^{-i \frac{p^2 t}{2m}} e^{ip(x-x^\prime)}
\]
Complete the square letting $p^\prime = p-\frac{m(x-x^\prime)}{t}$

\begin{align*}
K_0(x,x^\prime;t) &= e^{i\frac{m(x-x^\prime)^2}{2t}} \int_{-\infty}^\infty \frac{dp^\prime}{2\pi} \exp \left[ - \frac{i(p^\prime)^2 t}{2m} \right] \\ 
&= e^{i\frac{m(x-x^\prime)^2}{2t}} \sqrt{\frac{m}{2\pi i t}} 
\end{align*}

Note as $t\to 0$ 
\[
\lim_{t\to 0 }  K_0(x,x^\prime;t) = \delta(x-x^\prime)
\]
agreeing with $\braket{x^\prime|x} = \delta(x-x^\prime)$. For $V(\hat{x}) \neq 0$, we need small time steps. Although 
\[
e^{\hat{A}}e^{\hat{B}} = \exp\left( \hat{A} +\hat{B} + \frac{1}{2}\comm[\hat{A}]{\hat{B}} + \dots \right) \neq e^{\hat{A} + \hat{B}}
\]
For small $\eps$
\[
e^{\eps\hat{A}}e^{\eps\hat{B}} = \exp\left( \eps\hat{A} +\eps\hat{B} + O(\eps^2) \right)
\]
So 
\[
e^{\eps(\hat{A} + \hat{B} ) } = e^{\eps\hat{A}}e^{\eps\hat{B}} ( 1 + O(\eps^2) ) 
\]
\[
\Rightarrow e^{\hat{A} + \hat{B} } = \lim_{n\to\infty} \left( e^{\frac{\hat{A}}{n}} e^{\frac{\hat{B}}{n}} \right)^n
\]
hence
\begin{align*}
\braket{x_{r+1} | e^{-i \hat{H} \delta t } | x_r } &= e^{-i V(x_r) \delta t} \braket{x_{r+1} | e^{-i \frac{\hat{p}^2 \delta t}{2m}} | x_r } \\
&= \sqrt{\frac{m}{2\pi i \delta t}} \exp \left[ \frac{i}{2} m \left( \frac{x_{r+1} - x_r}{\delta t } \right)^2 \delta t - i V(x_r) \delta t \right ]
\end{align*} 
With $T=n\delta t$
\[
K(x,x_0;T) = \int \left( \prod_{r=1}^n dx_r \right) \left( \frac{m}{2\pi i \delta t} \right)^{\frac{n+1}{2}} \exp \left\{ i \sum_{r=0}^n \left[ \frac{m}{2} \left( \frac{x_{r+1} - x_r}{\delta t } \right)^2 - V(x_r) \right] \delta t \right\} 
\]
Take the limit $n\to\infty$, $\delta t \to 0$ with $T$ fixed then exponent 
\[
\sum_{r=0}^n \left[ \right] \delta t \to \int_0^T \left[ \frac{m}{2} \dot{x}^2 - V(x) \right] dt = \int_0^T L dt 
\]
$L$ is the classical Lagrangian. Hence 
\[
K(x,x_0;T) = \braket{x | e^{-i\hat{H} T} | x_0} = \int \mc{D}x \; e^{iS[x]}
\]
where $S[x] = \int_0^T L(x,\dot{x}) dt$ is the classical action and $\mc{D}x = \lim \sqrt{} \prod \sqrt{} dx_r$.\\
If we restore $\hbar$, noting $[S] = [E][t] = [\hbar]$ so 
\[
K(x,x_0;T)  = \int \mc{D}x \; e^{i\frac{S[x]}{\hbar}}
\]
In the limit $\hbar \to 0$, i.e. the classical limit, the lowest frequency dominates the integral (Riemann-Lebesgue lemma) so the minimal action dominates. This coincides with Hamilton's principle from classical dynamics. 
\end{document}