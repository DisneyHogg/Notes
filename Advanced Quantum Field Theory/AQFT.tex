\documentclass{article}

\usepackage{header}
%%%%%%%%%%%%%%%%%%%%%%%%%%%%%%%%%%%%%%%%%%%%%%%%%%%%%%%%
%Preamble

\title{Advanced Quantum Field Theory Notes}
\author{Linden Disney-Hogg}
\date{January 2019}

%%%%%%%%%%%%%%%%%%%%%%%%%%%%%%%%%%%%%%%%%%%%%%%%%%%%%%%%
%%%%%%%%%%%%%%%%%%%%%%%%%%%%%%%%%%%%%%%%%%%%%%%%%%%%%%%%
\begin{document}

\maketitle
\tableofcontents

%%%%%%%%%%%%%%%%%%%%%%%%%%%%%%%%%%%%%%%%%%%%%%%%%%%%%%%%
%%%%%%%%%%%%%%%%%%%%%%%%%%%%%%%%%%%%%%%%%%%%%%%%%%%%%%%%
\section{Introduction}
%%%%%%%%%%%%%%%%%%%%%%%%%%%%%%%%%%%%%%%%%%%%%%%%%%%%%%%%
\section{Path Integral in Quantum Mechanics}
(See Osborn's notes, section 1.2). Set $\hbar = 1 = c$ for now, though later when restoring $\hbar$ through dimensional analysis 
\[
[\hbar]=[E][t]=[p][x]
\]
Take a Hamiltonian in 1 dimension 
\[
\hat{H} = H(\hat{x},\hat{p})
\]
with 
\[
\comm[\hat{x}]{\hat{p}}=i
\]
Assume for simplicity here that 
\[
\hat{H} = \frac{\hat{p}^2}{2m} + V(\hat{x})
\]
\begin{definition}[Schr\"odinger's equation]
\[
i \pd{t} \ket{\psi(t)} = \hat{H} \ket{\psi(t)}
\]
The formal solution given $\ket{\psi(0)}$ is 
\[
\ket{\psi(t)} = e^{-i\hat{H}t} \ket{\psi(0)}
\]
\end{definition}

Consider position eigenstates $\ket{x,t}$
\[
\hat{X}(t) \ket{x,t} = x\ket{x,t} 
\]
Normalise these such that 
\[
\braket{x^\prime, t| x,t} = \delta(x^\prime - x)
\]
In the Schr\"odinger picture, states depend on time, operators are treated as constant. Hence we have a fixed basis of eigenstates $\set{\ket{x}}$ of $\hat{X}$. The wavefunction is then 
\[
\psi(x,t) = \braket{x | \psi(t)}
\]
\[
\hat{H}\psi(x,t) = \left( -\frac{1}{2m} \pds{x} +V(x) \right) \psi(x,t)
\]

%%%%%%%%%%%%%%%%%%%%%%%%%%%%%%%%%%%%%%%%%%%%%%%%%%%%%%%%
\subsection{Path Integral}
We wish to express time evolution as a sum over all  trajectories, appropriately weighted. 
\[
\psi(x,t) = \braket{x | e^{-i\hat{H}t} | \psi(0)}
\]
Insert a complete set of states $1 = \int dx_0 \ket{x_0} \bra{x_0}$ 
\begin{align*}
    \psi(x,t) &= \int dx_0 \braket{x| e^{-i\hat{H}t} | x_0} \braket{x_0 | \psi(0)} \\ 
    &= \int dx_0 \, K(x,x_0 ; t) \psi(x_0,0)
\end{align*}
To evaluate $K$, break it into discrete steps $0=t_0 < t_1 < \dots < t_n < t_{n+1}=T$
\[
e^{-i\hat{H}T} = e^{-i\hat{H}(t_{n+1}-t_n)} \dots e^{-i\hat{H}(t_1-t_0)}
\]
and again insert complete sets of states between each operator 
\[
K(x,x_0 ; T) = \int \left[ \prod_{r=1}^n dx_r \braket{x_{r+1}|e^{-i\hat{H}(t_{r+1}-t_r)}|x_r}      \right] \braket{x_1 | e^{-i\hat{H} t_1} | x_0}
\]
i.e integrate over all positions $x_r$ for each point in time. \\
Look at free theory first , $V(x)=0$. 
\[
K_0(x,x^\prime;t) = \braket{x | e^{-i\frac{\hat{p}^2}{2m}t} | x^\prime }
\]
Insert a complete set of momentum eigenstates $\ket{p}$ 
\[
\int \frac{dp}{2\pi} \ket{p}\bra{p} = 1
\]
Recall these are plane waves $\braket{x|p} = e^{ipx}$. Then 
\[
K_0(x,x^\prime;t) = \int \frac{dp}{2\pi} e^{-i \frac{p^2 t}{2m}} e^{ip(x-x^\prime)}
\]
Complete the square letting $p^\prime = p-\frac{m(x-x^\prime)}{t}$

\begin{align*}
K_0(x,x^\prime;t) &= e^{i\frac{m(x-x^\prime)^2}{2t}} \int_{-\infty}^\infty \frac{dp^\prime}{2\pi} \exp \left[ - \frac{i(p^\prime)^2 t}{2m} \right] \\ 
&= e^{i\frac{m(x-x^\prime)^2}{2t}} \sqrt{\frac{m}{2\pi i t}} 
\end{align*}

Note as $t\to 0$ 
\[
\lim_{t\to 0 }  K_0(x,x^\prime;t) = \delta(x-x^\prime)
\]
agreeing with $\braket{x^\prime|x} = \delta(x-x^\prime)$. For $V(\hat{x}) \neq 0$, we need small time steps. Although 
\[
e^{\hat{A}}e^{\hat{B}} = \exp\left( \hat{A} +\hat{B} + \frac{1}{2}\comm[\hat{A}]{\hat{B}} + \dots \right) \neq e^{\hat{A} + \hat{B}}
\]
For small $\eps$
\[
e^{\eps\hat{A}}e^{\eps\hat{B}} = \exp\left( \eps\hat{A} +\eps\hat{B} + O(\eps^2) \right)
\]
So 
\[
e^{\eps(\hat{A} + \hat{B} ) } = e^{\eps\hat{A}}e^{\eps\hat{B}} ( 1 + O(\eps^2) ) 
\]
\[
\Rightarrow e^{\hat{A} + \hat{B} } = \lim_{n\to\infty} \left( e^{\frac{\hat{A}}{n}} e^{\frac{\hat{B}}{n}} \right)^n
\]
hence
\begin{align*}
\braket{x_{r+1} | e^{-i \hat{H} \delta t } | x_r } &= e^{-i V(x_r) \delta t} \braket{x_{r+1} | e^{-i \frac{\hat{p}^2 \delta t}{2m}} | x_r } \\
&= \sqrt{\frac{m}{2\pi i \delta t}} \exp \left[ \frac{i}{2} m \left( \frac{x_{r+1} - x_r}{\delta t } \right)^2 \delta t - i V(x_r) \delta t \right ]
\end{align*} 
With $T=n\delta t$
\[
K(x,x_0;T) = \int \left( \prod_{r=1}^n dx_r \right) \left( \frac{m}{2\pi i \delta t} \right)^{\frac{n+1}{2}} \exp \left\{ i \sum_{r=0}^n \left[ \frac{m}{2} \left( \frac{x_{r+1} - x_r}{\delta t } \right)^2 - V(x_r) \right] \delta t \right\} 
\]
Take the limit $n\to\infty$, $\delta t \to 0$ with $T$ fixed then exponent 
\[
\sum_{r=0}^n \left[ \right] \delta t \to \int_0^T \left[ \frac{m}{2} \dot{x}^2 - V(x) \right] dt = \int_0^T L dt 
\]
$L$ is the classical Lagrangian. Hence 
\[
K(x,x_0;T) = \braket{x | e^{-i\hat{H} T} | x_0} = \int \mc{D}x \; e^{iS[x]}
\]
where $S[x] = \int_0^T L(x,\dot{x}) dt$ is the classical action and $\mc{D}x = \lim \sqrt{} \prod \sqrt{} dx_r$.\\
If we restore $\hbar$, noting $[S] = [E][t] = [\hbar]$ so 
\[
K(x,x_0;T)  = \int \mc{D}x \; e^{i\frac{S[x]}{\hbar}}
\]
In the limit $\hbar \to 0$, i.e. the classical limit, the lowest frequency dominates the integral (Riemann-Lebesgue lemma) so the minimal action dominates. This coincides with Hamilton's principle from classical dynamics. 

Consider analytically continuing, "rotating" $t \to -i\tau$. Then 
\[
\braket{x | e^{-\frac{H\tau}{\hbar}} | x_0 } = \int \mc{D}x e^{-\frac{S}{\hbar}}
\]
In the $\hbar \to 0$ limit, it is clear that the integral is dominated by the classical path $x(t)$ that minimises $S[x]$. Thus quantum mechanics corresponds to QFT in $0+1$ dimensions. 

%%%%%%%%%%%%%%%%%%%%%%%%%%%%%%%%%%%%%%%%%%%%%%%%%%%%%%%%
\subsection{Path Integral Methods} 
The main ideas here are identical in 0-dimensions. $\phi:\set{pt}\to\mbb{R}$, i.e. $\phi$ a real variable. As with imaginary time, we study the "partition function"
\[
Z = \int_\mbb{R} d\phi e^{-\frac{S(\phi)}{\hbar}}
\]
$S(\phi)$ will be a polynomial, with the highest power even for convergence. The  correlation functions or expectation values are given by 
\[
\braket{ f } = \frac{1}{Z} \int d\phi f(\phi) e^{-\frac{S(\phi)}{\hbar}}
\]
where $f(\phi)$ should not grow $\hbar$ fast to ensure convergence. 

%%%%%%%%%%%%%%%%%%%%%%%%%%%%%%%%%%%%%%%%%%%%%%%%%%%%%%%%
\subsubsection*{Free Field Theory}
Say we have $N$ real field $\set{\phi_a}$. For a free theory, the action is quadratic. 
\[
\Rightarrow S(\phi) = \frac{1}{2} M_{ab} \phi_a \phi_b = \frac{1}{2} \phi^T M \phi 
\]
where $M$ is $N\times N$ symmetric and positive definite. Thus we can diagonalise $M= P\Lambda P^T$. Thus 
\[
Z_0 = \int d^n \phi e^{-\frac{1}{2\hbar}\phi^T M \phi} = \frac{(2\pi\hbar)^{\frac{N}{2}}}{\sqrt{\det M}}
\]
Now introduce a source $J$ such that the action becomes 
\[
S(\phi) = \frac{1}{2} \phi^T M \phi  + J\cdot\phi
\]
Complete the square, writing $\tilde{\phi} = \phi + M^{-1}J$. Then 
\[
Z_0(J) = e^{\frac{1}{2\hbar}J^T M^{-1}J}Z_0
\]
This is called the \bam{generator function}.

\begin{example}
\begin{align*}
    \braket{ \phi_a \phi_b } &= \frac{1}{Z_0} \int d^N \phi \phi_a \phi_b \exp\left[ -\frac{1}{2\hbar}\phi^T M \phi - \frac{1}{\hbar} J\cdot \phi \right] \rvert_{J=0} \\
    &= \frac{1}{Z_0} \int d^N \phi \left( -\hbar \pd{J_a} \right) \left( -\hbar \pd{J_b } \right) \exp\left[ -\frac{1}{2\hbar}\phi^T M \phi - \frac{1}{\hbar} J\cdot \phi \right] \rvert_{J=0} \\
    &= \hbar^2 \frac{\del^2}{\del J_a \del J_b} \frac{1}{Z_0} Z_0 (J) |_{J=0} \\
    &= \hbar (M^{-1})_{ab}
\end{align*}
i.e. the two point function is the inverse of the quadratic part of the action, called the \bam{propagator}. \\
An alternative method is to consider 
\begin{align*}
    M_{ca} \braket{ \phi_c \phi_b } &= \frac{1}{Z_0} \int d^N \phi M_{ca} \phi_c \phi_b e^{ -\frac{1}{2\hbar}\phi^T M} \\
    &= \frac{-\hbar}{Z_0} \int d^N\phi \phi_b \pd{\phi_a} \left( e^{ -\frac{1}{2\hbar}\phi^T M} \right) \\
    &\text{Integrate by parts, assuming boundary terms vanish} \\
    &= \frac{\hbar}{Z_0} \int d^n \phi \pd[\phi_b]{\phi_a} e^{ -\frac{1}{2\hbar}\phi^T M} \\
    &= \hbar \frac{\delta_{ab}}{Z_0} Z_0 = \hbar \delta_{ab}
\end{align*}
\end{example}
More generally, let $\mc{L}(\phi)$ be some linear operator acting on $\phi_a$, i.e. $\mc{L}(\phi) = \ell_a \phi_a$. Then 
\[
\braket{ \mc{L}_1(\phi) \dots \mc{L}_p(\phi) } = (-\hbar)^p \prod_{i=1}^p  \left( \ell_i \pd{J} \right) \frac{Z_0(J)}{Z_0} \rvert_{J=0}
\]
\begin{itemize}
    \item If $p$ is odd, then the integrand is odd in at least one $\phi_a$ and the integral vanishes. 
    \item If $p=2k$ the terms that survive $J\to0$ involves $k$ factors of $M^{-1}$.
\end{itemize}

\begin{example}[4-point function]
\[
\braket{ \phi_a \phi_b \phi_c \phi_d } = \hbar^2 \left[ (M^{-1})_{ab}(M^{-1})_{cd}+(M^{-1})_{ac}(M^{-1})_{bd}+(M^{-1})_{ad}(M^{-1})_{bc} \right]
\]
This corresponds to the Feynmann diagram
\[
\tensor*[^a_c]{=}{^b_d} + \tensor*[^a_c]{|\phantom{x}|}{^b_d} + \tensor*[^a_c]{\times}{^b_d}
\]
\end{example}
In general the number of ways of pairing $2k$ elements is $\frac{(2k)!}{2^k k!}$.

\begin{remark}
For complex $\phi_a$, $M$ is Hermitian $\Rightarrow M^{-1}$ not symmetric $\Rightarrow$ order of indices matters. We denote 
\[
\braket{ \phi_a \phi_b^\ast } = \hbar (M^{-1})_{ab} = \tensor[_a]{\rightarrow}{_b}
\]
\end{remark}

%%%%%%%%%%%%%%%%%%%%%%%%%%%%%%%%%%%%%%%%%%%%%%%%%%%%%%%%
\subsubsection*{Interacting Theory}
Take a single, real scalar field $\varphi$. Let 
\[
S(\varphi) = \frac{1}{2} m^2 \varphi^2 + \frac{\lambda}{4!}\varphi^4
\]
Take $\lambda>0$ for stability, and $m^2>0$ s.t. minimum of S is at $\varphi = 0$. Expand about the minimum of $S$, equivalent to expanding   about $\hbar = 0$. 
\begin{align*}
Z &= \int d\varphi e^{-\frac{1}{\hbar}(\frac{1}{2} m^2 \varphi^2 + \frac{\lambda}{4!}\varphi^4)} \\
&= \int d\varphi e^{\frac{m^2\varphi^2}{2\hbar}} \sum_{n=0}^\infty \frac{1}{n!} \left( -\frac{\lambda}{\hbar 4!} \right)^n \varphi^{4n} \\
&= \frac{\sqrt{2\hbar}}{m} \sum_{n=0}^\infty \frac{1}{n!} \left( - \frac{\hbar \lambda}{4! m^4} \right)^n \cdot 2^{2n} \underbrace{\int_0^\infty dx e^{-x} x^{2n+\frac{1}{2}-1}}_{\Gamma(2n+1)=\frac{(4n)! \sqrt{\pi}}{4^{2n}(2n)!}} \quad \text{letting } x=\frac{m^2 \varphi^2}{2\hbar}\\
&= \frac{\sqrt{2\pi\hbar}}{m} \sum_{n=0}^N \left( -\frac{\lambda \hbar}{m^4}\right)^n \underbrace{\frac{1}{(4!)^n n!}}_{(1)} \underbrace{\frac{(4n)!}{2^{2n}(2n)!}}_{(2)}
\end{align*}
Recall stirling's approximation $ n! \approx e^{n\log n}$. This give $(1)\cdot(2) \approx e^{n\log n} \approx n!$. Factorial growth of coefficients give zero radius of convergence and hence the above is an asymptotic series. The true function can deviate from truncated series by some transcendental function. \\
Term $(1)$ comes from expanding $\frac{\lambda}{4! \hbar} \varphi^4$ term in $e^{-S}$. \\
Term $(2)$ is the number of ways of joining $4n$ elements into distinct pairs. \\
In Feynmann diagrams , we give a propagator a factor $-\frac{\hbar}{m^2}$ and a vertex a factor $-\frac{\lambda}{\hbar}$. 
$Z$ has no $\varphi$ dependence $\Leftrightarrow$ no external legs ("vacuum diagrams"). Let $D_n$ be the set of all labelled vacuum diagrams with $n$ vertices.
\begin{itemize}
    \item $D_1 = \set{\tensor*[^1_2]{\infty}{^3_4},\tensor*[^1_2]{8}{^3_4}, \tensor*[^1_2]{}{^3_4} } \Rightarrow |D_1| = 3$
\end{itemize}
Let $G_n$ be group which permutes each of the 4 fields at each vertex $(S_4)^n$, and also permutes the $n$ vertices $S_n$. Then 
\[
|G_n| = |S_4|^n |S_n| = (4!)^n n!
\]
\begin{align*}
\frac{Z}{Z_0} &= \sum_{n=0}^N \left(-\frac{\lambda\hbar}{m^4} \right)^n \frac{|D_n|}{|G_n|} \qquad Z_0 =\frac{\sqrt{2\pi\hbar}}{m} \\
&= 1 - \frac{\hbar\lambda}{8m^4} + \frac{35}{384}\frac{\hbar^2 \lambda^2}{m^8}+\dots
\end{align*}
\[
\frac{|D_n|}{|G_n|} = \frac{\text{sum over topologically distinct graphs}}{\text{symmetry factor}} = \sum_{\Gamma} \frac{1}{S_\Gamma}
\]
$\Gamma$ a distinct graph, free from labels, and $S_\Gamma$ the number of permutations of lines and vertices leaving $\Gamma$ invariant. 

\begin{example}
\begin{align*}
&\Gamma = 8 \Rightarrow S_\Gamma = \underbrace{2^2}_{\text{left/right for top/bottom}} \times \underbrace{2}_{\text{flip up/down}}=8 \\
&\Gamma = basketball \Rightarrow S_\Gamma = 2 \times 4! = 48 \\
&\Gamma = ooo \Rightarrow S_\Gamma = 16 \\
&\Gamma = 8 8 \Rightarrow S_\Gamma = 128
\end{align*}
\end{example}
Then 
\begin{align*}
    \frac{Z}{Z_0} &= \emptyset + 8 + basketball + ooo + 8 8 + \dots \\
    &= 1 - \frac{\hbar\lambda}{8m^4} + \frac{\hbar^2}{\lambda^2 m^6} \left( \frac{1}{48} + \frac{1}{16} + \frac{1}{128} \right) + \dots 
\end{align*}
In dimensions $d>0$, loops $\Rightarrow$ integrals over momenta. \\
With external source, the generating function 
\[
Z(j) = \int d\varphi \exp -\frac{1}{\hbar} \left( \frac{1}{2}m^2 \varphi^2 + \frac{\lambda}{4!}\varphi^4 + J\varphi \right) 
\]
\[
\braket{\varphi^2} = \frac{\hbar^2}{Z(0)} \pds{J} \left. Z(J) \right \rvert_{J=0}
\]
Rule for the source term: Expansion of $Z(J)$ includes $Z(0)$ vacuum diagrams, but also those ending here in $2,4,6,8,\dots$ sources. 
\begin{align*}
\braket{\varphi^2} &= \frac{\hbar^2}{Z_0} \pds{J} \left.\left( 8 + \dots + \mc{l} + \mc{l}o +\dots +x+\dots \right)\right\rvert_{J=0} \\
&= \frac{1}{Z(0} (1 + 10 + \dots + \text{disconnected vacuum bubble diagrams}) \\
&= \frac{1}{Z(0)} (\text{non bubble diagrams})\underbrace{(\text{vacuum bubbles})}_{=Z(0)} \\
&= (\text{non bubble diagrams})
\end{align*}
Also note 
\eq{
\braket{\varphi^4} = ( || + x + \dots ) 
}

%%%%%%%%%%%%%%%%%%%%%%%%%%%%%%%%%%%%%%%%%%%%%%%%%%%%%%%%
\subsection{Effective Actions}
The Wilsonian effective action is W, with 
\[
Z = e^{-\frac{W}{\hbar}}
\]
as $\sum(\text{all vacuum diagrams}) = \exp\left( -\frac{1}{\hbar} \sum (\text{connected diagrams}) \right)$.
Any diagram $D$ is the product of connected diagrams $C_I$ 
\[
D = \frac{1}{S_D} \prod_{I} (C_I)^{n_I}
\]
$I$ is the index over connected diagrams. Note $C_I$ \emph{includes} its own symmetry factor. 
\[
C_D = \text{number of ways of rearranging identical } C_I
\]
\[
\Rightarrow S_D = \prod_I (n_I)^1
\]
e.g. $D = 8 8 \infty 8 = 8 8 8 \infty \Rightarrow C_D = 3! \cdot 1! = 6$. 
\eq{
\Rightarrow \frac{Z}{Z_0} = \sum_{\set{n_I}} D &= \sum_{\set{n_I}} \prod_I \frac{1}{n_I!}(C_I)^{n_I} \\
&= \prod_I \sum_{n_I} \frac{1}{n_I!}(C_I)^{n_I} \\
&= \exp\left(\sum_I C_I \right) = e^{-\frac{W-W_0}{\hbar}}
}
Hence 
\[
W=W_0 - \hbar \sum_I C_I \quad \text{sum of connected diagrams}
\]
\\
Why is $W$ an "effective action"?. Consider a theory with two real scalar fields $\varphi, \chi$ 
\[
S(\varphi,\chi) = \frac{1}{2}m^2 \varphi^2 + \frac{1}{2}M^2 \chi^2 + \underbrace{\frac{1}{4} \varphi^2 \chi^2}_{\text{no factoral}} 
\]
The Feynmann rules are 
\begin{itemize}
    \item $\frac{\hbar}{m^2}$ for $\varphi$ propagator 
    \item $\frac{\hbar}{M^2}$ for $\chi$ propagator 
    \item $-\frac{\lambda}{\hbar}$ at vertex.
\end{itemize}
Then 
\eq{
-\frac{W}{\hbar} &= (\text{sum of bubble in $\varphi,\chi$}) \\
&= -\frac{\hbar\lambda}{4m^2M^2} + \frac{\hbar^2\lambda^2}{m^4 M^4} \left( \frac{1}{16} + \frac{1}{16} + \frac{1}{8} \right) + \dots
}
\eq{
\braket{\varphi^2} = \frac{\hbar}{m^2}+ \frac{\hbar^2\lambda}{2m^4M^2} + \frac{\hbar^3\lambda^2}{m^6 M^4} \left[ \frac{1}{4} + \frac{1}{4} + \frac{1}{2} \right] + \dots 
}
Say we want to remove explicit dependence on $\chi$, e.g. say $\chi$ is very massive ($M\gg m$). Then "integrate out" the heavy field. Define $W$ such that 
\eq{
e^{-\frac{W(\varphi)}{\hbar}} = \int d\chi e^{-\frac{S(\varphi,\chi)}{\hbar}}
}
The $\varphi^2\chi^2$ acts as a source term for $\chi^2$. \\
The correlation functions can then be expressed as 
\eq{
\braket{f(\varphi)} &= \frac{1}{Z} \int d\varphi d\chi f(\varphi) e^{-\frac{S(\varphi,\chi)}{\hbar}} \\
&= \frac{1}{Z} \int d\varphi f(\varphi) e^{-\frac{W(\varphi)}{\hbar}}
}

In our example, the integral can be done exactly, as 
\eq{
\int d\chi e^{-\frac{S(\varphi,\chi)}{\hbar}} = e^{-\frac{1}{2}m^2\varphi^2} \sqrt{\frac{2\pi\hbar}{M^2 + \frac{1}{2}\lambda\varphi^2}} \\
W(\varphi) = \frac{1}{2}m^2 \varphi^2 + \frac{\hbar}{2}\log\left(1+\frac{\lambda}{2M^2} \varphi^2 \right) + \underbrace{\frac{\hbar}{2} \log \frac{M^2}{2\pi\hbar}}_{\text{const so ignored}}
}
The constant piece won't affect QFT correlation functions. However, this term will contribute to the cosmological constant of the universe. Why is the observed cosmological constant $\Lambda$ so small? \\ 
Expand the log, 
\eq{
W(\varphi) &= \left( \frac{1}{2}m^2 + \frac{\hbar\lambda}{4M^2}\right) \varphi^2 - \frac{\hbar\lambda^2}{16M^4}\varphi^4 + \frac{\hbar\lambda^3}{48M^6} \varphi^6 + \dots \\
&= \frac{1}{2}m_{eff}^2\varphi^2 + \frac{\lambda_4}{4!} \varphi^4 + \frac{\lambda_6}{6!} \varphi^6 + \dots +\frac{\lambda_{2k}}{(2k)!} \varphi^{2k} + \dots
}
where 
\eq{
m_{eff} &= m^2 + \frac{\hbar\lambda}{2M^2} \\
\lambda_{2k} &= (-1)^{k+1}\hbar \frac{(2k)!}{2^{k+1}k} \frac{\lambda^k}{m^{2k}}
}
Note all the new terms are
\begin{itemize}
    \item  $\propto \hbar$ and hence are quantum effects.
    \item  Suppressed by $\sim \frac{1}{M^{2k}}$
\end{itemize}
Usually we find $W(\varphi)$ perturbatively. Again, we treat the $\frac{\lambda}{4} \varphi^2\chi^2$ term as a source, and use the Feynmann rules for the $\chi$ integration 
\begin{itemize}
    \item $\frac{\hbar}{M^2}$ for a $\chi$ propagator. 
    \item $-\frac{\lambda}{2}\frac{\varphi^2}{\hbar}$ for each source. 
\end{itemize}
Then the effective action is (sum of connected components)($\times -\hbar$) so 
\eq{
W(\varphi) &\sim -\hbar \left[ \text{source} + \text{1 source loop} + \dots \right] \\
&= \frac{m^2 \varphi^2}{2} + \frac{1}{2}\frac{\hbar\lambda}{2M^2} \varphi^2 - \frac{1}{4} \frac{\hbar\lambda^2}{4M^2}\varphi^4 + \dots 
}
so 
\eq{
\braket{\varphi^2} = \frac{1}{Z} \int d\varphi \varphi^2 e^{-\frac{W}{\hbar}} = \frac{h}{m_{eff}} - \frac{\lambda_4 \hbar^2}{2m_{eff}^6} + \dots
}
as before. 

%%%%%%%%%%%%%%%%%%%%%%%%%%%%%%%%%%%%%%%%%%%%%%%%%%%%%%%%
\subsubsection*{Quantum Effective action}
\begin{itemize}
    \item $W(J)$ is the Wilson effective action, is the average over quantum fluctuation of some degrees of freedom. 
    \begin{example}
    \eq{
    e^{-\frac{W(\phi)}{\hbar}} = \int d\chi e^{-\frac{S(\phi,\chi)}{\hbar}}
    }
    \end{example}
    \item $\Gamma(\Phi)$ is the quantum effective action averaged over all quantum fluctuations. 
\end{itemize}

Define an average field in the presesnce of some source $J$ as 
\eq{
\Phi = \pd[W]{J} &= -\frac{\hbar}{Z(J)} \pd{J} \int d\varphi e^{-\frac{S(\varphi) + J\varphi}{\hbar}} \\
&= braket{\varphi}_J
}

Legendre transform from $W(J) \to \Gamma(\Phi)$ is then 
\eq{
\Gamma(\Phi) = W(J) - \Phi
}
Note 
\eq{
\pd[\Gamma]{\Phi} &= \pd[W]{\Phi} - J - \Phi \pd[J]{\Phi} \\
&= \pd[W]{J} \pd[J]{\Phi} - J - \Phi  \pd[J]{\Phi} = -J \\
}
and 
\eq{
\pd[\Gamma]{\Phi} |_{J=0} = 0 
}i.e. in absence of a source, $J=0$, $\Phi=\braket{\varphi}$ corresponds to an extremum of $\Gamma(\Phi)$. In higher dimensions 
\eq{
\Gamma(\Phi) = \int d^d x \left[ - V(\Phi) \highlight{-} \frac{1}{2} \del^\mu \Phi \del_\mu \Phi + \dots \right] 
}
The first term $V(\Phi)$ is the \bam{effective potential}. 

\subsubsection*{analogy with statistical mechanics of magnetic systems}
Say we have a spin $s(x)$, an external magnetic field $h$ and Hamiltonian $\mc{H}$. Then 
\eq{
Z(h) = e^{-\beta F(h)} = \int \mc{D}s \exp \left[ -\beta \int d^x (\mc{H}(s) - hs)\right]
}
The magnetisation is 
\eq{
M = -\pd[F]{h} = \int d^d x \braket{s(x)}
}
and the Legendre transform is 
\eq{
G &= F + hm \\
\pd[G]{M} &= h
}
giving the Gibbs free energy from the Helmholtz free energy. When $h\to0$ the equilibrium $M$ is given by the minimum of $G$

%%%%%%%%%%%%%%%%%%%%%%%%%%%%%%%%%%%%%%%%%%%%%%%%%%%%%%%%
\subsection{Perturbative Calculation of Gamma}
Treat $\Phi$ as we did $\varphi$. A quantum path integral over $\Phi$
\begin{align} \label{eq:AQFT:1}
e^{-\frac{W_\Gamma(J)}{g}} = \int d\Phi e^{-\frac{\Gamma(\Phi)+J\Phi}{g}}
\end{align}
where $g$ is a fictional new Plancks constant. $W_\Gamma(J)$ is the sum of connected diagrams with $\Phi$ propagators and vertices. Expanding in $g$ gives 
\eq{
W_\Gamma(J) = \sum_{l=0}^\infty g^l W_\Gamma^{(l)} (J)
}
where each term corresponds to $l$ loops. \bam{Tree diagrams} are those composing $W^{(0)}_\Gamma(J)$. In the $g\to 0$ limit only these contribute i.e. 
\eq{
\lim_{g \to 0} W_\Gamma(J) = W^{(0)}_\Gamma(J)
}
Also, as $g\to 0 $, integral \ref{eq:AQFT:1} over $\Phi$ will be dominated by minima of the exponent i.e $\Phi$ such that 
\eq{
\pd[\Gamma]{\Phi} = -J \quad \text{(Steepest descent)} \\
\Rightarrow W_\Gamma(J) = W^{(0)}_\Gamma(J) = \Gamma(\Phi) + J\Phi = W(J)
}
Hence the sum of the connected diagrams $W(J)$ (action $S(\varphi)+J\varphi$) can be otained as the sum of \emph{tree diagrams} $W_\Gamma^{(0)}(J)$ (action $\Gamma(\Phi)+J\Phi$). 

\begin{definition}[Bridge]
An edge of a connected graph is a \bam{bridge} if cutting it would make the graph disconnected. 
\end{definition}

\begin{definition}
A connected graph is \bam{one particle irreducible (1PI)} if it contains no bridges. 
\end{definition}

$\Gamma(\Phi)$ sums the 1PI grphs of the theory with action $S(\Phi)$ yielding many vertices. Then correlation functions van be found using tree graphs with vertices from $\Gamma(\Phi)$

\begin{example}
For N component $\varphi$ 
\eq{
-\hbar \frac{\del^2 W}{\del J_a \del J_b} &= \braket{\varphi_a \varphi_b}_J^\text{connected} \\
&= \hbar \left( \frac{\del^2 \Gamma}{\del \Phi_a \del \Phi_b} \right)^{-1} = \hbar \times \text{inverse of quadratic part of } \Gamma
\braket{\varphi_a \varphi_b}_J^\text{connected} = \braket{\varphi_a \varphi_b} - \braket{\varphi_a}\braket{\varphi_b}
}
\end{example}

%%%%%%%%%%%%%%%%%%%%%%%%%%%%%%%%%%%%%%%%%%%%%%%%%%%%%%%%
\subsection{Fermions}
Let \bam{Grassman numbers} be n elements $\set{\theta_a}$ obeying 
\eq{
\theta_a \theta_b = - \theta_b \theta_a \\
\forall \varphi_n \text{ complex scalar } \theta_a \varphi_b = \varphi_b \theta_a
}
Note $\theta_a^2 = 0$, which implies any function of of $n$ Grassman variables can be written in finite form 
\eq{
F(\theta) = f + \varphi_a \theta_a + \frac{1}{2!} g_{ab} \theta_a \theta_b +\dots + \frac{1}{n!} h_{a_1 \dots a_n} \theta_{a_1} \dots \theta_{a_n}
}
where coeeficients $\varphi,g,\dots,h$ are totally antisymmetric under interchange of indices. Differentiation defined to anticommute as follows 
\eq{
\pd{\theta_a} \theta_b + \theta_b \pd{\theta_a} = \delta_{ab}
}
Integration defined such that for a single Grassmann variable $\theta$, with $F(\theta) = f + \varphi \theta$, we require "translational invariance" of the integrate so 
\eq{
\int d\theta (\theta+ \eta) = \int d\theta \theta \\
\Rightarrow \int d\theta = 0
}
and so choose the normalisation such that 
\eq{
\int d\theta \theta 1
}
These rules (the \bam{Berezin rules}) give 
\eq{
\int d\theta \pd{\theta} F(\theta) = 0 
}
For $n$ Grassmann variabls the only non vanishing integrals involve exactly 1 power of each integration variable. 
\eq{
\int d^n\theta \theta_1 \dots \theta_n = \int d\theta_n \dots d\theta_1 \theta_1 \dots \theta_n = 1
}
In general, 
\eq{
\int d^n \theta \theta_{a_1} \dots \theta_{a_n} = \eps^{a_1 \dots a_n} 
}
say we had a change of variables $\theta_a^\prime = A_{ab} \theta_b$. Then 
\eq{
\int d^n \theta \theta_{a_1}^\prime \dots \theta_{a_n}^\prime  &= A_{a_1 b_1} \dots A_{a_n b_n}\int d^n \theta \theta_{b_1} \dots \theta_{b_n} \\
&= A_{a_1 b_1} \dots A_{a_n b_n} \eps^{b_1 \dots b_n} \\
&= \det A \eps^{a_1 \dots a_n} \\
&= \det A \int d^n\theta^\prime \theta_{a_1}^\prime \dots \theta_{a_n}^\prime \\
\Rightarrow d^n\theta^\prime = \frac{d^n \theta}{\det A }
}
Note for scalars this is the opposite, $d^n\varphi = \frac{d^n\varphi^\prime}{\det A }$

%%%%%%%%%%%%%%%%%%%%%%%%%%%%%%%%%%%%%%%%%%%%
\subsection{Free fermion field theory}
Consider $d=0$, with 2 fermion fields $\theta_1,\theta_2$. The action must be scalar bosonic, so the only possible non constant action is 
\eq{
S(\theta) = \frac{1}{2} A \theta_1 \theta_2
}
for $A\in\mbb{R}$. 
Then
\eq{
Z_0 &= \int d^2\theta e^{-\frac{S(\theta)}{\hbar}} \\
&= \int d^2\theta (1-\frac{A}{2\hbar} \theta_1\theta_2 ) = -\frac{A}{2\hbar}
}
Now for $n=2m$ fermion fields 
\eq{
S = \frac{1}{2}A_{ab} \theta_a \theta_b
}
for $A$ antisymmetric, then 
\eq{
Z_0 &= \int d^{2m}\theta e^{-\frac{S(\theta)}{\hbar}} \\
&= \int d^{2m}\theta \sum_{j=0}^m \frac{(-1)^j}{(2\hbar)^j j!} (A_{ab} \theta_a \theta_b)^j \\
&= \frac{(-1)^m}{(2\hbar)^m m!} \int d^{2m}\theta A_{a_1 a_2}A_{a_3 a_3} \dots A_{a_{2m-1} a_{2m}} \theta_{a_1} \dots \theta_{a_{2m}} \\
&=  \frac{(-1)^m}{(2\hbar)^m m!} \eps^{a_1 \dots a_{2m}} A_{a_1 a_2}A_{a_3 a_3} \dots A_{a_{2m-1} a_{2m}} \\
&= \frac{(-1)^m}{\hbar^m} \Pfaff(A)
}
where 
\eq{
\Pfaff(A) = \highlight{\frac{1}{2^m m!}} \eps^{a_1 \dots a_{2m}} A_{a_1 a_2}A_{a_3 a_3} \dots A_{a_{2m-1} a_{2m}} 
}
\begin{ex}
Show $\Pfaff(A) = \pm \sqrt{\det A}$
\end{ex}

\begin{example}
\eq{
\Pfaff \begin{pmatrix} 0 & -a \\ a & 0 \end{pmatrix} = a
}
\end{example}
Then
\eq{
Z_0 &= \pm \sqrt{\frac{\det A }{\hbar^n}} \quad \text{(Fermionic)} \\
Z_0 &= \sqrt{\frac{(2\pi\hbar)^n }{\det M}} \quad \text{(Bosonic)} 
}
Introduce an external source function 
\eq{
S(\theta,\eta) = \frac{1}{2}A_{ab} \theta_a \theta_b + \eta_a \theta_a 
}
Complete the square as before to get 
\eq{
S(\theta,\eta) = \frac{1}{2}(\theta_a + \eta_c (A^{-1})_{ca}) A_{ab} (\theta_b + \eta_d (A^{-1})_{db}) + \frac{1}{2} \eta_a (A^{-1})_{ab} \eta_b 
}
so 
\eq{
Z_0 (\eta) = \exp \left( - \frac{1}{2\hbar} \eta^T (A^{-1}) \eta \right) Z_0(0)
}
and the propagator is 
\eq{
\braket{\theta_a \theta_b} = \frac{\hbar^2}{Z_0(0)} \frac{\del^2 Z_0(\eta)}{\del \eta_a \del \eta_b} \rvert_{\eta=0} = \hbar (A^{-1})_{ab}
}

\begin{remark}
In the first example sheet, there are two typos. 
\begin{itemize}
    \item In question 1, there should be no square in the denominator $\exp \frac{im(x-x_0)^2}{2(t-t_0)^{\highlight{2}}}$
    \item In question 2, the the numerator is should be $\exp\frac{\dots -\highlight{2}xx_0}{\dots}$ \bam{not} $\exp\frac{\dots -\highlight{9}xx_0}{\dots}$ 
\end{itemize}
\end{remark}

%%%%%%%%%%%%%%%%%%%%%%%%%%%%%%%%%
%%%%%%%%%%%%%%%%%%%%%%%%%%%%%%%%%
\section{LSZ (Lehmann-Symanzik-Zimmermann) reduction formula}

\begin{remark}
The Main results is this : there is a direct relation between scatterin amplitudes and correlation functions (Vacuum Expectation Values). 
\end{remark}

\section{2 to 2 scattering of real scalar fields}
Let look at the example of 2 to 2 scattering of real scalar fields
Consider first free scalar fields 
\eq{
\varphi(x) = \int \LImeas \left[ a(\bm{p}) e^{-ip\cdot x} + a^\dagger(\bm{p}) e^{ip\cdot x} \right]
}
Invert to find 
\eq{
\int  d^3 x e^{i p \cdot x} \varphi(x) &= \frac{1}{2E_{\bm{p}}} a(\bm{p}) + \frac{1}{2E_{\bm{p}}} e^{2iE_{\bm{p}}p} a^\dagger(-\bm{p})  \\
\int  d^3 x e^{i p \cdot x} \del_0\varphi(x) &= -\frac{i}{2} a(\bm{p}) + \frac{i}{2} e^{2iE_{\bm{p}}p} a^\dagger(-\bm{p}) \\
\Rightarrow a(\bm{p}) &= \int d^3 x e^{ip\cdot x}(i \del_0 \varphi(x) + E_{\bm{p}} \varphi(x) ) \\
a^\dagger(\bm{p}) = \int d^3 x e^{-ip\cdot x}(-i \del_0 \varphi(x) + E_{\bm{p}} \varphi(x) )
}

\subsubsection*{Initial state}
For a free theory 1 particle state 
\eq{
\ket{p} = a^\dagger(\bm{p}) \ket{0} 
}
where the vacuum $\ket{0}$ satisfies $\braket{0|0}=1$ and $a(\bm{p})\ket{0}=0$. Normalise to 
\eq{
\braket{p^\prime | p} = (2\pi)^3 (2E_{\bm{p}}) \delta^{(3)}(\bm{p}-\bm{p}^\prime)
}
Introduce a Guassian wavepacket 
\eq{
a_1^\dagger = \int d^3 p f_1(\bm{p}) a^\dagger(\bm{p})
}
with 
\eq{
f_1(\bm{p}) \propto \exp \left[ -\frac{(\bm{p}-\bm{p}_1)^2}{4\sigma^2}\right]
}
for some $\bm{p}_1,\sigma$. 
Similarly, for a second particle 
\eq{
a_2^\dagger = \int d^3 p f_2(\bm{p}) a^\dagger(\bm{p}) \\
f_1(\bm{p}) \propto \exp \left[ -\frac{(\bm{p}-\bm{p}_2)^2}{4\sigma^2}\right] \quad $\bm{p}_1 \neq \bm{p}_2$. 
}
Now evolve Gaussain wavepackets in the far distant past(/future), so the overlap between Gaussians is vanishingly small. Assume this work s even when interactions are present. The problem is that $a^\dagger$ becomes times dependent. We will assume that as $t\to \pm\infty$, $a_1^\dagger,a_2^\dagger$ coincide with free theory expressions.

Let the initial/final states be 
\eq{
\ket{i} &= \lim_{t\to -\infty} a_1^\dagger(t) a_2^\dagger(t) \ket{0} \\
\ket{f} &= \lim_{t\to \infty} a_{1^\prime}^\dagger(t) a_{2^\prime}^\dagger(t) \ket{0}
}
with 
\eq{
\braket{i|i} = 1 = \braket{f|f} \\
\bm{p}_1 \neq \bm{p}_2 \\
\bm{p}_{1^\prime} \neq \bm{p}_{2^\prime}
}
The scattering amplitude is the 
\eq{
\braket{f|i}
}
Noting 
\eq{
a_1^\dagger(\infty) - a_1^\dagger(-\infty) &= \int_{-\infty}^\infty \del_0 a_1^\dagger(t) \quad \text{(FTC)} \\
&= \int d^3 p f_1(p) \int d^4 x \del_0 [e^{-ip\cdot x}(-i\del_0 + E \varphi)] \\
&= -i \int d^3p f_1(p) \int d^4x e^{-ip\cdot x}(\del_0^2 + E^2) \varphi \\
&= -i \int d^3 p f_1(p) d^4 x e^{-ip\cdot x}(\del_0^2 + \underbrace{\bm{p}^2}_{-\nabla^2} + m^2)\varphi \\
&= -i \int d^3 p f_1(p) \int d^4 x (\del^2 + m^2) \varphi e^{-ip\cdot x} \\
&= 0 \quad \text{as for a free theory $\varphi$ satisfies Klein Gordon}
}
Now time ordering 
\eq{
\braket{f|i} = \braket{0| \mc{J} a_{1^\prime}(\infty)a_{2^\prime}(\infty)a_1(-\infty)a_2^(-\infty) | 0}
}
Use equations to substitute 
\eq{
a_j(-\infty) &= a_j(\infty) + i \int d^3p f_1(\bm{p}) \int d^4 x e^{-ip\cdot x}(\del^2 + m^2) \varphi \\
a_j^\dagger(-\infty) &= a_j^\dagger(\infty) + \dots
}
so 
\eq{
\text{(LSZ) : }\braket{f|i} &= (i)^4 \int d^4 x_1 d^4 x_2 d^4 x_{1^\prime} d^4 x_{2^\prime} e^{-ip_1 \cdot x_1} e^{-ip_2 \cdot x_2} e^{ip_{1^\prime} \cdot x_{1^\prime}} e^{ip_{2^\prime} \cdot x_{2^\prime}} \\
& \cdot (\del_1^2 + m^2)(\del_{1^\prime}^2 + m^2)(\del_2^2 + m^2)(\del_{2^\prime}^2 + m^2) \\
& \cdot \braket{0| \mc{J} \varphi(x_1) \varphi(x_2) \varphi(x_{1^\prime}) \varphi(x_{2^\prime})|0}
}
Having taken $\sigma \to 0 $ in the $f_i$ giving $\delta^{(3)}(\bm{p}-\bm{p}_i)$. \\
\newline

We may weaken the assumptions to  
\begin{itemize}
    \item Assume a unique ground state and first excited state is single particle. 
    \item Want $\varphi\ket{0}$ to be a single particle, i.e. $\braket{0|\varphi|0}=0$. If not shift $\varphi \to \varphi - \braket{0|\varphi|0}$
    \item Want $\varphi$ normalised such that $\braket{p | \varphi| 0 } = e^{i p \cdot x}$ as in the free case. With interactions we may need to rescale $\varphi \to Z_\varphi^\frac{1}{2} \varphi $
\end{itemize}
under which LSZ still applies. e.g. 
\eq{
\mc{L} &= \frac{1}{2} \del_\mu \varphi \del^\mu \varphi - \frac{1}{2}m^2 \varphi^2 + \dots \\
\text{renormalise} \\
&= \frac{1}{2} Z_\varphi \del_\mu \varphi \del^\mu \varphi - \frac{1}{2} Z_m m^2 \varphi^2 + \dots 
}

\end{document}