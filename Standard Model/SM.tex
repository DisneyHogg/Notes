\documentclass{article}
\usepackage{header}
%%%%%%%%%%%%%%%%%%%%%%%%%%%%%%%%%%%%%%%%%%%%%%%%%%%%%%%%
%Preamble

\title{Standard Model Notes}
\author{Linden Disney-Hogg}
\date{January 2019}

%%%%%%%%%%%%%%%%%%%%%%%%%%%%%%%%%%%%%%%%%%%%%%%%%%%%%%%%
%%%%%%%%%%%%%%%%%%%%%%%%%%%%%%%%%%%%%%%%%%%%%%%%%%%%%%%%
\begin{document}

\maketitle
\tableofcontents

\section{Introduction}
The Standard Model (SM) describes three fundamental forces (EM, weak, and strong). The forces are mediated by gauge bosons (spin $=1$).
\begin{itemize}
    \item EM (QED) : photon $\gamma$, (massless)
    \item Weak : $W$ and $Z$ bosons
    \item Strong : gluons $g$ (massless)
\end{itemize}
Matter content (spin $-\frac{1}{2}$ fermions) comes in three generations 
\begin{itemize}
    \item Neutrinos  $\nu_e, \nu_\mu \nu_\tau$ weak 
    \item Charged leptons $e, \mu, \tau$ weak and EM
    \item Quarks $u,d c,s,t,b$, weak, EM, and strong. 
\end{itemize}
There is also the Higgs boson $H$ (scalar, spin 0), responsible for generating the mass of W and Z bosons. 

Gauge bosons are manifestations of local symmetries. In SM the gauge group is 
\[
\underbrace{SU(3)_c}_{\text{color}} \times \underbrace{SU(2)_L}_{\text{chiral left}} \times \underbrace{U(1)_\gamma}_{\text{hypercharge}}
\]

%%%%%%%%%%%%%%%%%%%%%%%%%%%%%%%%%%%%%%%%%%%%%%%%%%%%%%%%
%%%%%%%%%%%%%%%%%%%%%%%%%%%%%%%%%%%%%%%%%%%%%%%%%%%%%%%%
\section{Chiral and Gauge symmetries}
%%%%%%%%%%%%%%%%%%%%%%%%%%%%%%%%%%%%%%%%%%%%%%%%%%%%%%%%
\subsection{Chiral symmetry}

Spin 1/2 particles are Dirac fermions with spinor field $\psi$ that satisfy the Dirac equation $(i\slashed{\del}-m)\psi=0$. The Dirac adjoint $\bar{\psi} = \psi^\dagger \gamma^0$ satisfies $\bar{\psi}(i\slashed{\del}^\leftarrow - m)=0$ where $\slashed{\del}^\leftarrow$ acts to the left. Dirac matrices satisfy 
\[
\acomm[\gamma^\mu]{\gamma^\nu}=2g^{\mu\nu}I
\]
where $g^{\mu\nu}=\diag(1,-1,-1,-1)$. Also define 
\[
\gamma^5 = i \gamma^0 \gamma^1 \gamma^2 \gamma^3
\]
which satisfies $(\gamma^5)^2=I$ and $\acomm[\gamma^\mu]{\gamma^5}=0$. In general the \emph{Chiral} or \emph{Weyl} basis is used, where 
\begin{align*}
    \gamma^0 &= \begin{pmatrix} 0 & I_2 \\ I_2 & 0 \end{pmatrix} \\
    \gamma^i &= \begin{pmatrix} 0 & \sigma_i \\ -\sigma_i & 0 \end{pmatrix} \\ 
\end{align*}
In this representation 
\[
\gamma^5=\begin{pmatrix} -I_2 & 0 \\ 0 & I_2 \end{pmatrix}
\]
Consider the massless limit of the Dirac equation
\[
\slashed{\del}\psi = 0 \Rightarrow \slashed{\del} ( \gamma^5\psi) = 0. 
\]
Define $P_{R,L}=\frac{1}{2} (1 \pm \gamma^5)$, projection operators. Then correspondingly let $\psi_{R,L} = P_{R,L} \psi$. Then $\gamma^5 \psi_{R,L} = \pm \psi_{R,L}$, so the projections are eigenstates of chirality. In the chiral basis, $\psi_{R,L}$ only contain lower/upper 2-component spinor degrees of freedom (d.o.f). As a result $\psi_{R,L}$ annihilates left/right handed chiral particles respectively. In addition 
\[
\bar{\psi}_{R,L} = (P_{R,L} \psi)^\dagger \gamma^0 = \psi^\dagger \frac{1}{2} (1\pm\gamma^5)\gamma^0=\bar{\psi} P_{L,R}
\]
A massless Dirac fermion has a \emph{global} $U(1)_L \times U(1)_R$ chiral symmetry. Under $U(1)_{R,L}$, $\psi_{R,L} \mapsto e^{i\alpha_{R,L}} \psi_{R,L}$, as can be seen from the Dirac Lagrangian 
\[
\mc{L}_D = \bar{\psi}(i\slashed{\del}-m)\psi = \bar{\psi}_L i \slashed{\del} \psi_L + \bar{\psi}_R i \slashed{\del} \psi_R - m(\bar{\psi}_R\psi_L + \bar{\psi}_L \psi_R) 
\]
The mass term explicitly breaks the chiral symmetry to a remaining vector symmetry where $\alpha_L = \alpha_R = \alpha$, so $\psi \mapsto e^{i\alpha} \psi$, $U(1)_L \times U(1)_R \to U(1)_V$. 

%%%%%%%%%%%%%%%%%%%%%%%%%%%%%%%%%%%%%%%%%%%%%%%%%%%%%%%%
\subsection{Review of Dirac Field}
Quantise 
\[
\psi(x) = \sum_{p,s} \left[ b^s(p) u^s(p) e^{-ip\cdot x} + {d^s}^\dagger(p) v^s(p) e^{ip\cdot x} \right]
\]
$s=\pm\frac{1}{2}$, and $\sum_p = \int \LImeas$, $\braket{p|q} = (2\pi)^2 (2E_p)\delta^3(\bm{p}-\bm{q})$ with $\ket{p} = b^\dagger(p) \ket{0}$. $u$ and $v$ are solutions of 
\[
(\slashed{p}-m)u=0 , \quad (\slashed{p}+m)v=0
\]
In the chiral basis these become 
\begin{align*}
    u^s(p) &= \begin{pmatrix} \sqrt{p\cdot \sigma} \xi^s \\ \sqrt{p\cdot\bar{\sigma}} \xi^s \end{pmatrix} \\ 
    v^s(p) &= \begin{pmatrix} \sqrt{p\cdot \sigma} \xi^s \\ -\sqrt{p\cdot\bar{\sigma}} \xi^s \end{pmatrix}
\end{align*}
with $\sigma^\mu = (1, \bm{\sigma})$, $\bar{\sigma}^\mu = (1, -\bm{\sigma})$. 

\begin{definition}[Helicity]
\bam{Helicity} is defined as the projection of angular momentum onto the linear momentum direction. 
\[
h = \bm{J} \cdot \hat{\bm{p}} = \bm{s} \cdot \hat{\bm{p}}
\]
as $\bm{J} = \bm{r}\times\bm{p} + \bm{s}$, and 
\[
s_i = \frac{i}{4}\eps_{ijk} \gamma^i \gamma^j = \frac{1}{2} \begin{pmatrix} \sigma_i & 0 \\ 0 & \sigma_i \end{pmatrix}
\]
\end{definition}
A massless spinor satisfies $\slashed{p}u=0$
so 
\[
hu(p) = \frac{\gamma^5}{2} u(p) 
\]
\[
\Rightarrow hu_{R,L} = \frac{\gamma^5}{2} u_{R,L} = \pm \frac{1}{2} u_{R,L}
\]
Note
\begin{itemize}
     \item Chiral states are only eigenstates of the Dirac equation when $m=0$
     \item Helicity is defined for $m=0$ and $m\neq0$, but it's not Lorentz Invariant when $m\neq 0$. 
     \item There is only a 1-1 correspondence between helicity and chirality when $m=0$. 
\end{itemize}

%%%%%%%%%%%%%%%%%%%%%%%%%%%%%%%%%%%%%%%%%%%%%%%%%%%%%%%%
\subsection{Gauge Symmetry}
Promoting $\alpha$ to a function of $x$, $\alpha(x)$, i.e. gauging the symmetry $\psi \to e^{i\alpha(x)} \psi$, the kinetic term is no longer invariant. 
\[
\bar{\psi} i \slashed{\del} \psi \to \bar{\psi} i \slashed{\del} \psi - (\bar{\psi}\gamma^mu \psi)(\del_\mu \alpha(x))
\]
Introduce a gauge covariant derivative $D_\mu$ such that
\[
D_\mu \psi(x) \to e^{i\alpha(x)} D_\mu \psi(x)
\]
To do this introduce a gauge field $A_\mu(x)$ so 
\[
D_\mu \psi = (\del_\mu + igA_\mu) \psi
\]
where $A_\mu \to A_\mu-\frac{1}{g} \del_\mu \alpha$ so $\bar{\psi} i \slashed{D} \psi$ is invariant. \\
Introduce a kinetic term for the gauge fields 
\begin{align*}
\mc{L}_{gauge} &= -\frac{1}{4} F_{\mu\nu} F^{\mu\nu} \\
F_{\mu\nu} &= \del_\mu A_\nu - \del_\nu A_\mu \\
ig F_{\mu\nu} &= \comm[D_\mu]{D_\nu}
\end{align*}
QED has a $U(1)$ gauge symmetry that treats LH and RH fields equivalently. The weak gauge bosons only couple to LH fields, but $U(1)$ is not the appropriate symmetry, we need $SU(2)$. (We will review non-abelian gauge symmetries later.) 

\subsection{Types of Symmetry}
Symmetries may manifest themselves in a variety of ways: 
\begin{itemize}
    \item Symmetry is intact e.g. $U(1)_{EM}$, and $SU(3)_c$ gauge symmetries. 
    \item Symmetry of $\mc{L}$ is broken by an \bam{anomaly} (holds classically but is broken by quantum loop effects). Not actually a true symmetry. E.g. global axial $U(1)$ symmetry in the SM.
    \item Symmetries can hold for some terms in $\mc{L}$ but not others. This is called "broken explicitly". It may be an approximate symmetry if the breaking terms are small. E.g. global 'isospin' symmetry relating $u$ and $d$ quarks in QCD.
    \item Hidden symmetries - respected by $\mc{L}$ but \emph{not} the vacuum. These can be: a) \bam{spontaneously broken symmetries}, vacuum expectation value from one or more scalar fields non-zero, e.g $SU(2)_L \times U(1)_\gamma \to U(1)_{EM}$, or b) Even without scalar fields get \bam{dynamical breaking} from quantum effects. e.g. $SU(2)_L \times SU(2)_R$ global symmetry in QCD (massless quarks). 
\end{itemize}

%%%%%%%%%%%%%%%%%%%%%%%%%%%%%%%%%%%%%%%%%%%%%%%%%%%%%%%%
%%%%%%%%%%%%%%%%%%%%%%%%%%%%%%%%%%%%%%%%%%%%%%%%%%%%%%%%
\section{Discrete Symmetries}
The discrete symmetries are 
\begin{itemize}
    \item Parity $P : (t,\bm{x}) \to (t,-\bm{x})$
    \item Time reversal $T : (t,\bm{x}) \to (-t,\bm{x})$
    \item Charge conjugation $C : \text{particles} \leftrightarrow \text{antiparticles} $
\end{itemize}
Parity and Time reversal are spacetime symmetries. These have properties such as 
\begin{itemize}
    \item $\bar{\psi}\gamma^\mu \psi$ couplings between gauge bosons and fermions,  e.g. QED and QCD, are invariant under $P$ and $C$ separately.  
    \item $\bar{\psi}\gamma^\mu (1-\gamma^5) \psi$ couplings to fermions, e.g. weak interactions, are not.
    \item Weak interaction violates $CP$, which leads to $T$ violation from the $CPT$ theorem. 
\end{itemize}
We will first investigate the consequences of $C,P,T$ symmetries in order to understand the above statements. 

%%%%%%%%%%%%%%%%%%%%%%%%%%%%%%%%%%%%%%%%%%%%%%%%%%%%%%%%
\subsection{Symmetry Operators} 

\begin{theorem}[Winger]
If physics is invariant under $\Psi \to \Psi^\prime$ (where $\Psi, \Psi^\prime \in \mc{H}$ some Hilbert space), then $\exists W$ an operator such that $\Psi^\prime = W\Psi$ where either W i) linear and unitary or ii) antilinear and antiunitary. 
\end{theorem}
\begin{proof}
Consider a Poincare transform 
\[
x^\mu \to \Lambda^\mu_\nu x^\nu + a^\mu
\]
Then for a parity transform 
\[
\Lambda^\mu_\nu = \mbb{P}^\mu_\nu = \diag(1,-1,-1,-1)
\]
and for time reversal
\[
\mbb{T}^\mu_\nu = \diag(-1,1,1,1)
\]
The corresponding operator can be expanded as 
\[
W(\Lambda,a) = W(1+1,\eps) = 1+\frac{i}{2} w_{\mu\nu} J^{\mu\nu} - i\eps_\mu p^\mu
\]
where the $J$ are the generators of boosts and rotations, and $p$ the generators of translation i.e $p^0=H=$Hamiltonian and $p^i=$momentum. Then
\begin{align*}
    \hat{P} &= W(\mbb{P},0) \\
    \hat{T} &= W(\mbb{T},0)
\end{align*}
Now from general composition rules 
\[
\hat{P} W(\Lambda,a) \hat{P}^{-1} = W(\mbb{P}\Lambda\mbb{P}^{-1},\mbb{P}a)
\]
Insert expansion of $W$ and compare coefficients of $-\eps_0$ to get 
\[
\hat{P} iH \hat{P}^{-1} = iH
\]
and doing likewise for $\hat{T}$ gives 
\be
\hat{T} iH \hat{T}^{-1} = -iH \label{eq:SM:1}
\ee
Suppose $\Psi$ is an energy eigenstate
\[
(\Psi, iH\Psi) = (\Psi, iE\Psi) = iE
\]
If $\hat{P},\hat{T}$ are symmetries then $\hat{P}\Psi,\hat{T}\Psi$ should also be energy eigenstates with the same energy. 
Suppose $\hat{P}$ is linear. Then
\begin{align*}
(\hat{P}\Psi,iH\hat{P}\Psi) &= (\hat{P}\Psi,\hat{P} iH\Psi) \text{ from \ref{eq:SM:1}} \\ 
&= (\hat{P}\Psi,\hat{P} iE\Psi) \\
&= iE(\hat{P}\Psi,\hat{P}\Psi) \text{ as $\hat{P}$ linear} \\
&= iE 
\end{align*}

Now suppose $\hat{T}$ linear and complete as above giving
\[
(\hat{T}\Psi,iH\hat{T }\Psi) = -iE(\hat{T}\Psi,\hat{T}\Psi)
\]
Contradiction. 
Supposing instead $\hat{T}$ antilinear gives 
\[
(\hat{T}\Psi,iH\hat{T }\Psi) = iE(\hat{T}\Psi,\hat{T}\Psi)
\]
Hence must have $\hat{P}$ linear and unitary, whereas $\hat{T}$ antilinear and antiunitary. 
\end{proof}

%%%%%%%%%%%%%%%%%%%%%%%%%%%%%%%%%%%%%%%%%%%%%%%
\subsection{Parity}

\subsubsection*{Scalar fields}
\begin{definition}[Scalar Fields]
A complex scalar field is 
\[
\phi(x) = \sum_p \left[ a(p) e^{-ip\cdot x} + c^\dagger(p) e^{ip\cdot x} \right]
\]
\end{definition}
$\hat{P}: \ket{p} \to {\eta^a}^\ast \ket{p_P}$ where $p_P = (p^0,-\bm{p})$ and ${\eta^a}^\ast$ is a complex phase. For later define in analogy $x_P = (x^0, -\bm{x})$
\[
\Rightarrow \hat{P} a^\dagger (p) \ket{0} = {\eta^a}^\ast a^\dagger (p_P) \ket{0}
\]
since $\hat{P}\hat{P}^{-1}=I$ and assuming $\hat{P}\ket{0}=\ket{0}$
\[
\hat{P} a^\dagger (p) \hat{P}^{-1} = {\eta^a}^\ast a^\dagger (p_P)
\]
To conserve normalisation, $\hat{P} a(p) \hat{P}^{-1} = \eta^{a} a(p_P)$. Similarly $\hat{P} c^\dagger(p) \hat{P}^{-1} = {\eta^c}^\ast c^\dagger(p_P)$. Thus 
\begin{align*}
\hat{P} \phi(x) \hat{P}^{-1} &= \sum_p \left[ \hat{P} a(p) \hat{P}^{-1} e^{-ip.x} + \hat{P} c^\dagger(p) \hat{P}^{-1} e^{ip\cdot x} \right] \\ 
&= \sum_p \left[ \eta^a a(p_P) e^{-ip\cdot x} + {\eta^c}^\ast c^\dagger(p_P) e^{ip\cdot x} \right] \\
\text{Relabelling $p \leftrightarrow p_P$ } &= \sum_{p_P} \left[ \eta^a a(p) e^{-ip_P\cdot x} + {\eta^c}^\ast c^\dagger(p) e^{ip_P \cdot x} \right] \\
[p_P \cdot x = p\cdot x_P] \Rightarrow &= \sum_{p_P} \left[ \eta^a a(p) e^{-ip\cdot x_P} + {\eta^c}^\ast c^\dagger(p) e^{ip \cdot x_P} \right]  \\
[\sum_p = \sum_{p_P}] \Rightarrow &= \sum_{p} \left[ \eta^a a(p) e^{-ip\cdot x_P} + {\eta^c}^\ast c^\dagger(p) e^{ip \cdot x_P} \right] 
\end{align*}
This does not look like $\phi(x_P)$ unless $\eta^a = {\eta^c}^\ast \equiv \eta_P$ and otherwise would note in general find $\comm[\phi(x)]{\hat{P}\phi^\dagger(y) \hat{P}^{-1}}$ vanishes for spacelike $x-y_p$. Hence
\[
\hat{P} \phi(x) \hat{P}^{-1} = \eta_P \phi(x_P)
\]
\\
Note that for a real scalar field $a=c$ and so $\eta^a = {\eta^a}^\ast \Rightarrow \eta_P = \pm 1$. For a complex field, we may not have real $\eta_P$, but if there is some conserved charge we can redefine $\hat{P}$ such that $\eta_P = \pm1$

\subsubsection*{Vecotr fields}
\begin{definition}[Vector Fields]
A vector field is 
\[
V^\mu(x) = \sum_{p,\lambda} \left[ \eps^\mu(\lambda,p) a^\lambda(p) e^{-ip\cdot x} +{\eps^\mu}^\ast(\lambda,p) {c^\lambda}^\dagger(p) e^{ip\cdot x} \right]
\]
$\lambda = 0, \pm1$ is helicity, $\eps^\mu$ polarisation vectors. 
\end{definition}
Use $\eps^\mu(\lambda, p_P) = -\mbb{P}^\mu_\nu \eps^\nu (\lambda,p)$. In analogy to the above treatment we find 
\[
\hat{P} V^\mu(x) \hat{P}^{-1} = -\eta_P \mbb{P}^\mu_\nu V^\nu(x_P)
\]
Vectos have $\eta_P = -1$, axial vectors have $\eta_P = 1$. \\

\subsubsection*{Dirac Field}
For a Dirac field, creation/annihilation operators should behave like those for bosons. The 3-momentum reverses direction, the spin component s is changed. 
\eq{
\hat{P} b^s(p) \hat{P}^{-1} &= \eta^b b^s(p_P) \\
\hat{P} {d^s}^\dagger(p) \hat{P}^{-1} &= {\eta^d}^\dagger {d^s}^\dagger(p_P)
}
Then
\eq{
\hat{P}\psi(x)\hat{P}^{-1} &= \sum_{p,s} \left[ \eta^b b^s(p_P) u^s(p) e^{-ip\cdot x} + {\eta^d}^\ast {d^s}^\dagger(p_P) v^s(p) e^{ip\cdot x} \right] \\
&= \sum_{p,s} \left[ \eta^b b^s(p) u^s(p_P) e^{-ip\cdot x_P} + {\eta^d}^\ast {d^s}^\dagger(p) v^s(p_P) e^{ip\cdot x_P} \right]
}
One can verify
\eq{
u^s(p_P) &= \gamma^0 u^s(p) \\
v^s(p_P) &= -\gamma^0 v^s(p)
}
so 
\eq{
\hat{P}\psi(x) \hat{P}^{-1} = \sum_{p,s} \left[ \eta^b b^s(p) \gamma^0 u^s(p) e^{-ip\cdot x_P} - {\eta^d}^\ast {d^s}^\dagger(p) \gamma^0 u^s(p) e^{ip\cdot x_P} \right]
}
Require $\eta^b = -{\eta^d}^\ast=\eta_P$ so that 
\eq{
\hat{P} \psi(X) \hat{P}^{-1} = \eta_P \gamma^0 \psi(x_P) \equiv \psi^P(x)
}
Then 
\[
\bar{\psi}^P(x) = \hat{P}\bar{\psi}(x) \hat{P}^{-1} = \eta_P^\ast \bar\psi(x_P) \gamma^0
\]
Note 
\begin{itemize}
    \item $\hat{P} \psi_L(x) \hat{P}^{-1} = \eta_P \gamma^0 \psi_R(x_P)$ 
    \item It can be checked that $\psi$ satisfies the Dirac equation $\Rightarrow \psi^P$ satisfies the Dirac equation.  
\end{itemize}
From the above we can determine the transformation properties of fermion bilinears. 
\eq{
\bar{\psi}(x) \psi(x) &\to \hat{P} \bar{\psi}(x) \hat{P}^{-1} \hat{P} \bar{\psi}(x) \hat{P}^{-1} = \bar{\psi}(x_P) \psi(x_P) \quad &\text{(scalar)} \\
\bar{\psi}(x) \gamma^5 \psi(x) &\to -\bar{\psi}(x_P) \gamma^5 \psi(x_P) \quad &\text{(pseudoscalar)} \\
\bar{\psi}(x) \gamma^\mu \psi(x) &\to \mbb{P}^\mu_\nu\bar{\psi}(x_P) \gamma^\nu \psi(x_P) \quad &\text{(vector)} \\
\bar{\psi}(x) \gamma^\mu \gamma^5 \psi(x) &\to  - \mbb{P}^\mu_\nu\bar{\psi}(x_P) \gamma^\nu  \gamma^5 \psi(x_P) \quad &\text{(axial vector)}
} 

%%%%%%%%%%%%%%%%%%%%%%%%%%%%%%
\subsection{Charge conjugation}
$\hat{C}$ is a linear and unitary operator sending particles $\leftrightarrow$ antiparticles 

\subsubsection*{Scalar Field}
Lorentz symmetry constrains the phases 
\eq{
\hat{C} a(p) \hat{C}^{-1} &= \eta_C c(p) \\
\hat{C} c(p) \hat{C}^{-1} &= {\eta_C}^\ast a(p)
}
Then 
\eq{
\hat{C} \ket{particle, p} &= \hat{C}a^\dagger(p) \ket{0} \\
&= {\eta_C}^\ast c^\dagger(p) \ket{0} \\
&= {\eta_C}^\ast \ket{antiparticle,p}
}
From the decomposition 
\eq{
\hat{C}\phi(x) \hat{C}^{-1} = \eta_C \phi^\dagger (x) \\
\hat{C}\phi^\dagger(x) \hat{C}^{-1} = {\eta_C}^\ast \phi(x)
}
For a real scalar field $\phi=\phi^\dagger$ and so $\eta_C = \pm1$. 

\subsubsection*{Vector field}
Photon field must obey $\hat{C} A_\mu(x) \hat{C}^{-1} = -A_\mu (x)$. Hence a $\pi^0$ meson can decay to $2\gamma \Rightarrow \eta_C^{\pi^0} = (-1)^2 = 1$.
For a complex field, $\eta_C$ is arbitrary. Say $\eta_C= e^{2i\beta}$, we can do a global $U(1)$ transform st $\phi \to \phi^\prime = e^{-i\beta}\phi$ such that $\eta_C^\prime = 1$.

\subsubsection*{Dirac Field}
Define the matrix $C$ such that $(\gamma^\mu C)^T = \gamma^\mu C$. In the chiral basis where 
\eq{
{\gamma^0}^T &= \gamma^0 \\
{\gamma^1}^T &= -\gamma^1 \\
{\gamma^2}^T &= \gamma^2 \\
{\gamma^3}^T &= -\gamma^3 
}
a suitable choice is 
\eq{
C = =i\gamma^0 \gamma^2 = \begin{pmatrix} i\sigma^2 & 0 \\ 0 & -i\sigma^2 \end{pmatrix}
}
giving 
\[
C = -C^T = -C^\dagger = -C^{-1}
\]
Then 
\eq{
(\gamma^\mu)^T &= -C^{-1} \gamma^\mu C \\
{\gamma^5}^T &= C^{-1} \gamma^5 C
}
similarly to bosons 
\eq{
\hat{C} b^s(p) \hat{C}^{-1} &= \eta_C d^s(p) \\
\underbrace{\hat{C} {d^s}^\dagger(p) \hat{C}^{-1}}_{\text{in } \psi} &= \underbrace{\eta_C {b^s}^\dagger (p)}_{\text{in } \bar{\psi}}
}
Now consider 
\eq{
\hat{C} \psi(x) \hat{C}^{-1} = \eta_C \sum_{p,s} \left[ d^s(p) u^s(p) e^{-ip\cdot x} + {b^s}^\dagger v^s(p) e^{ip\cdot x} \right]
}
and compare with 
\eq{
{\bar{\psi}}^{T} (x) = \sum_{p,s} \left[ {b^s}^{\dagger(p)} ({\bar{u}}^s)^{T} (p) e^{ip\cdot x} + d^s ({\bar{v}}^s)^{T} (p) e^{-ip\cdot x} \right]
}
Consider the spinors and take $\eta^s = i\sigma^2 {\xi^s}^\ast$ (choosing a basis for the spinors) we can write 
\eq{
v^s(p) &= C (\bar{u}^s)^T \\
u^s(p) &= C (\bar{v}^s)^T \\
}
and so 
\eq{
\psi^C(x) = \hat{C} \psi(x) \hat{C}^{-1} = \eta_C C\bar{\psi}^T (x)
}
similarly 
\eq{
\bar{\psi}^C(x) = \hat{C} \bar{\psi}(x) \hat{C}^{-1} = {\eta_C}^\ast \psi^T (x) C = -{\eta_C}^\ast \psi^T(x) C^{-1}
}
Note that $\psi(x)$ satsifes the Dirac eqn $\Rightarrow \; \psi^C(x)$ does. 
\begin{itemize}
    \item Majorana fermions have $b^s(p) = d^s(p) \Rightarrow$ particle is its own anti particle. In this case $\psi = \psi^C$. 
    \item Is is note known whether the only neutral fermions in the SM (neutrinos) are Majorana (c.f. neutrinoless double beta decay). 
\end{itemize}

\subsubsection*{Fermion Bilinears}
Note it is important to keep track of what's an operator ($\hat{C}$) nad what is a matrix in spinor space ($C$). 
\begin{example}
\eq{
j^\mu(x) &= \bar{\psi}(x) \gamma^\mu \psi(x) \\ 
\Rightarrow \hat{C} j^\mu \hat{C}^{-1 } &= \hat{C}\bar{\psi} \hat{C}^{-1} \gamma^\mu \hat{C} \psi \hat{C}^{-1} \\
&= -\eta_C^\ast \eta_C \psi^T C^{-1} \gamma^\mu C \bar{\psi}^T  \\
&= - \psi_\alpha (C^{-1} \gamma^\mu C)_{\alpha\beta} \bar{\psi}_\beta \\
&= \bar{\psi}_\beta (C^{-1} \gamma^\mu C)_{\alpha\beta} \psi_\alpha \quad \text{(fermions anticommute)} \\
&= \bar{\psi}_\beta (C^{-1} \gamma^\mu C)_{\beta\alpha}^T \psi_\alpha \\
&= \bar{\psi} (C^{-1} \gamma^\mu C)^T \psi \\
&= -\bar{\psi} \gamma^\mu \psi = -j^\mu
}
Similarly 
\eq{
\hat{C} {j^\mu}^5 \hat{C}^{-1} = {j^\mu}^5
}
where ${j^\mu}^5 = \bar{\psi} \gamma^\mu \gamma^5 \psi$.
\end{example}

%%%%%%%%%%%%%%%%%%%%%%%%%%%%%%%%%%%%%%%%%%%%%%%%%%%%%%%%%%%%
\subsection{Time Reversal}
Take the notation $x^\mu_T = (-x^0, \bm{x})$, $p^\mu_T = (p^0, -\bm{p})$. In T-symmetric theories, the physics is unchanged if time runs backwards. 

\subsubsection*{Boson Field}
\eq{
\hat{T} a(p) \hat{T}^{-1} &= \eta_T a(p_T) \\
\hat{T} c^\dagger(p) \hat{T}^{-1} &= \eta_T c^\dagger (p_T)
}
From the decomposition, recalling $\hat{T}$ antihermitian, 
\eq{
\hat{T} \phi(x) \hat{T}^{-1} &= \sum_p \left[ \hat{T} a(p) \hat{T}^{-1} e^{ip\cdot x} + \hat{T} c^\dagger (p) \hat{T}^{-1} e^{-ip\cdot x} \right] 
}
So using $p_T \cdot x  = -p\cdot x_T$ 
\eq{
\hat{T} \phi(x) \hat{T}^{-1} &= \eta_T \sum_p \left[ a(p) e^{-p\cdot x_T} + c^T(p) e^{ip\cdot x_T} \right]
}
\subsubsection*{Dirac Field}
$\hat{T}$ flips sign of any momentum. The creation/annihilation ops can be taken to transform as 
\eq{
\hat{T} b^s(p) \hat{T}^{-1} & = \eta_T (-1)^{\frac{1}{2}-s} b^{-s} (p_T) \\
\hat{T} {d^s}^\dagger(p) \hat{T}^{-1} & = \eta_T (-1)^{\frac{1}{2}-s} {d^{-s}}^\dagger (p_T)
}
It can be shown that 
\eq{
(-1)^{\frac{1}{2}-s} {u^{-s}}^\ast (p_T) &= -Bu^s(p) \\
(-1)^{\frac{1}{2}-s} {v^{-s}}^\ast (p_T) &= -Bv^s(p) 
}
where 
\eq{
B = C^{-1} \gamma^5 = \begin{pmatrix} i\sigma^2 & 0 \\ 0 & i\sigma^2 \end{pmatrix}
}
Then 
\eq{
\hat{T} \psi(x) \hat{T}^{-1} &= \eta_T \sum_{p,s} (-1)^{\frac{1}{2}-s} \left[ b^{-s}(p_T) {u^s}^\ast(p) e^{ip\cdot x} + {d^{-s}}^\dagger(p_T) {v^s}^\ast(p) e^{-ip\cdot x}   \right] \\
&= \eta_T \sum_{p,s} (-1)^{\frac{1}{2}-s+1} \left[ b^{s}(p) {u^{-s}}^\ast(p_T) e^{ip\cdot x_T} + {d^{s}}^\dagger(p) {v^{-s}}^\ast(p_T) e^{ip\cdot x_T}   \right] \\
&= \eta_T B \psi(x_T)
}
Similarly 
\eq{
\hat{T} \bar{\psi}(x) \hat{T}^{-1} = \eta_T^\ast \bar{\psi}(x_T) B^{-1}
}
Then some bilinears we have 
\eq{
\hat{T} \bar{\psi}(x) \psi(x) \hat{T}^{-1} &= \bar{\psi}(x_T) \psi(x_T) \\
\hat{T} \bar{\psi}(x) \gamma^\mu \psi(x) \hat{T}^{-1} &= \bar{\psi}(x_T) B^{-1} {\gamma^\mu}^\ast B \psi(x_T)
}
Now we can check 
\eq{
B^{-1} {\gamma^0}^\ast B &= \gamma^0 \\
B^{-1} {\gamma^0i}^\ast B &= -\gamma^i \\ 
\Rightarrow B^{-1} {\gamma^\mu}^\ast B &= -\mbb{T}^\mu_\nu \gamma^\nu
}

%%%%%%%%%%%%%%%%%%%%%%%%%%%%%%%%%%%%%%%%%%%%
\subsection{Scattering S-Matrix}
Define 
\eq{
\braket{p_1,p_2,\dots| S | k_A, k_B, \dots} &= \tensor[_o]{\braket{p_1,p_2,\dots| k_A, k_B, \dots}}{_i} \\
&= \lim_{T\to\infty} \braket{p_1,p_2,\dots | Te^{-i\int_{-T}^T V(t) dt} | k_A, k_B, \dots}
}
with 
\eq{
V(t) = -\int d^3 x \mc{L}_I
}
the potential energy term. 
\begin{example}
In QED 
\eq{
\mc{L}_I = -e \bar{\psi} \gamma^\mu A_\mu \psi
}
\end{example}
Now we have the table of transformations 
\begin{center}$
\begin{array}{cccc}
    \text{Quantity} & \hat{P}\cdot\hat{P}^{-1} & \hat{C}\cdot\hat{C}^{-1} & \hat{T}\cdot\hat{T}^{-1} \\
    \hline
    \hline
    \mc{L}_I(x) & \mc{L}_I(x_P) & \mc{L}_I(x) & \mc{L}_I(x_T) \\
    V(t) & V(t) & V(t) & V(-t) \\
    S & ? & ? & ? \\
\end{array}
$\end{center}

Now write 
\eq{
S = \sum_{n=0}^\infty (-i)^n \int_{-\infty}^\infty dt_1 \int_{-\infty}^{t_1} dt_2 \dots \int_{-\infty}^{t_{n-1}} dt_n V(t_1) V(t_2) \dots V(t_n) \\
\Rightarrow S_T \ \hat{T} S \hat{T}^{-1} =  \sum_{n=0}^\infty (i)^n \int_{-\infty}^\infty dt_1 \int_{-\infty}^{t_1} dt_2 \dots \int_{-\infty}^{t_{n-1}} dt_n V(-t_1) V(-t_2) \dots V(-t_n)
}
Substituting $\tau = -t_{n+1-i}$ 
\eq{
S_T &= \sum_{n=0}^\infty (i)^n \int_\infty^{-\infty} (-d\tau_n) \int_\infty^{-t_1} (-d\tau_{n-1}) \dots \int_\infty^{-t_{n-1}} (-d\tau_1) V(\tau_n) V(\tau_{n-1})\dots V(\tau_1) \\
&=\sum_{n=0}^\infty (i)^n \int_{-\infty}^\infty d\tau_n \int_{-t_1}^\infty d\tau_{n-1} \dots \int_{-t_{n-1}}^\infty d\tau_1 V(\tau_n) V(\tau_{n-1})\dots V(\tau_1) \\ 
&= \sum_{n=0}^\infty (i)^n \int_{-\infty}^\infty d\tau_n \int_{\tau_n}^\infty d\tau_{n-1} \dots \int_{\tau_2}^\infty d\tau_1 V(\tau_n) V(\tau_{n-1})\dots V(\tau_1)
}
Geometrically it can be seen 
\eq{
\int_{-\infty}^\infty d\tau_n \int_{\tau_n}^\infty d\tau_{n-1} = \int_{-\infty}^\infty d\tau_{n-1} \int_{-\infty}^{\tau_{n-1}} d\tau_n 
}
so successively swapping 
\eq{
S_T = \sum_{n=0}^\infty (i)^n \int_{-\infty}^\infty d\tau_1 \int_{-\infty}^{\tau_1} d\tau_2 \dots \int_{-\infty}^{\tau_{n-1}} d\tau_n V(\tau_n) V(\tau_{n-1})\dots V(\tau_1)
}
Now consider 
\eq{
S^\dagger &= \sum_{n=0}^\infty (i)^n \int_{-\infty}^\infty dt_1 \int_{-\infty}^{t_1} dt_2 \dots \int_{-\infty}^{t_{n-1}} dt_n [V(t_1) V(t_2) \dots V(t_n)]^\dagger \\
&= \sum_{n=0}^\infty (i)^n \int_{-\infty}^\infty dt_1 \int_{-\infty}^{t_1} dt_2 \dots \int_{-\infty}^{t_{n-1}} dt_n V(t_n) V(t_{n-1}) \dots V(t_1)
}
So we have shown 
\eq{
S_T = S^\dagger
}
Hence note ${S_T}^\dagger = S$. Now consider $\ket{\xi},\ket{\eta}$ with 
\eq{
\ket{\xi_T} &= \hat{T} \ket{\xi} \\
\ket{\eta_T} &= \hat{T} \ket{\eta}
}
Then 
\eq{
\braket{\eta_T | S \xi_T } &= (\hat{T} \eta , S_T^\dagger \hat{T} \xi) \\
&=(\hat{T} \eta , \hat{T} S^\dagger \xi) \\
&= (\eta, S^\dagger \xi)^\ast \quad \text{( $\hat{T}$ antiunitary)} \\
&= (S^\dagger \xi,\eta) \\ 
&= (\xi, S \eta) \\
&= \braket{\xi | S | \eta}
}
Hence if $\hat{T} \mc{L}_I (x) \hat{T}^{-1} = \mc{L}_I (x_T)$, S-matrix elements are equal form time reversed processes where initial and final states are swapped. 

%%%%%%%%%%%%%%%%%%%%%%%%%%%%%%%%%%%
\subsection{CPT theorem}

\begin{theorem}
Any Lorentz invariant $\mc{L}$ with a Hermitian Hamiltonian should be invariant under the product of P,C, and T. 
\end{theorem}
\begin{proof}
See Streater and Wightman "PCT, spin and statistics, and all that" (1989).
\end{proof}
All observations suggest that CPT is respected in nature. This means 
 a particle (positive charge, spin up) propagating forward in time cannot be distinguished from an antiparticle (negative charge, spin down) propagating backwards in time. 
 
 %%%%%%%%%%%%%%%%%%%%%%%%%%%%%%%%%%%
 \subsection{Baryogenesis}

\begin{definition}[Baryogenesis]
\bam{Baryogenesis} is the generation of matter-antimatter asymmetry in the universe. 
\end{definition}

There are three necessary conditions for Baryongenesis, the \bam{Sakarov conditions}
\begin{itemize}
    \item Baryon number violation: $X \to Y+B$, B excess baryons (or leptogenesis, i.e lepton number violation giving baryon number asymmetry through B+L violation). 
    \item Non-equilibrium : Otherwise $\Gamma(Y+B \to X) = \Gamma(X \to Y+B)$
    \item C and CP violation: Otherwise  
    \eq{
    \frac{dB}{dt}\propto \Gamma(X \to Y+B) - \Gamma(\bar{X} \to \bar{Y} + \bar{B})=0
    }
    with C-symmetry, or 
    \eq{
    \Gamma(x \to nq_L) + \Gamma(x \to n q_R) = \Gamma(\bar{x} \to n \bar{q}_R) + \Gamma(\bar{x} \to n \bar{q}_L)
    }
    with CP-symmetry.
\end{itemize}

%%%%%%%%%%%%%%%%%%%%%%%%%%%%%%%%%%
%%%%%%%%%%%%%%%%%%%%%%%%%%%%%%%%%%
\section{Spontaneous Symmetry Breaking (SSB)}
There are hidden symmetries present in $\mc{L}$ but not in observable. 

%%%%%%%%%%%%%%%%%%%%%%%%%%%%%%%%%%
\subsection{SSB of discrete symmetry}
Consider a real scalar field $\phi(x)$ with symmetric $V(\phi)$ and $\mc{L} = \frac{1}{2} \del_\mu \phi \del^\mu - V(\phi)$, e.g. $V(\phi) = \frac{1}{2} m^2 \phi^2 + \frac{\lambda}{4} \phi^4$, $\lambda > 0$ \\
We have either 
\begin{itemize}
    \item the typical case to analyze, $m^2 > 0$, where $V(\phi)$ has a minimum at $\phi=0$.
    \item$m^2< 0$, then $V(\phi) = \frac{\lambda}{4} (\phi^2 - v^2)^2$ up to a constant, where $v = \sqrt{\frac{-m^2}{\lambda}}$. Now $\phi=0$ is an unstable vacuum and there are two degenerate vacua at $\phi = \pm v$.
\end{itemize}
In the second case $\phi$ has a acquired a non zero \bam{Vacuum Expectation Value (vev)}. Wlog we may study small excitations about $\phi=v$ 
\eq{
\phi(x) = v + f(x) ]]
\mc{L} = \frac{1}{2} \del_\mu f \del^\mu f - \lambda ( v^2 f + vf^3 + \frac{1}{4} f^4) + \text{ constant}
}
Hence $f$ is a scalar field with mass $m_f = \sqrt{2\lambda v^2}$. This $\mc{L}$ is \emph{not} invariant under $f \to -f$. The symmetry of the original $\mc{L}$ is broken by the VEV of $\phi$. 

%%%%%%%%%%%%%%%%%%%%%%%%%%%%%%%%%
\subsection{SSB of continuous (global) symmetry}

Consider a real $N$- component scalar field $\phi=(\phi_1,\dots,\phi_N)^T$, with 
\eq{
\mc{L}= \frac{1}{2} (\del_\mu \phi)\cdot(\del^\mu \phi) - V(\phi) \\
V(\phi) = \frac{1}{2}m^2 \phi^2 + \frac{\lambda}{4} \quad \phi^2 = \phi \cdot \phi , \phi^4 = (\phi^2)^2,  \lambda > 0
}
invariant under a global $O(N)$ symmetry. We're interested in $m^2 < 0$ again. In this case 
\eq{
V(\phi) = \frac{\lambda}{4}(\phi^2-v^2)^2
}
Up to an irrelevant constant where 
\eq{
v^2 = - \frac{m^2}{\lambda}
}
This is the "Mexican hat" potential. 
There is then a continuum of vacua with $\phi^2=v^2$. Wlog choose $\phi_0 = (0,\dots,0,v)^T$ and study small fluctuations about this 
\eq{
\phi(x) = (\pi_1(x), \dots, \pi_{N-1}(x),v+\sigma(x))^T
}
then 
\eq{
\mc{L} = \frac{1}{2} (\del_\mu \pi)\cdot(\del^\mu \pi) + \frac{1}{2} \del_\mu \sigma \del^\mu \sigma - V(\pi,\sigma) \\
V(\pi,\sigma) = \frac{1}{2} m_\sigma^2 \sigma^2 + \lambda v (\sigma^2+ \pi^2)\sigma + \frac{\lambda}{4}(\sigma^2)+\pi^2)^2
}
The $\sigma$ field, which is a radial excitation in the potential, has mass $m_\sigma^2=2\lambda v^2$, but the $N-1$ $\pi$ fields, which are azimuthal excitations that see flat potential, are massless \\
\newline
Generalise to a symmetry group $G$ of $\mc{L}$ which is broken to a subgroup $H\subset G$ by the vacuum (we'll generally be considering normal subgroups). The transform is $\phi \to g\phi$ with $g\in G$ in some representation, and $\mc{L}(\phi) = \mc{L}(g\phi)$. Assume $G$ is spontaneously broken and hence the vacuum is not unique but a manifold\footnote{I suspect that this is necessarily a manifold as our configuration space is assume to be a manifold (in this case $\mbb{R}^N$) and then $\Phi_0$ is a closed subgroup for continuous $V$, so the closed subgroup theorem applies}. 
\eq{
\Phi_0 = \set{\phi_0 : V(\phi_0)=V_{min}}
}
The invariant subgroup (or stability group) $H\subset G$ is 
\eq{
H = \set{h \in G : h\phi_0 = \phi_0}
}
Different vacua are related by $\phi_0^\prime = g\phi_0$ for some $g\in G$. Stability groups for different vacua are isomorphic. For $
\phi_0^\prime$ the stability group is $H^\prime = gHg^{-1}$. Group elements that map one vacuum to another are in the coset space $\faktor{G}{H}$ and fall into equivalence classes 
\eq{
g_1 \sim g_2 \Leftrightarrow g_2^{-1}g_1 \in H 
}
which are the left cosets. Hence there's one equivalence class for each $\phi_0^\prime \in \Phi_0 \Rightarrow \Phi_0 \cong \faktor{G}{H}$. If $H$ is a normal subgroup the this is a group\footnote{I may prove this for fun if I find the time}. Now let's consider infinitesimal transforms $g\phi = \phi + \delta \phi$, $\delta \phi = i \alpha^a t^a \phi$, where $a=1,\dots,\dim G$ and $t^a$ are the generators of the Lie algebra of $G$ in the representation acting on $\phi$, and $\alpha^a$ are 'small' parameters. $G$ invariance means that $V(\phi) = V(\phi+\delta\phi)$, or 
\begin{align}\label{eq:SM:2}
V(\phi+\delta\phi) - V(\phi) = i\alpha^a (t^a \phi)_r \left(\pd[V]{\phi}\right)_r = 0 \quad \text{to first order}
\end{align}
where $r=1,\dots,N$ are indices of the components of $\phi$. If $\phi_0$ is a min of V, 
\eq{
V(\phi_0 + \delta \phi) - V(\phi_0) = \frac{1}{2} \delta\phi_r \underbrace{\frac{\del^2 V}{\del \phi_r \del \phi_s}}_{=M_{rs} \text{ (mass matrix)}} \delta \phi_s + \dots
}
Differentiate \ref{eq:SM:2} and evaluate at $\phi_0$ to get 
\eq{
\pd{\phi_s}\left[\left( t^a \phi)\right)_r \pd[V]{\phi_r} \right] = \pd{\phi_s} \left(t^a \phi\right)_r \pd[V]{\phi_r}|_{\phi_0} + (t^a \phi_0)_r M^2_{sr} = 0
}
Two cases 
\begin{itemize}
    \item Unbroken symmetry : $\forall g \in G \; g\phi_0=\phi_0 \Rightarrow \delta\phi=0 \Rightarrow \forall a \; t^a \phi_0 = 0$
    \item Brojen symmetry : $\exists g\in G \, s.t. \, \exists a \; t^a \phi_0 \neq 0 \Rightarrow t^a \phi_0$ is an eigenstate of $M^2_{rs}$ with eigenvalue 0.  \\
    Generatros of $H\subset G$ are $\tilde{t}^i \; i=1,\dots,\dim H$ and $\tilde{t}^i \phi_0 = 0$
\end{itemize}

Now a fact from SFP, for a compact semi-simple lie algebra of $G$ we can define a group invariant inner product and orthogonality. Choose a basis of the Lie algebra $t^a = \set{\tilde{t}^i, \theta^{\tilde{a}} }$ where $\theta^{\tilde{a}}$ are orthogonal to $\tilde{t}^i$ (i.e. $\tr {\tilde{t}}^i \theta^{\tilde{a}} = 0$. Then $\theta^{\tilde{a}} \phi_0$ is a unique zero eigenvector of $M^2_{sr}$ for $\tilde{a}=1,\dots,\dim G - \dim H \Rightarrow \dim G - \dim H$ massless modes  exists (\bam{Goldstone Bosons}) and in general $N-(\dim G - \dim H)$ massive modes exist. \\
This is the \emph{classical} proof of Goldstone's theorem

\begin{example}
For $O(N)$ model, $O(N) \to O(N-1)$ as $\Phi_0 = S^{N-1}$, so we expect
\eq{
\frac{1}N(N-1) - \frac{1}{2}(N-1)(N-2) = N-1 
}
massless modes, and  this is what was found. 
\end{example}

\subsubsection*{Insert on Group Theory}
Suppose a $\mc{L}$ written in terms of a complex $N\times N$ matrix field $M$ is invariant under $M \to AMB^{-1}$ where $A,B \in U(N)$. There should be only one identity element in the group, $(I_A,I_B)\in U(N)\times U(N)$ s.t. 
This is true when $M=I \Rightarrow I=I_A I_B^{-1} \Rightarrow I_A = I_B$. Hence \eq{
I_A M = M I_A \text{ for arbitrary } M
}
\begin{lemma}[Schur's lemma]
If $\forall g \in G \; SD(g) = D(g) S$ for $D$ some irreducible rep of $G$ then $S \propto I$. 
\end{lemma}
Schur's Lemma gives $I_A \propto I \Rightarrow I_A = e^{i\theta} I$ for $\theta\in\mbb{R}$. Thus these $I_A$ form a $U(1)$ normal subgroup. Hence the symmetry group is $\faktor{U(N)\times U(N)}{U(1)}$

\subsection{Goldstone's Theorem}
Now consider SSB in a fully quantum way. Suppose the symmetry group $G$ of $\mc{L}$ is psontaneously broken to $H\subset G$, i.e. $\phi$ gets a non-zero VEE $\braket{0| \phi | 0} = \phi_0 \neq 0$. The VEV is invariant under $h\in H$,but not under $g^\prime \in G\setminus H$. Let 
\begin{itemize}
    \item Lie algebra of G be $\set{t^a : a=1,\dots,\dim G}$
    \item Lie algebra of H be $\set{\tilde{t}^i : i=1,\dots,\dim H}$
\end{itemize}
$G$ is a symmetry of $\mc{L} \Rightarrow $ conserved currents from Noether's theorem 
\eq{
j^{a\mu}(x) = i \frac{\del \mc{L}}{\del(\del_\mu \phi)}t_a \phi
}
and charges 
\eq{
Q^a = \int d^3 x j^{a0}(x) = \int d^3 x \pi(x) t_a \phi(x)
}
These induce a representation on the Lie algebra 
\eq{
\delta \phi(0) = i\alpha^a t^a \phi(0) = i\comm[Q^a]{\phi(0)} \alpha^a 
}
Consider now 
\eq{
C^{a\mu} &= \braket{0 | \comm[j^{a\mu}(x)]{\phi(0)}|0} \\
&= \sum_n \left[ \braket{0|j^{a\mu}(x)|n}\braket{n|\phi(0)|0} - \braket{0|\phi(0)|n}\braket{n|j^{a\mu}(x)|0} \right] \\
&= i \int \frac{d^4 k}{(2\pi)^3 } \left[ \rho^{a\mu}(k) e^{-ik \cdot x} - \tilde{\rho}^{a\mu}(k) e^{ik \cdot x}   \right]
}
where 
\eq{
i \rho^{a\mu}(k) &= (2\pi)^3 \sum_n \delta^{(4)}(k-p_n) \braket{0|j^{a\mu}(0)|n}\braket{n|\phi(0)|0} \\
i \tilde{\rho}^{a\mu}(k) &= (2\pi)^3 \sum_n \delta^{(4)}(k-p_n) \braket{0|\phi(0)|n}\braket{n|j^{a\mu}(0)|0}
}
and recall 
\eq{
j^{a\mu}(x) = e^{ip \cdot x} j^{a\mu}(0) e^{-ip \cdot x}
}
This is the \bam{K\"allen Lehmann spectral representation}. Lorentz covariance gives $\rho^{a\mu} \propto k^\mu \propto \tilde{\rho}^{a\mu}$, physical states with $k^0 > 0$. Hence 
\eq{
\rho^{a\mu}(k) &= k^\mu \Theta(k^0)\rho^a(k^2) \\
\tilde{\rho}^{a\mu}(k) &= k^\mu \Theta(k^0)\tilde{\rho}^a(k^2)
}
So 
\eq{
C^{a\mu} &= - \del^\mu \int \frac{d^4 k}{(2\pi)^3 } \Theta(k^0) \left[ \rho^a(k^2) e^{-ik \cdot x} + \tilde{\rho}^a e^{ik \cdot x} \right]
}
Now consider the propagator 
\eq{
D(z-y;\sigma) &= \braket{0|\phi(z) \phi(y) | 0} \\
&= \int \frac{d^4 p}{(2\pi)^3} \, \Theta(p^0) \delta(p^2-\sigma) e^{-ip \cdot (z-y)}
}
and recognise 
\eq{
\rho(k^2) = \int d\sigma \, \rho(\sigma) \delta(k^2 - \sigma)
}
so 
\eq{
C^{a\mu} &= - \del^\mu \int  d\sigma \, \left[ \rho^a(\sigma) D(x;\sigma) + \tilde{\rho}^a D(-x,\sigma) \right]
}
For $x^2 < 0$ $D(x,\sigma) = D(-x,\sigma)$. The requiring $x^2 < 0 \Rightarrow C^{a\mu}=0$, i.e. causality, yields 
\eq{
\rho^a(\sigma) = -\tilde{\rho}^a(\sigma)
}
\begin{align}\label{eq:SM:3}
\Rightarrow C^{a\mu} = -\del^\mu \int d\sigma \, \rho^a(\sigma) i \Delta(x,\sigma)
\end{align}
where 
\eq{
i\Delta(x,\sigma) &= D(x,\sigma) - D(-x,\sigma) \\
&= \int \frac{d^4 k }{(2\pi)^3} \delta(k^2 - \sigma) \eps(k^0) e^{ik \cdot x} \\
\eps(k^0) &= \left\{ \begin{array}{cc} 1 & k^0 > 0 \\ -1 & k^0 < 0 \end{array} \right. 
}
Now 
\eq{
\del_\mu j^{a\mu} = 0 \Rightarrow -\del^2 \int d\sigma \, \rho^a(\sigma) i \Delta(x,\sigma) = 0
}
and the Klein Gordon equation gives 
\eq{
(\del^2 + \sigma) \Delta(x,\sigma) = 0 \Rightarrow \int d\sigma \, \sigma \rho^a(\sigma) i \Delta(x,\sigma) = 0 
}
For this second equation to hold $\forall x$, using that the norm of the states is positive definites so $\rho > 0$ gives 
\eq{
\sigma \rho(\sigma) = 0 
}
This gives two possibilites 
\begin{itemize}
    \item $\forall \sigma \, \rho(\sigma) = 0 \Rightarrow C^{a\mu} = 0 \Rightarrow t^a \phi = 0 $ (unbroken generator) 
    \item $\rho^a(\sigma) = N^a \delta(\sigma)$ where $N^a$ is a dimensionful non-zero constant. 
\end{itemize}
In the second case substitute into  \ref{eq:SM:3} to get 
\eq{
C^{a\mu} = -i N^a \del^\mu \Delta(x,\sigma) \\ 
\Rightarrow \braket{0| \comm[Q^a]{\phi(0)}|0} = -iN^a \int d^3 x \, \del^0 \Delta(x,0) = iN^a \\
\Rightarrow t^a \phi = \comm{0 | \comm[Q^a]{\phi(0)}|0} = iN^a
}
Now some states in $\rho^{a\mu},\tilde{\rho}^{a\mu}$ must be non zero. Label these $B(p)$ s.t. 
\eq{
\braket{0 | j^{a\mu}(0) | B(p)} = i F_B^a p^\mu \quad F_B^a \text{ a dim 1 constant} \\
\braket{B(p) | \phi(0) | 0} = Z^B \quad Z^B \text{ dim 0 constant}
}
$B(p)$ are spin 0 and massless as $\sigma = p^2 = 0$. Now 
\eq{
i \rho^{a\mu}(k) &= ik^\mu \Theta(k^0) N^a \delta(k^2) \\
&= \sum_B \int \frac{d^3 p }{2|\bm{p}|} \, \delta^{(4)}(k-p) \braket{0|j^{a\mu}(0)|B(p)}\braket{B(p) | \phi(0) | 0} \\
\Rightarrow \int \frac{d^3 p }{2|\bm{p}|} 
\delta^{(4)}(k-p) ik^\mu N^a &= \int \frac{d^3 p }{2|\bm{p}|} 
\delta^{(4)}(k-p) i p^\mu \sum_B F_B^a Z^B \\
\Rightarrow N^a &= \sum_B F_B^a Z^B
}
Hence we have $n$ $\rho^a$ which have non-zero contribution at $\sigma = 0$, so $F_B^a$ is a rank $n$ matrix gives we have $n$ \bam{Goldstone bosons}. \\

Note we've assumed Lorentz invariance in our theory with $>2$ spacetime dimensions, and also that states have positive definite norm. 

%%%%%%%%%%%%%%%%%%%%%%%%%%%%%%%%%%%%%%%%%%%%
\subsection{Abelian Higgs Mechanism}
Gauge theories can violate this theorem, e.g. in QED, imposing a Lorentz invariance gauge condition (Lorentz gauge) can lead to states with a negative norm. Hence a gauge with no negative norm states breaks Lorentz invariance. \\

Consider scalar electrodynamics with complex scalar $\phi(x)$ nad photon $A_\mu(x)$ 
\eq{
\mc{L} &= -\frac{1}{4} F_{\mu\nu}F^{\mu\nu} + (D_\mu \phi)^\ast (D^\mu \phi) - V(\phi^\ast \phi) \\
F_{\mu\nu} &= \del_\mu A_\nu - \del_\nu A_\mu \\
D_\mu &= \del_\mu + iq A_\mu 
}
With $U(1)$ gauge invariance, $\phi(x) \to e^{i\alpha(x)}\phi(x)$, $\alpha\in\mbb{R}$, and $A_\mu(x) \tp A_\mu(x) - \frac{1}{q} \del_\mu \alpha(x)$. Take 
\eq{
V(\phi^\ast \phi) = \mu^2 |\phi|^2 + \lambda |\phi|^4 \quad \lambda>0
}
Then 

\begin{itemize}
    \item $\mu^2 > 0 \Rightarrow |\phi|^2$ is usual mass term for $\phi$ and there is a unique vacuum at $\phi=0$. 
    \item $\mu^2 < 0 \Rightarrow $ minima at $|\phi_0|^2=-\frac{\mu^2}{2\lambda} = \frac{v^2}{2}$. 
\end{itemize}
Wlog expand around real $\phi_0$ 
    \eq{
    \phi(x) &= \frac{1}{\sqrt{2}} e^{i\frac{\theta(x)}{v}} \left( v + \eta(x) \right) \\
    \Rightarrow \mc{L} = \frac{1}{2}\left( \del_\mu \eta \del^\mu \eta - 2\lambda v^2 \eta^2 \right) + \frac{1}{2} (\del_\mu \theta) (\del^\mu \theta) - \frac{1}{4} F_{\mu\nu} F^{\mu\nu} + qv A_\mu \del^\mu \theta + \frac{q^2 v^2}{2} A_\mu A^\mu + \mc{L}_{int}
    }
where $\mc{L}_{int}$ are terms with $>2$ fields. Appear to have mass for $\eta,A_\mu$ but not $\theta$. Transform to unitary gauge $\alpha(x) = -\frac{1}{v} \theta(x) $
\eq{
\phi \to e^{-i\frac{\theta}{v}} \phi = \frac{1}{\sqrt{2}} \left( v + \eta \right) \\
A_\mu \to A_\mu + \frac{1}{vq} \del_\mu \theta \\
\mc{L} = \frac{1}{2} (\del_\mu \eta \del^\mu \eta - 2\lambda v^2 \eta^2) - \frac{1}{4} F_{\mu\nu} F^{\mu\nu} + \frac{q^2 v^2}{2} A_\mu A^\mu + \mc{L}_\int
}
Hence 
\begin{itemize}
    \item Photon has mass $m_A^2 = q^2 v^2$ 
    \item Scalar $\eta$ has mass $m_\eta^2 = 2\lambda v^2 = -2\mu^2$ 
    \item Goldstone modes $\theta$ has been 'eaten' to become longitudinal polarisation of $A_\mu$. 
\end{itemize}
Now 
\eq{
\mc{L}_{int} = \frac{q^2}{2} A_\mu A^\mu \eta^2 q m_A A_\mu A^\mu \eta - \frac{\lambda}{4} \eta^4 - m_\eta \sqrt{\frac{\lambda}{2}} \eta^3
}

%%%%%%%%%%%%%%%%%%%%%%%%%%%%%%%%%%%%%%%%%%%%
\subsection{Non Abelian Gauge Theories (SU(N))}
Consider the transform 
\eq{
\psi_i (x) \to U_{ij}(x) \psi_j (x) = \exp \left( i t^a \theta^a(x) \right)_{ij} \psi_j(x)
}
where the $U$ are matrices for an n-dimensional representation $R$ of a unitary Lie group, and $t^a$ are the hermitian generators of $R$ forming a Lie algebra. Then 
\eq{
\bar{\psi}_i(x) \to \bar{\psi}_j (x) (U^\dagger (x))_{ji} = \bar{\psi}_j (x) \exp\left( - i t^a \theta^a(x) \right)_{ji} 
}
Let the Lie algebra be defined by 
\eq{
\comm[t^a]{t^b} = i f^{abc}t^c
}
with 
\eq{
\tr(t^a t^b) = T(R) \delta^{ab}
}
as the normalisation, $T(R)$ the \bam{Dynkin index} of the representation. (Note for the fundamental rep of SU(N) $T(R)=\frac{1}{2}$). The covariant derivative is 
\eq{
(D_\mu)_{ij} = \del_\mu \delta_{ij} + ig (t^a A_\mu^a)_{ij} \\
}
we want 
\eq{
 (D_\mu \psi)_i \to ( U D_\mu \psi)_i \\
 \text{s.t} \quad \mc{L} = \bar{\psi}_i (i \slashed{D}_{ij} - m\delta_{ij}) \psi_j
}
Hence the gauge field transformation 
\eq{
t^a A_\mu^a \to U t^a A_\mu^a U^{-1} + \frac{i}{g} (\del_\mu U) U^{-1}
}
The infinitesimal transform is 
\eq{
\delta A_\mu^a = -\frac{1}{g} \del_\mu \theta^a - f^{abc} \theta^b A_\mu^c
}



\end{document} 