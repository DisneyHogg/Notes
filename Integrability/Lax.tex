\documentclass{article}

\usepackage{header}
%%%%%%%%%%%%%%%%%%%%%%%%%%%%%%%%%%%%%%%%%%%%%%%%%%%%%%%%
%Preamble

\title{Lax Pairs and Spectral Curves}
\author{Linden Disney-Hogg}
\date{December 2019}

%%%%%%%%%%%%%%%%%%%%%%%%%%%%%%%%%%%%%%%%%%%%%%%%%%%%%%%%
%%%%%%%%%%%%%%%%%%%%%%%%%%%%%%%%%%%%%%%%%%%%%%%%%%%%%%%%
\begin{document}

\maketitle
\tableofcontents

%%%%%%%%%%%%%%%%%%%%%%%%%%%%%%%%%%%%%%%%%%%%%%%%%%%%%%%%
%%%%%%%%%%%%%%%%%%%%%%%%%%%%%%%%%%%%%%%%%%%%%%%%%%%%%%%%
\section{Introduction}
These are some notes, based on \cite{BeisertIntroductionNotes}, for me to cement in example my thoughts on Lax pairs and spectral curves. 

%%%%%%%%%%%%%%%%%%%%%%%%%%%%%%%%%%%%%%%%%%%%%%%%%%%%%%%%
%%%%%%%%%%%%%%%%%%%%%%%%%%%%%%%%%%%%%%%%%%%%%%%%%%%%%%%%
\section{'Theory' and discussion}

As usual we will fix a phase space $M$, dimensions $2n$, with some Hamiltonian flow giving some time evolution 

\begin{definition}
A \bam{Lax pair} is a pair of two $n \times n$ matrices, functions on $M$, obeying 
\eq{
\dot{L} = \comm[M]{L}
}
\end{definition}

\begin{prop}
Let $L,M$ be a lax pair. Then 
\eq{
I_k \equiv \tr L^k
}
are conserved quantities, of which at most $n$ are independent. Equivalently $L$ is isospectral. 
\end{prop}

Recall for integrability we need not only conserved quantities, but that these quantities be involution. This condition can be encoded in the existence of a \bam{classical r matrix}, which is $r_{12} \in \End(\mbb{C}^n \otimes \mbb{C}^n)$ such that 
\eq{
\pbrace{L_1,L_2} = \comm[r_{12}]{L_1} - \comm[r_{21}]{L_2}
}
where 
\eq{
L_1 &= L \otimes 1 \\
L_2 &= 1 \otimes L \\
r_{21} &= (r_{12})^T
}

We can now extend the Lax pair to include a spectral parameter $u$. Suppose $L(u)$ is holomorphic on $\mbb{C}_\infty$. The eigenvalues of $L$, $\pbrace{\lambda_k(u)}$, are given by the zeros of the equation 
\eq{
\det(L(u) - \lambda I) = 0
}
This defines a spectral curve $\tilde{\Sigma}=\pbrace{(u,\lambda)}$. We can label the points of degeneracy of these eigenvalues as $\hat{u}_k$, and then we have an $n$-sheeted cover of $\mbb{C}_\infty$ given by the $\lambda_k$, with branch points at the $\hat{u}_k$ where the sheets coincide. We can instead reinterpret this as a single valued function $\lambda(z)$ defined on a Riemann surface $\Sigma$ with the singularities resolved, which projects onto $\tilde{\Sigma}$.
\begin{center}
    \begin{tikzcd}
    \Sigma \arrow[r] & \tilde{\Sigma} \arrow[d,"u"] \\ & \mbb{C}_\infty
    \end{tikzcd}
\end{center}


\begin{example}
Consider the case $n=2$ and the matrix 
\eq{
L = \begin{pmatrix} a & b \\ c & d \end{pmatrix}
}
Then the eigenvalues are given by 
\eq{
\lambda_\pm = \frac{1}{2}(a+d) \pm \sqrt{\frac{1}{4}(a-d)^2 + bc}
}
These are equal at $\hat{u}$ if there 
\eq{
\frac{1}{4}(a-d)^2 + bc = 0
}
and this gives that about $\hat{u}$
\eq{
\lambda_\pm (u) = \frac{1}{2} \psquare{a(\hat{u}) + d(\hat{u})} \pm \alpha\sqrt{u- \hat{u}} + O(u-\hat{u})
}
where 
\eq{
\alpha^2 = \ev{\psquare{\frac{1}{2}(a-d)(a^\prime - d^\prime) + bc^\prime + b^\prime c}}{u= \hat{u}}
}
\end{example}
The spectral curve will give information on the conserved quantities of the system - equivalently the action - but does not give any dynamical information. We will need machinery to give us this. \\
Suppose at a given branch point $\hat{u}$, two eigenvectors coincide, i.e. the geometric multiplicity of the eigenspace drops to 1 even as the algebraic multiplicity remains at two. This can be seen to occur in the $2 \times 2$ case, as if the geometric multiplicity was still 2 we could diagonalise wrt to a basis of eigenvectors and so 
\eq{
a(\hat{u}) &= \lambda(\hat{u}) = d(\hat{u}) \\
b(\hat{u}) &= 0 = c(\hat{u})
}
giving $\alpha = 0$ from before, in which case there would be no singularity. Hence we can again consider the eigenvectors to be a single valued function $\psi(z)$ on $\Sigma$. We can fix the scaling of the eigenvectors to remove redundancy, for example with the choice 
\eq{
\forall z \in \Sigma , \; v \cdot \psi(z) = 1
}
for some constant vector $v$. Note that the poles of $\psi$ are then exactly when $\psi$ is orthogonal to $v$, so the choice of $v$ is important.  

\begin{definition}
A pole of $\psi$ is $z^\times$ where a component of $\psi$ diverges as $(z-z^\times)^{-1}$. The set of poles $\pbrace{z_k^\times}$ is called the \bam{dynamical divisor}. 
\end{definition}

\begin{prop}
The dynamical divisor consists of $n-g+1$ points, where $g$ is the genus of $\Sigma$. 
\end{prop}

%%%%%%%%%%%%%%%%%%%%%%%%%%%%%%%%%%%%%%%%%%%%%%%%%%%%%%%%
%%%%%%%%%%%%%%%%%%%%%%%%%%%%%%%%%%%%%%%%%%%%%%%%%%%%%%%%
\section{SHO}
We will now demonstrate all the objects above for the case of the simpe harmonic oscillator with Hamiltonian 
\eq{
H = \frac{1}{2} (p^2+ \omega^2 q^2)
}
A Lax pair is given by 
\eq{
L &= \begin{pmatrix} p & \omega q \\ \omega q & -p \end{pmatrix} \\
M &= \begin{pmatrix} 0 & -\frac{1}{2} \omega \\ \frac{1}{2} \omega & 0 \end{pmatrix} 
}
A suitably classical r-matrix is given by 
\eq{
r_{12} = \frac{1}{q} \begin{pmatrix} 0 & 0 \\ 1 & 0 \end{pmatrix} \otimes \begin{pmatrix} 0 & 1 \\ 0 & 0 \end{pmatrix} - \frac{1}{q} \begin{pmatrix} 0 & 1 \\ 0 & 0 \end{pmatrix} \otimes \begin{pmatrix} 0 & 0 \\ 1 & 0 \end{pmatrix}
}
The Lax matrix can be given a spectral parameter by modifying to 
\eq{
L(u) = \begin{pmatrix} p + u\omega q & \omega q - up \\ \omega q - up & -p-u\omega q \end{pmatrix} 
}
which then gives 
\eq{
\lambda_\pm(u) = \pm\sqrt{(p+u\omega q)^2 + (\omega q - up)^2} = \pm\sqrt{2E}\sqrt{1+u^2}
}
This gives $\pbrace{\hat{u}} = \pbrace{\pm i, \infty}$. The spectral curve is given by 
\eq{
\lambda^2 = 2E(1+u^2)
}
Making the coordinate change 
\eq{
u(z) = -i \frac{z - \sfrac{1}{z}}{z + \sfrac{1}{z}}
}
gives 
\eq{
\lambda(z) = \sqrt{2E} \frac{2}{z + \sfrac{1}{z}}
}
Moreover the eigenvector function can be found to be 
\eq{
\psi_\pm &\sim \begin{pmatrix} up - \omega q \\ u \omega q + p - \lambda_\pm(u) \end{pmatrix} \\
\Rightarrow \psi(z) &\sim \begin{pmatrix} -iz(p-i\omega q) + i z^{-1}(p+i\omega q) \\z(p-i\omega q) + z^{-1}(p+i\omega q) - 2\sqrt{2E} \end{pmatrix}
}
which can be fixed by normalisation to be 
\eq{
\psi(z) &\sim \begin{pmatrix} 1 \\-i \frac{p+iq - \sqrt{2E}z}{p+iq + \sqrt{2E}z} \end{pmatrix}
}
Hence we have the $n-g+1 =1 $ pole at 
\eq{
z^\times = - \frac{p+iq}{\sqrt{2E}}
}
Along the Hamiltonian flow, $q,p$ evolve as 
\eq{
q(t) &= \sqrt{2E}\sin t \\
p(t) &= \sqrt{2E}\cos t
}
and so 
\eq{
z^\times = -e^{it}
}
%%%%%%%%%%%%%%%%%%%%%%%%%%%%%%%%%%%%%%%%%%%%%%%%%%%%%%%%
%%%%%%%%%%%%%%%%%%%%%%%%%%%%%%%%%%%%%%%%%%%%%%%%%%%%%%%%
\bibliographystyle{plain}
\bibliography{references.bib}


\end{document}