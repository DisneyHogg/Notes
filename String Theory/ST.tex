\documentclass{article}

\usepackage{header}
%%%%%%%%%%%%%%%%%%%%%%%%%%%%%%%%%%%%%%%%%%%%%%%%%%%%%%%%
%Preamble

\title{String Theory Notes}
\author{Linden Disney-Hogg}
\date{January 2019}

%%%%%%%%%%%%%%%%%%%%%%%%%%%%%%%%%%%%%%%%%%%%%%%%%%%%%%%%
%%%%%%%%%%%%%%%%%%%%%%%%%%%%%%%%%%%%%%%%%%%%%%%%%%%%%%%%
\begin{document}

\maketitle
\tableofcontents


\section{Book Keeping}
Suggested books on the material are the "easier" books 
\begin{itemize}
    \item Schomerus (CUP) 'Primer in string theory'
    \item Becker, Becker, and Schwarz (CUP) 'String theory and M theory'
\end{itemize}
and the "harder" books 
\begin{itemize}
    \item Polchinski, Vol 1 (CUP) 
    \item Lust and Thiesen (Springer) 
    \item Green, Schwarz, and Witten (CUP)
\end{itemize}

\section{Introduction}
\subsection{Material}
The course will follow 
\begin{itemize}
    \item Classical theory + canonical quantisation 
    \item Path integral quantisation
    \item Conformal Field Theory (CFT) + BRST quantisation
    \item Scattering amplitudes 
    \item Advanced topics...
\end{itemize}

\subsection{Expectation Management}
The need for a new theory can be explained by 
\begin{itemize}
    \item What sets the parameters for the standard model? 
    \item What fixed the cosmological constant? 
    \item Failure of perturbative General Relativity (GR).
    \item Black Hole (BH) information paradox. 
    \item How do you quantise a theory in the absence of an existing causal structure? 
\end{itemize}

\subsection{What is String Theory}
ST "seems" to be perturbation theory around classical solutions to a unique quantum theory. We imagine replacing particles with strings. Harmonics of the strings the correspond to different particles, including the graviton. We assume we are close to some well understood solution with metric $\eta^{\mu\nu}$ and take \[
g_{\mu\nu}(x) = \eta_{\mu\nu}+h_{\mu\nu}(x)+\dots
\]
\subsection{Interactions}
We will consider 
\begin{itemize}
    \item Propagators 
    \item Interactions vertices
\end{itemize}
which can be combined in analogue to Feynmann diagrams. Consider the Lagrangian density 
\[
\mc{L} = -\frac{1}{4}F^2 - \bar{\psi} (\slashed{D}-m)\psi
\]
then in QFT it is known how to convert this into Feynmann diagrams. In string theory it is not known where the "Feynmann diagrams" come from. In a "particle limit" this may look like a field theory. 


%%%%%%%%%%%%%%%%%%%%%%%%%%%%%%%%%%%%%%%%%%%%%%%%%%%%%%%%
%%%%%%%%%%%%%%%%%%%%%%%%%%%%%%%%%%%%%%%%%%%%%%%%%%%%%%%%
\section{Classical Theory}
In quantum mechanics we have the important parameter time $t$ and position $\hat{\bm{X}}$ is an operator. In quantum field theory we take $\hat{\phi}(t,\bm{x})$ as operators where $(t,\bm{x})$ are parameters.  \\
Alternatively, we could look for a formalism in which $\hat{X}^\mu=(\hat{T},\hat{\bm{X}})$ are operators. 

\begin{example}[Worldline Formalism]
Imagine we have a massive particle propagating on a flat spacetime with metric $\eta_{\mu\nu}$. A suitable action for this theory is 
\[
S[X] = - m\int_{s_1}^{s_2} \, ds
\]
Note that working with $\hbar=c=1$ requires that we have a parameter with mass dimensions outside the integral to make the action dimensionless. \\
We can parametrise the worldline such that 
\be \label{eq:ST:1}
S[X] = -m \int_{s_1}^{s_2} d\tau \, \sqrt{-\eta_{\mu\nu} \dot{X}^\mu \dot{X}^\nu}
\ee
The conjugate momentum is 
\[
P_\mu (\tau) = \frac{-m\dot{X}_\mu}{\sqrt{-\dot{X}^2}} \Rightarrow p^2+m^2 = 0
\]
so this is an "on-shell" formalism. We can vary with respect to $X^\mu$ to find the e.o.m. \\
\end{example}

%%%%%%%%%%%%%%%%%%%%%%%%%%%%%%%%%%%%%%%%%%%%%%%%%%%%%%%%
\subsection{Alternative Perspective}
An alternative action for this theory is 
\[
S[X,e]=\frac{1}{2} \int d\tau (e^{-1} \eta_{\mu\nu} \dot{X}^\mu \dot{X}^\nu - em^2 )
\]
where $e$ is an auxiliary field. The $X^\mu$ e.o.m. is 
\[
\frac{d}{d\tau}(e^{-1} \dot{X}^\mu ) = 0 
\]
and the $e$ e.o.m is 
\[
\dot{X}^2 + e^2m^2=0
\]
Note $e(\tau)$ appears only algebraically, and so we can substitute back into the action to recover the previous formulation, equation \ref{eq:ST:1}. \\

%%%%%%%%%%%%%%%%%%%%%%%%%%%%%%%%%%%%%%%%%%%%%%%%%%%%%%%%
\subsection{Symmetries}
Consider the local symmetry $\tau \to \tau + \xi(\tau)$. Then 
\begin{align}
    \delta X^\mu &= \xi \dot{X}^\mu \\
    \delta e &= \frac{d}{d\tau}(\xi e) 
\end{align}
We can use the one arbitrary degree of freedom to gauge fix $e(\tau)$ to a convenient value. 

\begin{definition}[Rigid Symmetry]
The rigid symmetry map is 
\[
X^\mu \to \Lambda^\mu_\nu X^\nu + a^\mu
\]
This is Poincare invariance in the background spacetime. \end{definition}
For a massless particle we can write the action as 
\[
S[X,e] = \frac{1}{2} \int_L d\tau \, e^{-1} g_{\mu\nu} \dot{X}^\mu \dot{X}^\nu
\]
and then the classical equations of motion reproduce the geodesic equation 
\[
\ddot{X}^\mu + \Gamma^{\mu}_{\nu\lambda} \dot{X}^\nu \dot{X}^\lambda = 0
\]
and the $e$ equations of motion give constraints. 

%%%%%%%%%%%%%%%%%%%%%%%%%%%%%%%%%%%%%%%%%%%%%%%%%%%%%%%%
%%%%%%%%%%%%%%%%%%%%%%%%%%%%%%%%%%%%%%%%%%%%%%%%%%%%%%%%
\section{Strings}
As a string moves through some spacetime $\mc{M}$ with metric $\eta_{\mu\nu}$ it sweeps out a \bam{worldsheet} $\Sigma$. Assume the string is closed, i.e. $\sigma \sim \sigma+2n\pi$ for $n\in\mbb{Z}$ ($\sigma$ is the 'angle' parameter of the string). The embedding of the worldsheet is given by the specification 
\[
X^\mu (\sigma,\tau)
\]
We will call the $X^\mu$ "embedding fields. 
\[
X : \Sigma \to \mc{M}
\]
The area of $\Sigma$ may be given by 
\[
A = -\frac{1}{2\pi \cst} \int d\tau d\sigma \sqrt{-\det{\eta_{\mu\nu}\del_a X^\mu \del_b X^\nu}}
\]
where $\del_a = \pd{\sigma^a}$, $\sigma^a = (\tau, \sigma)$. , and $\cst$ is a free parameter. \\
We often refer to the \bam{string length} 
\[
l_s = 2 \pi \sqrt{\cst}
\]
or the \bam{Tension} 
\[
T = \frac{1}{2\pi \cst}
\]
The object 
\[
G_{ab} = \eta_{\mu\nu}\del_a X^\mu \del_b X^\nu
\]
is an induced metric on $\Sigma$. $S[X]$ is called the \bam{Nambu-Goto action}.

\subsection{Polyakov Action}
Consider instead the action 
\[
S[X,h] = -\frac{1}{4\pi \cst} \int_\Sigma d^2\sigma \sqrt{-h} h^{ab} \eta_{\mu\nu} \del_a X^\mu \del_b X^\nu
\]
\begin{idea}
This looks like free Klein Gordon field theory in two dimensions.
\end{idea}
This action is classically equivalent to the Nambu-Goto action. The $h_{ab}$ equations of motion are given by
\[
-\frac{2\pi}{\sqrt{-h}} \frac{\delta S}{\delta h^{ab}} = 0
\]
These e.o.m give the vanishing of the \bam{stress tensor} 
\[
T_{ab}= 0 
\]
where
\[
T_{ab} = -\frac{1}{\cst} \left( \del_a X^\mu \del_b X_\mu - \frac{1}{2}h_{ab} \del_c X^\mu \del_d X_\mu h^{cd} \right) 
\]
Note
\begin{itemize}
    \item In two dimensions $T_{ab}h^{ab}=0$, i.e. $T$ is traceless.
    \item The $X^\mu$ e.o.m are $\frac{1}{\sqrt{-h}} ( \del_a \sqrt{-h} h^{ab} \del_b X^\mu ) =0 $, i.e. $\square X^\mu = 0$.
\end{itemize}

\subsection{Symmetries}
The Polyakov action has the following symmetries: 

\begin{definition}[Rigid Symmetry]
\[
X^\mu \to \Lambda^\mu_\nu X^\nu + a^\mu
\]
i.e the actions has Poincare invariance. 
\end{definition}

The action also has reparametrisation invariance i.e. $\sigma^a \to {\sigma^\prime}^a$. Under such a transform the embedding fields transform 
\[
{X^\prime}^\mu(\sigma^\prime,\tau^\prime) = {X}^\mu(\sigma,\tau)
\]
\[
h_{ab}(\sigma,\tau) = \pd[{\sigma^\prime}^c]{\sigma^a} \pd[{\sigma^\prime}^d]{\sigma^b} h^\prime_{cd} (\sigma^\prime, \tau^\prime)
\]
So infinitseimally $\sigma^a \to \sigma^a - \xi^a$
\begin{align*}
    \delta X^\mu = \xi^a \del_a X^\mu \\
    \delta h_{ab} = \xi^c \del_c h_{ab} + \del_a \xi^c h_{cb} + \del_b \xi^c h_{ca} = \nabla_a\xi_b + \nabla_b \xi_a \\
    \delta \sqrt{-h} = \del_a(\xi^a \sqrt{-h})
\end{align*}

\begin{definition}
\[
X^\mu \to X^\mu
\]
\[
h^\prime_{ab}(\sigma,\tau) = e^{2\Lambda(\sigma,\tau)} h_{ab}(\sigma,\tau)
\]
i.e. 
\begin{align*}
    \delta x^\mu = 0 \\
    \delta h_{ab} = 2\Lambda h_{ab}
\end{align*}
\end{definition}

We now have three arbitrary degrees of freedom in $(\xi^a,\Lambda)$, and we can use them to fix the three d.o.f in $h_{ab}$. 

\subsection{Classical solutions}
Let us use reparametrisation invariance to fix 
\begin{align*}
    h_{ab} = e^{2\phi} \eta_{ab} \\
    \eta_{ab} = \begin{pmatrix} -1 & 0 \\ 0 & 1 \end{pmatrix}
\end{align*}
The action becomes 
\[
S[X] = -\frac{1}{4\pi\cst} \int_\Sigma d^2\sigma ( -\dot{X}^2 + {X^\prime}^2 ) 
\]
where $\dot{X}^mu = \pd[X^\mu]{\tau}$ and ${X^\prime}^\mu = \pd[X^\mu]{\sigma}$. \\
The $X^\mu$ e.o.m becomes the wave equation in 2D, so the solutions are of the form 
\[
X^\mu(\sigma,\tau) = X^\mu_R(\tau-\sigma) + X^\mu_L(\tau+\sigma)
\]
It is useful to introduce modes $\alpha^\mu_n, \bar{\alpha}^\mu_n$. Then 
\[
X^\mu_R(\tau-\sigma) = \frac{1}{2} x^\mu + \frac{\cst}{2} p^\mu (\tau-\sigma) + i\sqrt{\frac{\cst}{2}} \sum_{n\neq0} \frac{1}{n} \alpha^\mu_n e^{-in(\tau-\sigma)}
\]
\[
X^\mu_L(\tau+\sigma) = \frac{1}{2} x^\mu + \frac{\cst}{2} p^\mu (\tau+\sigma) + i\sqrt{\frac{\cst}{2}} \sum_{n\neq0} \frac{1}{n} \bar{\alpha}^\mu_n e^{-in(\tau+\sigma)}
\]
where $x^\mu, p^\mu$ are constants independent of $\tau$ and $\sigma$. Set 
\[
\alpha^\mu_0 = \bar{\alpha}^\mu_0 = \sqrt{\frac{\cst}{2}} p^\mu
\]
Let us call 
\[
G_{ab} = \del_a X^\mu \del_b X_\mu
\]
Then 
\[
T_{ab}=0 \Rightarrow G_{ab}=\frac{1}{2} h_{ab} (h^{cd}G_{cd})
\]
so 
\[
\det G_{ab} = \left(\frac{1}{2}h^{cd}G_{cd} \right)^2 \det h_{ab} = \frac{1}{4}(h^{cd}G_{cd})^2 h
\]
so 
\[
2\sqrt{-\det G_{ab}} = (h^{cd}G_{cd}) \sqrt{-h} = \sqrt{-h} h^{ab} \del_a X^\mu \del_b X_\mu
\]\footnote{Here it was pointed out that spacetime indices will be raised and lowered with $h$, whereas Lorentz indices will be raised and lowered with $\eta$}
Substituting back into the action yields 
\[
S[X] = \frac{1}{2\pi\cst}\int d^2\sigma \sqrt{-\det G_{ab}}
\]

%%%%%%%%%%%%%%%%%%%%%%%%%%%%%%%%%%%%%%%%%%%%%%%%%%%%%%%
\subsection{Stress Tensor}
Recall that the conjugate momentum to $X^\mu$ is 
\[
P_\mu = \frac{1}{2\pi\cst} \dot{X}_\mu
\]
We can define a Hamiltonian density 
\[
H = P_\mu \dot{X}^\mu - L = \frac{1}{4\pi\cst}(\dot{X}^2 + {X^\prime}^2)
\]
We shall introduce Poisson brackets $\pb[]{}$. Given $F,G$ defined on the phase space we have 
\[
\pb[F]{G} = \int_0^{2\pi} d\sigma \left( \frac{\delta F}{\delta X^\mu(\sigma)} \frac{\delta G}{\delta P_\mu(\sigma)} - \frac{\delta F}{\delta P_\mu(\sigma)}\frac{\delta G}{\delta X^\mu(\sigma)} \right) 
\]
In particular $\pb[X^\mu(\tau,\sigma)]{P_\nu(\tau,\sigma^\prime}=\delta^\mu_\nu \delta(\sigma-\sigma^\prime)$.\\
We introduced a mode expansion 
\[
X^\mu(\sigma,\tau) = X^\mu_R(\tau-\sigma) + X^\mu_L(\tau+\sigma)
\]
\[
X^\mu_R(\tau-\sigma) = \frac{1}{2} x^\mu + \frac{\cst}{2} p^\mu (\tau-\sigma) + i\sqrt{\frac{\cst}{2}} \sum_{n\neq0} \frac{1}{n} \alpha^\mu_n e^{-in(\tau-\sigma)}
\]
\[
X^\mu_L(\tau+\sigma) = \frac{1}{2} x^\mu + \frac{\cst}{2} p^\mu (\tau+\sigma) + i\sqrt{\frac{\cst}{2}} \sum_{n\neq0} \frac{1}{n} \bar{\alpha}^\mu_n e^{-in(\tau+\sigma)}
\]
The Poisson bracket, acting on the modes $\alpha^\mu_n, \bar{\alpha}^\mu_n$ are 
\begin{align} \label{eq:ST:2}
\pb[\alpha^\mu_m]{\alpha^\nu_n} &= -im \delta_{m,-n} \eta^{\mu\nu} = \pb[\bar{\alpha}^\mu_m]{\bar{\alpha}^\nu_n} \\
\pb[\alpha^\mu_m]{\bar{\alpha}^\nu_n} &= 0
\end{align}
for $n,m\neq0$. If we define $\alpha^\mu_0=\bar{\alpha}^\mu_0 = \sqrt{\frac{\cst}{2}}p^\mu$, $\pb[x^\mu]{p_\nu}=\delta^\mu_\nu$
Let's see why this might be reasonable. Set $\tau=0$ wlog. 
\begin{align*}
X^\mu(\sigma) &= x^\mu + i\sqrt{\frac{\cst}{2}} \sum_{n\neq0} \frac{1}{n} \left( \alpha^\mu_n e^{in\sigma} + \bar{\alpha}^\mu_n e^{-in\sigma} \right) \\ 
P^\mu(\sigma^\prime) &= \frac{p^\mu}{2\pi} + \frac{1}{2\pi}\sqrt{\frac{1}{2\cst}} \sum_{m\neq0} \left( \alpha^\mu_n e^{in\sigma^\prime} + \bar{\alpha}^\mu_n e^{-im\sigma^\prime} \right)
\end{align*}
So 
\[
\pb[X^\mu(\sigma)]{P_\nu(\sigma^\prime)} = \frac{1}{2\pi} \pb[x^\mu]{p^\nu} - \frac{1}{4\pi} \sum_{n,m\neq0} \frac{1}{m} \left( \pb[\alpha^\mu_m]{\alpha^\nu_n} e^{i(m\sigma+n\sigma^\prime}+\pb[\bar{\alpha}^\mu_m]{\bar{\alpha}^\nu_n}e^{-i(m\sigma+n\sigma^\prime)} \right)
\]
using \ref{eq:ST:2} and also the periodic delta function 
\[
\delta(\sigma-\sigma^\prime)=\frac{1}{2\pi} \sum_{m=-\infty}^{m=\infty} e^{im(\sigma-\sigma^\prime)} \quad \text{("Dirac Comb")}
\]
one can show that 
\[
\pb[X^\mu(\sigma)]{P_\nu(\sigma^\prime)} = \eta^{\mu\nu} \delta(\sigma-\sigma^\prime)
\]

%%%%%%%%%%%%%%%%%%%%%%%%%%%%%%%%%%%%%%%%%%%%%%%%%%%%%%%
\subsection{The Wit Algebra}
It will be useful to use light-cone coordinates
\[
\sigma^\pm=\tau\pm\sigma
\]
so 
\[
ds^2=-d\tau^2 + d\sigma^2 = (d\sigma^+, d\sigma^-)\begin{pmatrix} 0 & -\frac{1}{2} \\ -\frac{1}{2} & 0 \end{pmatrix} \begin{pmatrix} d\sigma^+ \\ d\sigma^- \end{pmatrix}
\]
Also $\del_\pm=\pd{\sigma^\pm}=\frac{1}{2}\left( \del_\tau \pm \del_\sigma \right)$. The action and e.o.m become 
\[
S[X] = -\frac{1}{2\pi\cst} \int_\Sigma d\sigma^+ d\sigma^- \del_+ X^\mu \del_- X_\mu
\]
\[
\del_+ \del_- X^\mu =0
\]
The stress tensor becomes 
\begin{align*}
T_{++} &= -\frac{1}{\cst} \del_+ X^\mu \del_+ X_\mu \\
T_{--} &= -\frac{1}{\cst} \del_- X^\mu \del_- X_\mu
\end{align*} 
$T_{+-}=0$ as this is nothing more than the trace of $T_{ab}$. We can also introduce modes for the stress tensor $l_m,\bar{l}_m$
\begin{align*}
    l_m &= -\frac{1}{2\pi} \in_0^{2\pi} d\sigma T_{--}e^{-im\sigma} \\
    \bar{l}_m &= -\frac{1}{2\pi} \in_0^{2\pi} d\sigma T_{++}e^{im\sigma}
\end{align*}
Using the fact that 
\[
\del_- X^\mu = \sqrt{\frac{\cst}{2}} \sum_n \alpha^\mu_n e^{-in\sigma}
\]
where $\alpha^\mu_0 = \sqrt{\frac{\cst}{2}} p^\mu$.\\
We shall see that $l_m, \bar{l}_m$ are conserved particles on the space $T_{ab}=0$. Using 
\[
\del_- X^\mu (\sigma) = \sqrt{\frac{\cst}{2}} \sum_n \alpha_n^\mu e^{-in\sigma}
\]
where $\alpha_0^\mu = \sqrt{\frac{\cst}{2}} p^\mu$. We fine $l_n$ in terms of the $\alpha)m^\mu$s.
\begin{example}
\eq{
l_n &= \frac{1}{2\pi\cst} \int_0^{2\pi} d\sigma \del_- X \cdot \del_-X e^{in\sigma} \\
&= \frac{1}{4\pi} \sum_{m,p} \alpha_m \cdot\alpha_p \int_0^{2\pi} d\sigma e^{i(m+p-n)\sigma} \\
&= \frac{1}{4\pi} \sum_{m,p} \alpha_m \cdot\alpha_p 2\pi \delta_{m+p,n} \\
&= \frac{1}{2} \sum_m \alpha_{n-m} \cdot \alpha_m 
}
Similarly 
\[
\bar{l}_n = \frac{1}{2} \sum_m \bar{\alpha}_{n-m} \cdot \bar{\alpha}_m 
\]
\end{example}
Using these expressions and the Possion bracket relations it can be shown 
\eq{
\pb[l_m]{l_n} &= (m-n) l_{m+n} \\
\pb[\bar{l}_m]{\bar{l}_n} &= (m-n) \bar{l}_{m+n} \\
\pb[l_m]{\bar{l_n}} = 0
}
This is often called the \bam{Wit algebra}. Note $l_0, l_{\pm1},\bar{l}_0,\bar{l}_{\pm1}$ generate the algebra $\mf{sl}(2,\mbb{C})$. \\
The Hamiltonian may be written as  
\eq{
H = \frac{1}{2\pi\cst} \int_0^{2\pi} d\sigma \left[ (\del_+ X)^2 + (\del_- X)^2 \right] &= \frac{1}{2} \sum_n (\alpha_{-n}\cdot\alpha_n + \bar{\alpha}_{-n}\cdot\bar{\alpha}_n \\
&= l_0 + \bar{l_0}
}
Call the $l$ \bam{Virasoro generators}. \\

On the contraint surface $l_n=0$, one can show that $\frac{dl_n}{d\tau}=\pb[H]{l_n}=-nl_n = 0$ on this surface.


%%%%%%%%%%%%%%%%%%%%%%%%%%%%%%%%%%%%%%%%%%%
%%%%%%%%%%%%%%%%%%%%%%%%%%%%%%%%%%%%%%%%%%%
\section{Canonical Quantisation}

Our classical theory takes the form of a Poisson bracket $\pb[X^\mu]{P_\nu}$ and a constraint $T_{ab}=0$. To quantise we could possibly 
\begin{itemize}
    \item Impose the constraint on our functions to get some brackets $\pb[q^\mu]{\pi_\nu}^\prime$ and quantise by $\pb[ ]{ }^\prime \to i\comm[ ]{ }$ giving a light cone Hilbert space $\mc{H}_{l.c.}$
    \item Directly quantise $\pb[ ]{ } \to i\comm[ ]{ }$, leading to a quantum theory on which we then impose the constraint by $T_{ab}\ket{0}=0$ to get a Hilbert space $\mc{H}_Q$
\end{itemize}
The hope is that $\mc{H}_{l.c.} = \mc{H}_Q$ (and in fact is true, though not shown here). \\
We start by replaicng the \emph{fundamental} Poisson bracket relations with canonical commutation relations 
\[
\pb[X^\mu]{P_\nu} \to -i\comm[X^\mu]{P_\nu}
\]
and similarly for the modes. We introduce the \bam{Virasoro operators} 
\[
L_n = \frac{1}{2} \sum_m \alpha_{n-m} \cdot \alpha_m \quad n\neq 0
\]
where we might distinguish the $L_n$ from the classical $l_n$. and we introduce a vacuum state $\ket{0}$
\[
\forall m\geq0 \; \alpha^\mu_m \ket{0} = 0 
\]
We think of $\alpha^\mu_n$ as annihilation operators for $n>0$ and creation operators for $n<0$. \\
We notice an ambiguity in the definition of $L_0$ and $\bar{L}_0$. 
\eq{
L_0 = \frac{1}{2} \alpha_0^2 + \sum_{n>0} \alpha_{-n} \cdot \alpha_n
}
so the order of the $\alpha$ changes $L_0$ up to a constant. We define \bam{normal ordering} $::$ in the usual way and define composite operators using the ordering 
\begin{example}
\[
T_{--}(\sigma^-) = -\frac{1}{\cst} : \del_- X^\mu \del_- X_\mu :
\]
\end{example}

%%%%%%%%%%%%%%%%%%%%%%%%%%%%%%%%%%%%%%%%%%%
\subsection{Physical State Conditions}
We define the number operator $N_n$ by 
\[
n N_n = \alpha_{-n} \cdot \alpha_n
\]
and the total number operators as 
\eq{
N &= \sum_n nN_n \\
\bar{N} &= \sum_n n\bar{N}_n
}
The $L_0, \bar{L}_0$ may be written as 
\eq{
L_0 &= \frac{\cst}{4} p^2 + N \\
\bar{L}_0 &= \frac{\cst}{4} p^2 + \bar{N}
}
We will impose the conditions 
\eq{
L_n \ket{\phi} &= 0 = \bar{L_n} \ket{\phi} & n>0 \\
(L_0 - a) \ket{\phi} &= 0 = (\bar{L}_0-a)\ket{\phi} & a\in\mbb{R}
}
for $\ket{\phi}$ to be a physical state, where $a$ quantises the ambiguity in $L_0$. \\
We shall see later that the theory is consistent only if $D=26, a=1$. From now on we shall assume $a=1$. \\
It will be useful to define  
\eq{
L_0^\pm = L_0 \pm \bar{L}_0
}
and so we have
\eq{
(L_0^+-2)\ket{\phi} &= 0 \\
L_0^- &= 0 \\
L_n \ket{\phi} &= 0 = \bar{L_n} \ket{\phi}
}
Recall 
\eq{
L_0 &= \frac{\cst}{4} p^2 + N \\
\bar{L}_0 &= \frac{\cst}{4} p^2 + \bar{N}
}

%%%%%%%%%%%%%%%%%%%%%%%%%%%%%%%%%%%%%%%%%%%
\subsection{The Spectrum}
We shall look atthe lowest lying modes of the theory. 

\subsubsection*{The Tachyon}
The simplest state we can write down is the momentum eigenstate 
\eq{
\ket{k} = e^{ik\cdot x}\ket{0} \quad k_\mu \text{ some 4-vector}
}
and the action of the momentum $p_\mu$ is 
\eq{
p_\mu \ket{k} = k_\mu \ket{k}
}
We could also define a ket by a weighted sum of such states, 
\eq{
\ket{T} = \int d^D k \, T(k) \ket{k} 
}
where $T(k)$ is some function of the momentum space. The $L_0^-\ket{\phi} = 0$ condition imposes $N=\bar{N}$\footnote{This is called the "Level Matching" Condition. This is roughly the idea that the number of left and right moving operators must be the same, and this is the only condition that relates the two sides. You can think of this as the left and right sector agreeing on the mass of the state.}. Now 
\eq{
(L_0^+ - 2) T(k) \ket{k} = \left(\frac{\cst}{2}p^2 + N +\bar{N} -2 \right)T(k) \ket{k} 
}
In this case $N=\bar{N} = 0$
\eq{
\Rightarrow & \left( \frac{\cst}{2} k^2 -2 \right) T(k) = 0 \\
\Rightarrow & (k^2 + m^2) T(k) = 0 
}
where 
\eq{
m^2 = -\frac{4}{\cst}
}
This is just a mass shell condition for the momentu, space field $T(k)$. We notice that the field $T(k)$ is \bam{Tachyonic} as it is \emph{spacelike}. 
\eq{
L_n \ket{T} = 0 = \bar{L}_n \ket{T} \quad \forall n>0
}
is satisfied trivially. 

\subsubsection*{Massless states}
Next we consider states ofthe form  
\eq{
\ket{\eps} = \eps_{\mu\nu}(k) \alpha_{-1}^\mu \bar{\alpha_{-1}^\nu} \ket{k}
}
Level matching is clearly satisfies and the condition $(L_0^+-2) \ket{\eps} = 0$ gives 
\[
m^2 = 0
\]
Trivially for $n>1$ $L_n\ket{\eps}=0$. However for $L_1$ 
\eq{
L_1 \ket{\eps} &= \frac{1}{2} \sum_n \alpha_{1-n} \cdot \alpha_n \eps_{\mu\nu}(k) \alpha_{-1}^\mu \bar{\alpha_{-1}^\nu} \ket{k} \\
&= \eps_{\mu\nu}(k) \alpha_0 \cdot \alpha_1 \alpha_{-1}^\mu \bar{\alpha_{-1}^\nu} \ket{k} \\
&= \highlight{\sqrt{\frac{2}{\cst}}} \eps_{\mu\nu}(k) k_\lambda \alpha_1^\lambda alpha_{-1}^\mu \bar{\alpha_{-1}^\nu} \ket{k} \\
&= \highlight{\sqrt{\frac{2}{\cst}}} \eps_{\mu\nu}(k) k_\lambda \left( \comm[\alpha_1^\lambda]{\alpha_{-1}^\mu} + \alpha_{-1}^\mu \alpha_1^\lambda \right) \bar{\alpha}_{-1}^\nu \ket{k} \\
&= \highlight{\sqrt{\frac{2}{\cst}}} \eps_{\mu\nu}(k) k_\lambda \eta^{\lambda\mu} \bar{\alpha}_{-1}^\nu \ket{k} =0 \\
\Rightarrow \eps_{\mu\nu}(k) k^\mu = 0
}
Two states related by $\eps_{\mu\nu}(k) \to \eps_{\mu\nu}(k) + k_\mu \xi_
nu$ are physically equivalent and
\eq{
\bar{L}_1 \ket{\eps} = 0 \Rightarrow k^\nu \eps_{\mu\nu}(k) = 0
}
It is useful to decompose $\eps_{\mu\nu}(k)$ as 
\eq{
\eps_{\mu\nu}(k) &= \tilde{g}_{\mu\nu}(k) + \tilde{B}_{\mu\nu}(k) + \eta_{\mu\nu} \tilde{\phi}(k)
}
where $\tilde{g}$ tracless and symmetric, $\tilde{B}$ antisymmetric. We find that $\tilde{g}_{\mu\nu}(\eps)$ is a momentum space metric perturbation 
\eq{
\tilde{g}_{\mu\nu}(k) \sim \tilde{g}_{\mu\nu}(k) + k_\mu \xi_\nu + \xi_\mu k_\nu
}
This is simply linearised diffeomorphism invariance. \\
Also, the 'B field', which corresponds to the momentum spacetime field $B_{\mu\nu}=-B_{\nu\mu}$\footnote{"Kinda like a photon with an extra index".} where
\eq{
\tilde{B}_{\mu\nu}(k) \sim \tilde{g}_{\mu\nu}(k) + k_\mu \\lambda_\nu - \lambda_\mu k_\nu
}
In spacetime, this is a gauge invariance 
\eq{
B_{\mu\nu} \sim B_{\mu\nu} + \del_\mu \lambda_\nu - \del_\nu \lambda_\mu \quad \text{(gerbe)}
}

\subsubsection*{Massless Modes}
Let 
\eq{
\ket{g} &= h_{\mu\nu} \alpha_{-1}^{(\mu} \bar{\alpha}_{-1}^{\nu)}\ket{k} \\
\ket{B} &= B_{\mu\nu} \alpha_{-1}^{[\mu} \bar{\alpha}_{-1}^{\nu]}\ket{k} \\
\ket{\phi} = \phi \alpha_{-1}^\mu \bar{\alpha}_{-1 \mu}\ket{k}
}
These are a graviton ($g_{\mu\nu} = \eta_{\mu\nu} + h_{\mu\nu}$), B field $B_{\mu\nu} = -B_{\nu\mu}$, and Dilaton respectively. \\
One can show that these fields arise as a linear approximation to the theory described by the action
\begin{align} \label{eq:ST:3}
S = -\frac{1}{2k^2} \int d^D x \sqrt{-g} e^{-2\phi} \left( R - 4\del_\mu \phi \del^\mu \phi + \frac{1}{12} H_{\mu\nu\lambda}H^{\mu\nu\lambda}\right)
\end{align}
where $H_{\mu\nu\lambda}=\del_{[\mu}B_{\nu\lambda]}$.
We could consider the more general starting point 
\eq{
S_1[X,h] = -\frac{1}{4\pi\cst} \int_\Sigma d^2\sigma \sqrt{-h} h^{ab} \del_a X^\mu \del_b X^\nu g_{\mu\nu}(X)
}
The quantum theory has Weyl symmetry if $g_{\mu\nu}$ satisfies 
\eq{
R_{\mu\nu}=0
}
to first order in $\cst$. Higher orders give corrections to this result to next order in $\cst$
\eq{
R_{\mu\nu} + \frac{\cst}{2} R_{\mu\rho\lambda\sigma}R\indices{_\nu^\rho^\lambda^\sigma}=0
}
We can also add another term 
\eq{
S_2[X,h] &= -\frac{1}{4\pi\cst} \int_\Sigma d^2\sigma \sqrt{-h} \eps^{ab} \del_a X^\mu \del_b X^\nu B_{\mu\nu} \\
S_3 &= \frac{1}{4\pi\cst} \int_\Sigma d^2\sigma \sqrt{-h} \phi(x) R_2
}
where $R_2$ is the worldsheet Ricci scalar. The condition that the action 
\eq{
S = S_1 + S_2 + S_3
}
gives a Weyl-invariant quantum theory are what we might call equations of motion in spacetime for $g_{\mu\nu}, B_{\mu\nu}, \phi$. To leasing order in $\cst$ these equation of motion may be derived from the action \ref{eq:ST:3}

%%%%%%%%%%%%%%%%%%%%%%%%%%%%%%%%%%%%%%%%%%%
%%%%%%%%%%%%%%%%%%%%%%%%%%%%%%%%%%%%%%%%%%%
\section{Path Integral Quantisation}
Path integrals give us a conceptually different way to think about calculating transition amplitudes in QM and QFT. 
\eq{
\braket{x_f,t_f | x_i, t_i} = \int_{x(t_i)}^{x(t_f)} \mc{D}X e^{-iS[X]}
}
for 
\eq{
S[X] = \int_{t_i}^{t_f} dt L(X,\dot{X})
}
We will be interested in the path integral quantisation of the Polyakov action. 
Given some inital and final string state $\Psi_{i,f}$ the path integral is  
\eq{
\braket{\Psi_f | \Psi_i} = \int_i^f \mc{D}X \mc{D}h e^{iS[X,h]}
}
a weighted sum over all possible worldsheets with fixed inital and final states. 





\end{document}