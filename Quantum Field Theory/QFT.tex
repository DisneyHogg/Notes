\documentclass{article}

%%%%%%%%%%%%%%%%%%%%%%%%%%%%%%%%%%%%%%%%%%%%%%%%%%%%%%%%
%Packages

\usepackage[utf8]{inputenc}
\usepackage{amsmath}
\usepackage{amsthm}
\usepackage{amssymb}
\usepackage{braket} %Bras, kets, sets, caps or not. 
\usepackage{hyperref} % References and links. 
\usepackage{bm} % Bold vectors. 
\usepackage{faktor} % Write quotients nicely.
\usepackage{xfrac} % Slanted fractions with \sfrac
\usepackage{slashed} % 

%%%%%%%%%%%%%%%%%%%%%%%%%%%%%%%%%%%%%%%%%%%%%%%%%%%%%%%%
%Definitions

%Environments
\newtheorem{theorem}{Theorem}[subsection]
\newtheorem{corollary}{Corollary}[theorem]
\newtheorem{lemma}[theorem]{Lemma}
\newtheorem*{remark}{Remark}
\newtheorem{definition}{Definition}[subsection]
\newtheorem{fact}{Fact}[subsection]
\newtheorem*{idea}{Idea}
\newtheorem{example}{Example}[subsection]

%Text math operators.
\DeclareMathOperator{\spn}{Span}
\DeclareMathOperator{\tr}{Tr}
\DeclareMathOperator{\SU}{SU}
%\DeclareMathOperator{\dim}{dim}
\DeclareMathOperator{\rank}{rank}
\DeclareMathOperator{\diag}{diag}

%Symbolic shortcuts 
\newcommand{\eps}{\epsilon}
\newcommand{\curl}{\bm{\nabla}\times}
\newcommand{\divergence}{\bm{\nabla}\cdot}
\newcommand{\grad}{\bm{\nabla}}
\newcommand{\del}{\partial}
\newcommand{\tmeas}{\frac{d^3\bm{p}}{(2\pi)^3}} %Momentum measure in 3 dimensions 

%Text type shortcuts
\newcommand{\bam}[1]{\textbf{#1}}
\newcommand{\mf}[1]{\mathfrak{#1}}
\newcommand{\mbb}[1]{\mathbb{#1}}
\newcommand{\mc}[1]{\mathcal{#1}}

%Functions shortcuts.
\newcommand{\comm}[2][]{\left[ #1, #2 \right]} %Commutator .
\newcommand{\acomm}[2][]{\left\{ #1, #2 \right\}} %Anticommutator .
\newcommand{\pd}[2][]{\frac{\partial#1}{\partial#2}} %Partial derivative. 
\newcommand{\pds}[2][]{\frac{\partial^2 #1}{\partial #2^2}} %Second partial derivative.

\newcommand{\be}{\begin{equation}}  
\newcommand{\ee}{\end{equation}}
%Used like \[ \] but with tags.

%Small matrix used in text.
\newenvironment{psmallmatrix}
  {\left(\begin{smallmatrix}}
  {\end{smallmatrix}\right)}

%%%%%%%%%%%%%%%%%%%%%%%%%%%%%%%%%%%%%%%%%%%%%%%%%%%%%%%%
%Preamble

\title{Quantum Field Theory Revision Notes}
\author{Linden Disney-Hogg}
\date{January 2019}

%%%%%%%%%%%%%%%%%%%%%%%%%%%%%%%%%%%%%%%%%%%%%%%%%%%%%%%%
%%%%%%%%%%%%%%%%%%%%%%%%%%%%%%%%%%%%%%%%%%%%%%%%%%%%%%%%
\begin{document}

\maketitle
\tableofcontents

\section{Introduction}
A brief overview of some key ideas, concepts, and facts that I find useful in revising QFT. 

%%%%%%%%%%%%%%%%%%%%%%%%%%%%%%%%%%%%%%%%%%%%%%%%%%%%%%%%
%%%%%%%%%%%%%%%%%%%%%%%%%%%%%%%%%%%%%%%%%%%%%%%%%%%%%%%%
\section{Preliminaries}

%%%%%%%%%%%%%%%%%%%%%%%%%%%%%%%%%%%%%%%%%%%%%%%%%%%%%%%%
\subsection{Miscellaneous}

\begin{definition}
In these notes attention will only be paid to $3+1$ dimension spacetime. As such Greek indices (e.g $\mu$, $\nu$) will run from $0$ to $3$, while Latin indices (e.g. $i$, $j$) will run from $1$ to $3$. 
\end{definition}

\begin{definition}[Field]
In these notes a \bam{field} will be a function of spacetime $\phi=\phi(x)$ where $x=(t,\bm{x})$. 
\end{definition}

\begin{definition}[Minkowski Metric]
In these notes the mostly-minus metric convention will be adopted, so the \bam{Minkowski metric} is 
\[
\eta^{\mu\nu}=\diag(1, -1, -1, -1)
\]
\end{definition}

\begin{definition}[Natural Units]
In QFT, \bam{natural units} are used where
\[
c=1=\hbar
\]
$[c]=LT^{-1}$, $[\hbar]=L^2 M T^{-1}$, so in natural units
\[
L=T=M^{-1}
\]
Units are therefore given in \bam{mass dimension}, e.g. if $[f]=M^d$, write $[f]=d$.
\end{definition}

\begin{fact}
A table of common values and their mass dimension is given below. 
\begin{center}$
\begin{array}{ccc}
    \text{Quantity} & \text{Symbol} & \text{Mass Dimension} \\
    \hline
    \hline
    \text{Energy} & E & 1 \\
    \text{Action} & S & 0 \\
    \text{Lagrangian} & \mc{L} & 4 \\
    \text{Spacetime derivative} & \del_\mu & 1 \\
    \text{Spacetime measure} & d^4x & -4 \\
\end{array}
$\end{center}
\end{fact}

%%%%%%%%%%%%%%%%%%%%%%%%%%%%%%%%%%%%%%%%%%%%%%%%%%%%%%%%
\subsection{Lorentz Transformations}

\begin{definition}[Lorentz Group]
The \bam{Lorentz Group} is $O(1,3)$, the group of transformations that preserve the Minkowski metric, i.e.
\[
\forall \Lambda\in O(1,3) \quad \Lambda^\mu_\rho \Lambda^\nu_\sigma \eta^{\rho\sigma}=\eta^{\mu\nu}
\]
The \bam{proper orthochronous Lorentz group} is the connected component containing the identity $SO^+(1,3)$. These preserve orientation and the direction of time. 
\end{definition}

\begin{theorem}
Under a Lorentz transformation $\Lambda$, $x\to x^\prime=\Lambda x$, a scalar field $\phi=\phi(x)$ transforms as 
\[
\phi\to\phi^\prime, \quad \phi^\prime(x^\prime)=\phi(x)
\]
Vector fields $A^\mu(x)$ transform as 
\[
A\to A^\prime, \quad \left(A^\prime\right)^\mu(x^\prime)=\Lambda^\mu_\nu A^\nu (x)
\]
In general a field transforms as 
\[
\phi\to\phi^\prime, \quad (\phi^\prime)^\mu(x^\prime)=D(\Lambda)^\mu_\nu \phi^\nu(x)
\]
where $D$ is some representation of the Lorentz group. 
\end{theorem}

\begin{theorem}
Let $\Lambda^\mu_\nu$ be a Lorentz transform with infinitesimal expansion 
\[
\Lambda^\mu_\nu=\delta^\mu_\nu+\eps \omega^\mu_\nu +\mathcal{O}(\eps^2) 
\]
Then 
\[
\omega^{\mu\nu}+\omega^{\nu\mu}=0
\]
\end{theorem}

\begin{definition}
Define the matrix $M^{\rho\sigma}$ by 
\[
\left( M^{\rho\sigma} \right)^{\mu\nu} = \eta^{\rho\mu}\eta^{\sigma\nu}-\eta^{\sigma\mu}\eta^{\rho\nu}
\]
These matrices for generate all antisymmetric tensors by \[
\omega^{\mu\nu}=\frac{1}{2}\left(\Omega_{\rho\sigma}M^{\rho\sigma}\right)^{\mu\nu}
\]
summation convention used. Hence $\set{M^{\rho\sigma}}$ is a generating set for the Lorentz transforms. 
\end{definition}

\begin{definition}[Lorentz Algebra]
The matrices $m^{\rho\sigma}$ satisfy the \bam{Lorentz algebra}
\[
\comm[M^{\rho\sigma}]{M^{\mu\nu}}=\eta^{\sigma\mu}M^{\rho\nu}-\eta^{\rho\mu}M^{\sigma\nu}-\eta^{\sigma\nu}M^{\rho\mu}+\eta^{\rho\nu}M^{\sigma\mu}
\]
\end{definition}

%%%%%%%%%%%%%%%%%%%%%%%%%%%%%%%%%%%%%%%%%%%%%%%%%%%%%%%%
\subsection{Pauli Matrices}

\begin{definition}[Pauli Matrices]
The \bam{Pauli matrices} are

\begin{align*}
\sigma_1 &= \begin{pmatrix} 0 & 1 \\ 1 & 0\end{pmatrix}  \\
\sigma_2 &= \begin{pmatrix} 0 & -i \\ i & 0\end{pmatrix}  \\
\sigma_3 &= \begin{pmatrix} 1 & 0 \\ 0 & -1\end{pmatrix}  
\end{align*}

Note they are all Hermitian and traceless.
\end{definition}

\begin{fact}
\[
\sigma_i \sigma_j = \delta_{ij}I +i\epsilon_{ijk}\sigma_k
\]
Hence 
\begin{align*}
    \comm[\sigma_i]{\sigma_j} &= 2i\eps_{ijk}\sigma_k \\
    \acomm[\sigma_i]{\sigma_j} &= 2\delta_{ij}I
\end{align*}
\end{fact}

\begin{fact}
It will be common to notate $\bm{\sigma}=(\sigma_1, \sigma_2, \sigma_3)$, and $\sigma^\mu=(I,\bm{\sigma})$, $\bar{\sigma}^\mu=(I, -\bm{\sigma})$
\end{fact}
%%%%%%%%%%%%%%%%%%%%%%%%%%%%%%%%%%%%%%%%%%%%%%%%%%%%%%%%
%%%%%%%%%%%%%%%%%%%%%%%%%%%%%%%%%%%%%%%%%%%%%%%%%%%%%%%%
\section{Field Theories}

%%%%%%%%%%%%%%%%%%%%%%%%%%%%%%%%%%%%%%%%%%%%%%%%%%%%%%%%
\subsection{Basics}

\begin{definition}[Free Theory]
A \bam{free field theory} is one that contains no interaction terms, and so there are no terms containing multiple different fields. 
\end{definition}

\begin{definition}[Euler-Lagrange Equations]
Given a Lagrangian density $\mathcal{L}=\mathcal{L}(\phi_a,\del_\mu \phi_a)$ the \bam{Euler-Lagrange (E-L) equations} are 
\[
\del_\mu \pd[\mc{L}]{(\del_\mu \phi_a)}-\pd[\mc{L}]{\phi_a}=0
\]
These are necessary and sufficient conditions for the action 
\[
S=\int \mc{L}(x) \, d^4x
\]
to be stationary, given sufficiently decaying boundary terms. 
\end{definition}

\begin{definition}[Lorentz Invariance]
A field theory is \bam{Lorentz invariant} if the action is unchanged by a Lorentz transformation. 
\end{definition}

\begin{idea}
It is common to look for theories that exhibit Lorentz invariance, as it is believed to be a true symmetry of the universe.
\end{idea}

\begin{definition}[Klein-Gordon Field]
The \bam{Klein-Gordon (K-G) field} is the real field $\phi$ with Lagrangian density
\[
\mc{L}=\frac{1}{2}\eta^{\mu\nu} \del_\mu\phi \del_\nu\phi-\frac{1}{2}m^2\phi^2
\]
It is a free field theory, and will be the prototypical field theory. The E-L equation is the \bam{Klein-Gordon equation}
\[
\left( \del_\mu \del^\mu + m^2 \right)\phi=0
\]
The corresponding action is Lorentz invariant. 
\end{definition}


%%%%%%%%%%%%%%%%%%%%%%%%%%%%%%%%%%%%%%%%%%%%%%%%%%%%%%%%
\subsection{Conserved Quantities}

\begin{theorem}[Noether's Theorem]
Every continuous symmetry of the action of the form $\phi\to\phi+\alpha\Delta\phi$, that induces the transform on the Lagrangian density 
\[
\mc{L} \to \mc{L}+\alpha \del_\mu X^\mu
\]
gives rise to a conserved current $j^\mu$ s.t. $\del_\mu j^\mu = 0$, given by
\[
j^\mu = \pd[\mc{L}]{(\del_\mu \phi)}\Delta\phi-X^\mu
\]
and a corresponding conserved charge
\[
Q=\int_{\mbb{R}^3} j^0 \, d^3x
\]
\end{theorem}

\begin{example}[Complex Scalar Field]
The Lagrangian for a complex scalar field $\psi$ is 
\[
\mc{L}=\del_\mu \psi^\dagger \del^\mu \psi - V\left(|\psi|^2\right)
\]
This is invariant under the continuous symmetry $\psi \to e^{i\alpha}\psi$. The corresponding conserved current is 
\[
j^\mu=i\left( \psi\del^\mu\psi^\dagger-\psi^\dagger\del^\mu\psi \right)
\]
\end{example}

\begin{definition}[Energy-Momentum Tensor]
For a given Lagrangian density the \bam{energy momentum tensor} is defined as  
\[
T^{\mu\nu}=\pd[\mc{L}]{(\del_\mu \phi)} \del^\nu \phi - \eta^{\mu\nu}\mc{L}
\]
\end{definition}

\begin{theorem}
If the Lagrangian density depends only on $x$ implicitly, then translation $x^\mu \to x^\mu-\alpha\eps^\mu$ is a continuous symmetry, and hence for each $\nu$, $T^{\mu\nu}$ is a conserved current. 
\end{theorem}

\begin{definition}[Energy and Momentum]
The energy $E$ is defined as 
\[
E=\int_{\mbb{R}^3} T^{00} \, d^3x
\]
and the momentum $\bm{P}$ is defined with components 
\[
P^i = \int_{\mbb{R}^3} T^{0i} \, d^3
\]

\end{definition}

%%%%%%%%%%%%%%%%%%%%%%%%%%%%%%%%%%%%%%%%%%%%%%%%%%%%%%%%
\subsection{Hamiltonian}

\begin{definition}[Conjugate Momentum]
The \bam{conjugate momentum} to a field $\phi$ is defined as 
\[
\pi = \pd[\mc{L}]{(\del_0\phi)} = \pd[\mc{L}]{\dot{\phi}}
\]
\end{definition}

\begin{definition}[Hamiltonian Density]
Given a Lagrangian density $\mc{L}$, the corresponding \bam{Hamiltonian density} is defined as 
\[
\mc{H}=\pi \dot{\phi}-\mc{L}
\]
where $\mc{H}$ is a function of $\pi$ instead of $\dot{\phi}$
\end{definition}

\begin{definition}[Hamilton's Equations]
Given a Hamiltonian density $\mc{H}$, \bam{Hamilton's equations} are 
\[
\dot{\phi}=\pd[\mc{H}]{\pi} \text{ and } \dot{\pi}=-\pd[\mc{H}]{\phi} 
\]
\end{definition}

\begin{example}
For the Klein-Gordon Lagrangian the conjugate momentum is 
\[
\pi=\dot{\phi}
\]
and so the Hamiltonian density becomes 
\[
\mc{H}=\frac{1}{2}\pi^2+\frac{1}{2}|\grad\phi|^2+\frac{1}{2}m^2\phi^2
\]
\end{example}
%%%%%%%%%%%%%%%%%%%%%%%%%%%%%%%%%%%%%%%%%%%%%%%%%%%%%%%%
%%%%%%%%%%%%%%%%%%%%%%%%%%%%%%%%%%%%%%%%%%%%%%%%%%%%%%%%
\section{Quantisation}

\begin{definition}[Quantum Field]
A \bam{quantum field} is an operator valued function that obeys the commutation relations 
\begin{align*}
    \comm[\phi_a(x)]{\phi_b(y)} &= 0 \\
    \comm[\pi^a(x)]{\pi^b(y)} &= 0 \\
    \comm[\phi_a(x)]{\pi^b(y)} &= i\delta(x-y) \delta_a^b
\end{align*}
\end{definition}

\begin{example}
To quantise a real scalar field that satisfies K-G, write 
\[
\phi(t,\bm{x})=\int\tmeas e^{i\bm{p}\cdot\bm{x}} \hat{\phi}(t,\bm{p})
\]
using Fourier transform. Then the K-G equation forces 
\[
\left[\pds{t}+(|\bm{p}|^2+m^2)\right]\hat{\phi}(t,\bm{p})=0
\]
Hence letting $\omega_{\bm{p}}=\sqrt{|\bm{p}|^2+m^2}$ yields solutions 
\[
\hat{\phi}(t,\bm{p})=\frac{a_{\bm{p}}}{\sqrt{2\omega_p}} e^{-i\omega_{\bm{p}} t}
\]
Where $a_{\bm{p}}$ is some operator, and the factor of 
$\frac{1}{\sqrt{2\omega_p}}$ is arbitrary. Then, ensuring that $\phi$ is real gives 
\[
\phi(t,\bm{x})=\int\tmeas  \frac{1}{\sqrt{2\omega_p}} \left( a_{\bm{p}} e^{i\bm{p}\cdot\bm{x}-i\omega_{\bm{p}} t} + a_{\bm{p}}^\dagger e^{-i\bm{p}\cdot\bm{x}+i\omega_{\bm{p}} t} \right)
\]
Hence 
\[
\pi(t,\bm{x})=\int\tmeas  (-i)\sqrt{\frac{\omega_p}{2}} \left( a_{\bm{p}} e^{i\bm{p}\cdot\bm{x}-i\omega_{\bm{p}} t} - a_{\bm{p}}^\dagger e^{-i\bm{p}\cdot\bm{x}+i\omega_{\bm{p}} t} \right)
\]
Here $a_{\bm{p}}$ is call an \bam{annihilation operator} and $a_{\bm{p}}^\dagger$ a \bam{creation operator}. Note this quantisation has occured in the Heisenberg picture, where the operator is time dependent. In the Schr\"odinger picture time dependence needs to be removed.
\end{example}

\begin{definition}[Quantising a Real Scalar Field]
Given a real scalar field $\phi$ in the Schr\"odinger picture, satisfying the K-G equation, with conjugate momentum $\pi$, it shall be defined to be quantised as
\[
\phi(\bm{x})=\int\tmeas  \frac{1}{\sqrt{2\omega_p}} \left( a_{\bm{p}} e^{i\bm{p}\cdot\bm{x}} + a_{\bm{p}}^\dagger e^{-i\bm{p}\cdot\bm{x}} \right)
\]
and
\[
\pi(\bm{x})=\int\tmeas  (-i)\sqrt{\frac{\omega_p}{2}} \left( a_{\bm{p}} e^{i\bm{p}\cdot\bm{x}} - a_{\bm{p}}^\dagger e^{-i\bm{p}\cdot\bm{x}} \right)
\]
\end{definition}

\begin{theorem}
Given the above quantisation of a real scalar quantum field, the following commutation relations are induced on the operators $a_{\bm{p}}$, $a_{\bm{p}}^\dagger$. 
\begin{align*}
    \comm[a_{\bm{p}}]{a_{\bm{q}}} &= 0 \\
    \comm[a_{\bm{p}}^\dagger]{a_{\bm{q}}^\dagger} &= 0 \\
    \comm[a_{\bm{p}}]{a_{\bm{q}}^\dagger} &= (2\pi)^3 \delta(\bm{p}-\bm{q})
\end{align*}
\end{theorem}

\begin{definition}[Normal Ordering]
Given a product of a string of operator $\phi_1(\bm{x}_1)\dots\phi_n(\bm{x}_n)$ the \bam{normal ordering} of the product is denoted as 
\[
:\phi_1(\bm{x}_1)\dots\phi_n(\bm{x}_n):
\]
and is defined by moving all the annihilation operators to the right of the product, and all the creation operators to the left 
\end{definition}

\begin{theorem}
The corresponding Hamiltonian is 
\begin{align*}
    H &= \int d^3x \, \frac{1}{2}\pi^2+\frac{1}{2}|\grad\phi|^2+\frac{1}{2}m^2\phi^2 \\
    &= \frac{1}{2} \int \tmeas \omega_{\bm{p}} (a_{\bm{p}}a_{\bm{p}}^\dagger+a_{\bm{p}}^\dagger a_{\bm{p}})
\end{align*}
After normal ordering this is 
\[
:H:=\int\tmeas \omega_{\bm{p}} a_{\bm{p}}^\dagger a_{\bm{p}}
\]
\end{theorem}

\begin{definition}[Vacuum]
The \bam{vacuum state} $\ket{0}$ is defined such that 
\[
\forall \bm{p} \quad a_{\bm{p}} \ket{0} = 0
\]
\end{definition}


%%%%%%%%%%%%%%%%%%%%%%%%%%%%%%%%%%%%%%%%%%%%%%%%%%%%%%%%
%%%%%%%%%%%%%%%%%%%%%%%%%%%%%%%%%%%%%%%%%%%%%%%%%%%%%%%%
\section{Perturbation Theory}


%%%%%%%%%%%%%%%%%%%%%%%%%%%%%%%%%%%%%%%%%%%%%%%%%%%%%%%%
%%%%%%%%%%%%%%%%%%%%%%%%%%%%%%%%%%%%%%%%%%%%%%%%%%%%%%%%
\section{Fermions}
This section will begin to discuss spinor fields, which will turn out to represent fermions when quantised.
%%%%%%%%%%%%%%%%%%%%%%%%%%%%%%%%%%%%%%%%%%%%%%%%%%%%%%%%
\subsection{Clifford Algebra}
The \bam{Clifford algebra} is $\set{ \gamma^\mu }$ with the anticommutation relation 
\[
\acomm[\gamma^\mu]{\gamma^\nu} = 2\eta^{\mu\nu}
\]

\begin{definition}[$\gamma^5$]
Define
\[
\gamma^5 = i \gamma^0 \gamma^1 \gamma^2 \gamma^3
\]
\end{definition}

\begin{theorem}
The following are properties of $\gamma^\mu$:
\begin{itemize}
    \item $\forall p\in\mbb{N}_0, \; \tr \left( \prod _ { i = 1 } ^ { 2 p + 1 } \gamma ^ { \mu _ { i } } \right)$
    \item $\tr(\gamma^\mu \gamma^\nu)=4\eta^{\mu\nu}$
    \item $\comm[\gamma^\mu\gamma^\nu]{\gamma^\rho\gamma^\sigma}=2\eta^{\nu\rho}\gamma^\mu\gamma^\sigma-2\eta^{\mu\rho}\gamma^\nu\gamma^\sigma+2\eta^{\nu\sigma}\gamma^\rho\gamma^\mu-2\eta^{\mu\sigma}\gamma^\rho\gamma^\nu$
    \item $\acomm[\gamma^\mu]{\gamma^5}=0$
    \item $(\gamma^5)^2=I$
    \item $\tr{\gamma^5}=0$
\end{itemize}
\end{theorem}




\begin{definition}[Chiral Representation]
The \bam{chiral} or \bam{Weyl representation} of the Clifford Algebra is the 4 dimensional representation
\begin{align*}
    \gamma^0 &= \begin{pmatrix} 0 & I_2 \\ I_2 & 0 \end{pmatrix} \\
    \gamma^i &= \begin{pmatrix} 0 & \sigma_i \\ -\sigma_i & 0 \end{pmatrix} \\ 
\end{align*}
In this representation 
\[
\gamma^5=\begin{pmatrix} -I_2 & 0 \\ 0 & I_2 \end{pmatrix}
\]
\end{definition}

%%%%%%%%%%%%%%%%%%%%%%%%%%%%%%%%%%%%%%%%%%%%%%%%%%%%%%%%
\subsection{Spinors}

\begin{definition}[$S^{\mu\nu}$]
Define
\[
S^{\mu\nu}=\frac{1}{4}\comm[\gamma^\mu]{\gamma^\nu}=\frac{1}{2}\left( \gamma^\mu \gamma^\nu -\eta^{\mu\nu} \right)
\]
\end{definition}

\begin{theorem}
$S^{\mu\nu}$ satisfies the commutation relations
\begin{align*}
\comm[S^{\mu\nu}]{\gamma^\rho} &= \gamma^\mu \eta^{\nu\rho}-\gamma^\nu \eta^{\rho\mu} \\
\comm[S^{\rho\sigma}]{S^{\mu\nu}} &= \eta^{\sigma\mu}S^{\rho\nu}-\eta^{\rho\mu}S^{\sigma\nu}-\eta^{\sigma\nu}S^{\rho\mu}+\eta^{\rho\nu}S^{\sigma\mu}
\end{align*}
and hence gives a representation of the Lorentz algebra.
\end{theorem}

\begin{definition}[Spinor]
A \bam{Dirac spinor} is a 4-component vector $\psi_\alpha$ that transforms under a Lorentz transform 
\[
\Lambda=\exp\left(\frac{1}{2}\Omega_{\rho\sigma}M^{\rho\sigma}\right)
\]
as 
\[
\psi^{\alpha}(x)\to S[\Lambda]_{\beta}^{\alpha}\psi^{\beta}\left(\Lambda^{-1}x\right)
\]
where
\[
S[\Lambda]=\exp\left(\frac{1}{2}\Omega_{\rho\sigma}S^{\rho\sigma}\right)
\]
\end{definition}

\begin{theorem}
No such representation of the Lorentz group is unitary.
\end{theorem}
\begin{proof}
$S$ is unitary $\iff$ $S^{\rho\sigma}$ is anti-Hermitian, i.e. $\left( S^{\rho\sigma} \right)^\dagger = -S^{\rho\sigma}$. Now 
\[
\left( S^{\rho\sigma} \right)^\dagger = -\comm[{\gamma^\mu}^\dagger]{{\gamma^\nu}^\dagger}
\]
Hence for $S^{\rho\sigma}$ to be anti-Hermitian, all $\gamma^\mu$ must be Hermitian or all must be anti-Hermitian. Now $(\gamma^0)^2=I$ so $\gamma^0$ has real eigenvalues, hence cannot be anti-Hermitian. Also, $(\gamma^i)^2=-I$ so $\gamma^i$ has imaginary eigenvalues, hence cannot be Hermitian. 
\end{proof}

\begin{example}
Calculating explicitly in the chiral representation yields 
\begin{align*}
    S^{ij} &= \frac{-i}{2}\eps_{ijk} \begin{pmatrix} \sigma_k & 0 \\ 0 & \sigma_k \end{pmatrix} \\
    S^{0i} &= \frac{1}{2} \begin{pmatrix} -\sigma_i & 0 \\ 0 & \sigma_i \end{pmatrix}
\end{align*}
Hence if $\Lambda$ is a pure rotation, i.e. the only non-zero $\Omega_{\rho\sigma}$ are $\Omega_{ij}=-\eps_{ijk}\phi^k$, then 
\[
S[\Lambda]=\begin{pmatrix} e^{\frac{i\bm{\phi}\cdot\bm{\sigma}}{2}} & 0 \\ 0 & e^{\frac{i\bm{\phi}\cdot\bm{\sigma}}{2}} \end{pmatrix}
\]
Alternatively, if $\Lambda$ is a pure boost, i.e the only non-zero $\Omega_{\rho\sigma}$ are $\Omega_{0i}=-\Omega_{i0}=\chi_i$, then
\[
S[\Lambda]=\begin{pmatrix} e^{-\frac{\bm{\chi}\cdot\bm{\sigma}}{2}} & 0 \\ 0 & e^\frac{\bm{\chi}\cdot\bm{\sigma}}{2} \end{pmatrix}
\]
\end{example}

\begin{definition}[Chiral Spinors]
As the chiral representation matrices are block diagonal, the representation is reducible into the direct sum of two irreducible 2-dimensional representations. Write the representation space as $\mbb{C}^4=\mbb{C}^2\oplus\mbb{C}^2$ letting 
\[
\psi=\begin{pmatrix} U_L \\ U_R \end{pmatrix}
\]
$U_{L/R}$ are called \bam{chiral} or \bam{Weyl spinors}. Under rotations and boosts they transform as 
\begin{align*}
    U_{L/R} &\to e^{\frac{i\bm{\phi}\cdot\bm{\sigma}}{2}} U_{L/R} \\ 
    U_{L/R} &\to e^{\mp\frac{\bm{\chi}\cdot\bm{\sigma}}{2}} U_{L/R}
\end{align*}
\end{definition}

\begin{definition}[Projection Operators]
Define the operators 
\begin{align*}
    P_L &= \frac{1}{2}(I-\gamma^5) \\
    P_R &= \frac{1}{2}(I+\gamma^5) 
\end{align*}
It can be verified that $P_{L/R}^2=P_{L/R}$ and $P_L P_R =0$ so they are orthogonal \bam{projection operators}
\end{definition}

\begin{definition}[Left and Right Handed Spinors]
Define a \bam{left handed spinor} as 
\[
\psi_L=P_L \psi
\]
and similarly a \bam{right handed spinor} as 
\[
\psi_R=P_R \psi
\]
\end{definition}

%%%%%%%%%%%%%%%%%%%%%%%%%%%%%%%%%%%%%%%%%%%%%%%%%%%%%%%
\subsection{Parity}

\begin{fact}
The connected components of the Lorentz group have a Klein 4 group structure induced by the maps 
\[
T: (t,\bm{x}) \mapsto (-t,\bm{x}) \quad \text{(Time reversal)}
\]
and
\[
P: (t,\bm{x}) \mapsto (t, -\bm{x}) \quad \text{(Parity transform)}
\]
\end{fact}
\begin{fact}
Let $R$ be the generator of a rotation, i.e 
\[
R=-\frac{1}{2}\eps_{ijk}\phi_k M^{ij}
\]
and $B$ the generator of a boost, i.e. 
\[
B=\frac{1}{2}\chi_i (M^{0i}-M^{i0})
\]
then
\begin{align*}
    PR &= RP \\
    PB &= -BP 
\end{align*}
and so, after a parity transform, under rotation or boost
\begin{align*}
    U_{L/R} &\to e^{\frac{i\bm{\phi}\cdot\bm{\sigma}}{2}} U_{L/R} \\ 
    U_{L/R} &\to e^{\pm\frac{\bm{\chi}\cdot\bm{\sigma}}{2}} U_{L/R}
\end{align*}
Hence 
\[
P:\psi_{L/R}(t,\bm{x}) \mapsto \psi_{R/L}(t,-\bm{x})
\]
\end{fact}

\begin{fact}
For a Dirac spinor, parity transform is implemented as 
\[
P: \psi(t,\bm{x}) \mapsto \gamma^0 \psi(t,-\bm{x})
\]
\end{fact}

\begin{definition}[Vector-Like and Chiral Theories]
If a theory has symmetry under $\psi_L \leftrightarrow \psi_R$ then it is called \bam{vector-like}. If not, and $psi_L, \psi_R$ appear differently, then the theory is \bam{chiral}. 
\end{definition}

%%%%%%%%%%%%%%%%%%%%%%%%%%%%%%%%%%%%%%%%%%%%%%%%%%%%%%%
\subsection{Lorentz Invariant Spinor Action}

\begin{fact}
In the chiral representation, ${\gamma^0}^\dagger=\gamma^0$, ${\gamma^i}^\dagger=-\gamma^i$. Hence
\[
\gamma^0 \gamma^\mu \gamma^0={\gamma^\mu}^\dagger
\]
so 
\[
(S^{\mu\nu})^\dagger=-\gamma^0 S^{\mu\nu} \gamma^0
\]
yielding
\[
S[\Lambda]^\dagger=\gamma^0 S[\Lambda]^{-1} \gamma^0
\]
\end{fact}

\begin{definition}[Dirac Adjoint]
The \bam{Dirac adjoint} of a spinor $\psi$ is 
\[
\bar{\psi}=\psi^\dagger \gamma^0
\]
\end{definition}

\begin{theorem}
The following are Lorentz invariant combinations
\begin{center}$
\begin{array}{ccc}
    \text{Coupling} & \text{Type} & \text{Quantity} \\
    \hline
    \hline
    \bar{\psi}\psi & \text{scalar} & 1 \\
    \bar{\psi}\gamma^\mu\psi & \text{vector} & 4 \\
    \bar{\psi} S^{\mu\nu} \psi & \text{tensor} & 6 \\
    \bar{\psi}\gamma^5\psi & \text{pseudoscalar} & 1 \\
    \bar{\psi}\gamma^5\gamma^\mu\psi & \text{pseudovector} & 4 \\
\end{array}
$\end{center}
\end{theorem}

\begin{definition}[Slash Notation]
The contraction $\gamma^\mu A_\mu=\gamma_\mu A^\mu$ is written in \bam{slash notation} as $\slashed{A}$
\end{definition}

\begin{definition}[Dirac Lagrangian]
The action
\[
S=\int d^{4}x \underbrace{\bar{\psi}(x)\left(i\gamma^\mu\del_\mu-m\right)\psi(x)}_{\mc{L}_D}
\]
Is Lorentz invariant. The Lagrangian $\mc{L}_D$ is the \bam{Dirac Lagrangian}
\end{definition}

\begin{definition}[Dirac Equation]
The Euler-Lagrane equation corresponding to the Dirac Lagrangian is the \bam{Dirac equation}
\[
\left(i\gamma^\mu\del_\mu-m\right)\psi=0
\]
In slashed notation this is 
\[
(i \slashed{\del}-m)\psi=0
\]
\end{definition}

\begin{fact}
If $\psi$ satisfies the Dirac equation then each component of $\psi$ also satisfies the Klein-Gordon equation. 
\end{fact}

\begin{example}
The equations of motion for a spinor field are first order, and so only an initial value needs to be specified. Hence is can be specified that $\mc{L}_D=0$ and this remains true. 
The energy-momentum tensor for the Dirac Lagrangian is then  
\[
T^{\mu\nu} = i\bar{\psi}\gamma^\mu\del^\nu\psi-\eta^{\mu\nu}\mc{L}_D = i\bar{\psi}\gamma^\mu\del^\nu\psi
\]
There is also a conserved current arising from Lorentz invariance. Consider 
\[
\psi^\alpha \to S[\Lambda]^\alpha_\beta \psi^\beta (x^\mu-\omega^\mu_\nu x^\nu )
\]
with $\omega^\mu_\nu = \frac{1}{2} \Omega_{\rho\sigma}(M^{\rho\sigma})^\mu_\nu$. Note $\Omega$ can be taken to be antisymmetric as the symmetric part will give $0$ contribution, so $\omega^{\mu\nu}=\Omega^{\mu\nu}$. Then 
\begin{align*}
\delta \psi^\alpha &= -\omega^\mu_\nu x^\nu \del_\mu \psi^\alpha + \frac{1}{2}\Omega_{\rho\sigma}(S^{\rho\sigma})^\alpha_\beta \psi^\beta \\
&= -\omega^{\mu\nu} \left[ x_\nu \del_\mu \psi^\alpha -\frac{1}{2} (S_{\mu\nu})^\alpha_\beta \psi^\beta \right]
\end{align*}
giving the conserved current
\[
(J^\mu)^{\rho\sigma}=x^\rho T^{\mu\sigma}-x^\sigma T^{\mu\rho}-i\bar{\psi} \gamma^\mu S^{\rho\sigma} \psi
\]
In addition, there are internal symmetries due to phase of the field. Considering the symmetry $\psi \to e^{i\alpha} \psi$ gives the conserved current 
\[
j^\mu = \bar{\psi} \gamma^\mu \psi
\]
\end{example}

\begin{definition}[Weyl Equation]
Expanding the Dirac equation in terms of chiral spinors gives 
\begin{align*}
    i \bar{\sigma}^\mu \del_\mu U_L &= 0 \\
    i \sigma^\mu \del_\mu U_R &= 0
\end{align*}
\end{definition}

%%%%%%%%%%%%%%%%%%%%%%%%%%%%%%%%%%%%%%%%%%%%%%%%%%%%%%%%
\subsection{Plane Wave Solutions}

\begin{theorem}
Plane wave solutions to the Dirac equation of the form 
\[
\psi(x) = u_p e^{-i p\cdot x }
\]
are
\[
u_p = \begin{pmatrix} \sqrt{p\cdot\sigma} \xi \\ \sqrt{p\cdot\bar{\sigma}} \xi \end{pmatrix}
\]
for any two component spinor $\xi$ normalised such that $\xi^\dagger \xi = 1$. There are also negative frequency solutions of the form 
\[
\psi(x) = v_p e^{+i p\cdot x }
\]
with 
\[
v_p = \begin{pmatrix} \sqrt{p\cdot\sigma} \xi \\ -\sqrt{p\cdot\bar{\sigma}} \xi \end{pmatrix}
\]
\end{theorem}

\begin{definition}[Helicity Operator]
The \bam{helicity operator} $h$ projects the angluar momentum along the direction of motion 
\[
h=\hat{\bm{p}} \cdot \bm{s} = \frac{1}{2}\hat{p}_i \begin{pmatrix} \sigma_i & 0 \\ 0 & \sigma_i \end{pmatrix}
\]
\end{definition}

%%%%%%%%%%%%%%%%%%%%%%%%%%%%%%%%%%%%%%%%%%%%%%%%%%%%%%%%
%%%%%%%%%%%%%%%%%%%%%%%%%%%%%%%%%%%%%%%%%%%%%%%%%%%%%%%%
\section{Quantum Electrodynamics (QED)}


\end{document}