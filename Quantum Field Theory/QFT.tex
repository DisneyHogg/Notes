\documentclass{article}

%%%%%%%%%%%%%%%%%%%%%%%%%%%%%%%%%%%%%%%%%%%%%%%%%%%%%%%%
%Packages

\usepackage[utf8]{inputenc}
\usepackage{amsmath}
\usepackage{amsthm}
\usepackage{amssymb}
\usepackage{hyperref} % References and links. 
\usepackage{bm} % Bold vectors. 
\usepackage{faktor} % Write quotients nicely.
\usepackage{xfrac} % Slanted fractions with \sfrac

%%%%%%%%%%%%%%%%%%%%%%%%%%%%%%%%%%%%%%%%%%%%%%%%%%%%%%%%
%Definitions

%Environments
\newtheorem{theorem}{Theorem}[subsection]
\newtheorem{corollary}{Corollary}[theorem]
\newtheorem{lemma}[theorem]{Lemma}
\newtheorem*{remark}{Remark}
\newtheorem{definition}{Definition}[subsection]
\newtheorem{fact}{Fact}[subsection]
\newtheorem*{idea}{Idea}
\newtheorem{example}{Example}[subsection]

%Text math operators.
\DeclareMathOperator{\spn}{Span}
\DeclareMathOperator{\tr}{Tr}
\DeclareMathOperator{\SU}{SU}
%\DeclareMathOperator{\dim}{dim}
\DeclareMathOperator{\rank}{rank}
\DeclareMathOperator{\diag}{diag}

%Symbolic shortcuts 
\newcommand{\eps}{\epsilon}
\newcommand{\curl}{\bm{\nabla}\times}
\newcommand{\divergence}{\bm{\nabla}\cdot}
\newcommand{\grad}{\bm{\nabla}}
\newcommand{\del}{\partial}

%Text type shortcuts
\newcommand{\bam}[1]{\textbf{#1}}
\newcommand{\mf}[1]{\mathfrak{#1}}
\newcommand{\mbb}[1]{\mathbb{#1}}
\newcommand{\mc}[1]{\mathcal{#1}}

%Functions shortcuts.
\newcommand{\comm}[2][]{\left[ #1, #2 \right]} %Commutator .
\newcommand{\acomm}[2][]{\left\{ #1, #2 \right\}} %Anticommutator .
\newcommand{\pd}[2][]{\frac{\partial#1}{\partial#2}} %Partial derivative. 
\newcommand{\pds}[2][]{\frac{\partial^2 #1}{\partial #2^2}} %Second partial derivative.
\newcommand{\set}[1]{\lbrace #1 \rbrace} 


\newcommand{\be}{\begin{equation}} %Used like \[ \] but with tags. 
\newcommand{\ee}{\end{equation}}

%%%%%%%%%%%%%%%%%%%%%%%%%%%%%%%%%%%%%%%%%%%%%%%%%%%%%%%%
%Preamble

\title{Quantum Field Theory Revision Notes}
\author{Linden Disney-Hogg}
\date{January 2019}

%%%%%%%%%%%%%%%%%%%%%%%%%%%%%%%%%%%%%%%%%%%%%%%%%%%%%%%%
%%%%%%%%%%%%%%%%%%%%%%%%%%%%%%%%%%%%%%%%%%%%%%%%%%%%%%%%
\begin{document}

\maketitle
\tableofcontents

\section{Introduction}
A brief overview of some key ideas, concepts, and facts that I find useful in revising QFT. 

%%%%%%%%%%%%%%%%%%%%%%%%%%%%%%%%%%%%%%%%%%%%%%%%%%%%%%%%
%%%%%%%%%%%%%%%%%%%%%%%%%%%%%%%%%%%%%%%%%%%%%%%%%%%%%%%%
\section{Preliminaries}

%%%%%%%%%%%%%%%%%%%%%%%%%%%%%%%%%%%%%%%%%%%%%%%%%%%%%%%%
\subsection{Miscellaneous}

\begin{definition}
In these notes attention will only be paid to $3+1$ dimension spacetime. As such Greek indices (e.g $\mu$, $\nu$) will run from $0$ to $3$, while Latin indices (e.g. $i$, $j$) will run from $1$ to $3$. 
\end{definition}

\begin{definition}[Field]
In these notes a \bam{field} will be a function of spacetime $\phi=\phi(x)$ where $x=(t,\bm{x})$. 
\end{definition}

\begin{definition}[Minkowski Metric]
In these notes the mostly-minus metric convention will be adopted, so the \bam{Minkowski metric} is 
\[
\eta^{\mu\nu}=\diag(1, -1, -1, -1)
\]
\end{definition}

\begin{definition}[Natural Units]
In QFT, \bam{natural units} are used where
\[
c=1=\hbar
\]
$[c]=LT^{-1}$, $[\hbar]=L^2 M T^{-1}$, so in natural units
\[
L=T=M^{-1}
\]
Units are therefore given in \bam{mass dimension}, e.g. if $[f]=M^d$, write $[f]=d$.
\end{definition}

\begin{fact}
A table of common values and their mass dimension is given below. 
\begin{center}$
\begin{array}{ccc}
    \text{Quantity} & \text{Symbol} & \text{Mass Dimension} \\
    \hline
    \hline
    \text{Energy} & E & 1 \\
     
\end{array}
$\end{center}
\end{fact}

%%%%%%%%%%%%%%%%%%%%%%%%%%%%%%%%%%%%%%%%%%%%%%%%%%%%%%%%
\subsection{Lorentz Transformations}

\begin{definition}[Lorentz Group]
The \bam{Lorentz Group} is $O(1,3)$, the group of transformations that preserve the Minkowski metric, i.e.
\[
\forall \Lambda\in O(1,3) \quad \Lambda^\mu_\rho \Lambda^\nu_\sigma \eta^{\rho\sigma}=\eta^{\mu\nu}
\]
The \bam{proper orthochronous Lorentz group} is the connected component containing the identity $SO^+(1,3)$. These preserve orientation and the direction of time. 
\end{definition}

\begin{theorem}
Under a Lorentz transformation $\Lambda$, $x\to x^\prime=\Lambda x$, a scalar field $\phi=\phi(x)$ transforms as 
\[
\phi\to\phi^\prime, \quad \phi^\prime(x^\prime)=\phi(x)
\]
Vector fields $A^\mu(x)$ transform as 
\[
A\to A^\prime, \quad \left(A^\prime\right)^\mu(x^\prime)=\Lambda^\mu_\nu A^\nu (x)
\]
In general a field transforms as 
\[
\phi\to\phi^\prime, \quad (\phi^\prime)^\mu(x^\prime)=D(\Lambda)^\mu_\nu \phi^\nu(x)
\]
where $D$ is some representation of the Lorentz group. 
\end{theorem}

\begin{theorem}
Let $\Lambda^\mu_\nu$ be a Lorentz transform with infinitesimal expansion 
\[
\Lambda^\mu_\nu=\delta^\mu_\nu+\eps \omega^\mu_\nu +\mathcal{O}(\eps^2) 
\]
Then 
\[
\omega^{\mu\nu}+\omega^{\nu\mu}=0
\]
\end{theorem}

\begin{definition}
Define the matrix $M^{\rho\sigma}$ by 
\[
\left( M^{\rho\sigma} \right)^{\mu\nu} = \eta^{\rho\mu}\eta^{\sigma\nu}-\eta^{\sigma\mu}\eta^{\rho\nu}
\]
These matrices for generate all antisymmetric tensors by \[
\omega^{\mu\nu}=\frac{1}{2}\left(\Omega_{\rho\sigma}M^{\rho\sigma}\right)^{\mu\nu}
\]
summation convention used. Hence $\set{M^{\rho\sigma}}$ is a generating set for the Lorentz transforms. 
\end{definition}

\begin{definition}[Lorentz Algebra]
The matrices $m^{\rho\sigma}$ satisfy the \bam{Lorentz algebra}
\[
\comm[M^{\rho\sigma}]{M^{\mu\nu}}=\eta^{\sigma\mu}M^{\rho\nu}-\eta^{\rho\mu}M^{\sigma\nu}-\eta^{\sigma\nu}M^{\rho\mu}+\eta^{\rho\nu}M^{\sigma\mu}
\]
\end{definition}

%%%%%%%%%%%%%%%%%%%%%%%%%%%%%%%%%%%%%%%%%%%%%%%%%%%%%%%%
\subsection{Pauli Matrices}

\begin{definition}[Pauli Matrices]
The \bam{Pauli matrices} are

\begin{align*}
\sigma_1 &= \begin{pmatrix} 0 & 1 \\ 1 & 0\end{pmatrix}  \\
\sigma_2 &= \begin{pmatrix} 0 & -i \\ i & 0\end{pmatrix}  \\
\sigma_3 &= \begin{pmatrix} 1 & 0 \\ 0 & -1\end{pmatrix}  
\end{align*}

Note they are all Hermitian and traceless.
\end{definition}

\begin{fact}
\[
\sigma_i \sigma_j = \delta_{ij}I +i\epsilon_{ijk}\sigma_k
\]
Hence 
\begin{align*}
    \comm[\sigma_i]{\sigma_j} &= 2i\eps_{ijk}\sigma_k \\
    \acomm[\sigma_i]{\sigma_j} &= 2\delta_{ij}I
\end{align*}
\end{fact}
%%%%%%%%%%%%%%%%%%%%%%%%%%%%%%%%%%%%%%%%%%%%%%%%%%%%%%%%
%%%%%%%%%%%%%%%%%%%%%%%%%%%%%%%%%%%%%%%%%%%%%%%%%%%%%%%%
\section{Field Theories}

%%%%%%%%%%%%%%%%%%%%%%%%%%%%%%%%%%%%%%%%%%%%%%%%%%%%%%%%
\subsection{Basics}

\begin{definition}[Free Theory]
A \bam{free field theory} is one that contains no interaction terms, and so there are no terms containing multiple different fields. 
\end{definition}

\begin{definition}[Klein-Gordon Field]
The \bam{Klein-Gordon field} is the real field $\phi$ with Lagrangian density
\[
\mc{L}=\frac{1}{2}\eta^{\mu\nu} \del_\mu\phi \del_\nu\phi-\frac{1}{2}m^2\phi^2=0
\]
It is a free field theory, and will the prototypical field theory.
\end{definition}

\begin{definition}[Lorentz Invariance]
A field theory is \bam{Lorentz invariant} if the action is unchanged by a Lorentz transformation. 
\end{definition}

\begin{definition}[Euler-Lagrange Equations]
Given a Lagrangian density $\mathcal{L}=\mathcal{L}(\phi_a,\del_\mu \phi_a)$ the \bam{Euler-Lagrange equations} are 
\[
\del_\mu \pd[\mc{L}]{(\del_\mu \phi_a)}-\pd[\mc{L}]{\phi_a}=0
\]
\end{definition}

%%%%%%%%%%%%%%%%%%%%%%%%%%%%%%%%%%%%%%%%%%%%%%%%%%%%%%%%
\subsection{Conserved Quantities}

\begin{theorem}[Noether's Theorem]
Every continuous symmetry of the action of the form $\phi\to\phi+\alpha\Delta\phi$, that induces the transform on the Lagrangian density 
\[
\mc{L} \to \mc{L}+\alpha \del_\mu X^\mu
\]
gives rise to a conserved current $j^\mu$ s.t. $\del_\mu j^\mu = 0$, given by
\[
j^\mu = \pd[\mc{L}]{(\del_\mu \phi)}\Delta\phi-X^\mu
\]
and a corresponding conserved charge
\[
Q=\int_{\mbb{R}^3} j^0 \, d^3x
\]
\end{theorem}

\begin{definition}[Energy-Momentum Tensor]
For a given Lagrangian density the \bam{energy momentum tensor} is defined as  
\[
T^{\mu\nu}=\pd[\mc{L}]{(\del_\mu \phi)} \del^\nu \phi - \eta^{\mu\nu}\mc{L}
\]
\end{definition}

\begin{theorem}
If the Lagrangian density depends only on $x$ implicitly, then translation $x^\mu \to x^\mu-\alpha\eps^\mu$ is a continuous symmetry, and hence for each $\nu$, $T^{\mu\nu}$ is a conserved current. 
\end{theorem}

\begin{definition}[Energy and Momentum]
The energy $E$ is defined as 
\[
E=\int_{\mbb{R}^3} T^{00} \, d^3x
\]
and the momentum $\bm{P}$ is defined with components 
\[
P^i = \int_{\mbb{R}^3} T^{0i} \, d^3
\]


\end{definition}
%%%%%%%%%%%%%%%%%%%%%%%%%%%%%%%%%%%%%%%%%%%%%%%%%%%%%%%%
%%%%%%%%%%%%%%%%%%%%%%%%%%%%%%%%%%%%%%%%%%%%%%%%%%%%%%%%
\section{Quantisation}

\begin{definition}[Conjugate Momentum]
The \bam{conjugate momentum} to a field $\phi$ is defined as 
\[
\pi = \pd[\mc{L}]{(\del_0\phi)} = \pd[\mc{L}]{\dot{\phi}}
\]
\end{definition}

\begin{definition}[Hamiltonian Density]
Given a Lagrangian density $\mc{L}$, the corresponding \bam{Hamiltonian density} is defined as 
\[
\mc{H}=\pi \dot{\phi}-\mc{L}
\]
where $\mc{H}$ is a function of $\pi$ instead of $\dot{\phi}$
\end{definition}
%%%%%%%%%%%%%%%%%%%%%%%%%%%%%%%%%%%%%%%%%%%%%%%%%%%%%%%%
%%%%%%%%%%%%%%%%%%%%%%%%%%%%%%%%%%%%%%%%%%%%%%%%%%%%%%%%
\section{Perturbation Theory}


%%%%%%%%%%%%%%%%%%%%%%%%%%%%%%%%%%%%%%%%%%%%%%%%%%%%%%%%
%%%%%%%%%%%%%%%%%%%%%%%%%%%%%%%%%%%%%%%%%%%%%%%%%%%%%%%%
\section{Fermions}

%%%%%%%%%%%%%%%%%%%%%%%%%%%%%%%%%%%%%%%%%%%%%%%%%%%%%%%%
\subsection{Clifford Algebra}
The \bam{Clifford algebra} is $\set{ \gamma^\mu }$ with the anticommutation relation 
\[
\acomm[\gamma^\mu]{\gamma^\nu} = 2\eta^{\mu\nu}
\]

\begin{definition}[$\gamma^5$]
Define
\[
\gamma^5 = i \gamma^0 \gamma^1 \gamma^2 \gamma^3
\]
\end{definition}

\begin{theorem}
The following are properties of $\gamma^\mu$:
\begin{itemize}
    \item $\forall p\in\mbb{N}_0, \; \tr \left( \prod _ { i = 1 } ^ { 2 p + 1 } \gamma ^ { \mu _ { i } } \right)$
    \item $\comm[\gamma^\mu\gamma^\nu]{\gamma^\rho\gamma^\sigma}=2\eta^{\nu\rho}\gamma^\mu\gamma^\sigma-2\eta^{\mu\rho}\gamma^\nu\gamma^\sigma+2\eta^{\nu\sigma}\gamma^\rho\gamma^\mu-2\eta^{\mu\sigma}\gamma^\rho\gamma^\nu$
\end{itemize}
\end{theorem}




\begin{definition}[Chiral Representation]
The \bam{chiral} or \bam{Weyl representation} of the Clifford Algebra is the 4 dimensional representation
\begin{align*}
    \gamma^0 &= \begin{pmatrix} 0 & I_2 \\ I_2 & 0 \end{pmatrix} \\
    \gamma^i &= \begin{pmatrix} 0 & \sigma_i \\ -\sigma_i & 0 \end{pmatrix} \\ 
\end{align*}
In this representation 
\[
\gamma^5=\begin{pmatrix} -I_2 & 0 \\ 0 & I_2 \end{pmatrix}
\]
\end{definition}

%%%%%%%%%%%%%%%%%%%%%%%%%%%%%%%%%%%%%%%%%%%%%%%%%%%%%%%%
\subsection{Spinors}

\begin{definition}[$S^{\mu\nu}$]
Define
\[
S^{\mu\nu}=\frac{1}{4}\comm[\gamma^\mu]{\gamma^\nu}=\frac{1}{2}\left( \gamma^\mu \gamma^\nu -\eta^{\mu\nu} \right)
\]
\end{definition}

\begin{theorem}
$S^{\mu\nu}$ satisfies the commutation relation
\[
\comm[S^{\rho\sigma}]{S^{\mu\nu}}=\eta^{\sigma\mu}S^{\rho\nu}-\eta^{\rho\mu}S^{\sigma\nu}-\eta^{\sigma\nu}S^{\rho\mu}+\eta^{\rho\nu}S^{\sigma\mu}
\]
\end{theorem}

\begin{definition}[Spinor]
A \bam{Dirac spinor} is a 4-component vector $\psi_\alpha$ that transforms under a Lorentz transform 
\[
\Lambda=\exp\left(\frac{1}{2}\Omega_{\rho\sigma}M^{\rho\sigma}\right)
\]
as 
\[
\psi^{\alpha}(x)\to S[\Lambda]_{\beta}^{\alpha}\psi^{\beta}\left(\Lambda^{-1}x\right)
\]
where
\[
S[\Lambda]=\exp\left(\frac{1}{2}\Omega_{\rho\sigma}S^{\rho\sigma}\right)
\]
\end{definition}

\begin{example}
Calculating explicitly in the chiral representation yields 
\begin{align*}
    S^{ij} &= \frac{-i}{2}\eps_{ijk} \begin{pmatrix} \sigma_k & 0 \\ 0 & \sigma_k \end{pmatrix} \\
    S^{0i} &= \frac{1}{2} \begin{pmatrix} -\sigma_i & 0 \\ 0 & \sigma_i \end{pmatrix}
\end{align*}
Hence if $\Lambda$ is a pure rotation, i.e. the only non-zero $\Omega_{\rho\sigma}$ are $\Omega_{ij}=-\eps_{ijk}\phi^k$, then 
\[
S[\Lambda]=\begin{pmatrix} e^{\frac{i\bm{\phi}\cdot\bm{\sigma}}{2}} & 0 \\ 0 & e^{\frac{i\bm{\phi}\cdot\bm{\sigma}}{2}} \end{pmatrix}
\]
Alternatively, if $\Lambda$ is a pure boost, i.e the only non-zero $\Omega_{\rho\sigma}$ are $\Omega_{0i}=-\Omega_{i0}=\chi_i$, then
\[
S[\Lambda]=\begin{pmatrix} e^{-\frac{\bm{\chi}\cdot\bm{\sigma}}{2}} & 0 \\ 0 & e^\frac{\bm{\chi}\cdot\bm{\sigma}}{2} \end{pmatrix}
\]
\end{example}
%%%%%%%%%%%%%%%%%%%%%%%%%%%%%%%%%%%%%%%%%%%%%%%%%%%%%%%%
%%%%%%%%%%%%%%%%%%%%%%%%%%%%%%%%%%%%%%%%%%%%%%%%%%%%%%%%
\section{Quantum Electrodynamics (QED)}


\end{document}