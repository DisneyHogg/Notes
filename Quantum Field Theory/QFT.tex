\documentclass{article}

%%%%%%%%%%%%%%%%%%%%%%%%%%%%%%%%%%%%%%%%%%%%%%%%%%%%%%%%
%Packages

\usepackage[utf8]{inputenc}
\usepackage{amsmath}
\usepackage{amsthm}
\usepackage{amssymb}
\usepackage{hyperref} % Used for refernces and links. 
\usepackage{bm} % Used to bold vectors. 
\usepackage{faktor} % Used to write quotients nicely.

%%%%%%%%%%%%%%%%%%%%%%%%%%%%%%%%%%%%%%%%%%%%%%%%%%%%%%%%
%Definitions

\newtheorem{theorem}{Theorem}[subsection]
\newtheorem{corollary}{Corollary}[theorem]
\newtheorem{lemma}[theorem]{Lemma}
\newtheorem*{remark}{Remark}
\newtheorem{definition}{Definition}[subsection]
\newtheorem{fact}{Fact}[subsection]
\newtheorem*{idea}{Idea}

\DeclareMathOperator{\spn}{Span}
\DeclareMathOperator{\tr}{Tr}
\DeclareMathOperator{\SU}{SU}
%\DeclareMathOperator{\dim}{dim}
\DeclareMathOperator{\rank}{rank}
\DeclareMathOperator{\eps}{\epsilon}

\newcommand{\bam}[1]{\textbf{#1}}
\newcommand{\mf}[1]{\mathfrak{#1}}
\newcommand{\mbb}[1]{\mathbb{#1}}
\newcommand{\comm}[2][]{\left[ #1, #2 \right]}
\newcommand{\be}{\begin{equation}}
\newcommand{\ee}{\end{equation}}
\newcommand{\set}[1]{\lbrace #1 \rbrace}

%%%%%%%%%%%%%%%%%%%%%%%%%%%%%%%%%%%%%%%%%%%%%%%%%%%%%%%%
%Preamble

\title{Quantum Field Theory Revision Notes}
\author{Linden Disney-Hogg}
\date{January 2019}

%%%%%%%%%%%%%%%%%%%%%%%%%%%%%%%%%%%%%%%%%%%%%%%%%%%%%%%%
%%%%%%%%%%%%%%%%%%%%%%%%%%%%%%%%%%%%%%%%%%%%%%%%%%%%%%%%
\begin{document}

\maketitle
\tableofcontents

\section{Introduction}
A brief overview of some key ideas, concepts, and facts that I find useful in revising QFT. 

%%%%%%%%%%%%%%%%%%%%%%%%%%%%%%%%%%%%%%%%%%%%%%%%%%%%%%%%
%%%%%%%%%%%%%%%%%%%%%%%%%%%%%%%%%%%%%%%%%%%%%%%%%%%%%%%%
\section{Field Theories}
\begin{definition}[Free Theory]
A \bam{free field theory} is one that contains no interaction terms, and so there are no terms containing multiple different fields. 
\end{definition}

\begin{definition}[Natural Units]
In QFT, \bam{natural units} are used where
\[
c=1=\hbar
\]
$[c]=LT^{-1}$, $[\hbar]=L^2 M T^{-1}$, so in natural units
\[
L=T=M^{-1}
\]
Units are therefore given in \bam{mass dimension}, e.g. if $[x]=M^d$, write $[x]=d$.
\end{definition}

\begin{fact}
A table of common values and their mass dimension is given below. 
\begin{center}$
\begin{array}{ccc}
    \text{Quantity} & \text{Symbol} & \text{Mass Dimension} \\
    \hline
    \hline
    \text{Energy} & E & 1 \\
     
\end{array}
$\end{center}
\end{fact}

%%%%%%%%%%%%%%%%%%%%%%%%%%%%%%%%%%%%%%%%%%%%%%%%%%%%%%%%
%%%%%%%%%%%%%%%%%%%%%%%%%%%%%%%%%%%%%%%%%%%%%%%%%%%%%%%%
\section{Quantisation}


%%%%%%%%%%%%%%%%%%%%%%%%%%%%%%%%%%%%%%%%%%%%%%%%%%%%%%%%
%%%%%%%%%%%%%%%%%%%%%%%%%%%%%%%%%%%%%%%%%%%%%%%%%%%%%%%%
\section{Perturbation Theory}


%%%%%%%%%%%%%%%%%%%%%%%%%%%%%%%%%%%%%%%%%%%%%%%%%%%%%%%%
%%%%%%%%%%%%%%%%%%%%%%%%%%%%%%%%%%%%%%%%%%%%%%%%%%%%%%%%
\section{Fermions}

%%%%%%%%%%%%%%%%%%%%%%%%%%%%%%%%%%%%%%%%%%%%%%%%%%%%%%%%
\subsection{Clifford Algebra}

\begin{theorem}
Let $\Lambda^\mu_\nu$ be a Lorentz transform with infinitesimal expansion 
\[
\Lambda^\mu_\nu=\delta^\mu_\nu+\eps \omega^\mu_\nu +\mathcal{O}(\eps^2) 
\]
Then 
\[
\Lambda^\mu_\rho \Lambda^\nu_\sigma \eta^{\rho\sigma}=\eta^{\mu\nu} \Rightarrow \omega^{\mu\nu}+\omega^{\nu\mu}=0
\]
\end{theorem}

\begin{definition}
Define the matrix $M^{\rho\sigma}$ by 
\[
\left( M^{\rho\sigma} \right)^{\mu\nu} = \eta^{\rho\mu}\eta^{\sigma\nu}-\eta^{\sigma\mu}\eta^{\rho\nu}
\]
These matrices for generate all antisymmetric tensors by \[
\omega^{\mu\nu}=\frac{1}{2}\left(\Omega_{\rho\sigma}M^{\rho\sigma}\right)^{\mu\nu}
\]
summation convention used. Hence $\set{M^{\rho\sigms}}$ is a generating set for the Lorentz transforms. 
\end{definition}
%%%%%%%%%%%%%%%%%%%%%%%%%%%%%%%%%%%%%%%%%%%%%%%%%%%%%%%%
%%%%%%%%%%%%%%%%%%%%%%%%%%%%%%%%%%%%%%%%%%%%%%%%%%%%%%%%
\section{Quantum Electrodynamics (QED)}


\end{document}