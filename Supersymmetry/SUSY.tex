\documentclass{article}

\usepackage{header}
%%%%%%%%%%%%%%%%%%%%%%%%%%%%%%%%%%%%%%%%%%%%%%%%%%%%%%%%
%Preamble

\title{Supersymmetry Notes}
\author{Linden Disney-Hogg}
\date{January 2019}

%%%%%%%%%%%%%%%%%%%%%%%%%%%%%%%%%%%%%%%%%%%%%%%%%%%%%%%%
%%%%%%%%%%%%%%%%%%%%%%%%%%%%%%%%%%%%%%%%%%%%%%%%%%%%%%%%
\begin{document}

\maketitle
\tableofcontents

%%%%%%%%%%%%%%%%%%%%%%%%%%%%%%%%%%%%%%%%%%%%%%%%%%%%%%%%
%%%%%%%%%%%%%%%%%%%%%%%%%%%%%%%%%%%%%%%%%%%%%%%%%%%%%%%%
\section{Introduction - What is SUSY and Why do we care?}

%%%%%%%%%%%%%%%%%%%%%%%%%%%%%%%%%%%%%%%%%%%%%%%%%%%%%%%%
\subsection{What is SUSY}
In any quantum theory involving fermions, we can always split true Hilbert space into \emph{bosonic} and \emph{fermionic} parts:
\[
\mc{H} = \mc{H}_N \oplus \mc{H}_F
\]
Here, $\mc{H}_B$ contains an even number of fermionic excitations and $\mc{H}_F$ contains an odd number of fermion excitations, since a pair of fermions is a boson. \\
A theory is \bam{supersymmetric} if there is a fermionic operator $G"\mc{H}_B \to \mc{H}_F$, $G:\mc{H}_F \to \mc{H}_B$, such that $\acomm[Q]{Q^\dagger} = 2H$, where $H$ is the Hamiltonian, and $Q^\dagger$ is the adjoint of $Q$ with respect to the inner product on $\mc{H}$. We also require $Q^2 = 0 = (Q^\dagger)^2$. 

The consequences of this are 
\begin{itemize}
    \item \begin{align*}
    2\comm[H]{Q} &= \comm[ {\acomm[Q]{Q^\dagger}} ]{Q} \\ 
    &= \comm[QQ^\dagger + Q^\dagger Q]{Q} \\ 
    &= QQ^\dagger Q - QQ^\dagger Q \; \text{ since }Q^2=0 \\
    &= 0
\end{align*}
Hence $Q$ is \emph{conserved} and the transformations it generates will be \emph{symmetries}. These symmetries are parametrised by fermionic parameters (since $Q$ is a fermionic operator) which are called \bam{supersymmetries}. $Q$ is known as the \bam{supercharge}.
\item For any state $\ket{\psi}\in\mc{H}$ we have 
\begin{align*}
    2\braket{\psi|H|\psi} &= \braket{\psi| {\acomm[Q]{Q^\dagger}} | \psi} \\
    &= \braket{\psi | QQ^\dagger | \psi} + \braket{\psi | Q^\dagger Q | \psi} \\
    &= ||Q^\dagger \ket{\psi}||^2+||Q\ket{\psi}||^2 \\
    &\geq 0
\end{align*}
$\Rightarrow$ all states have non-negative energy. There is equality iff the ground state obeys $Q\ket{\psi}=0=Q^\dagger \ket{\psi}$. In otherwords, $\ket{0}$ has zero energy iff it is \emph{supersymmetric}.
\end{itemize}

%%%%%%%%%%%%%%%%%%%%%%%%%%%%%%%%%%%%%%%%%%%%%%%%%%%%%%%%
\subsection{Why do we care?}
\begin{itemize}
    \item \bam{Phenomenological}: In the Standard Model, the energy scale (i.e. the ultraviolet cutoff) versus the coupling constants looks like (...) so the forces don't meet in the plot. 
    Matter in the SM actually transforms in representation of $SO(10) \supset SU(5) \supset G_{SM}=SU(3) \times SU(2) \times U(1)$. Perhaps there was some unification, then the symmetries were broken? This does not happen in the Standard Model.
    If, however, we take a supersymmetric SM we get new particles $\Rightarrow$ changes scale dependence of coupling constant, allowing for the possibility of grand unification. 
    \item Another phenomenological reason is that it provides a \emph{dark matter} candidate. 
    \item (?) Perhaps related to the mass of the Higgs particle. The SM has a puzzle, why does the Higgs particle take the mass it does? We expect large quantum corrections to our predictions, but it turns out that we don't need them. What "protects" the mass of the Higgs? Could be SUSY... \emph{but} ruled out by CERN. 
    \item \bam{Theoretical Motivations}: SUSY helps understand QFT. Usually in QM, we've started from idealised systems with lots of symmetry, \emph{then} we perturb to more realistic cases. In QFT, the \emph{only} idealised case met so far is free theory. 
    SUSY allows us to do better. We can compute some quantities \emph{exactly}. Better yet, these often reveal deep connections between QFT and geometry/topology. 
\end{itemize}

%%%%%%%%%%%%%%%%%%%%%%%%%%%%%%%%%%%%%%%%%%%%%%%%%%%%%%%%
%%%%%%%%%%%%%%%%%%%%%%%%%%%%%%%%%%%%%%%%%%%%%%%%%%%%%%%%
\section{Path Integrals in QFT}
In QFT, we are interested in computing 
\[
z = \int_\mc{C} e^{-\frac{S[X]}{\hbar}} \mc{D}X \; \text{(Euclidean signature)}
\]
where $X$ is some field and $\mc{C}$ is the space of all field configurations. This integral is \emph{not} well defined, and is formidably hard to compute. 
As a toy, suppose the whole universe is just a single point: $\mc{M} = \set{pt}$. Then a field is a map $X:\set{pt} \to \mbb{R}$, and the integral is now just 
\[
z = \int_\mbb{R} e^{-\frac{S(X)}{\hbar}} dX
\]
Suppose $S(X) = \frac{1}{2}mX^2 + \frac{1}{6}\lambda X^4 +\frac{1}{6!} gX^6$, say. Even with this simple form of the integral, the integral is hard for the choice of $S$. \\
As $\hbar \to 0$, taking the semi classical limit, we can obtain an asymptotic series as $\hbar \to 0^+$: 
\[
z \sim \frac{e^{-\frac{S[X_0]}{\hbar}}}{\sqrt{\pds[S]{X}(X_0)}}(1+A\hbar + B\hbar^2+\dots)
\]
where $S$ has an isolated minimum at $X_0\in\mbb{R}$, where isolated $\Rightarrow \pds[S]{X}(X_0) \neq0$. We find (in AQFT) that the term \[
\frac{e^{-\frac{S[X_0]}{\hbar}}}{\sqrt{\pds[S]{X}(X_0)}}
\]
corresponds to \emph{tree level diagrams}, the $\mc{O}(\hbar)$ corresponds to one-loop corrections, etc. \\
This is at \emph{most} an asymptotic series. If it were to converge it would have to converge in a disk, but if $\hbar < 0$ the integral obviously diverges. 

%%%%%%%%%%%%%%%%%%%%%%%%%%%%%%%%%%%%%%%%%%%%%%%%%%%%%%%%
%%%%%%%%%%%%%%%%%%%%%%%%%%%%%%%%%%%%%%%%%%%%%%%%%%%%%%%%
\section{SUSY in d=0 dimensions}
%%%%%%%%%%%%%%%%%%%%%%%%%%%%%%%%%%%%%%%%%%%%%%%%%%%%%%%%
\subsection{Grassman variables}
These are a set of $n$ elements $\psi^a$ obeying the algebra 
\[
\acomm[\psi^a]{\psi^b} = 0
\]
In particular, $(\psi^a)^2=0$. We call the elements \bam{Grassman variables} or \bam{fermions}. If $x^b$ is a bosonic variable, $\comm[\psi^a]{x^b}=0$.\footnote{Note $\acomm[\psi^\alpha(\bm{x})]{\psi^\beta(\bm{y})} = 0$ for the Dirac field in QFT. Further, the exterior product on differential forms is antisymmetric: $dx^a \wedge dx^b = -dx^b \wedge dx^a$} \\
If we take a function of the Grassman variables, it has an expansion that must eventually terminate: 
\[
F(\psi) = f + e_a \psi^a + \phi_{ab} \psi^a \psi^b + \dots +g \psi^1\psi^2\dots\psi^n
\]
Once we have all $n$ fermions, any other term will give zero contribution. (Note $\phi_{ab} = -\phi_{ba}$). \\
If $F(\psi)$ is \emph{bosonic} (i.e. commuting) then $f$, $\phi_{ab},\dots$ (coefficients of even powers) must also be bosonic, whereas $e_a,\dots$ (coefficients of odd powers) must be fermionic. \\
We define derivatives by:
\[
\pd{\psi^a}\left(\psi^b \dots \right) = \delta^b_a ( \dots) - \psi^b \pd{\psi^a} (\dots) \; \text{(Negative Leibniz rule)}
\]
We can also define integration simply by defining $\int 1 d\psi$ and $\int \psi d\psi$. We want our integral to be transaltion invariant, i.e 
\[
\int (\psi + \eta) d\psi = \int d\psi
\]
\[
\Rightarrow \int 1 d\psi = 0 
\]
We then want to normalise by choosing 
\[
\int \psi d\psi = 1 \quad \text{(Berezin integration)}
\]
Note, suppose we have $n$ fermions $\psi^1, \dots, \psi^n$. with 
\[
\int \psi^1 \dots \underbrace{\psi^n d\psi^1 \dots d\psi^n}_{d^n \psi} = 1 
\]
and 
\[
\int \psi^{a_1} \dots \psi^{a_n} d^n \psi = \eps^{a_1 \dots a_n}
\]
Now let ${\psi^\prime}^a=N^a_b \psi^b$ for $N\in GL(n)$. We have 
\begin{align*}
    \int {\psi^\prime}^a {\psi^\prime}^b \dots {\psi^\prime}^d d^n \psi &= N^a_e N^b_f \dots N^d_g \int \psi^e \psi^f \dots \psi^g d^n \psi \\
    &= N^a_e N^b_f \dots N^d_g \eps^{ef\dots g} = \det N \eps^{ab\dots d} \\
    &= \det N \int {\psi^\prime}^a {\psi^\prime}^b \dots {\psi^\prime}^d d^n \psi^\prime 
\end{align*}
Hence we see $d^n \psi^\prime = \frac{1}{\det N} d^n \psi$ (opplsite way round to ususal).

In QFT, we will often need Gaussian integral. Suppose $\psi^1, \psi^2$ are fermions and let $S(\psi^a) = \frac{1}{2}\psi^1 M \psi^2$. Then 
\[
\int e^{-iS(\psi^a)} d\psi^1 d\psi^2 = \int \left( 1- \frac{1}{2} \psi^1 M \psi^2 \right) d\psi^1 d\psi^2 = \frac{1}{2}M
\]
More generally for $2m$ fermions we action $S(\psi^a) = \frac{1}{2} \psi^a M_{ab} \psi^b$ ($M_{ab}=-M_{ba})$. 
\begin{align*}
    \int e^{-S(\psi)} d^{2m}\psi &= \int \sum_{k=0}^\infty \frac{(-1)^k}{k!} \frac{1}{2^k} \left( \psi^a M_{ab} \psi^b \right)^k d^{2m}k \\
    &=\frac{(-1)^m}{2^m m!} \int \left( \psi^a M_{ab} \psi^b \right)^{2m} \\
    &= \frac{(-1)^m}{2^m m!} \eps^{a_1 b_1 a_2 b_2 \dots a_m b_m} M_{a_1 b_1} M_{a_2 b_2} \dots M_{a_n b_n} \\
    &= \sqrt{\det M} \equiv \Pfaff(M)
\end{align*}
(c.f. for bosons we have $\int e^{-\frac{1}{2} x^a M_{ab} x^b} d^{2m} x = \frac{(2\pi)^m}{\sqrt{\det M}} $)

\subsection{Supersymmetric Integrals and Localisation}
Consider a $d=0$ theory of one bosonic variable, and 2 fermions  $\psi^1,\psi^2$. Take $S(x,\psi) = V(x)-\psi^1 \psi^2 U(x)$ as our action. Even in $d=0$, for generic $V,U$ the integral $\int e^{-S(x,\psi^i)}dxd\psi^2 d\psi^2$ is difficult. \\
However, suppose we choose a polynomial $W(x)$ and take $S(x,\psi) = \frac{1}{2} (\del W)^2 - \bar{\psi} \psi \del^2 W$ (taking $\psi = \psi^1 + i \psi^2, \bar{\psi} = \psi^1 -i \psi^2$. This $S(x,\psi \bar{\psi})$ is invariant under 
\begin{align*}
\delta x &= \eps \psi - \bar{\eps} \bar{\psi} \\
\delta \psi &= \bar{\eps} \del W \\
\delta \bar{\psi} &= \eps \del W 
\end{align*}
where $\eps,\bar{\eps}$ are fermionic parameters. Check that 
\[
\delta_\eps S = \del W \del^2 W \eps \psi-\eps\del W \psi \del^2 W - \bar{\psi}\psi(\eps \psi \del^3 W) = 0 
\]
We write $\delta = \eps Q + \bar{\eps} \bar{Q}$ where $Q,\bar{Q}$ are called \emph{supercharges}
\begin{align*}
    Qx = \psi &\qquad \bar{Q}x = -\bar{\psi} \\
    Q \psi = 0 &\qquad \bar{Q} \psi = \del W \\
    Q \bar{\psi} = \del W &\qquad \bar{Q} \bar{\psi} = 0
\end{align*}
Or 
\begin{align*}
    Q &= \psi \pd{x} + \del W \pd{\bar{\psi}} \\
    \bar{Q} &= -\bar{\psi} \pd{x} + \del W \pd{\psi}
\end{align*}
These Generators obey $\acomm[Q]{\bar{Q}}=0$. Note $H=0$ in $d=0$ so this is a supersymmetry algebra.

The supersymmetric 'path' integral $\int e^{-S(x,\psi,\bar{\psi}} dx d\psi d\bar{\psi}$ in fact \bam{localises}. Suppose we rescale $W \to \lambda W $ for $\lambda \in \mbb{R}_{\geq 0}$ both $S \to S_\lambda$ and in the susy transform $Q\to Q_\lambda, \bar{Q} \to \bar{Q}_\lambda $. Then 
\[
I(\lambda) = \int e^{-S_\lambda(x,\psi,\bar{\psi}} dx d^2 \psi
\]
actually obeys $\frac{dI}{d\lambda}=0$
\begin{proof}
\begin{align*}
    \frac{dI}{d\lambda} &= \int \pd{\lambda} e^{-S_\lambda} dx d^2\psi \\
    &= -\int \left[ \lambda (\del W)^2 - \bar{\psi}\psi \del^2 W \right] e^{-S_\lambda} dx d^2\psi \\
    &= -\int \bar{Q}_\lambda (\del W \psi) e^{-S_\lambda} dx d^2 \psi \\
    &= -\int \bar{Q}_\lambda (\del W \psi e^{-S_\lambda}) dx d^2 \psi \\
\end{align*}
Since $\bar{Q}_\lambda = \bar{\psi} \pd{x} + (\lambda \del W) \pd{\psi}$, this vanishes. 
\end{proof}
Thus $I(1) = \lim_{\lambda \to \infty} I(\lambda)$. As $\lambda \to \infty$, $e^{-\frac{\lambda^2}{2}(\del W)^2}$ suppreses the integral except near where $\del W = 0$. The integral localises to critical points of $W(x)$. 

Suppose we have a group $G$ acting freely\footnote{This means something, and isn't just a turn of phrase} on our space of fields, and suppose the action and integration measure are $G$-invariant. 

\begin{example}
\eq{
\int_{\mbb{R}^2 \setminus \set{0} } e^{-S(x,y)} \, dxdy 
}
with $G=SO(2)$ and $S$ just a function of $r=\sqrt{x^2+y^2}$. 
\end{example}
In this case decompose $\mc{C}$ as $G \times \faktor{\mc{C}}{G}$ and integrate over $G$ to obtain $\vol G$. \\
If $G$ is a fermionic group, then $\vol G = 0$ since $0 = \int_G 1 d^{\dim G} \theta$.\\
More generally, if $G \circlearrowleft \mc{C}$ has some fixed points, we can only get contributions form a nbhd of these fixed points. In our case $\delta \psi = \bar{\eps} \del W, \delta \bar{\psi} = -\eps \del W$ so ficed points of our susy are critical points of $W(x)$. Away from such critical points define 
\eq{
y = x - \frac{\bar{\psi}\psi}{\del W} \\
\chi = \psi \sqrt{\del W} \\
 \bar{\chi} = \bar{\psi}
}
\begin{ex}
Show $dxd^2 \psi = \sqrt{\del W(y)} dy d^2\chi$
\end{ex}
The point is
\begin{itemize}
\item \eq{
\delta y = \eps \psi - \bar{\eps} \bar{\psi} - \frac{\eps \del W \psi}{\del W} - \frac{\bar{\psi} \bar{\eps} \del W}{\del W} = 0
} 
\item \eq{
S(y,0,0) &= \frac{1}{2} ( \del W(y) )^2 \\
&= \frac{1}{2} ( \del W(x) )^2 - \del W \del^2 W \frac{\bar{\psi}{\psi}}{\del W} \\
&= S(x,\psi,\bar{\psi})
}
\end{itemize}
Hence 
\eq{
\int_{U^c} e^{-S(x,\psi,\bar{\psi})} dx d^2\psi = \int e^{-S(y,0,0)} \sqrt{\del W(y)} dy d^2\chi = 0
}
where $U$ is an open neighborhood of $\set{ \del W = 0}$ and $U^c = \mc{C}\setminus U$. Near any isolated critical point $x_\ast$ such that 
\eq{
W(x) = W(x_\ast) + \frac{c_\ast}{2} (x-x_\ast)^2 +\dots
}
the higher order terms will be irrelevant, so 
\eq{
S^{(2)}(x,\psi,\bar{\psi}) = \frac{c_\ast^2}{2}(x-x_\ast)^2 - \bar{\psi}\psi c_\ast
}
Hence 
\eq{
I &= \int e^{-S(x,\psi,\bar{\psi})} \frac{dx}{\sqrt{2\pi}} d\psi \\
&= \int e^{-\frac{c_\ast^2}{2}(x-x_\ast)^2} (-1 + \bar{\psi}\psi c_\ast ) dx d^2\psi \\
&= \frac{c_\ast}{\sqrt{2\pi}} \int_\mbb{R} e^{-\frac{c_\ast^2}{2}(x-x_\ast)^2} dx = \frac{c_\ast}{|c_\ast|} = \pm 1
}
If $W$ has several critical points the 
\eq{
I = \sum_{x_\ast : \del W = 0} \frac{c_\ast}{|c_\ast|}
}
Hence 
\begin{itemize}
    \item $I=0$ if $W$ is an odd degree polynomial
    \item $I=-1$ if $W$ has even degree and $\lim_{|x|\to\infty} W(x) = -\infty$
    \item $I=1$ if $W$ has even degree and $\lim_{|x|\to\infty} W(x) = \infty$
\end{itemize}

%%%%%%%%%%%%%%%%%%%%%%%%%%%%%%%%%%%%%%%%%%%%%%%%%%
%%%%%%%%%%%%%%%%%%%%%%%%%%%%%%%%%%%%%%%%%%%%%%%%%%
\section{d=0 Landua Ginzburg Theory}
Now take $z\in\mbb{C}$ and two fermions $\psi_1,\psi_2$. Choose a holomorphic $W(z)$
\eq{
S(z,\psi) = |\del W|^2 + \del^2 W \psi_2 \psi_2 - \bar{\del^2 W} \bar{\psi_1} \bar{\psi_2}
}
This is invariant under 
\eq{
\delta z &= \eps_1 \psi_1 + \eps_2 psi_ 2 \\
\delta \psi_1 &= \eps_2 \bar{\del W } \\
\delta \psi_2 &= -\eps_1 \bar{\del W} \\
\bar{\delta} \bar{z} &= \bar{\eps}_1 \bar{\psi}_1 + \bar{\eps}_2 \bar{psi}_ 2 \\
\bar{delta} \bar{\psi}_1 &= \bar{\eps}_2 \del W \\
\bar{\delta} \bar{\psi}_2 &= -\bar{\eps}_1 \del W
}
and 
\eq{
\bar{\delta} z &= 0 = \bar{\delta} \psi_i \\
\delta \bar{z} &= 0 =\delta \bar{\psi}_i
}
You may check $\acomm[Q_i]{\bar{Q}_j}=0$ and $\acomm[Q_i]{Q_j}=0=\acomm[\bar{Q}_i]{\bar{Q}_j}$ %only holds "on shell", i.e. $\del^2 W = 0 = \bar{\del^2 W}$.\\
Again by rescaling $W \to \lambda W$ for $\lambda \in \mbb{R}_{\geq 0}$, localise $I = \int e^{-S} d^2z d^4\psi$ to its critical points of $W$ where 
\eq{
W(z) = W(z_\ast) + \frac{\alpha_\ast}{2} (z-z_\ast)^2 + \dots
}
\eq{
S^{(2)}(z,\psi_i) = |\alpha_\ast|^2 |z-z_\ast|^2 + \alpha_\ast \psi_1 \psi_2 - \bar{\alpha}_\ast \bar{\psi}_1 \bar{\psi}_2 + \dots
}
so 
\eq{
I &= \frac{1}{2\pi} \int e^{-S} d^2z d^4\psi \\
&= \sum_{z_\ast} \frac{1}{2\pi} \int e^{-|\alpha_\ast (z-z_ast)|^2} |\alpha_\ast|^2 \psi_1 \psi_2 \bar{\psi}_1 \bar{\psi}_2 d^2z d^4\psi \\
&= \sum_{z_\ast} \frac{|\alpha_\ast|^2}{|\alpha_\ast|^2} = \sum_{z_\ast} 1
}
counting the critical points. More generally let $f(z)$ be any holomorphic functions, then 
\eq{
\braket{f(z)} = \int e^{-S} f(z) d^2z d^4\psi
}
is still invariant under $\bar{\delta}$ transform, to again localises to critical points of $\bar{W}$. Then 
\eq{
\braket{f(z)} &= \sum_{z_\ast} f(z_\ast) \frac{1}{2\pi} \int e^{-S^{(2)}} d^2z d^4\psi \\
&= \sum_{z_\ast} f(z_\ast) 
}
the sum of the values at critical points. The key fact was $\bar{\delta}f=0$. Since $\bar{Q}_i^2=0$, one way to construct a $\bar{Q}_i$-invariant function is to take $\bar{Q}_i \Lambda(z,\bar{z},\psi_i,\bar{\psi}_j)$ for some general $\Lambda$ . However, if $F=\bar{Q}\Lambda$, 
\eq{
\braket{F} &= \braket{ \bar{Q}\Lambda } \\
&= \int (\bar{Q}\Lambda) e^{-S} \frac{d^2zd^4\psi}{2\pi} \\
&= \int \bar{Q} (\Lambda e^{-S} ) \frac{d^2zd^4\psi}{2\pi} = 0
}
So interesting functions are in 
\eq{
H_\bar{Q} = \faktor{\ker \bar{Q}}{\image \bar{Q}}
}
Now 
\eq{
\braket{F + \bar{Q} \Lambda} = \braket{F} + \braket {\bar{Q} \Lambda} = \braket{F} 
}
and supposing $F_i = \bar{Q} \Lambda$

\eq{
\braket{\prod_{i=1}^n F_i } &= \braket{\bar{Q} \Lambda \prod_{i=2}^n F_i } \\
&= \braket{\bar{Q} (\Lambda \prod_{i=2}^n F_i )}=0
}

Non trivial corelators from $O \in H_{\bar{Q}}$
\eq{
\braket{(\bar{Q}\Lambda) \prod_i O_i } = \braket{ \bar{Q} (\Lambda \prod_i O_i)}
}

\begin{example}

 the transform $\bar{\delta}\bar{\psi}_i = \bar{\eps}_i \del W$ shows that $\delta W$ is itself $\bar{Q}(\dots)$. Hence if our operator contain $\delta W$ as a factor, their correlators vanish 

For example $W(z) = \frac{1}{n} z^{n-1}-az$, $\del W = z^n - a$

Then we have non trivial $\bar{Q}$ invariant operators that are polynomials subject to $z^n = a$. Hence the operators form a \bam{ring} generated by $\set{1,z,\dots,z^n}$. 
\end{example}

The ring of non trivial supersymmetric operators is often called the \bam{chiral ring}. 

%%%%%%%%%%%%%%%%%%%%%%%%%%%%%%%%%%%%%%%%%%%%%%%%%%
%%%%%%%%%%%%%%%%%%%%%%%%%%%%%%%%%%%%%%%%%%%%%%%%%%
\section{Supersymmetric Quantum Mechanics}
There are 2 perspectives: canonical frameworks and tpath integral framework. \\
Take a worldline theory of a single bosonic field $x(t)$ and a single $\mbb{C}$- fermion $\psi(t)$
, and choose the action 
\eq{
S[x,\psi,\bar{\psi}] = \int \frac{1}{2}\dot{x}^2 + \frac{1}{2}(\bar{\psi} \dot{\psi} - \dot{\bar{\psi}} \psi) - \frac{1}{2}(\del h)^2 - \bar{\psi}\psi \del^2 h 
}
where $h=h(x(t))$ plays the role of a potential. $S$ is invariant under 
\eq{
\delta x &= \eps \bar{\psi} - \bar{\eps} \psi \\
\delta \psi &= \eps (i\dot{x} + \delta h) \\
\delta \bar{\psi} = \bar{\eps} ( -i\dot{x} + \del h)
}
By the Noether procedure, promoting $\eps \to \eps(t)$. find that 
\eq{
\delta S = -i \int (\dot{\eps} Q + \dot{\bar{\eps}} \bar{Q})
}
where the charges are 
\eq{
Q &= \bar{\psi} (i\dot{x} + \del h ) \\
\bar{Q} &= \psi ( -i \dot{x} + \del h ) 
}
and they obey
\eq{
\acomm[Q]{\bar{Q}} x &= (Q\bar{Q} + \bar{Q} Q )x \\
&= -Q \psi + \bar{Q} \bar{\psi} \\
&= - (i\dor{x} + \del h ) + (-i \dot{x} + \del h ) \\
&= -2i\dot{x} \\
\acomm[Q]{\bar{Q}} \psi = \bar{Q}(i\dot{x} + \del h) \\
&= -i\dot{\psi} - \psi \del^2 h \\
&= -2i\dot{\psi} \quad \text{on e.o.m} \dot{\psi} = - i \psi \del^2 h \\
\acomm[Q]{\bar{Q}} \bar{\psi} \approx -2i\dot{\bar{\psi}}
}
Then, up to fermionic e.o.m, the anticommutator of the supercharges generates time translation, so must be $\propto H$. \\
To canonically quantise, have 
\eq{
p = \frac{\delta L}{\delta \dot{x}} = \dot{x} \\
\pi = \frac{\delta L }{\delta \dot{\psi}} = i\bar{\psi}
}
so 
\eq{
H = p\dot{x} + \pi \dot{\psi} - L = \frac{1}{2}p^2 + (\del h )^2 + \frac{1}{2}\del^2 h (\bar{\psi}\psi - \psi \bar{\psi})
}
From quantisation, we have 
\eq{
\comm[\hat{x}]{\hat{p}} &= i \\
\acomm[\hat{\psi}]{\hat{\bar{\psi}}} = 1
}
For $x$ as usual take the Hilbert space $\mc{H} = L^2 ( \mbb{R}, dx)$ in which case $\hat{x} \Psi(x) = x\Psi(x)$ and $\hat{p} \Psi(x) = -i \pd[\Psi]{x}$. The relations $\acomm[\hat{\psi}]{\hat{\bar{\psi}}}$ are reminiscient of $\comm[a]{a^\dagger}=1$ in a SHO. Lets define a fermionic number operator $\hat{F} = \hat{\bar{\psi}}\hat{\psi}$. Then 
\eq{
\comm[\hat{F}]{\hat{\psi}} &= -\hat{\psi} \\
\comm[\hat{F}]{\hat{\bar{\psi}}} = \hat{\bar{\psi}}
}
We also let the vacuum of the fermionic system be $\ket{0}$ defined by $\hat{\psi}\ket{0}=0$. The 1st excited state is $\hat{\bar{\psi}}\ket{0} = \ket{1}$, but since $\acomm[\hat{\bar{\psi}}{\hat{\bar{\psi}}}=0$, there are no other excited states. Hence 
\eq{
\mc{H} &= L^2(\mbb{R},dx) \ket{0} \oplus L^2(\mbb{R},dx)\ket{1} \\
&= \mc{H}_B \oplus \mc{H}_F
}
In the qauntume theorey, we have 
\eq{
\hat{Q} &= \hat{\bar{\psi}}(i\hat{p} +\del h ) \\
\hat{\bar{Q}} &= \hat{\psi} ( -i\hat{p} + \del h ) \\
\hat{H} = \frac{1}{2}\hat{p}^2 + (\del h )^2 + \frac{1}{2} \del^2 h (\hat{\bar{\psi}} \hat{\psi} - \hat{\psi} \hat{\bar{\psi}} )
}
We have immediately, (dropping hats from now on ) $\acomm[Q]{Q} = 0 = \acomm[\bar{Q}]{\bar{Q}}$, but $\acomm[Q]{\bar{Q}}=2H$
\begin{ex}
Provie this result
\end{ex}
This is why we chose the particular ordering in $H$. 

%%%%%%%%%%%%%%%%%%%%%%%%%%%%%%%%%%%%%%%%%%%%%%%%%%
\subsection{Supersymmetric ground state}
As before $\braket{\Psi|H|\Psi}\geq = 0$ with equality iff $Q\ket{\Psi}=0=\bar{Q}\ket{\Psi}$, so a ground state of zero energy in SQM must be supersymmetrically invariant and will then be a ground state. If we represent 
\[
\ket{0} = \begin{pmatrix} 1 \\ 0 \end{pmatrix} \quad \ket{1} = \begin{pmatrix} 0 \\ 1 \end{pmatrix}
\]
Then 
\eq{
Q\ket{\Psi} &= \begin{pmatrix} 0 & 0 \\ \frac{d}{dx} + \del h & 0 \end{pmatrix}\begin{pmatrix} f(x) \\ g(x) \end{pmatrix} = 0\\
\bar{Q}\ket{\Psi} &= \begin{pmatrix} 0 & -\frac{d}{dx} + \del h  \\ 0 & 0 \end{pmatrix}\begin{pmatrix} f(x) \\ g(x) \end{pmatrix} = 0
}
which gives, for our ground state $\ket{\Psi}$
\eq{
\ket{\Psi} = \begin{pmatrix} A e^{-h(x)} \\ Be^{h(x)} \end{pmatrix}
}
We want a normalisable state, so 
\begin{itemize}
    \item $\lim_{|x|\to\infty} h(x) = \infty \Rightarrow B=0 \Rightarrow \ket{\Psi} = \begin{pmatrix} A e^{-h(x)} \\ 0 \end{pmatrix}$ 
    \item $\lim_{|x|\to\infty} h(x) = -\infty \Rightarrow A=0 \Rightarrow \ket{\Psi} = \begin{pmatrix} 0 \\ B e^{h(x)} \end{pmatrix}$ 
    \item $\lim_{x\to\infty} h(x) = \pm\infty \; \lim_{x\to -\infty} h(x) = \mp\infty\Rightarrow A=0=B \Rightarrow \ket{\Psi} = \begin{pmatrix} 0 \\ 0 \end{pmatrix}$ 
\end{itemize}
In the third case the is no zero energy state, so the ground state will have higher energy and SUSY is \bam{spontaneously broken}. 

\end{document}