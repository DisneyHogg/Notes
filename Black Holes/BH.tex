\documentclass{article}

\usepackage{header}
%%%%%%%%%%%%%%%%%%%%%%%%%%%%%%%%%%%%%%%%%%%%%%%%%%%%%%%%
%Preamble

\title{Black Holes Notes}
\author{Linden Disney-Hogg}
\date{January 2019}

%%%%%%%%%%%%%%%%%%%%%%%%%%%%%%%%%%%%%%%%%%%%%%%%%%%%%%%%
%%%%%%%%%%%%%%%%%%%%%%%%%%%%%%%%%%%%%%%%%%%%%%%%%%%%%%%%
\begin{document}

\maketitle
\tableofcontents

\section{Introduction}
\subsection{Conventions}
\[
c = G = 1
\]
The signature used it $(-,+,+,+)$. \\
Greek indices will be used for coordinate dependent expressions \\
Latin indices will be used for coordinate invariant expressions (e.g $R=g^{ab}R_{ab}$.\\ 
The Riemann tensor is fixed by 
\[
R(X,Y) = \grad_X \grad_Y Z - \grad_Y \grad_X Z - \grad_{\comm[X]{Y}} Z
\]
\subsection{Notes}
\begin{itemize}
    \item Harvey Reall's notes on BH and GR 
    \item Wald's "General Relativity"
    \item Witten's review "Light rays, singularities, and all that"
\end{itemize}

\section{Spherical Stars}
The \bam{Pauli exclusion principle} gives the \bam{degeneracy pressure}. If the degeneracy pressure of electrons is enough to balance the gravitational collapse (possible for mass  $<1.4 \msolar$ then a \bam{white dwarf} forms.\\
If $1.4 \msolar < M < 3 \msolar$ then neutron degeneracy pressure is required and a \bam{neutron star} forms. Above this mass a \bam{black hole} forms. 

\subsection{Spherical Symmetry}
A round sphere is invariant under rotations in $SO(3)$. 

\begin{definition}[Line element of sphere]
the \bam{line element} on the sphere is 
\[
d\Omega_2^2 = d\theta^2 + \sin^2\theta d\phi^2
\]
Also invariant under $\theta \to \pi-\theta$, so the sphere has $O(3)$ symmetry. 
\end{definition}

\begin{definition}[Spherically Symmetric Spacetime]
A spacetime manifold $(\mc{M},g)$ is \bam{spherically symmetric} is it posses the same symmetries as a round 2-sphere. 
\end{definition}

\begin{idea}
Note that the symmetry group of the sphere is bigger than $O(3)$, so just possessing an $O(3)$ symmetry group is not sufficient. In practice if a factor of $d\Omega^2$ appears in the line element then it will be sphericall symmetric (under some conditions.)
\end{idea}

\begin{definition}[Radius]
In a spherically symmetric spacetime
\[
r : \mc{M} \to \mbb{R}^+
\]
is defined by 
\[
r(p) = \sqrt{\frac{A(p)}{4\pi}}
\]
$r$ is the \bam{radius}. 
\end{definition}

\subsection{Time Independence}

\begin{definition}[Stationary]
 A spacetime $(\mc{M},g)$ is \bam{stationary} if it admits a Killing vector field $K^a$ that is everywhere timelike. i.e 
 \[
  K^a K^b g_{ab} < 0 
 \]
\end{definition}
Pick a hypersurface $\Sigma$ nowhere tangent to $K$. Pick coordinate $x^i$ on $\Sigma$, and assign coordinates $(t,x^i)$ on $\mc{M}$ to the point a parameter distance $t$ along the integral curve of $K$ through the point $x^i$ on $\Sigma$. Then in these coordinate 
\[
K^a = \left( \pd{t} \right)^a
\]
Recall the Killing equation $\mc{L}_K g = 0$ This implication of this is then that $g$ is time independent. Hence
\[
ds^2 = g_{tt}(x^k) dt^2 + 2g_{ti}(x^k) dt dx^i + g_{ij}(x^k) dx^i dx^j 
\]
The assumption of stationarity gives 
\[
g_{tt}(x^k) < 0
\]

\begin{lemma}
Take $\Sigma=\set{x: f(x)=0}$ for some $f:\mc{M} \to \mbb{R}$ with $df \neq 0$. Then $df$ is orthogonal to $\Sigma$.  
\end{lemma}
\begin{proof}
Take $Z^a$ to be tangent to be $\Sigma$. Then 
\[
df(Z) = Z(f) = Z^\mu \del_\mu f = 0
\]
on $\Sigma$.
\end{proof}

Now take a general 1-form normal to $\Sigma$. 
\[
m = g df + f m^\prime
\]
$m^\prime$ an arbitrary smooth vector, $g$ an arbitrary smooth function. Then 
\[
dm = dg \wedge df + df \wedge m^\prime + f \wedge dm^\prime \\
\Rightarrow dm \rvert_\Sigma = (dg-m^\prime) \wedge df 
\]
Hence
\[
m\rvert_\Sigma = g df \Rightarrow (m \wedge dm) \rvert_\Sigma =0
\]
So, if $\Sigma$ is hypersurface-orthogonal, $(m\wedge dm)_\Sigma = 0$

\begin{theorem}[Frobenius]
If $m$ is a non zero form such that $(m\wedge dm) = 0$ everywhere then $\exists g$ such that $m = g df$
\end{theorem}

\begin{definition}[Static]
A spacetime is \bam{static} if it admits a hypersurface- orthogonal timelike Killing vector field. 
\end{definition}

\begin{idea}
In a static spacetime we know that $K$ is hypersurface-orthogonal, so when defining coordinates we choose $\Sigma$ to be orthogonal to $K$. Hence in the 'nice' coordintate $\Sigma$ is the surface $t=0$ Hence $K \propto (1,\bm{0}) \Rightarrow K_i = 0$. Now as $K^a = \left( \pd{t} \right)^a \Rightarrow g_{ti} = K_i = 0$.
\end{idea}

\subsection{Static and spherically symmetric}
Certainly $\exists K = (\pd{t})^a$ an orthogonal Killing vector field. Now for a point $p$ on the orthogonal hypersurface $\Sigma_t$, consider the $S^2$ closed orbit of $p$ ensures by spherical symmetry, and take coordinates $(\theta,\phi)$. Then giving points on that sphere coordinates $(\theta,\phi,R(p))$ gives $\Sigma_t$ the line element 
\[
ds^2_{\Sigma_t} = e^{2\psi(r)}dr^2 + r^2 d\Omega_2^2
\]
The entire line element is then 
\[
ds^2 = e^{2\Phi(r)}dt^2 + ds^2_{\Sigma_t}
\]
For fluids
\[
T_{ab} = (\rho + p)u_a u_b + pg_{ab}
\]
where $\rho(r)$ is the energy density and $p(r)$ the pressure. Take $u^a u_a = -1 \Rightarrow u_t^2 g^{tt}=-1 \Rightarrow u^a = e^{-\Phi}(\pd{t})^a$. Impose that $\rho,p >0$. Finally, as a star has finite extent
\[
\exists R \; s.t. \; \forall r > R \; \rho(r) = p(r) =0.
\]
Recall further energy momentum conservation give 
\[
\nabla^a T_{ab}
\]
Now the Einstein equation is 
\[
R_{ab} - \frac{1}{2}R g_{ab} = 8 \pi T_{ab}
\]
As the divergence of the LHS is 0, the equations of motion for the fluid are entirely captured by the einstein equations, so they must give the equations for fluid motion. 

\begin{example}
Take 
\[
G_{tt} = \frac{e^{2(\Phi-\psi)}}{r^2} \left[ e^{2\psi} + 2r\psi^\prime-1 \right]
\]
where $\phantom{\psi}^\prime = \pd{r}$. We need also the $rr$ and $\theta\theta$ terms, but then the $\phi\phi$ term is given from the $\theta\theta$ by spherical symmetry. 
\begin{definition}[m(r)]
Define 
\[
e^{2\psi} = \left[ 1- \frac{2m(r)}{r} \right]^{-1}
\]
\end{definition}
\begin{align*}
    tt &: m^\prime = 4\pi r^2 \rho \\ 
    rr &: \Phi^\prime = \frac{m+4\pi r^3 p}{r[r-2m]} \\
    \theta\theta &: p^\prime = -(\rho+p) \frac{m+4\pi r^3 p}{r[r-2m]}
\end{align*}
\begin{definition}[Equation of state]
For a cold system, the equation of state $p(\rho,T)$ gives $p=p(\rho)$
\end{definition}
Now outside the star $\rho=p=0$. So $m^\prime = 0 \Rightarrow m(r) = M$
\[
\Rightarrow \psi(r) = -\frac{1}{2}\log(1-\frac{2M}{r}) = -\Phi(r)
\]
Then this gives the line element 
\[
ds^2 = -(1-\frac{2M}{r}) dt^2 + (1-\frac{2M}{r})^{-1}dr^2 + r^2 d\Omega_2^2
\]
The mass of the system is $M$, given by perturbation about flat space. Due to the singularity in the metric, we see physical stars need radius $R>2m$. For our sun $2\msolar = 3km$, $R\approx 700000 km$. \\
Inside the star 
\[
m(r) = 4\pi \int_0^r \rho(\tilde{r}) \tilde{r}^2 d\tilde{r} + m_\ast
\]
where $m_\ast$ is an integration constant. Regularity demands $m_\ast = 0$. At the surface of the start, the metric must be continuous. Hence 
\[
M = 4\pi \int_0^R \rho(r) r^2 dr
\]
The energy that the stress energy tensor generates is 
\[
E = \int_{Vol_R} \rho r^2 \sin\theta e^\psi > M 
\]
Now imputting units into the radius bound we see 
\[
\frac{GM}{c^2 R} < \frac{1}{2}
\]
Note that there is no Newtonian analogue of this bound. 
\end{example}
Now for reasonable matter 
\[
\frac{dp}{d\rho} > 0
\]
and 
\[
\frac{dp}{dr}\leq 0 \Rightarrow \frac{d\rho}{dr} \leq 0
\]
Hence we may bound the $\theta\theta$ component of the Einstein equation  to get 
\[
\frac{m(r)}{r} < \frac{2}{9} \left[ 1 - 6\pi r^2 p + (1 + 6\pi r^2 p)^\frac{1}{2} \right]
\]
Evaluating this at $r=R$ gives 
\[
R > \frac{9}{4} M \quad \text{(Buchdal bound)}
\]

\begin{theorem}
The solutions for a 1 parameter group. 
\end{theorem}
\begin{proof}
Consider the equations 
\begin{align*}
    m^\prime = 4\pi r^2 \rho \\
    p^\prime = -(\rho + p) \frac{m+4\pi r^3 p}{r[r-2m]}
\end{align*}
These are a system of first order equations in $\rho$ and $m$. The solutions are determined by two constants, but we know $m(0)=0$ so the solution is determined by $\rho(0)$. We can integrate out to the surface of the star, so we know $R$ is determined by when $\rho(R) = 0$ so $R=R(\rho(0))$. Then at the surface we have also defined $M=M(\rho(0))$. All other functions are then determined by integrating backward. Hence done. 
\end{proof}

Now to determine these solutions, equations of state should be known, but we do not understand these correctly at high densities. Say for some region $0<r<r_0$ we do not know the equation of state, but for $r_0 <r < R$ (the envelope) we do know the equation of state. \\
At $\rho=\rho_0$, $m_0 \geq \frac{4\pi}{3} r_0^3 \rho_0$ (seen by looking at the $m^\prime$ equation). Now using the Buchdal bound we get 
\[
\frac{m_0}{r_0} < \frac{4}{9}
\]
If $\rho_0$ is atomic nuclei density, then we combine the bounds to get $m_0 < 5 \msolar$. It is a numerical fact that $m_0 \approx M $ to within $5\%$. 

%%%%%%%%%%%%%%%%%%%%%%%%%%%%%%%%%%%%%%%%%%%%%%%%%%%%%%%
%%%%%%%%%%%%%%%%%%%%%%%%%%%%%%%%%%%%%%%%%%%%%%%%%%%%%%%
\section{Schwarzschild Black Hole}
In Schwarzschild coordinates $(t,r,\theta,\phi)$, we required $r>2M$ and had 
\[
ds^2 = -(1-\frac{2M}{r}) dt^2 + (1-\frac{2M}{r})^{-1} dr^2 + r^2 d\Omega_2^2.
\]
The radius $r=2M$ is called the \bam{Schwarzschild radius}. 
\begin{theorem}[Birkhoff]
Any spherically symmetric solution of the vacuum Einstein equations is isometric to Schwarzschild. 
\end{theorem}
\begin{proof}
\[
ds^2 = -f(t,r)[dt + \chi(t,r)dr]^2 + \frac{dr^2}{g(t,r)} + r^2 d\Omega^2_2
\]
We have freed to transform $t \to E(t,R)$, so we may set $\chi=0$. Hence 
\[
ds^2 = -f(t,r)dt^2 + \frac{dr^2}{g(t,r)} + r^2 d\Omega^2_2
\]
We now only have the freedom $t \to p(t)$. Now from the $tr$ component of the Einstein equation
\[
0 = R_{tr}- \frac{1}{2} R g_{tr} \Rightarrow \del_t g = 0 \Rightarrow g=g(r)
\]
From the $tt$ component
\[
1-g-rr^\prime = 0 \Rightarrow g(r) = 1-\frac{2M}{r}
\]
From the $rr$ component 
\[
1-\frac{1}{g} + r\frac{f^\prime}{f} = 0 \Rightarrow f(t,r) = C(t)(1-\frac{2M}{r})
\]
We may then redefine $t$ so $C(t)=1$
\end{proof}


\subsection{Gravitational Redshift}
Consider two observers Alice and Bob at radii $r_A,r_B>@M$ respectively sharing angles. Let us assume Alice sends signals to Bob and that they are separated by a Schwarzschild coordinate time $\Delta t$. The proper time between photons emitted by A/B and measured by A/B is 
\[
\Delta\tau_{A/B} = \sqrt{1-\frac{2M}{r_{A/B}}}
\]
And thus 
\[
\frac{\Delta \tau_B}{\Delta\tau_A}=\sqrt{\frac{1-\frac{2M}{r_B}}{1-\frac{2M}{r_A}}}>1 \text{ for } r_B > r_A
\]
The interval is a proxy for wavelength $\lambda_b > \lambda_A$. If B is really far i.e. $r_B \gg 1$
\[
1+z = \frac{\lambda_B}{\lambda_A} = \frac{1}{\sqrt{1-\frac{2M}{r_A}}}
\]

\subsection{Geodesics of Schwarzschild}
Let $x^\mu(\lambda)$ be an affinely parametrised geodesic with tangent vector $u^\mu = \frac{dx^\mu}{d\lambda}$. Since $K=\pd{t}$ and $m=\pd{\phi}$ are Killing vectors. Then
\begin{align*}
    E &= -K\cdot u = (1-\frac{2m}{r}) \frac{dt}{d\tau} \\
    h &- m\cdot u = r^2 \sin^2\theta \frac{d\phi}{d\tau}. 
\end{align*}
are constant as, suppose $u^a \nabla_a u^b=0$ and $\nabla_a k_b + \nabla_b k_a = 0$, then 
\[
u^c \nabla_c (u^a k_a) = (u^c \nabla_c u^a)k_a + u^a u^c \nabla_c k_a = 0 
\]
For timelike particles if $\tau$ is proper time, then $E$ has the interpretation of energy and $h$ is angular momentum per unit mass. For null geodesics, only $b = \left\lvert \frac{h}{E} \right\rvert$ is physical, and has the interpretation of the impact factor\footnote{As a result, when working with calculation of null geodesics, where proper time has no interpretation, the answer should only depend on the impact factor}.
We also have $u^a u_a = \pm1,0$. \\
Now consider the action 
\[
S = \int d\tau \dot{x}^a \dot{x}^b g_{ab}
\]
Taking Euler Lagrange equations 
\[
\frac{d}{d\tau} \left( \pd[L]{\dot{\theta}} \right) - \pd[L]{\theta} = 0 \Rightarrow r^2 \frac{d}{d\tau}(r^2 \dot{\theta}) - \frac{\cos\theta}{\sin^3\theta} h^2
\] 
Now wlog we can ensure $\theta(0)=\frac{\pi}{2}$ and $\dot{\theta}(0)=0 \Rightarrow \ddot{\theta}(0)=0$. Then as the e.o.m is second order we know $\theta=\frac{\pi}{2}$ for all time. 
Hence substituting
\begin{align*}
g_{\mu\nu}\dot{x}^\mu \dot{x}^\nu = -\sigma \Rightarrow \frac{1}{2}\dot{r}^2 + V(r) = \frac{1}{2}E^2 \\
V(r) = \frac{1}{2} \left( \sigma +\frac{h^2}{r^2} \right) \left( 1-\frac{2M}{r} \right)
\end{align*}


\subsection{Eddington Finkelstein Coordinates}
Take $h=0$ and $E=1 \Rightarrow \dot{t} = (1-\frac{2M}{r})^{-1}, \dot{r} = \pm1$. For ther upper sign 
\[
\frac{\dot{t}}{\dot{r}} > 0, r>2M (outgoing)
\]
and for the lower 
\[
\frac{\dot{t}}{\dot{r}} < 0, r>2M (ingoing)
\]
From $\dot{r}=-1$, we can reach $r=2M$ without a problem. In Schwarzschild coordinates 
\[
\frac{dt}{dr} = \pm (1-\frac{2m}{r} )^{-1}
\]
We define Reggie-Wheeler (tortoise) coordinates
\[
dr_\ast = (1-\frac{2m}{r} )^{-1} dr \Rightarrow r_\ast = r+2M\log \left\lvert \frac{r}{2M} -1 \right\rvert
\]
\end{document}