\documentclass{article}

\usepackage{header}
%%%%%%%%%%%%%%%%%%%%%%%%%%%%%%%%%%%%%%%%%%%%%%%%%%%%%%%%
%Preamble

\title{Classical and Quantum Solitons Revision Notes}
\author{Linden Disney-Hogg}
\date{May 2019}

%%%%%%%%%%%%%%%%%%%%%%%%%%%%%%%%%%%%%%%%%%%%%%%%%%%%%%%%
%%%%%%%%%%%%%%%%%%%%%%%%%%%%%%%%%%%%%%%%%%%%%%%%%%%%%%%%
\begin{document}

\maketitle
\tableofcontents

\section{Introduction}
A brief overview of some key ideas, concepts, and facts that I find useful in revising CQS. 
%%%%%%%%%%%%%%%%%%%%%%%%%%%%%%%%%%%%%%%%%%%%%%%%%%%%%%%%
\subsection{Preliminaries}

\begin{definition}[Minkowski Metric]
In these notes the convention for the Mkinowski metric will be 
\eq{
\eta_{\mu\nu} = \diag(1,-1,\dots,-1)
}
i.e. the \bam{mostly minus metric}. 
\end{definition}

\begin{definition}[Soliton]
Solitons are stable, spatially localised, smooth, exact solutions of classical field equations in QFT. In relativistic theory they have energy $E$ and momentum $\bm{P}$ satisfying $E^2-\abs{\bm{P}}^2 = M^2$, $M$ the mass. As such they are particles. To get solitons we solve the equation exactly and quantise perturbatively.
\end{definition}

\begin{definition}[Moduli]
Solitons often occur in families of solutions described by a finite number of parameters. These are called the \bam{moduli} or \bam{collective coordinates}. The set of parameters corresponding to a family of solitons is called the \bam{moduli space} $\mc{M}$.
\end{definition}

\begin{definition}[Vacuum]
In this course we will consider Lagrangian densities with a potential term $U(\Psi)$. Assume that $U$ is bounded below by $0$. Values of $\Psi$ where $U(\Psi) = 0$ are called \bam{vacua}. The set 
\eq{
\mc{V} = \pbrace{\Psi_0 : U(\Psi_0) = 0}
}
is called the \bam{vacuum manifold}. 
\end{definition}

\begin{theorem}[Derrick's Theorem]
Let $\Psi$ be a non-vacuum field configuration defined of $\mbb{R}^n$ and let $E = E[\Psi]$ be the energy functional of the system. Define $\Psi_\lambda$ on $\mbb{R}^n$ for $\lambda\in\mbb{R}_{>0}$ by $\Psi_\lambda(\bm{x}) = \Psi(\lambda\bm{x})$ and $e(\lambda) = E[\Psi_\lambda]$. Then for $\Psi$ to be a static solution to the field equations it is necessary that 
\eq{
\frac{d}{d\lambda} e(\lambda) \rvert_{\lambda = 1} = 0
}
Hence, if $e(\lambda)$ has no stationary points, there are no soliton solutions. 
\end{theorem}

\begin{definition}[Bogomolny Equations]
If the energy can be bounded below using a sum-of-squares argument, the resulting energy bound is the \bam{Bogomolny energy bound}. The equations satisfied when the bound is saturated are called the \bam{Bogomolny equations}.
\end{definition}

\begin{definition}[Topological Charge]
Topological solitons come with an associated charge conserved topologically by the system. This gives stability to the solution. 
\end{definition}

%%%%%%%%%%%%%%%%%%%%%%%%%%%%%%%%%%%%%%%%%%%%%%%%%%%%%%%%
%%%%%%%%%%%%%%%%%%%%%%%%%%%%%%%%%%%%%%%%%%%%%%%%%%%%%%%%
\section{Kinks}

In 1+1 dimensions with scalar field $\phi:\mbb{R}^{1+1} \to \mbb{R}, \phi=\phi(x,t)$, we take the Lagrangian 
\eq{
L = \int_{-\infty}^\infty \psquare{\frac{1}{2} \del_\mu \phi \del^\mu \phi - U(\phi)} \, dx
}
for some potential function Split it as $L=T-V$ with 
\eq{
T &= \int \frac{1}{2} \dot{\phi}^2 \, dx \\ 
V &= \int \psquare{\frac{1}{2} {\phi^\prime}^2 + U(\phi)} \, dx 
}
where $\dot{\phi} = \pd[\phi]{t}, \phi^\prime = \pd[\phi]{x}$. The Euler-Lagrange equation is 
\eq{
\del_\mu \del^\mu \phi + \frac{dU}{d\phi} = 0 \quad \text{(nonlinear Klein Gordon equation)}
}
Using Noether's theorem, the energy can be found to be 
\eq{
E = T+V = \int \psquare{\frac{1}{2} \dot{\phi}^2 + \frac{1}{2} {\phi^\prime}^2 + U(\phi)} \, dx
}

\begin{prop}
Finite energy field configurations must tend to values in the vacuum manifold, i.e. 
\eq{
\lim_{x \to \pm\infty} \phi(x,t) \in \mc{V}
}
\end{prop}

\begin{definition}[Kinks]
If the vacuum manifold is discrete, \bam{kinks} are finite energy field configurations that connect different vacua. 
\end{definition}

\begin{definition}[Winding Number]
The current $j^\mu = \eps^{\mu\nu}\del_\nu \phi$, where $\eps^{\mu\nu}$ is the 2d alternating tensor, is conserved topologically, as $\del_\mu j^\mu = 0$ independently of the field equations. As such the \bam{winding number}
\eq{
Q = \int j^0 \, dx = \int \del_x \phi \, dx
}
is a conserved topological charge.
\end{definition}

\begin{prop}
When the target space of $\phi$,$\mc{M}$, is taken to be a manifold of finite volume, the winding number is found as the integral of the pullback of the normalised volume form $\omega$ on $\mc{M}$ under $\phi$. 
\end{prop}

\begin{example}
If $\phi:\mbb{R} \to S^1$, then the normalised volume form on $S^1$, $\frac{1}{2\pi}d\theta$, pulls back to $\frac{1}{2\pi}\frac{d\phi}{dx}dx$. We then have 
\eq{
Q = \int \frac{1}{2\pi} \frac{d\phi}{dx} \, dx
}
\end{example}

%%%%%%%%%%%%%%%%%%%%%%%%%%%%%%%%%%%%%%%%%%%%%%%%%%%%%%%%
\subsection{Static Kinks}

For a static solution, the field equations become
\eq{
\frac{d^2 \phi}{dx^2} = \frac{dU}{d\phi}
}

\begin{theorem}[Derrick's Theorem for Kinks]
For a static kink solutions, 
\eq{
\int \frac{1}{2} {\phi^\prime}^2 \, dx = \int U(\phi) \, dx
}
\end{theorem}

\begin{prop}[Bogomolny Equations for Kinks]
Assume we can write $U=\frac{1}{2}\pround{\frac{dW}{d\phi}}^2$. Then for a static field the energy is bounded below by
\eq{
E \geq \pm \psquare{W(\phi(\infty)) - W(\phi(-\infty))}
}
The bound is saturated when $\phi^\prime = \pm \frac{dW}{d\phi} = \pm\sqrt{2U}$. This is the Bogomolny energy bound, and the corresponding equation the Bogomolny equation
\end{prop}
\begin{proof}
We can write 
\eq{
E &= \frac{1}{2} \int \psquare{{\phi^\prime}^2 + \pround{\frac{dW}{d\phi}}^2} \, dx \\
&= \frac{1}{2} \int \pround{\phi^\prime \mp \frac{dW}{d\phi}}^2 \, dx \pm \int \frac{d\phi}{dx} \frac{dW}{d\phi} \, dx \\
&= \frac{1}{2} \int \pround{\phi^\prime \mp \frac{dW}{d\phi}}^2 \, dx \pm \psquare{W(\phi(\infty)) - W(\phi(-\infty))}
}
$E$ is then minimised when $\phi^\prime = \pm \frac{dW}{d\phi} = \pm\sqrt{2U}$, at which point it takes the value 
\eq{
E = \pm \psquare{W(\phi(\infty)) - W(\phi(-\infty))}
}
\end{proof}

\begin{remark}
Note that the Bogomolny equation implies Derrick's theorems for kinks, as $\frac{1}{2}{\phi^\prime}^2 = U(\phi)$ pointwise. 
\end{remark}

\begin{prop}
The Bogomolny equation for a static kink are equivalent to the equations of motion. 
\end{prop}

\begin{prop}
There are no static kink solutions that connect non-adjacent vacua.
\end{prop}

\begin{prop}
The Bogomolny equation is a first order differential equation, and so comes with an associated constant of integration. This corresponds to the location of the kinks, and is the modulus of the family of static solutions. Hence in this case $\mc{M} = \mbb{R}$. 
\end{prop}
%%%%%%%%%%%%%%%%%%%%%%%%%%%%%%%%%%%%%%%%%%%%%%%%%%%%%%%%
\subsection{Dynamic Kinks}
Sometimes exact solutions for non-static kinks can be obtained. Approximations can be obtained through an adiabatic point of view for low velocities $v \ll 1$ of the form 
\eq{
\phi(x,t) = \phi_0(x-a(t))
}
where $\phi_0$ is the static solution, and $a(t)$ is the time dependent modulus. This gives $\dot{\phi} = -\dot{a}\phi^\prime$, and so a reduced Lagrangian 
\eq{
L = \frac{1}{2} M \dot{a}^2 + \text{const}
}
The Euler-Lagrange equation for this reduced Lagrangian give
\eq{
M \ddot{a} &= 0 \\
\Rightarrow a(t) &= vt + \text{const}
}
From Noether's theorem, the momentum is 
\eq{
P = - \int \dot{\phi} \phi^\prime \, dx 
}
which under the moduli space approximation becomes 
\eq{
P = M\dot{a}
}
This agrees with the definition of momentum as the conjugate to the location given by 
\eq{
P = \pd[L]{\dot{a}} - L = \frac{P^2}{2M}
}
%%%%%%%%%%%%%%%%%%%%%%%%%%%%%%%%%%%%%%%%%%%%%%%%%%%%%%%%
\subsection{Quantisation}

The classical Hamiltonian corresponding to the reduced Lagrangian is 
\eq{
H = P \dot{a}
}
To quantise, replace $P$ with $-i\hbar \pd{a}$ so 
\eq{
H = - \frac{\hbar^2}{2M}\pds{a}
}
Stationary states $\psi = \psi(a)$ satisfy $H\psi = E\psi$. Hence 
\eq{
\psi(a) &= e^{ika} \\
P &= \hbar k \\
E &= \frac{\hbar^2 k^2}{2M} = \frac{P^2}{2M}
}
%%%%%%%%%%%%%%%%%%%%%%%%%%%%%%%%%%%%%%%%%%%%%%%%%%%%%%%%
%%%%%%%%%%%%%%%%%%%%%%%%%%%%%%%%%%%%%%%%%%%%%%%%%%%%%%%%
\section{Vortices}

\begin{definition}[Abelian Higgs Model for Vortices]
The \bam{abelian Higgs model} has Lagrangian density 
\eq{
\mc{L} = - \frac{1}{4} f_{\mu\nu}f^{\mu\nu} + \frac{1}{2} (D_\mu \phi)^\ast (D^\mu \phi) - \frac{\lambda}{8} (1 - \phi^\ast \phi)^2
}
where 
\begin{itemize}
    \item $\phi:\mbb{R}^2 \to \mbb{C}$
    \item $a_\mu : \mbb{R}^2 \to U(1)$ is a gauge field
    \item $f_{\mu\nu} = \del_\mu a_\nu - \del_\nu a_\mu$
    \item $D_\mu = \del_\mu - ia_\mu$
\end{itemize}
and the gauge transformation acts as
\eq{
\phi &\mapsto e^{i\alpha} \phi \\
a_\mu &\mapsto a_\mu + \del_\mu \alpha
}
for $\alpha : \mbb{R}^2 \to \mbb{R}$. 
\end{definition}

\begin{prop}
The Euler-Lagrange field equations for the abelian Higgs model are 
\eq{
\del_\mu f^{\mu\nu} &= \frac{i}{2} \psquare{(D^\nu \phi)^\ast \phi - \phi^\ast (D^\nu \phi)} \\
D_\mu D^\mu \phi &= \frac{\lambda}{2} (1-\phi^\ast \phi) \phi
}
\end{prop}

\begin{prop}[Ginzburg Landau Energy]
The energy functional in the abelian Higgs model is the \bam{Ginzburg Landau energy}
\eq{
E = \int d^2x \, \pbrace{\frac{1}{2}B^2 + \frac{1}{2} (D_1 \phi)^\ast(D_1 \phi) + \frac{1}{2} (D_2 \phi)^\ast(D_2 \phi) +\frac{\lambda}{8} (1 - \phi^\ast \phi)^2 }
}
where $B = \del_1 a_2 - \del_2 a_1$ is the \bam{magnetic field}. Note that the vacuum manifold is $\mc{V} = \pbrace{\abs{\phi} = 1} = S^1$. Hence a finite energy field configuration requires that as $\abs{\bm{x}}\to\infty$, $B\to 0$, $\abs{\phi}\to 0$, and $D\phi \to 0$. 
\end{prop}

\begin{definition}[Critical Coupling]
The behaviour of the system depends on the value of $\lambda$. 
\begin{itemize}
    \item $\lambda > 1 \Rightarrow$ Type II behaviour
    \item $\lambda < 1 \Rightarrow$ Type I behaviour
    \item $\lambda = 1 \Rightarrow$ \bam{Critical coupling}, get soliton states. 
\end{itemize}
\end{definition}

\begin{lemma}
We can write the spatial part of the gauge field as 
\eq{
a = a_1 dx^1 + a_2 dx^2
}
which gives 
\eq{
f = da = (\del_1 a_2 - \del_2 a_1) dx^1 \wedge dx^2
}
but 
\eq{
f = f_{12} dx^1 \wedge dx^2
}
and so $f_{12} = B$. In polar coordinates we get therefore $f_{r\theta} = r f_{12} = rB$. 
\end{lemma}

\begin{definition}[Winding Number]
Fixing the gauge of the solutions, let 
\eq{
\phi_\infty(\theta) = \lim_{r\to\infty} \phi(r,\theta) \,,
}
which exists. As $\abs{\phi_\infty} = 1$, define $\chi(\theta)$ such that $\phi_\infty(\theta) = e^{\chi(\theta)}$ and $\chi$ is continuous. Then for $\phi_\infty$ to be continuous we require 
\eq{
\chi(\theta+2\pi) &= \chi(\theta) \quad \text{(mod $2\pi$)} \\
\Rightarrow \chi(\theta+2\pi) &= \chi(\theta) + 2\pi N
}
for $N\in\mbb{Z}$. $N$ is the \bam{winding number} of $\phi$. 
\end{definition}

\begin{prop}
Winding number is gauge invariant. 
\end{prop}

\begin{prop}
$N$ is the number of isolated zeros of $\phi$ counted with multiplicity. They correspond to vortex centres. 
\end{prop}

\begin{prop}
\eq{
\int_{\mbb{R}^2} B \, d^2x = 2\pi N
}
\end{prop}
\begin{proof}
First note that as $D_\theta \phi \to 0$ as $r\to\infty$, 
\eq{
\del_\theta \phi_\infty - ia_\theta\rvert_{r=\infty} \phi_\infty &= 0 \\
\Rightarrow i(\del_\theta \chi - a_\theta \rvert_{r=\infty} )\phi_\infty &= 0 \\
\Rightarrow a_\theta\rvert_{r=\infty} &= \del_\theta \chi
}
Hence
\eq{
\int_{\mbb{R}^2} B \, d^2x = \int_{\mbb{R}^2} f &= \int_0^\infty \int_0^{2\pi} f_{r\theta} \, dr \, d\theta \\
&=  \int_0^\infty \int_0^{2\pi} (\del_r a_\theta - \del_\theta a_r) \, dr \, d\theta \\
&= \int_0^{2\pi} a_\theta \,d\theta \rvert_{r=\infty} \quad \text{(by Green's theorem)} \\ 
&= \int_0^{2\pi} \del_\theta \chi \, d\theta = \chi(2\pi) - \chi(0) = 2\pi N
}
\end{proof}
\begin{corollary}
Each vortex has flux $2\pi$. 
\end{corollary}

%%%%%%%%%%%%%%%%%%%%%%%%%%%%%%%%%%%%%%%%%%%%%%%%%%%%%%%%
\subsection{Critical Coupling}

\begin{prop}[Bogomolny Equations for Critically Coupled Vortices]
At critical coupling the Bogomolny energy bound is $E \geq \pi N$, saturated when the Bogomolny equations
\eq{
B - \frac{1}{2}(1- \phi^\ast \phi) &= 0 \\
D_1 \phi + i D_2 \phi &= 0 
}
are satisfied.
\end{prop}
\begin{proof}
\eq{
E = &  \frac{1}{2} \int_{\mbb{R}^2} \pbrace{B^2 + (D_1 \phi)^\ast (D_1 \phi) + (D_2 \phi)^\ast (D_2 \phi) + \frac{1}{4}(1-\phi^\ast \phi)^2} \, d^2x \\
= & \frac{1}{2} \int_{\mbb{R}^2} \pbrace{\psquare{B - \frac{1}{2}(1-\phi^\ast \phi)}^2 + (D_1 \phi + i D_2 \phi)^\ast (D_1 \phi + i D_2 \phi)} \, d^2 x \\
& + \frac{1}{2}\int_{\mbb{R}^2} \pbrace{B(1-\phi^\ast \phi) + i\psquare{(D_2 \phi)^\ast (D_1 \phi) - (D_1 \phi)^\ast (D_2\phi)}} \, d^2 x 
}
We will require the following two results 
\begin{lemma}[Covariant Leibniz Rule]
\eq{
\partial _ { i } \left( \phi ^ { * } D _ { j } \phi \right) = \left( D _ { i } \phi \right) ^ { * } D _ { j } \phi + \phi ^ { * } D _ { i } D _ { j } \phi
}
\end{lemma}
\begin{lemma}
\eq{
\left[ D _ { i } , D _ { j } \right] \phi = - i f _ { i j } \phi
}
\end{lemma}
The lemmas give that the second integral can be written as 
\eq{
\frac{1}{2} \int_{\mbb{R}^2} B \, d^2x + \frac{i}{2} \int_{\mbb{R}^2} \pbrace{ \del_2 (\phi^\ast D_1 \phi) - \del_1 (\phi^\ast D_2 \phi)} = \pi N
}
as the first integral is half of the flux, and the second term is 0 by stokes theorem. 
\end{proof}

\begin{theorem}[Taubes' Theorem]
The Bogomolny equation can be reduced to 
\eq{
\nabla^2 u - e^u + 1 = 4\pi \sum_{r=1}^N \delta^{(2)}(\bm{x}- \bm{x}_r) \quad \text{(\bam{Taubes' equation})}
}
where $\phi = e^{\frac{1}{2}u + i \chi}$, $\chi$ real, which has a unique solution for given $\pbrace{\bm{x}_r}$. 
\end{theorem}

\begin{prop}[Properties of Bogomolny Vortices]
Bogomolny vortices satisfy 
\begin{enumerate}
    \item Total magnetic flux is $2\pi N$, total energy is $\pi N$
    \item $B=\frac{1}{2}$, its maximum value at vortex centres. Hence each vortex has approximate area $4\pi$. 
    \item Far from the vortex centres, $u \to 0$, so Taubes' equation linearises to $\nabla^2 u - u = 0$, giving exponential decay. 
    \item $u$ is everywhere non-positive
\end{enumerate}
\end{prop}
\begin{proof}
We will only prove the fourth property. Note that $u$ has a maximum if $\nabla^2 u \leq 0$. At this point,
\eq{
e^u - 1 = \nabla^2 u &\leq 0 \\
\Rightarrow e^u &\leq 1 \\
\Rightarrow u &\leq 0
}
$u \leq 0 $ at maxima $\Rightarrow u \leq 0$ everywhere. 
\end{proof}

\begin{prop}
The moduli space of the Bogomolny vortices is $\mc{M} = \mbb{C}^N$.
\end{prop}
\begin{proof}
The moduli of the solution are the $\pbrace{\bm{x}_r}$ under permutation.
Let $z_r = (\bm{x}_r)_1 + i(\bm{x}_r)_2$, and define the complex polynomial 
\eq{
p(z) &= (z-z_1) \cdots (z-z_N) \\
&= z^N - (z_1 + \dots + z_N)z^{N-1} + (-1)^N(z_1 \cdots z_N)
}
The coefficients of the polynomial are invariant under permutation, and they are independent. Hence result. 
\end{proof}
%%%%%%%%%%%%%%%%%%%%%%%%%%%%%%%%%%%%%%%%%%%%%%%%%%%%%%%%
\subsection{Static Bogomolny Vortices on Curved Surfaces}
Let $\Sigma$ be a Riemann surface with a metric comptible with the complex structure. We required that, if $\Sigma$ has a boundary $\del \Sigma$, $\abs{\phi}\rvert_{\del\Sigma} = 1$

\begin{definition}
The energy in the abelian Higgs model on a curved surface $\Sigma$ with metric $g_{ij}$ is 
\eq{
E = \int _ { \Sigma } \left\{ \frac { 1 } { 4 } f _ { i j } f _ { k l } g ^ { i k } g ^ { j l } + \frac { 1 } { 2 } \left( D _ { i } \phi \right) ^ { * } D _ { j } \phi g ^ { i j } + \frac { \lambda } { 8 } \left( 1 - \phi ^ { * } \phi \right) ^ { 2 } \right\} \sqrt { \operatorname { det } g } d y ^ { 1 } d y ^ { 2 }
}
\end{definition}

\begin{definition}[Isothermal Coordinates]
Coordinates $y$ in which the metric takes the form 
\eq{
ds^2 = \Omega(y^1,y^2) \psquare{(dy^1)^2 + (dy^2)^2}
}
are called \bam{isothermal}. Corresponding complex coordinates are $z = y^1 + iy^2$, so 
\eq{
ds^2 = \Omega(z,\bar{z}) dz d\bar{z}
}
\end{definition}

\begin{prop}
Isothermal coordinates always exist for a surface. 
\end{prop}

\begin{prop}
The area element on $\Sigma$ in isothermal coordinates is 
\eq{
dA = \Omega \, d^2 y 
}
\end{prop}

\begin{prop}
The energy in the abelian Higgs model on $\Sigma$ in isothermal coordinates is  
\eq{
E = \frac{1}{2} \int_{\Sigma} \pbrace{ \Omega^{-1} B^2 + \abs{D_1 \phi}^2 + \abs{D_2\phi}^2 + \frac{1}{4} \Omega(1-\abs{\phi}^2)^2} \, d^2y
}
\end{prop}

\begin{definition}[1\textsuperscript{st} Cern Number]
The \bam{1\textsuperscript{st} Cern Number} is 
\eq{
N = \frac{1}{2\pi} \int_\Sigma f = \frac{1}{2\pi} \int_\Sigma B \, d^2 y 
}
\end{definition}

\begin{prop}
The Bogomolny equations on $\Sigma$ are 
\eq{
B - \frac{1}{2}\Omega(1-\abs{\phi}^2) &= 0 \\
D_1 \phi + i D_2 \phi &= 0 
}
and the corresponding Bogomolny energy is $E=\pi N$
\end{prop}
\begin{corollary}[Bradlow Bound on N]
Assume $\Sigma$ is closed with area $A$. Then 
\eq{
4\pi N \leq A
}
with equality if $\abs{\phi} = 0$ everywhere, and $B = \frac{1}{2}\Omega$. 
\end{corollary}
\begin{proof}
Integrate the first Bogomolny equation to get 
\eq{
\underbrace{2\int_\Sigma B \, d^2 y}_{4\pi N} + \underbrace{\int_\Sigma \abs{\phi}^2 \Omega \, d^2 y}_{\geq 0 } = \int_\Sigma \Omega \, d^2y = A
}
\end{proof}

\begin{prop}
Taubes' equation on $\Sigma$ is 
\eq{
\nabla^2 u - \Omega e^u + \Omega = 4\pi \sum_{r=1}^N \delta^{(2)}(\bm{y}-\bm{y}_r)
}
\end{prop}

\begin{prop}
Introduce the modified metric on $\Sigma$
\eq{
d\tilde{s}^2 = \tilde{\Omega}(y^1,y^2) \psquare{(dy^1)^2 + (dy^2)^2}
}
where $\tilde{\Omega} = \Omega \abs{\phi}^2 = \Omega e^u$. Then 
\eq{
\pround{\tilde{K} + \frac{1}{2}}e^u = K + \frac{1}{2}
}
i.e. 
\eq{
\pround{\tilde{K} + \frac{1}{2}} d\tilde{s}^2 = \pround{K + \frac{1}{2}} ds^2 
}
where $K,\tilde{K}$ are the corresponding Gaussian curvatures of the metrics on $\Sigma$. 
\end{prop}
\begin{proof}
We will need the following lemma.
\begin{lemma}
The Gaussian curvature on $\Sigma$ is 
\eq{
K = - \frac{1}{2\Omega} \nabla^2 \log \Omega
}
where $\nabla^2 = \del_1^2 + \del_2^2$. 
\end{lemma}
Then 
\eq{
\tilde{K} &= - \frac{1}{2\tilde{\Omega}}\nabla^2 \log \tilde{\Omega} \\
&= - \frac{1}{2\Omega e^u} \nabla^2(\log \Omega + u) \\
&= \frac{K}{e^u} - \frac{1}{2\Omega e^u} (\Omega e^u - \Omega) \\
&= \frac{1}{e^u}\pround{K + \frac{1}{2}} - \frac{1}{2}
}
\end{proof}

%%%%%%%%%%%%%%%%%%%%%%%%%%%%%%%%%%%%%%%%%%%%%%%%%%%%%%%%
%%%%%%%%%%%%%%%%%%%%%%%%%%%%%%%%%%%%%%%%%%%%%%%%%%%%%%%%
\section{Skyrmions}

\end{document}