\documentclass{article}

\usepackage{header}
%%%%%%%%%%%%%%%%%%%%%%%%%%%%%%%%%%%%%%%%%%%%%%%%%%%%%%%%
%Preamble

\title{Boundary Value Problems in Linear PDEs Revision Notes}
\author{Linden Disney-Hogg}
\date{December 2018}

%%%%%%%%%%%%%%%%%%%%%%%%%%%%%%%%%%%%%%%%%%%%%%%%%%%%%%%%
%%%%%%%%%%%%%%%%%%%%%%%%%%%%%%%%%%%%%%%%%%%%%%%%%%%%%%%%
\begin{document}

\maketitle
\tableofcontents

%%%%%%%%%%%%%%%%%%%%%%%%%%%%%%%%%%%%%%%%%%
%%%%%%%%%%%%%%%%%%%%%%%%%%%%%%%%%%%%%%%%%%
\section{Introduction}
A brief overview of some key ideas, concepts, and facts that I find useful in revising BVPs. 

\section{Complex Tools}
\begin{definition}[Fourier Transform]
Given some suitable function $f:\Omega \to \mbb{C}$ that is $L_1$ and $L_2$, its \bam{Fourier transform} is 
\eq{
\hat{f} : \mbb{R} \to \mbb{C} \\
\hat{f}(\lambda) = \int_\Omega e^{-i\lambda x} f(x) \, dx
}
and the inverse transform is 
\[
f(x) = \frac{1}{2\pi} \int_\mbb{R} e^{i\lambda x} \hat{f}(\lambda) \, d\lambda
\]
\end{definition}

\begin{lemma}[Jordan's Lemma]
Let $\gamma_R$ be the contour in $\mbb{C}$ for $R\in\mbb{R}_{geq 0}$
\eq{
& \gamma_R : [0,\pi] \to \mbb{C} \\
& \gamma_R(t) = Re^{it}
}
Then for $f:\mbb{C}\to\mbb{C}$ a holomorphic function satisfying 
\eq{
\forall z\in \gamma_R \quad |f(z)| < K(R) 
}
for some $K(R)$ s.t. $\lim_{R\to\infty} K(R) = 0$ then 
\eq{
\forall \lambda \in \mbb{R}_{>0} \quad  \lim_{R\to\infty} \int_{\gamma_R} e^{i\lambda z} f(z) \, dz = 0
}
An analogous result hold for $\lambda <0$ for a semicircular contour in the LHP. 
\end{lemma}

\begin{theorem}\label{thm:BVP:AnalyticIntegral}
Let $f : \mbb{C} \times \mbb{C} \to \mbb{C}, f=f(z,t)$ and $\gamma$ a contour in $\mbb{C}$. Define
\eq{
& F : \mbb{C} \to \mbb{C} \\
& F(z) = \int_\gamma f(z,t) \, dt
}
Then the conditions for $F$ to be defined and analytic in $D \subset \mbb{C}$ are 
\begin{itemize}
    \item $f$ is continuous in $z,t$,
    \item $F$ converges uniformly in $D$,
    \item $\forall t \; f(z,t)$ is analytic in $z$. 
\end{itemize}
\end{theorem}
%%%%%%%%%%%%%%%%%%%%%%%%%%%%%%%%%%%%%%%%%%
%%%%%%%%%%%%%%%%%%%%%%%%%%%%%%%%%%%%%%%%%%
\section{Heat Equation}
In this section the standard approach to a PDE 1st order in time, and higher order in space is discussed. At all points the heat equation will be used as the "lab rat" PDE the method will be tried out on. \\
\begin{definition}[Heat Equation]
The \bam{heat equation} is the partial differential equation for $u:\Omega \times [0,\infty) \to \mbb{R}$ $u=u(x,t)$
\[
u_t = u_{xx}
\]
Typical domains $\Omega$ are 
\begin{itemize}
    \item The half line $\Omega = [0,\infty)$
    \item The  finite interval $\Omega = [0,L]$ for $L\in\mbb{R}$
\end{itemize}
The problem will come with conditions on the boundary $\del \Omega \times [0,\infty) \cup \Omega\times \set{0}$. Standard notation is 
\eq{
u(x,0) &= u_0(x) \\
(\del_x)^j u(0,t) &= g_j(t) \\
(\del_x)^j u(L,t) &= h_j(t)
}
and it is always assumed that for fixed $t$, $u$ decays at spatial infinity.
\end{definition}

\begin{theorem}
The solution to the heat equation with boundary conditions $g_1, g_0$ and initial condition $u_0$ specified is 
\eq{
u ( x , t ) = \frac { 1 } { 2 \pi } \int _ { - \infty } ^ { \infty } e ^ { i \lambda x - \lambda ^ { 2 } t } \hat { u } _ { 0 } ( \lambda ) d \lambda - \frac { 1 } { 2 \pi } \int _ { \partial D ^ { + } } e ^ { i \lambda x - \lambda ^ { 2 } t } \left[ \tilde { g } _ { 1 } \left( \lambda ^ { 2 } , t \right) + i \lambda \tilde { g } _ { 0 } \left( \lambda ^ { 2 } , t \right) \right] \, d \lambda
}
where 
\begin{itemize}
    \item 
    \eq{
    \hat{u}_0(\lambda) &= \int_\mbb{R} e^{-i\lambda x}u_0(x) dx
    }
    \item 
    \eq{
    \tilde{g}_j(\lambda,t) &= \int_0^t e^{\lambda\tau} g_j(\tau) d\tau
    }
    \item 
    \eq{
    D^+ = \set{\lambda : \Im \lambda \geq 0, \Re \lambda^2 < 0 }
    }
\end{itemize}
\end{theorem}
\begin{proof}
By assuming sufficient "niceness" of the initial and boundary conditions, we apply Theorem \ref{thm:BVP:AnalyticIntegral} to see
\begin{itemize}
    \item As the range of the integral is finite, $\tilde{g}_j(\lambda,t)$ is analytic $\forall \lambda,t$. 
    \item As the integral for $\hat{u}_0$ is over all $x\in\mbb{R}$, for uniform convergence of the integral we require 
    \eq{
    \left| e ^ { - i \lambda x } \right| = \left| e ^ { - i \lambda _ { R } x + \lambda _ { I } x } \right| = e ^ { \lambda _ { I } x }
    }
    to remain bounded, which means
    \eq{
    \Im \lambda < 0 \,,
    }
    so $\hat{u}_0$ is analytic in the LHP. 
\end{itemize}
Now, we may utilise integration by parts to get 
\eq{
\hat{u}_t &= \int_0^\infty e^{-i\lambda x} u_t \, dx \\
&= \int_0^\infty e^{-i\lambda x} u_{xx} \, dx \\
&= \psquare{u_x e^{-i\lambda x} }_0^\infty - \int_0^\infty -i\lambda e^{-i\lambda x}u_x \, dx \\
&= \psquare{u_x e^{-i\lambda x} }_0^\infty +i\lambda \psquare{ u e^{-i\lambda x}}_0^\infty - i\lambda\int_0^\infty -i\lambda e^{-i\lambda x}u \, dx\\
&= \psquare{u_x e^{-i\lambda x} }_0^\infty +i\lambda \psquare{ u e^{-i\lambda x}}_0^\infty - \lambda^2 \hat{u} }
and so 
\eq{
\Rightarrow&& \hat{u}_t + \lambda^2 \hat{u} &= -g_1 -i\lambda g_0 \\
\Rightarrow&& (\hat{u} e^{\lambda^2 t} )_t &= -e^{\lambda^2 t} [g_1 +i\lambda g_0] \\
\Rightarrow&&  e^{\lambda^2 t}\hat{u}(\lambda,t) &= \hat{u}_0(\lambda) - \int_0^t e^{\lambda^2 \tau} [g_1(\tau) +i\lambda g_0(\tau)] \, d\tau \\
\Rightarrow&& e^{\lambda^2 t}\hat{u}(\lambda,t) &= \hat{u}_0(\lambda) - \psquare{\tilde{g}_1(\lambda^2,t) +i\lambda \tilde{g}_0(\lambda^2,t)}
}
This expression is called the \bam{Global Relation (GR)}. \\
The GR can be inverted then to find 
\eq{
u(x,t) = \int_\mbb{R} e^{i\lambda x - \lambda^2 t}\hat{u}_0(\lambda) \, d\lambda - \frac{1}{2\pi} \int_\mbb{R} e^{i\lambda x - \lambda^2 t} \psquare{\tilde{g}_1(\lambda^2,t) +i\lambda \tilde{g}_0(\lambda^2,t)} \, d\lambda
}
We are now nearly done, except for the contour on the second integral.
First consider 
\eq{
e^{i\lambda x - \lambda^2 t}\tilde{g}_1(\lambda^2,t) &= e^{i\lambda x - \lambda^2 t}\tilde{g}_1(\lambda^2,t)\int_0^t e^{\lambda^2 \tau} g_1(\tau) \, d\tau \\
&= e^{i\lambda x} \int_0^t e^{-\lambda^2 (t-\tau)} g_1(\tau) \, d\tau \\
\Rightarrow \; \abs{e^{i\lambda x - \lambda^2 t}\tilde{g}_1(\lambda^2,t)} &\leq e^{-x \Im \lambda }  \abs{\int_0^t g_1(\tau) \, d\tau} \sup_{\tau \in [0,t]} \pround{e^{- (t-\tau)\Re\lambda^2}}
}
It can be seen that, if $\Im \lambda \geq 0$ and $\Re \lambda^2 \geq 0$, then the term will be bounded as $\abs{\lambda} \to \infty$. Then integrating by parts
\eq{
\int_0^t e^{-\lambda^2 (t-\tau)} g_1(\tau) \, d\tau &=   \psquare{\frac{e^{-\lambda^2 (t-\tau)}}{\lambda^2} g_1(\tau)}_{\tau = 0}^{\tau = t}-\frac{1}{\lambda^2}\int_0^t e^{-\lambda^2 (t-\tau)} g_1^\prime(\tau) \\
&= \frac{g_1(t) - g_1(0)e^{-\lambda^2 t}}{\lambda^2} - \psquare{\frac{e^{-\lambda^2 (t-\tau)}}{\lambda^4} g_1^\prime(\tau)}_{\tau = 0}^{\tau = t} + \frac{1}{\lambda^4}\int_0^t e^{-\lambda^2 (t-\tau)} g_1^{\prime\prime}(\tau) \\
&\sim \frac{g_1(t)}{\lambda^2} \quad \text{as } \abs{\lambda}\to\infty
}
By the same argument, $e^{i\lambda x - \lambda^2 t} \cdot i\lambda\tilde{g}_0(\lambda^2,t)$ is bounded and has asymptotic expansion $\frac{ig_0(t)}{\lambda}$ as $\abs{\lambda}\to\infty$. 
\end{proof}

%%%%%%%%%%%%%%%%%%%%%%%%%%%%%%%%
%%%%%%%%%%%%%%%%%%%%%%%%%%%%%%%%
%%%%%%%%%%%%%%%%%%%%%%%%%%%%%%%%
%%%%%%%%%%%%%%%%%%%%%%%%%%%%%%%%
\begin{idea}
Note we deformed onto a new contour. The reason for this is that, when only some boundary conditions are given, this contour will be necessary for a substitution. See below. 
\end{idea}

%%%%%%%%%%%%%%%%%%%%%%%%%%%%%%
%%%%%%%%%%%%%%%%%%%%%%%%%%%%%%
% Generalisation 
\section{Generalisation}
This can be extended in general for a PDE on the half line
\eq{
u_t &= \psquare{\sum_{i=0}^n a_i (\del_x)^i} u
}
to \hl{FIX THIS BIT}
\eq{
\hat{u} &= \sum_{i=0}^n a_i \int_0^\infty  e^{-i\lambda x} (\del_x)^i u \\
&= \sum_{i=0} a_i \sum_{j=1}^{i} -(-i\lambda)^j g_{}
}
gives some polynomial $\omega(-i\lambda)$. \\
The solution for $u$ is then obtained via inverse Fourier transform 
\eq{
u(x,t) = \frac{1}{2\pi} \int_{\mbb{R}} e^{i \lambda x} \hat{u}(\lambda,t) \, d\lambda
}

Subsequently, in many examples\footnote{is this numerical integration we are referring to?} it will be easier to deform onto a contour in the complex plane to complete the $\lambda$ integral. This is possible if 
\begin{enumerate}
    \item The integrand is analytic through the region it is deformed through (requirement for Cauchy's theorem)
    \item The contribution from the end of the contours at $\infty$ is 0 (for which we use Jordan's lemma). 
\end{enumerate}
%%%%%%%%%%%%%%%%%%%%%%%%%%%%%%
%%%%%%%%%%%%%%%%%%%%%%%%%%%%%%r

\end{document}